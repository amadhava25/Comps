\documentclass{article}
\usepackage{../header}
\title{Geometry Comps Practice}
\author{\me}
\date{January 2, 2026}

\begin{document}
	\maketitle
	\fullline 
	\tableofcontents
	\halfline 
	\newpage 
	
\begin{prb}{2019-J-II-6}
	Let $X$ and $Y$ be vector fields on $\mathbb{R}^{3}$, defined by 
		\begin{equation}
			X = \frac{\partial}{\partial x} + x \frac{\partial}{\partial y} + y\frac{\partial}{\partial z} \quad \text{ and } \quad Y =y\frac{\partial}{\partial x} + z \frac{\partial}{\partial y} + \frac{\partial}{\partial z}. 
		\end{equation}
	Is there a coordinate chart $\varphi = (x_{1}, x_{2}, x_{3}): U \to \mathbb{R}^{3}$ of the origin $0 \in \mathbb{R}^{3}$ such that 
		\begin{equation}
			X|_{U} = \frac{\partial}{\partial x^{1}} \quad \text{ and } \quad Y|_{U} = \frac{\partial}{\partial x^{2}}. 
		\end{equation}
\end{prb}
\begin{solutions}
	No, there exists no coordinate chart containing the origin $0 \in \mathbb{R}^{3}$ that satisfies the above conditions. To show this, we will compute the Lie Brackets of the given vector fields; let $\ti{X} = \partial/\partial x^{1}$ and $\ti{Y} = \partial/\partial x^{2}$. First, we observe that 
		\begin{align}
			\begin{split}
				[X, Y] &= \frac{\partial}{\partial x}(y, z, 1) + x\frac{\partial}{\partial y}(y, z, 1) + y\frac{\partial}{\partial z}(y, z, 1) - y\frac{\partial}{\partial x}(1, x, y) -z \frac{\partial}{\partial y}(1, x, y) - \frac{\partial}{\partial z}(1, x, y) \\
				&= x(1,0,0) + y(0,1,0) - y(0,1,0) - z(0,0,1) \\
				&= x\frac{\partial}{\partial x} - z\frac{\partial}{\partial z}. 
			\end{split}
		\end{align}
	This means that the Lie Bracket of $X$ and $Y$ is not identically zero on any neighborhood of the origin. On the other hand, it is straightforward to see that the Lie Bracket of $\ti{X}$ and $\ti{Y}$ is identically zero on \textit{all} of $U$. This is a contradiction. Therefore, such a coordinate chart cannot exist. 
\end{solutions}	
\begin{prb}{2019-A-I-5}
	Let $H^{3}$ be the 3-dimensional Heisenberg group consisting of upper triangular $3 \times 3$ matrices, with 1's on the diagonal, and with the group operation being matrix multiplication. Let $\Gamma \subset H^{3}$ be the subgroup consisting of matrices all of whose entries are integers. Show that the quotient space $N = H^{3}/\Gamma$ is a closed 3-dimensional manifold. Show that there is a fiber bundle projection $P: N \to \mathbb{T}^{2}$ to a 2-dimensional torus $\mathbb{T}^{2} = S^{1} \times S^{1}$ with fiber $S^{1}$. Hint: Consider the center $Z$ of $H^{3}$. 
\end{prb}
\begin{solutions}
	Let $H^{3}$ be the 3-dimensional Heisenberg group consisting of upper triangular $3 \times 3$ matrices, with 1's on the diagonal, and with the group operation being matrix multiplication, and let $\Gamma$ be the subgroup consisting of matrices all of whose entries are integers. We proceed with the proof over several steps: 
		\begin{enumerate}[itemsep =-2pt,label = (\textbf{\arabic{*}})]
			\item ($H^{3}$ is a Lie Group): We claim that there is a one-to-one correspondence between the elements of $H^{3}$ and $\mathbb{R}^{3}$, given by the map 
				\begin{equation}
					\Phi: (a, b, c) \in \mathbb{R}^{3} \longmapsto \begin{pmatrix} 1 & a & b \\ 0 & 1 & c \\ 0 & 0 & 1 \end{pmatrix} 
				\end{equation}
			It is straightforward to verify that $\Phi$ is a bijection. In particular, since the coordinate functions of $\Phi$ are polynomials (as is for the inverse of $\Phi$), it follows that $\Phi$ is actually a diffeomorphism. Therefore, since $\mathbb{R}^{3}$ is a smooth manifold, $H^{3}$ must also be a smooth manifold. Now consider the product of two matrices in $H^{3}$:
				\begin{equation}
					\begin{pmatrix}
						1 & a & b \\ 0 & 1 & c \\ 0 & 0 & 1
					\end{pmatrix}
					\cdot 
					\begin{pmatrix}
						1 & a' & b' \\ 0 & 1 & c' \\ 0 & 0 & 1 
					\end{pmatrix}
					= 
					\begin{pmatrix}
						1 & a + a' & b + b' + ac' \\ 0 & 1 & c + c' \\ 0 & 0 & 1
					\end{pmatrix}.
				\end{equation}
			By means of the one-to-one correspondence $\Phi$, the multiplication becomes 
			\begin{equation}
				m: \mathbb{R}^{3} \times \mathbb{R}^{3}, \qquad m((a,b,c) \cdot (a', b',c')) = (a + a', b + b' + ac', c + c'). 
			\end{equation}
			Since all of the coordinate functions are polynomials, $m$ is a smooth map. Therefore, by way of $\Phi$, the multiplication map on $H^{3}$ is also smooth. Likewise, the inverse map can be seen to be a smooth map on $H^{3}$. Therefore, we conclude that $H^{3}$ is three-dimensional Lie group. 
			\item ($\Gamma$ is a discrete Lie Group): Trivially, every discrete group is a Lie group so that $\Gamma$ is a discrete Lie subgroup of $H^{3}$. 
			\item (Action of $\Gamma$ on $H^{3}$): Let $\Gamma$ act on the Lie group $H^{3}$ by left multiplication. First, we show that this action is free: 
				\begin{equation}
					\begin{pmatrix}
						1 & a & b \\ 0 & 1 & c \\  0 & 0 & 1 
					\end{pmatrix}
					\cdot 
					\begin{pmatrix}
						1 & a' & b' \\ 0 & 1 & c' \\ 0 & 0 & 1 
					\end{pmatrix}
					= \begin{pmatrix}
						1 & a' & b' \\ 0 & 1 & c' \\ 0 & 0 & 1 
					\end{pmatrix}
					\iff 
					\begin{cases}
						a + a' = a', & \\
						b + b' + ac' = b', & \\
						c + c' = c' & 
					\end{cases}
					\iff 
					a = 0, b = 0, c = 0. 
				\end{equation}
			Hence, $\begin{pmatrix}1 & a & b \\ 0 & 1 & c \\ 0 & 0 & 1 \end{pmatrix}$ is the identity. This proves that the action is free. That the group action is smooth follows trivially from the fact that matrix multiplication is smooth. Finally, we show that the group action is properly discontinuous. I.e., for a compact set $K \subset \mathbb{R}^{3}$, we want to show that the set $\brac*{\gamma \in \Gamma:(\gamma \cdot K) \cap K \neq \varnothing}$ is a finite set. By the one-to-one correspondence between $H^{3}$ and $\mathbb{R}^{3}$ we demonstrated in (\textbf{1}), our goal is to show that there exist finitely many 3-tuples $(m, n, p) \in \mathbb{Z}^{3}$ such that if $K$ is a compact set and $[-R, R]^{3}$ is a cube containing $K$, then $(m,n,p) \cdot K \cap K \neq \varnothing$. If $(m,n, p) \cdot K$ intersects $K$, then for some $(x, y, z) \in K$, 
				\begin{align}
					\begin{split}
						&-R \leq x + m \leq R \implies -2R \leq m \leq 2R. \\
						&-R \leq z + p \leq R \implies -2R \leq p \leq 2R. \\
						&-R \leq n + y + mz \leq R \implies -2R \leq n + mz\leq 2R.
					\end{split}
				\end{align}
			Since $m, n, p$ are integers and $R < \infty$, there exist only finitely many 3-tuples $(m,n,p)$ that satisfy the above conditions. Therefore, it follows that the action is properly continuous. 
		\end{enumerate}
	Therefore, by the Quotient Manifold Theorem (see Lee \textit{Introduction to Smooth Manifolds}, Theorem 9.16), $N = H^{3}/\Gamma$ is a smooth manifold of dimension $\dim{H^{3}} - \dim{\Gamma} = 3 - 0 = 3$. \textcolor{red}{[!! Complete Later !!]}
\end{solutions}
\begin{prb}{2023-A-II-5}
	Let $(t, x, y, z)$ be the standard coordinate system on $\mathbb{R}^{4}$, and let $\phi$ be the non-zero smooth 1-form on $\mathbb{R}^{4}$ defined by 
		\begin{equation*}
			\Phi = dt + y\;dx + z\;dy. 
		\end{equation*}
	Let $D$ be the 3-plane field on $\mathbb{R}^{4}$ that consists of tangent vectors $V$ such that $\Phi(V) = 0$. Is $D$ Frobenius integrable? Support your answer with a proof. 
\end{prb}
\begin{solutions}
	Let $D = \ker{\Phi} \subset T\mathbb{R}^{4}$, which must be a smooth 3-plane field on $\mathbb{R}^{4}$ since $\phi$ is nowhere zero. Note that since $\dim{T_{p}\mathbb{R}^{4}} = 4$ at any $p$ and $\dim{D_{p}} = \dim{\ker{\Phi_{p}}} = 4 - 1 = 3$, it follows that $\operatorname{codim}{D} = 3$. By the Frobenius Theorem, a codimension-one distribution $D = \ker{\Phi}$ is integrable iff $\Phi \wedge d\Phi = 0$. Computing $d\Phi$ first, we observe that: 
		\begin{align}
			\begin{split}
				d\Phi &= d(dt) + dy \wedge dx + dz\wedge dy \\
				&= -dx\wedge dy + dz\wedge dy. 
			\end{split}
		\end{align}
	Therefore, 
		\begin{align}
			\begin{split}
				\Phi \wedge d\Phi &= (dt + y\;dx + z\;dy) \wedge (-dx \wedge dy + dz\wedge dy) \\
				&= -dt \wedge dx \wedge dy + dt \wedge dz \wedge dy + y\;dx \wedge dz \wedge dy,  
			\end{split}
		\end{align}
	which is not identically zero everywhere on $\mathbb{R}^{4}$. Therefore, since $\Phi \wedge d\Phi \neq 0$, the Frobenius integrability condition fails, and so $D$ is not Frobenius integrable. 
\end{solutions}
\textbf{Remark:} Here, $D$ had codimension one. \textit{If} $D$ was a smooth distribution of codimension $k$, we can write $D = \ker{\phi^{1}, \ldots, \phi^{k}}$, where $\phi^{1}, \ldots, \phi^{k}$ are $k$ smooth 1-forms that are pointwise linearly independent, then $D$ is Frobenius integrable if and only if $\phi^{1}\wedge \dotsm \wedge \phi^{k} \wedge d\phi^{i} = 0$ for all $i$. 

\newpage 
\begin{prb}{2023-A-I-2}
	Let $f: T^{2} \to S^{2}$ be a smooth map from the 2-torus to the 2-sphere. Can $f$ be an immersion. If the answer is yes, give an explicit example. If the answer is no, then give a proof. 
\end{prb}
\begin{solutions}
	Let $f: T^{2} \to S^{2}$ be a smooth map from the 2-torus to the 2-sphere, and assume for the sake of an argument, that $f$ is an immersion. Since $\dim{T^{2}} = \dim{S^{2}} = 2$, $f$ must have constant rank 2. That is, for each $p \in T^{2}$, 
		\begin{equation}
			df_{p}: T_{p}T^{2} \to T_{p}S^{2} 
		\end{equation}
	is an isomorphism. Hence, by the Inverse Function Theorem, $f$ is a local diffeomorphism near $p$. Since local diffeomorphisms are open maps, it follows that $f(T^{2})$ is an open subset of $S^{2}$. On the other hand, since the impact of compact sets under continuous maps is compact, and $T^{2}$ is compact, $f(T^{2})$ is a compact, hence closed, subset of $S^{2}$. Since $S^{2}$ is connected, this implies that $f(T^{2}) = S^{2}$. Therefore, $f$ is a surjective local diffeomorphism, which means that $f$ is a covering map. Because $S^{2}$ is simply connected, any covering map onto $S^{2}$ must be a diffeomorphism. This implies that $T^{2} \cong S^{2}$. However, this is a contradiction since $\pi_{1}(T^{2}) \cong \mathbb{Z}^{2}$ and $\pi_{1}(S^{2}) = \{0\}$ and diffeomorphisms preserve fundamental groups. Hence, by contradiction, $f$ cannot be an immersion. 
\end{solutions}

\begin{prb}{2023-J-I-3}
	Show that if $M$ is a closed manifold that has an even dimensional sphere $S^{2n}$ as its universal cover, then its fundamental group $\pi_{1}(M)$ is either trivial or $\mathbb{Z}_{2}$. 
\end{prb}


\begin{prb}{2018-J-I-6}
	Consider the distribution in $\mathbb{R}^{3}$ spanned by the two vector fields 
		\begin{equation*}
			V = \partial_{x} + 2xy\partial_{z}, \qquad W = x\partial_{x} + \partial_{y} + (2x^{2}y + x^{2} - 2y)\partial_{z}. 
		\end{equation*}
	Show that this distribution is integrable and find an explicit formula for the integral submanifold passing through the point $(0,0,z_{0})$. 
\end{prb}
\begin{solutions}
	Let $D$ be the distribution in $\mathbb{R}^{3}$ spanned by the two vector fields $V$ and $W$ describe above. By the Frobenius Theorem, to show that this distribution is integrable, it suffices to show that the distribution is involutive. In other words, we merely have to show that the Lie Bracket of $V$ and $W$ is a smooth local section of $D$. For ease of notation, we denote $V$ and $W$ as $(1, 0, 2xy)$ and $(x, 1, (2x^{2}y + x^{2} - 2y))$, respectively. Then their Lie Bracket is:
		\begin{align}
			\begin{split}
				[V, W] &= \brac*{(1,0,2xy) \cdot (x, 1, (2x^{2}y + x^{2} - 2y))} - \brac*{(x, 1, (2x^{2}y + x^{2} - 2y))\cdot (1, 0, 2xy)} \\
				&= (1,0,4xy + 2x) - (0,0,2xy) - (0,0,2x) = (1,0,2xy) = V. 
			\end{split}
		\end{align}
	Hence, this shows that the distribution $D$ is involutive, and hence completely integrable. To find an explicit formula for the integral submanifold at some point $p \in \mathbb{R}^{3}$, since a 2-dimensional distribution in $\mathbb{R}^{3}$ is the kernel of a single 1-form $\omega$ (because the codimension of the distribution is 1), we start by finding an annihilator 1-form; i.e., a 1-form such that $\omega(V) = \omega(W) =0$. Let $\omega = A\;dx + B\;dy + dz$. Then 
		\begin{align}
			\begin{split}
				0 = \omega(V) &= A + 2xy \implies A = -2xy. \\
				0 = \omega(W) &= Ax + B + (2x^{2}y + x^{2} - 2y) \implies B = 2x^{2}y - 2x^{2}y - x^{2} + 2y = -x^{2} + 2y. 
			\end{split}
		\end{align}
	Hence, our annihilator 1-form is 
		\begin{equation}
			\omega = -2xy\;dx -(x^{2} - 2y)\;dy + dz. 
		\end{equation}
	On integral surfaces, $\omega = 0$. Hence, we observe that 
		\begin{equation}
			dz = 2xy\;dx + (x^{2} - 2y)\;dy \implies \frac{\partial z}{\partial x} = 2xy \text{ and }\frac{\partial z}{\partial y} = x^{2} - 2y. 
		\end{equation}
	From the first differential equation, we find 
		\begin{equation}
			z = x^{2}y + f(y). 
		\end{equation}
	From the second equation, 
		\begin{equation}
			x^{2} + \frac{df}{dy} = x^{2} - 2y \implies \frac{df}{dy} = -2y \implies f(y) = -y^{2} + c. 
		\end{equation}
	Therefore, we obtain 
		\begin{equation}
			z = x^{2}y - y^{2} + c. 
		\end{equation}
	Plugging in the point $(0,0,z_{0})$, we obtain $c = z_{0}$. Hence, the explicit formula for the integral submanifold passing through the point $(0,0,z_{0})$ is given by 
		\begin{equation}
			x^{2}y - y^{2} -z = z_{0}. 
		\end{equation}
\end{solutions}
\begin{prb}{2019-J-11-6}
	Let $X$ and $Y$ be vector fields on $\mathbb{R}^{3}$, defined by 
		\begin{equation*}
			X = \frac{\partial}{\partial x} + x\frac{\partial}{\partial y} + y\frac{\partial}{\partial z} \qquad \text{and} \qquad Y = y\frac{\partial}{\partial x} + z\frac{\partial}{\partial y} + \frac{\partial}{\partial z}. 
		\end{equation*}
	Is there a coordinate chart $\varphi = (x_{1}, x_{2}, x_{3}): U \to \mathbb{R}^{3}$ of the neighborhood of the origin $0 \in \mathbb{R}^{3}$ such that 
		\begin{equation*}
			X|_{U} = \frac{\partial}{\partial x_{1}} \qquad \text{and} \qquad Y|_{U} = \frac{\partial}{\partial x_{2}}.
		\end{equation*}
\end{prb}
\begin{solutions}
	We claim that there does \textit{not} exist such a coordinate chart. Suppose to the contrary that there do exist such coordinates on a neighborhood $U$ of the origin $0 \in \mathbb{R}^{3}$. We observe that 
		\begin{equation}
			[X|_{U}, Y|_{U}] = \frac{\partial}{\partial x_{1}}\left(\frac{\partial}{\partial x_{2}}\right) - \frac{\partial}{\partial x_{2}}\left(\frac{\partial}{\partial x_{1}}\right) = 0. 
		\end{equation}
	I.e., the Lie Bracket of the vector fields in these coordinates vanish everywhere on $U$. On the other hand, computing the Lie Bracket of $X$ and $Y$ in the original coordinates, 
		\begin{align}
			\begin{split}
				[X, Y] &= (1, x, y) \cdot (y, z, 1)  - (y, z, 1) \cdot (1,x,y)\\
				&= (0,0,0) + (x,0,0) + (0,y,0) - (0,y,0) - (0,0,z) - (0,0,0) \\
				&= (x,0,-z) = x\frac{\partial}{\partial x} - z\frac{\partial}{\partial z}. 
			\end{split}
		\end{align}
	Since the Lie Bracket of $X$ and $Y$ is not identically zero on $U$, we have reached a contradiction. Therefore, by contradiction, we see that such a coordinate chart cannot exist. 
\end{solutions}

\begin{prb}{(Comps Lemma)}
	Let $M, N$ be smooth connected $n$-manifolds and let $f: M \to N$ be a (smooth) immersion. If $M$ is compact and nonempty, then $N$ is compact and $f$ is a (smooth) covering map. 
\end{prb}
\begin{solutions}
	Let $M, N$ be smooth connected $n$-manifolds and let $f: M \to N$ be an immersion. By definition, $df_{p}: T_{p}M \to T_{f(p)}N$ is injective for each $p$, so that the map $df_{p}$ is an isomorphism for each $p$. The inverse function theorem implies that $f$ is a local diffeomorphism. Then $f(M)$ is both open (since local diffeomorphisms are open maps) and closed (compact subset of a Hausdorff space). Therefore, $f(M)$ must be the whole space since $N$ is connected and $f(M)$ is nonempty, which shows that $N$ must be compact. 
	
	For $f$ to be a covering map, it remains to show that $N$ is evenly covered. Let $q \in N$. Since $f$ is a local diffeomorphism, $f^{-1}(q) \subset M$ is closed and discrete. Hence, $f^{-1}(q)$ is a finite set $\{x_{1}, \ldots, x_{s}\}$. Let $U_{i} \subset M$ be an open neighborhood of $x_{i}$ such that $U_{i} \cap U_{j} = \varnothing$; such a collection is possible to find since $M$ is Hausdorff; shrink each $U_{j}$ if needed so that $f|_{U_{j}} \to U_{j}$ is a diffeomorphism for each $j$. Set $V = \bigcap_{1}^{s}f(U_{i})$. Then $V$ is an evenly covered neighborhood of $q$ in $N$. 
\end{solutions}
\begin{prb}{2019-A-I-4}
	Let $f: \mathbb{RP}^{3} \to \mathbb{T}^{3} = S^{1} \times S^{1} \times S^{1}$ be a smooth map. Show that $f$ is not an immersion. 
\end{prb}
\begin{solutions}
	Let $f: \mathbb{RP}^{3} \to \mathbb{T}^{3} = S^{1} \times S^{1} \times S^{1}$ be a smooth map. Assume to the contrary that $f$ is an immersion. First, we prove the \textit{comps lemma}, which we will use to develop our argument. 
	\begin{quote}
		(\textbf{Comps Lemma}) Let $M$ and $N$ be smooth connected $n$-manifolds, and $f: M \to N$ a (smooth) immersion. If $M$ is compact and nonempty, then $N$ is compact and $f$ is a (smooth) covering map. 
	\end{quote}
	\begin{proof}
		Let $M, N$ be smooth connected $n$-manifolds, $f: M \to N$ an immersion, and $M$ compact and nonempty. Since $f$ is an immersion, the map $df_{p}: T_{p}M \to T_{f(p)}N$ is injective at each $p \in M$ so that $f$ is a local diffeomorphism. This implies that $f(M)$ is open in $N$ (continuous image of an open set) and $f(M)$ is closed in $N$ (continuous image of a compact set is compact, and compact subsets of Hausdorff spaces are closed). Since $N$ is connected and $M$ is nonempty, $f(M) = N$. Therefore, $N$ is compact. 
		
		To show that $f$ is a covering map, it remains to be shown that $N$ is evenly covered. Let $q \in N$ be arbitrary but fixed. Since $f$ is a local diffeomorphism, $f^{-1}(q) \subset M$ is closed and discrete, which means $f^{-1}(q) = \{x_{1}, \ldots, x_{s}\}$ for some finite $s$ and $x_{j} \in M$. Since $M$ is Hausdorff, we may pick a collection $\brac*{U_{j}}_{j = 1}^{s}$ of open subsets of $M$ such that $x_{j} \in U_{j}$ for each $j$, and $U_{i} \cap U_{j} = \varnothing$ for each $i \neq j$; shrink each $U_{j}$ if needed so that $f|_{U_{j}} \to f(U_{j})$  is a diffeomorphism. Set $V = \bigcap_{j = 1}^{s}f(U_{j})$ so that $V$ is an evenly covered neighborhood of $q \in N$. 
	\end{proof}
	
	In our case, $\mathbb{RP}^{3}$ is a smooth connected compact nonempty 3-manifold, and $\mathbb{T}^{3}$ is a smooth connected 3-manifold. By the Comps Lemma, $\mathbb{T}^{3}$ is compact and $f$ is a smooth covering map. Consider the induced \textit{injective} homomorphism $f_{\ast}: \pi_{1}(\mathbb{RP}^{3}) \to \pi_{1}(\mathbb{T}^{3})$. Since $\pi_{1}(\mathbb{RP}^{3}) = \mathbb{Z}/2\mathbb{Z}$, $\pi_{1}(\mathbb{T}^{3}) = \mathbb{Z} \times \mathbb{Z} \times \mathbb{Z}$, and the former has torsion while the latter does not since each subgroup of $\mathbb{Z}^{3}$ is free abelian, there cannot exist such an injective homomorphism. Hence, by contradiction, we must have that $f$ cannot be a immersion. 
\end{solutions}
\begin{prb}{2017-J-II-1}
	Let $f: M \to \mathbb{R}$ be a smooth function on a smooth manifold $M$. In an arbitrary smooth local coordinate chart $\pi: U \to \mathbb{R}^{n}$ of $M$, define 
		\begin{equation}
			\mathscr{D}f \coloneqq \sum_{i = 1}^{n}\frac{\partial f}{\partial x^{i}}\frac{\partial}{\partial x^{i}}. 
		\end{equation}
	Does $\mathscr{D}f$ give a well-defined vector field on $M$? 
\end{prb}
\begin{solutions}
	No, $\mathscr{D}f$ does \textit{not} give a well-defined vector field on $M$. In fact, we claim that $\mathscr{D}f$ does not transform covariantly. Let $(U, (x^{i}))$ and $(V, (\ti{x}^{i}))$ be smooth local coordinate charts on $M$, and let $p \in U \cap V$. In the remainder of the proof, we shall use Einstein Summation Convention. We find that 
		\begin{align}
			\begin{split}
				\mathscr{D}F &= \frac{\partial f}{\partial x^{i}}(p)\frac{\partial}{\partial x^{i}}\bigg|_{p} \\
				&= \left(\frac{\partial f}{\partial \ti{x}^{j}}(\hat{p})\frac{\partial \ti{x}^{j}}{\partial x^{i}}\bigg|_{p}\frac{\partial \ti{x}^{k}}{\partial x^{i}}\right)\frac{\partial}{\partial \ti{x}^{k}}\bigg|_{\hat{p}} \\
				&\neq \frac{\partial f}{\partial \ti{x}^{k}}(\hat{p})\frac{\partial}{\partial \ti{x}^{k}}\bigg|_{\hat{p}} = \mathscr{D}f, 
			\end{split}
		\end{align}
	which is a contradiction. Therefore, $\mathscr{D}f$ does not give a well-defined vector field on $M$. 
\end{solutions}
\begin{prb}{2012-J-I-4}
	Let $x, y, z$ be the usual coordinates on $\mathbb{R}^{3}$. Consider the 1-form on $\mathbb{R}^{3}$ given by 
		\begin{equation*}
			\varphi = dx + ydz. 
		\end{equation*}	
	Is it possible to find smooth functions $u$ and $v$ on $\mathbb{R}^{3}$ such that $\varphi = udv$? Why? 
\end{prb}
\begin{solutions}
	Let $\varphi_{1} = dx + ydz$ and $\varphi_{2} = udv$ for some smooth functions $u$ and $v$ on $\mathbb{R}^{3}$. Then, we observe that 
		\begin{align}
			\begin{split}
				d\varphi_{1} &= 
			\end{split}
		\end{align}
\end{solutions}


\end{document}