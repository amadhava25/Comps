\documentclass{article}
\usepackage{../header}
\usepackage{notomath}
\setcounter{secnumdepth}{0}
\title{Comps Practice}
\author{\me}
\date{January 2, 2026}

\begin{document}
	\maketitle
	\fullline 
	\tableofcontents
	\halfline 
	\newpage 
	
\subsection{Analysis}
\begin{prb}{2019-J-I-4}
	Let $(X, \mu)$ be a measure space and $f \in L^{1}(X, \mu) \cap L^{\infty}(X, \mu)$. Show that $f \in L^{q}(X, \mu)$ for all $q > 1$ and $$\norm{f}_{\infty} = \lim_{q \to \infty}\norm{f}_{q}.$$
\end{prb}
\begin{solutions}
	Let $(X, \mu)$ be a measure space, and assume $f \in L^{1}(X) \cap L^{\infty}(X)$. Let $q > 1$. Then since $\abs{f}^{q} = \abs{f} \cdot \abs{f}^{q - 1}$ and $\abs{f} \leq \norm{f}_{\infty}$ a.e., we observe that 
		\begin{equation}
			\int \abs{f}^{q} = \int \abs{f} \cdot \abs{f}^{q - 1} \leq \norm{f}_{\infty}^{q - 1}\int \abs{f} = \norm{f}_{1}\norm{f}_{\infty}^{q - 1}. 
		\end{equation}
	Taking the $q\textsuperscript{th}$ roots on both sides: 
		\begin{equation}
			\norm{f}_{q} = \left[\int \abs{f}^{q}\right]^{1/q} \leq \norm{f}_{1}^{1/q}\norm{f}_{\infty}^{1 - 1/q}.
		\end{equation}
	Since $f \in L^{1} \cap L^{\infty}$, $\norm{f}_{1}, \norm{f}_{\infty} < \infty$ so that $\norm{f}_{q} < \infty$. Therefore, $f \in L^{q}(X, \mu)$. From the above expression, it is clear to see that 
		\begin{equation}
			\lim_{q \to \infty}\norm{f}_{q} \leq \lim_{q \to \infty}\norm{f}_{1}^{1/q} \norm{f}_{\infty}^{1 - 1/q} = \norm{f}_{\infty}. 
		\end{equation} 
	Hence, it suffices to show that the reverse inequality is satisfied. Let $\epsilon > 0$ and define the set 
		\begin{equation}
			A_{\epsilon} = \brac*{x: \abs{f(x)} \geq \norm{f}_{\infty} - \epsilon}. 
		\end{equation}
	By definition of the essential supremum, $A_{\epsilon}$ is a set of positive measure. Therefore, 
		\begin{equation}
			\norm{f}_{q}^{q} = \int \abs{f}^{q} \geq \int_{A_{\epsilon}}\abs{f}^{q} \geq \mu(A_{\epsilon}) \cdot (\norm{f}_{\infty} - \epsilon)^{q}. 
		\end{equation}
	Taking the $q\textsuperscript{th}$ roots on both sides, 
		\begin{equation}
			\norm{f}_{q} \geq \mu(A_{\epsilon})^{1/q}\cdot (\norm{f}_{\infty} - \epsilon). 
		\end{equation}
	Taking the limit $q \to \infty$ on both sides, 
		\begin{equation}
			\lim_{q \to \infty}\norm{f}_{q} \geq \norm{f}_{\infty} - \epsilon. 
		\end{equation}
	Since $\epsilon > 0$ was arbitrary, taking $\epsilon \to 0$ gives us the reverse inequality; namely, that $\lim_{q \to \infty}\norm{f}_{q} \geq \norm{f}_{\infty}$. 
\end{solutions}
\begin{prb}{2024-J-I-6}
	Let $f$ and $g$ be Lebesgue-measurable functions on $\mathbb{R}$. Define the convolution 
		\begin{equation*}
			(f \ast g)(x) = \int_{\mathbb{R}}f(x -y)g(y)\;dy
		\end{equation*}
	for all $x$ such that the integral exists. Prove that if $f \in L^{p}(\mathbb{R})$ and $g \in L^{q}(\mathbb{R})$ with $p, q \in (1, \infty)$ satisfying $\frac{1}{p} + \frac{1}{q} =1$, then $f \ast g$ is a bounded continuous function on $\mathbb{R}$. 
\end{prb}
\begin{solutions}
	Let $f \in L^{p}(\mathbb{R})$ and $g \in L^{q}(\mathbb{R})$, where $p, q \in (1, \infty)$ are conjugate exponents. Furthermore, define the convolution $(f \ast g)(x)$ as described above. Then by H\"older's inequality, for any fixed $x \in \mathbb{R}$, 
		\begin{equation}
			\abs{(f \ast g)(x)} = \abs{\int_{\mathbb{R}}f(x - y)g(y)\;dy} \leq \int_{\mathbb{R}}\abs{f(x - y)}\abs{g(y)}\;dy \leq \norm{f(x - \cdot)}_{p}\norm{g}_{q}. 
		\end{equation}
	Since $L^{p}$ norms are translation invariant, $\norm{f(x - \cdot)}_{p} = \norm{f}_{p}$. Therefore, for any given $x \in \mathbb{R}$, $\abs{(f \ast g)(x)} \leq \norm{f}_{p}\norm{g}_{q} = M < \infty$, where $M$ is finite since $f \in L^{p}$ and $g \in L^{q}$. Hence, this shows that $\norm{f \ast g}_{\infty}\leq M$, which means that $f \ast g$ is bounded. To show continuity, we use the fact that translations are continuous in the $L^{p}$ norm; i.e., if $\tau_{y}f(x) = f(x - y)$ denote the translation of $f(x)$, then $\lim_{y \to 0}\norm{\tau_{y}f - f}_{p} \to 0$ for all $p \in (1, \infty)$. In our case, 
		\begin{align}
			\begin{split}
				\norm{\tau_{y}(f \ast g) - (f \ast g)}_{\infty} &= \norm{(\tau_{y}f - f) \ast g}_{\infty} \leq \norm{\tau_{y}f - f}_{p}\norm{g}_{q} \longrightarrow 0 \text{ as $y \to 0$}, 
			\end{split}
		\end{align}
	which shows that $f \ast g$ is uniformly continuous, and hence continuous on $\mathbb{R}$. 
\end{solutions}
\begin{prb}{2023-A-I-6}
	Suppose $f: [-1,1] \to \mathbb{R}$ is odd and $C^{1}$. Show 
		\begin{equation*}
			\int_{-1}^{1}\abs{f(x)}^{2}\;dx \leq \int_{-1}^{1}\abs{f'(x)}^{2}\;dx. 
		\end{equation*}
\end{prb}
\begin{solutions}
	Let $f: [-1,1] \to \mathbb{R}$ be odd and $C^{1}$. Then since $|f(x)|^{2}$ is even, 
		\begin{equation}
			\int_{-1}^{1}|f(x)|^{2}\;dx = 2\int_{0}^{1}|f(x)|^{2}\;dx. 
		\end{equation}
	Hence, it suffices to examine the integral $\int_{0}^{1}|f(x)|^{2}\;dx$. Using Cauchy-Schwarz, we observe that 
		\begin{align}
			\begin{split}
				\abs{f(x)}^{2} &= \abs{\int^{x}f'(t)\;dt}^{2} \leq \left(\int^{x}1^{2}\;dt\right) \cdot \left(\int^{x}|f'(t)|^{2}\;dt\right) \\
				&= x \cdot \int^{x}\abs{f'(t)}^{2}\;dt. 
			\end{split}
		\end{align}
	Integrating over $[0,1]$, 
		\begin{align}
			\begin{split}
				\int_{0}^{1}\abs{f(x)}^{2}\;dx &= \int_{0}^{1}x\int_{0}^{x}\abs{f'(t)}^{2}\;dt\;dx \\
				&= \int_{0}^{1}\abs{f'(t)}^{2}\int_{t}^{1}x\;dx\;dt \\
				&= \int_{0}^{1}\abs{f'(t)}^{2} \cdot \frac{1 - t^{2}}{2}\;dt \leq \frac{1}{2}\int_{0}^{1}\abs{f'(t)}^{2}\;dt \\
				\implies \int_{-1}^{1}\abs{f'(x)}^{2}\;dx = 2\int_{0}^{1}\int \abs{f(x)}^{2}\;dx  &\leq \int_{0}^{1}\abs{f'(t)}^{2}\;dt \leq \int_{-1}^{1}\abs{f'(t)}^{2}\;dt \eqqcolon \int_{-1}^{1}\abs{f'(x)}^{2}\;dx, 
			\end{split}
		\end{align}
	where we used Fubini/Tonelli to swap the integrals in the second line. 
\end{solutions}
\begin{prb}{2024-J-II-5}
	Let $P$ be the vector space over $\mathbb{R}$ of (finite degree) polynomials in the variable $x \in (-\infty,\infty)$. Show that $P$ cannot be a Banach space with respect to any norm, that is, if $\norm{\empspace}$ is some norm on $P$, then $P$ is not complete under this norm. Hint: You may use the Baire Category Theorem. 
\end{prb}
\begin{solutions}
	Let $P$ be the vector space over $\mathbb{R}$ of finite-degree polynomials in the variable $x \in (-\infty, \infty)$. We may write 
		\begin{equation}
			P = \bigcup_{n \in \mathbb{N}}P_{n}, 
		\end{equation}
	where $P_{n}$ denotes the space of all polynomials in $P$ with degree at most $n$. Assume to the contrary that $P$ is a Banach space with respect to the norm $\norm{\empspace}$. Then since any norm induces a metric, $P$ is a complete metric space. By the Baire Category Theorem, there exists at least one positive integer $m$ such that $P_{m}$ is not nowhere dense in $P$; i.e., its closure $\overline{P}_{m}$ has nonempty interior. Since $P_{m}$ is a finite-dimensional vector subspace of the normed space $P$, it follows that $P_{m}$ is closed in $P$. Therefore, $P_{m} = \overline{P}_{m}$ has nonempty interior. Let $p \in P_{m}$; since $P_{m}$ has nonempty interior, it contains a ball $B(p, r) = \brac*{q \in P: \norm{p - q} < r}$ for some $r > 0$. Let $u \in P\setminus\{0\}$ be an arbitrary finite-degree polynomial, and set 
		\begin{equation}
			u' = p + \frac{r \cdot u}{2\norm{u}}, 
		\end{equation}
	so that $u' \in B(p, r) \subset P_{m}$. But since $P_{m}$ is closed, it follows that $u = (u' - p) \cdot \frac{2\norm{u}}{r} \in P_{m}$. Since $u \in P$ was arbitrary and $P_{m} \subseteq P$, it follows that $P = P_{m}$, which is impossible. Therefore, every $P_{n}$ has empty interior, which  contradicts Baire's Category Theorem. Hence, $P$ cannot be a Banach space with respect to the norm $\norm{\empspace}$. 
\end{solutions}
\begin{prb}{(Steinhaus' Theorem)}
	Let $E \subset \mathbb{R}^{n}$ be a set of positive Lebesgue measure. Then the set $E - E = \brac*{a - b: a, b \in E}$ contains a neighborhood around the origin. 
\end{prb}
\begin{solutions}
	Let $E \subset \mathbb{R}^{n}$ be a subset of positive Lebesgue measure, and let $E - E \coloneqq \{e_{1} - e_{2}: e_{1},e_{2} \in E\}$.  Since $E$ has positive Lebesgue measure, by the Lebesgue Differentiation Theorem, there exists $x \in E$ such that for any $\epsilon > 0$, there exists a $r > 0$ satisfying the following inequality: 
	\begin{equation}
		m(E \cap \b{r}) \geq (1 - \epsilon)m(\b{r}) \implies m(\b{r}\setminus E) \leq \epsilon m(\b{r}). 
	\end{equation}
	Assume to the contrary that $E - E$ does \textit{not} contain a neighborhood about the origin, which means that there exists a sequence $x_{n} \to 0$ so that $x_{n} \not \in E - E$ for any $n$; equivalently $(E + x_{n}) \cap E = \varnothing$ for all $n$. Therefore, our goal is to force a contradiction to show that $(E + x_{n}) \cap E \neq \varnothing$ for some $n$. First, observe that 
	\begin{align}
		\begin{split}
			0 = \lim_{n \to \infty}m(\b{r}\Delta (\b{r} - x_{n})) &= \lim_{n \to \infty}m(\b{r} \cup (\b{r} - x_{n})) \\
			&\qquad- \lim_{n \to \infty}m(\b{r}\cap  (\b{r} - x_{n})) \\
			&= m(B_{r}(x)) - \lim_{n \to \infty}m(\b{r} \cap (\b{r} - x_{n})) \\
			\implies m(\b{r}) &= \lim_{n \to \infty}m(\b{r} \cap (\b{r} - x_{n})). 
		\end{split}
	\end{align}
	Therefore, we may choose $n$ large enough so that $$m(\b{r} \cap (\b{r} - x_{n})) \geq (1 - \epsilon)m(\b{r}).$$
	We will use these bounds to place a nonzero lower bound on $m(\b{r} \cap E \cap (E + x_{n}))$. Observe that since 
	\begin{equation}
		E \cap (E + x_{n}) \cap \b{r} \supset E \cap (E + x_{n}) \cap \b{r} \cap (\b{r} - x_{n}), 
	\end{equation}
	we have 
	\begin{equation}
		m(E \cap (E + x_{n}) \cap \b{r}) \geq m\left(E \cap (E + x_{n}) \cap \b{r} \cap (\b{r} -x_{n})\right). 
	\end{equation}
	Now, we want to simplify the right side. Remember that if $A, B$ are finite measured sets, $m(B) = m(B\setminus A) + m(B \cap A)$ so that $m(B \cap A) = m(B) - m(B \setminus A)$. Applying this above, 
	\begin{align}
		\begin{split}
			m\left(\left[\b{r} \cap (\b{r} - x_{n})\right] \cap \left[E \cap (E + x_{n}\right]\right) &= m(\b{r} \cap (\b{r}-x_{n})) \\
			&\quad - m(\left[\b{r} \cap (\b{r} - x_{n})\right] \setminus \left[E \cap (E + x_{n}\right]) \\
			&\geq m(\b{r} \cap (\b{r} -x_{n})) \\
			&\quad- m([\b{r} \cap (\b{r} - x_{n})] \setminus E) \\
			&\quad -m([\b{r} \cap (\b{r} - x_{n})] \setminus (E + x_{n})).
		\end{split}
	\end{align}
	Now, since 
	\begin{equation}
		\b{r} \cap (\b{r} - x_{n}) \subset \b{r}, 
	\end{equation}
	we have the inequality, 
	\begin{align}
		\begin{split}
			m(\left[\b{r} \cap (\b{r} - x_{n})\right]\setminus E) &\leq m(\b{r} \setminus E) = m(\b{r}) - m(E \cap \b{r}) \\
			&\leq \epsilon m(\b{r}). 
		\end{split}
	\end{align}
	Likewise, since $\b{r} \cap (\b{r} - x_{n}) \subset \b{r} - x_{n}$ and $(\b{r} - x_{n})\setminus (E + x_{n}) = (\b{r}\setminus E) - x_{n}$ and translation invariance tells us that $m((\b{r}\setminus E) - x_{n}) = m(\b{r} \setminus E)$, 
	\begin{equation}
		m([\b{r} \cap (\b{r} - x_{n})]\setminus (E + x_{n})) \leq m(\b{r}) - m(\b{r} \cap E) \leq \epsilon m(\b{r}). 
	\end{equation}
	Therefore, 
	\begin{equation}
		m([\b{r} \cap (\b{r} - x_{n})] \cap [E \cap (E + x_{n})]) \geq (1 - 3\epsilon)m(\b{r}). 
	\end{equation}
	Hence, this implies that 
	\begin{equation}
		m(E \cap (E + x_{n}) \cap \b{r}) \geq (1 - 3\epsilon)m(\b{r}) > 0
	\end{equation}
	for $\epsilon$ small enough. This contradicts our hypothesis that $E \cap (E + x_{n}) = \varnothing$ for all $n$. Hence, by contradiction, $E - E$ contains a neighborhood of the origin. 
\end{solutions}


\begin{prb}{2023-J-II-1}
	Suppose that $f: \mathbb{R} \to \mathbb{R}$ is continuous. Show that $f^{-1}(y) = \brac*{x \in \mathbb{R}: f(x) = y}$ has Lebesgue measure zero for Lebesgue almost $y$. 
\end{prb}
\begin{solutions}
	Let $f: \mathbb{R} \to \mathbb{R}$ be continuous, and let
		\begin{equation}
			\Gamma = \brac*{(x, y) \in \mathbb{R}^{2}: f(x) = y}
		\end{equation}
	be the graph of $f$. Since $f$ is a continuous function, $\Gamma$ is a measurable set. Define the function $g: \mathbb{R} \to \mathbb{R}$ as follows: for each $y \in \mathbb{R}$, $g(y) = m(f^{-1}(y))$. Our goal is to conclude that $m(f^{-1}(y)) = 0$ for almost all $y \in \mathbb{R}$. Define the following $x$- and $y$- slices of $\Gamma$, respectively, as follows: 
		\begin{equation}
			\Gamma_{x} = \brac*{y \in \mathbb{R}: f(x) = y} \qquad \text{ and } \qquad \Gamma^{y} = \brac*{x \in \mathbb{R}: f(x) = y} = f^{-1}(y). 
		\end{equation}		
	We observe that since $f$ is a function, for each $x \in \mathbb{R}$, $\Gamma_{x}$ is the singleton $\{f(x)\}$. Hence, $m(\Gamma_{x}) = 0$ for all $x \in \mathbb{R}$. By Fubini-Tonelli, we know that 
		\begin{equation}
			m(\Gamma) = \int_{\mathbb{R}}\int_{\mathbb{R}}\chi_{\Gamma}(x, y)\;d(x \times y) = \int_{\mathbb{R}}\left(\int_{\mathbb{R}}\chi_{\Gamma}(x, y)\;dx\right)\;dy = \int_{\mathbb{R}}\left(\int_{\mathbb{R}}\chi_{\Gamma}(x,y)\;dy\right)\;dx. 
		\end{equation}
	We observe that 
		\begin{equation}
			\int_{\mathbb{R}}\chi_{\Gamma}(x, y)\;dy = m(\Gamma_{x})= 0 \implies m(\Gamma) =0. 
		\end{equation}
	On the other hand, 
		\begin{equation}
			\int_{\mathbb{R}}\chi_{\Gamma}(x, y)\;dx = m(\Gamma^{y}) = m(f^{-1}(y)) = g(y). 
		\end{equation}
	Hence, we find that 
		\begin{equation}
			\int_{\mathbb{R}}g(y)\;dy = 0 \implies m(f^{-1}(y)) = g(y) = 0 \text{ for almost all $y \in \mathbb{R}$}. 
		\end{equation}
	The proof concludes. 
\end{solutions}
\begin{prb}{2023-J-II-2}
	Suppose that $f$ is continuous on $[0,1]$ and $\int_{0}^{1}f(x)x^{k}\;dx = 0$ for $k = 0, \ldots, n$. Prove that either $f$ is identically zero or $f$ must change sign at least $n + 1$ times. We say that $f$ changes sign $n$ times if there are points $x_{1} < \dotsm < x_{n + 1}$ so that $f(x_{j})f(x_{j + 1}) < 0$ for $j = 1, \ldots, n$. 
\end{prb}
\begin{solutions}
	Let $f$ be continuous on $[0,1]$ and $\int_{0}^{1}f(x)\;dx$ and assume $\int_{0}^{1}f(x)x^{k}\;dx = 0$ for $k = 0, \ldots, n$. If $f$ is identically zero, then the claim is straightforward to see. So assume $f$ is not identically zero. Assume to the contrary that $f$ changes sign at most $m \leq n$ times, and let $x_{1} < \dotsm < x_{m + 1}$ be the $m + 1$ points such that $f(x_{j})f(x_{j + 1}) < 0$ for $j = 1, \ldots, m$. For each $j$, let $y_{j} = f(x_{j})$, which gives us a collection of points $\{(x_{j}, y_{j})\}_{j = 1}^{n} \subset [0,1]\times \mathbb{R}$. Let $L(x)$ be the Lagrange polynomial of degree $\leq n$ such that $L(x_{j}) = y_{j}$ for each $j$. \textcolor{red}{[!! Complete Later !!]}
\end{solutions}

\newpage 
\begin{prb}{(Folland 6.1.3)}
	If $1 \leq p < r \leq \infty$, $L^{p} \cap L^{r}$ is a Banach space with norm $\norm{f} = \norm{f}_{p} + \norm{f}_{r}$, and if $p < q < r$, the inclusion map $L^{p} \cap L^{r} \to L^{q}$ is continuous. 
\end{prb}
\begin{solutions}
	Since $L^{p}$ and $L^{r}$ are vector spaces, and the intersection of vector spaces is a vector space, $L^{p} \cap L^{r}$ is a vector space. Define the map $\norm{\empspace}: L^{p} \cap L^{r} \to [0, \infty)$ as follows: $\norm{f} = \norm{f}_{p} + \norm{f}_{r}$. First, we will prove that this map is a norm. 
		\begin{enumerate}[itemsep =-2pt,label = (\roman{*})]
			\item $\norm{f} = 0$ iff $\norm{f}_{p} + \norm{f}_{r} = 0$ iff $\norm{f}_{p} = 0$ and $\norm{f}_{r} = 0$ (since $\norm{\empspace}_{p}, \norm{\empspace}_{r}$ are nonnegative) iff $f = 0$ a.e. 
			\item Let $a \in K$ and $f \in L^{p} \cap L^{r}$. Then 
				\begin{align}
					\begin{split}
						\norm{af} &= \norm{af}_{p} + \norm{af}_{r} \\
						&= \abs{a}\norm{f}_{p} + \abs{a}\norm{f}_{r} \\
						&= \abs{a}\norm{f}. 
					\end{split}
				\end{align}
			\item Let $f, g \in L^{p} \cap L^{r}$. Then 
				\begin{align}
					\begin{split}
						\norm{f + g} &= \norm{f + g}_{p} + \norm{f + g}_{r} \\
						&\leq \norm{f}_{p} + \norm{g}_{p} + \norm{f}_{r} + \norm{g}_{r} \\
						&= \norm{f} + \norm{g}. 
					\end{split}
				\end{align}
		\end{enumerate}
	Hence, $\norm{\empspace}$ is indeed a norm. Finally, we will show that $L^{p} \cap L^{r}$ is closed under this norm. Let $\{f_{n}\} \subset L^{p} \cap L^{r}$ such that $\sum \norm{f_{n}} < \infty$. This implies that 
		\begin{equation}
			\infty > \sum \norm{f_{n}} = \sum \norm{f_{n}}_{p} + \sum \norm{f_{n}}_{r} \implies \sum \norm{f_{n}}_{p}, \sum \norm{f_{n}}_{r} < \infty. 
		\end{equation}
	Since $L^{p}$ and $L^{r}$ are Banach spaces, $\sum^{m} f_{n} \longrightarrow f \in L^{p}$ and $\sum^{m} f_{n}\longrightarrow g \in L^{r}$ as $m \to \infty$. Since convergence is unique, $f = g$ a.e. and $f, g \in L^{p} \cap L^{r}$. Therefore, all that remains is to show that $\sum^{m} f_{n}\to f$ in $L^{p} \cap L^{r}$. But this is straightforward since 
		\begin{equation}
			\norm{\sum^{m} f_{n} - f} = \norm{\sum^{m} f_{n} - f}_{p} + \norm{\sum^{m} f_{n} - f}_{r} \longrightarrow 0 \text{ as $m \to \infty$}. 
		\end{equation}
	Hence, $L^{p} \cap L^{r}$ is a Banach space with the specified norm. Now we will show that the inclusion map $i: L^{p} \cap L^{r}\hookrightarrow L^{q}$ is continuous. It suffices to show that $i$ is bounded since continuity of linear maps between $L^{p}$ spaces is equivalent to boundedness. Let $f \in L^{p} \cap L^{r}$ such that $\norm{f} = 1$. Since $1 = \norm{f} = \norm{f}_{p} + \norm{f}_{r}$, we must have $\norm{f}_{p}, \norm{f}_{r} \leq 1$. Therefore, 
		\begin{equation}
			\norm{if}_{q} = \norm{f}_{q} \leq \norm{f}_{p}^{\lambda}\norm{f}_{r}^{1 - \lambda} \leq 1 = \norm{f}, 
		\end{equation}
	where $\lambda$ is defined to be the constant $q^{-1} = \lambda^{-1}p^{-1} + (1 - \lambda)r^{-1}$. Now if $f \in L^{p} \cap L^{r}$ so that $\norm{f} = a$, since $f = a \cdot \frac{f}{\norm{f}}$, where $\norm{f/\norm{f}} =1$, 
		\begin{equation}
			\norm{if}_{q} = \norm{a \cdot \frac{f}{\norm{f}}}_{q} \leq |a|^{\lambda}|a|^{1 -\lambda}\norm{\frac{f}{\norm{f}}}^{\lambda}_{p} \norm{\frac{f}{\norm{f}}}_{r}^{1 - \lambda} \leq |a| \cdot 1= |a| \cdot \norm{\frac{f}{\norm{f}}} = \norm{f}
		\end{equation}
	Hence, $i$ is bounded, with a bounding constant of $1$. This concludes the proof that $i$ is continuous. 
\end{solutions}
\begin{prb}{(Folland 6.1.4)}
	If $1 \leq p < r \leq \infty$, $L^{p} + L^{r}$ is a Banach space with norm $\norm{f} = \inf\brac*{\norm{g}_{p} + \norm{h}_{r}: f = g + h}$, and if $p < q < r$, the inclusion map $L^{q} \to L^{p} + L^{r}$ is continuous. 
\end{prb}

\newpage 
\subsection{Differential Geometry}
\begin{prb}{2019-J-II-6}
	Let $X$ and $Y$ be vector fields on $\mathbb{R}^{3}$, defined by 
		\begin{equation}
			X = \frac{\partial}{\partial x} + x \frac{\partial}{\partial y} + y\frac{\partial}{\partial z} \quad \text{ and } \quad Y =y\frac{\partial}{\partial x} + z \frac{\partial}{\partial y} + \frac{\partial}{\partial z}. 
		\end{equation}
	Is there a coordinate chart $\varphi = (x_{1}, x_{2}, x_{3}): U \to \mathbb{R}^{3}$ of the origin $0 \in \mathbb{R}^{3}$ such that 
		\begin{equation}
			X|_{U} = \frac{\partial}{\partial x^{1}} \quad \text{ and } \quad Y|_{U} = \frac{\partial}{\partial x^{2}}. 
		\end{equation}
\end{prb}
\begin{solutions}
	No, there exists no coordinate chart containing the origin $0 \in \mathbb{R}^{3}$ that satisfies the above conditions. To show this, we will compute the Lie Brackets of the given vector fields; let $\ti{X} = \partial/\partial x^{1}$ and $\ti{Y} = \partial/\partial x^{2}$. First, we observe that 
		\begin{align}
			\begin{split}
				[X, Y] &= \frac{\partial}{\partial x}(y, z, 1) + x\frac{\partial}{\partial y}(y, z, 1) + y\frac{\partial}{\partial z}(y, z, 1) - y\frac{\partial}{\partial x}(1, x, y) -z \frac{\partial}{\partial y}(1, x, y) - \frac{\partial}{\partial z}(1, x, y) \\
				&= x(1,0,0) + y(0,1,0) - y(0,1,0) - z(0,0,1) \\
				&= x\frac{\partial}{\partial x} - z\frac{\partial}{\partial z}. 
			\end{split}
		\end{align}
	This means that the Lie Bracket of $X$ and $Y$ is not identically zero on any neighborhood of the origin. On the other hand, it is straightforward to see that the Lie Bracket of $\ti{X}$ and $\ti{Y}$ is identically zero on \textit{all} of $U$. This is a contradiction. Therefore, such a coordinate chart cannot exist. 
\end{solutions}	
\begin{prb}{2019-A-I-5}
	Let $H^{3}$ be the 3-dimensional Heisenberg group consisting of upper triangular $3 \times 3$ matrices, with 1's on the diagonal, and with the group operation being matrix multiplication. Let $\Gamma \subset H^{3}$ be the subgroup consisting of matrices all of whose entries are integers. Show that the quotient space $N = H^{3}/\Gamma$ is a closed 3-dimensional manifold. Show that there is a fiber bundle projection $P: N \to \mathbb{T}^{2}$ to a 2-dimensional torus $\mathbb{T}^{2} = S^{1} \times S^{1}$ with fiber $S^{1}$. Hint: Consider the center $Z$ of $H^{3}$. 
\end{prb}
\begin{solutions}
	Let $H^{3}$ be the 3-dimensional Heisenberg group consisting of upper triangular $3 \times 3$ matrices, with 1's on the diagonal, and with the group operation being matrix multiplication, and let $\Gamma$ be the subgroup consisting of matrices all of whose entries are integers. We proceed with the proof over several steps: 
		\begin{enumerate}[itemsep =-2pt,label = (\textbf{\arabic{*}})]
			\item ($H^{3}$ is a Lie Group): We claim that there is a one-to-one correspondence between the elements of $H^{3}$ and $\mathbb{R}^{3}$, given by the map 
				\begin{equation}
					\Phi: (a, b, c) \in \mathbb{R}^{3} \longmapsto \begin{pmatrix} 1 & a & b \\ 0 & 1 & c \\ 0 & 0 & 1 \end{pmatrix} 
				\end{equation}
			It is straightforward to verify that $\Phi$ is a bijection. In particular, since the coordinate functions of $\Phi$ are polynomials (as is for the inverse of $\Phi$), it follows that $\Phi$ is actually a diffeomorphism. Therefore, since $\mathbb{R}^{3}$ is a smooth manifold, $H^{3}$ must also be a smooth manifold. Now consider the product of two matrices in $H^{3}$:
				\begin{equation}
					\begin{pmatrix}
						1 & a & b \\ 0 & 1 & c \\ 0 & 0 & 1
					\end{pmatrix}
					\cdot 
					\begin{pmatrix}
						1 & a' & b' \\ 0 & 1 & c' \\ 0 & 0 & 1 
					\end{pmatrix}
					= 
					\begin{pmatrix}
						1 & a + a' & b + b' + ac' \\ 0 & 1 & c + c' \\ 0 & 0 & 1
					\end{pmatrix}.
				\end{equation}
			By means of the one-to-one correspondence $\Phi$, the multiplication becomes 
			\begin{equation}
				m: \mathbb{R}^{3} \times \mathbb{R}^{3}, \qquad m((a,b,c) \cdot (a', b',c')) = (a + a', b + b' + ac', c + c'). 
			\end{equation}
			Since all of the coordinate functions are polynomials, $m$ is a smooth map. Therefore, by way of $\Phi$, the multiplication map on $H^{3}$ is also smooth. Likewise, the inverse map can be seen to be a smooth map on $H^{3}$. Therefore, we conclude that $H^{3}$ is three-dimensional Lie group. 
			\item ($\Gamma$ is a discrete Lie Group): Trivially, every discrete group is a Lie group so that $\Gamma$ is a discrete Lie subgroup of $H^{3}$. 
			\item (Action of $\Gamma$ on $H^{3}$): Let $\Gamma$ act on the Lie group $H^{3}$ by left multiplication. First, we show that this action is free: 
				\begin{equation}
					\begin{pmatrix}
						1 & a & b \\ 0 & 1 & c \\  0 & 0 & 1 
					\end{pmatrix}
					\cdot 
					\begin{pmatrix}
						1 & a' & b' \\ 0 & 1 & c' \\ 0 & 0 & 1 
					\end{pmatrix}
					= \begin{pmatrix}
						1 & a' & b' \\ 0 & 1 & c' \\ 0 & 0 & 1 
					\end{pmatrix}
					\iff 
					\begin{cases}
						a + a' = a', & \\
						b + b' + ac' = b', & \\
						c + c' = c' & 
					\end{cases}
					\iff 
					a = 0, b = 0, c = 0. 
				\end{equation}
			Hence, $\begin{pmatrix}1 & a & b \\ 0 & 1 & c \\ 0 & 0 & 1 \end{pmatrix}$ is the identity. This proves that the action is free. That the group action is smooth follows trivially from the fact that matrix multiplication is smooth. Finally, we show that the group action is properly discontinuous. I.e., for a compact set $K \subset \mathbb{R}^{3}$, we want to show that the set $\brac*{\gamma \in \Gamma:(\gamma \cdot K) \cap K \neq \varnothing}$ is a finite set. By the one-to-one correspondence between $H^{3}$ and $\mathbb{R}^{3}$ we demonstrated in (\textbf{1}), our goal is to show that there exist finitely many 3-tuples $(m, n, p) \in \mathbb{Z}^{3}$ such that if $K$ is a compact set and $[-R, R]^{3}$ is a cube containing $K$, then $(m,n,p) \cdot K \cap K \neq \varnothing$. If $(m,n, p) \cdot K$ intersects $K$, then for some $(x, y, z) \in K$, 
				\begin{align}
					\begin{split}
						&-R \leq x + m \leq R \implies -2R \leq m \leq 2R. \\
						&-R \leq z + p \leq R \implies -2R \leq p \leq 2R. \\
						&-R \leq n + y + mz \leq R \implies -2R \leq n + mz\leq 2R.
					\end{split}
				\end{align}
			Since $m, n, p$ are integers and $R < \infty$, there exist only finitely many 3-tuples $(m,n,p)$ that satisfy the above conditions. Therefore, it follows that the action is properly continuous. 
		\end{enumerate}
	Therefore, by the Quotient Manifold Theorem (see Lee \textit{Introduction to Smooth Manifolds}, Theorem 9.16), $N = H^{3}/\Gamma$ is a smooth manifold of dimension $\dim{H^{3}} - \dim{\Gamma} = 3 - 0 = 3$. \textcolor{red}{[!! Complete Later !!]}
\end{solutions}
\begin{prb}{2023-A-II-5}
	Let $(t, x, y, z)$ be the standard coordinate system on $\mathbb{R}^{4}$, and let $\phi$ be the non-zero smooth 1-form on $\mathbb{R}^{4}$ defined by 
		\begin{equation*}
			\Phi = dt + y\;dx + z\;dy. 
		\end{equation*}
	Let $D$ be the 3-plane field on $\mathbb{R}^{4}$ that consists of tangent vectors $V$ such that $\Phi(V) = 0$. Is $D$ Frobenius integrable? Support your answer with a proof. 
\end{prb}
\begin{solutions}
	Let $D = \ker{\Phi} \subset T\mathbb{R}^{4}$, which must be a smooth 3-plane field on $\mathbb{R}^{4}$ since $\phi$ is nowhere zero. Note that since $\dim{T_{p}\mathbb{R}^{4}} = 4$ at any $p$ and $\dim{D_{p}} = \dim{\ker{\Phi_{p}}} = 4 - 1 = 3$, it follows that $\operatorname{codim}{D} = 3$. By the Frobenius Theorem, a codimension-one distribution $D = \ker{\Phi}$ is integrable iff $\Phi \wedge d\Phi = 0$. Computing $d\Phi$ first, we observe that: 
		\begin{align}
			\begin{split}
				d\Phi &= d(dt) + dy \wedge dx + dz\wedge dy \\
				&= -dx\wedge dy + dz\wedge dy. 
			\end{split}
		\end{align}
	Therefore, 
		\begin{align}
			\begin{split}
				\Phi \wedge d\Phi &= (dt + y\;dx + z\;dy) \wedge (-dx \wedge dy + dz\wedge dy) \\
				&= -dt \wedge dx \wedge dy + dt \wedge dz \wedge dy + y\;dx \wedge dz \wedge dy,  
			\end{split}
		\end{align}
	which is not identically zero everywhere on $\mathbb{R}^{4}$. Therefore, since $\Phi \wedge d\Phi \neq 0$, the Frobenius integrability condition fails, and so $D$ is not Frobenius integrable. 
\end{solutions}
\textbf{Remark:} Here, $D$ had codimension one. \textit{If} $D$ was a smooth distribution of codimension $k$, we can write $D = \ker{\phi^{1}, \ldots, \phi^{k}}$, where $\phi^{1}, \ldots, \phi^{k}$ are $k$ smooth 1-forms that are pointwise linearly independent, then $D$ is Frobenius integrable if and only if $\phi^{1}\wedge \dotsm \wedge \phi^{k} \wedge d\phi^{i} = 0$ for all $i$. 

\newpage 
\begin{prb}{2023-A-I-2}
	Let $f: T^{2} \to S^{2}$ be a smooth map from the 2-torus to the 2-sphere. Can $f$ be an immersion. If the answer is yes, give an explicit example. If the answer is no, then give a proof. 
\end{prb}
\begin{solutions}
	Let $f: T^{2} \to S^{2}$ be a smooth map from the 2-torus to the 2-sphere, and assume for the sake of an argument, that $f$ is an immersion. Since $\dim{T^{2}} = \dim{S^{2}} = 2$, $f$ must have constant rank 2. That is, for each $p \in T^{2}$, 
		\begin{equation}
			df_{p}: T_{p}T^{2} \to T_{p}S^{2} 
		\end{equation}
	is an isomorphism. Hence, by the Inverse Function Theorem, $f$ is a local diffeomorphism near $p$. Since local diffeomorphisms are open maps, it follows that $f(T^{2})$ is an open subset of $S^{2}$. On the other hand, since the impact of compact sets under continuous maps is compact, and $T^{2}$ is compact, $f(T^{2})$ is a compact, hence closed, subset of $S^{2}$. Since $S^{2}$ is connected, this implies that $f(T^{2}) = S^{2}$. Therefore, $f$ is a surjective local diffeomorphism, which means that $f$ is a covering map. Because $S^{2}$ is simply connected, any covering map onto $S^{2}$ must be a diffeomorphism. This implies that $T^{2} \cong S^{2}$. However, this is a contradiction since $\pi_{1}(T^{2}) \cong \mathbb{Z}^{2}$ and $\pi_{1}(S^{2}) = \{0\}$ and diffeomorphisms preserve fundamental groups. Hence, by contradiction, $f$ cannot be an immersion. 
\end{solutions}

\begin{prb}{2023-J-I-3}
	Show that if $M$ is a closed manifold that has an even dimensional sphere $S^{2n}$ as its universal cover, then its fundamental group $\pi_{1}(M)$ is either trivial or $\mathbb{Z}_{2}$. 
\end{prb}


\begin{prb}{2018-J-I-6}
	Consider the distribution in $\mathbb{R}^{3}$ spanned by the two vector fields 
		\begin{equation*}
			V = \partial_{x} + 2xy\partial_{z}, \qquad W = x\partial_{x} + \partial_{y} + (2x^{2}y + x^{2} - 2y)\partial_{z}. 
		\end{equation*}
	Show that this distribution is integrable and find an explicit formula for the integral submanifold passing through the point $(0,0,z_{0})$. 
\end{prb}
\begin{solutions}
	Let $D$ be the distribution in $\mathbb{R}^{3}$ spanned by the two vector fields $V$ and $W$ describe above. By the Frobenius Theorem, to show that this distribution is integrable, it suffices to show that the distribution is involutive. In other words, we merely have to show that the Lie Bracket of $V$ and $W$ is a smooth local section of $D$. For ease of notation, we denote $V$ and $W$ as $(1, 0, 2xy)$ and $(x, 1, (2x^{2}y + x^{2} - 2y))$, respectively. Then their Lie Bracket is:
		\begin{align}
			\begin{split}
				[V, W] &= \brac*{(1,0,2xy) \cdot (x, 1, (2x^{2}y + x^{2} - 2y))} - \brac*{(x, 1, (2x^{2}y + x^{2} - 2y))\cdot (1, 0, 2xy)} \\
				&= (1,0,4xy + 2x) - (0,0,2xy) - (0,0,2x) = (1,0,2xy) = V. 
			\end{split}
		\end{align}
	Hence, this shows that the distribution $D$ is involutive, and hence completely integrable. To find an explicit formula for the integral submanifold at some point $p \in \mathbb{R}^{3}$, since a 2-dimensional distribution in $\mathbb{R}^{3}$ is the kernel of a single 1-form $\omega$ (because the codimension of the distribution is 1), we start by finding an annihilator 1-form; i.e., a 1-form such that $\omega(V) = \omega(W) =0$. Let $\omega = A\;dx + B\;dy + dz$. Then 
		\begin{align}
			\begin{split}
				0 = \omega(V) &= A + 2xy \implies A = -2xy. \\
				0 = \omega(W) &= Ax + B + (2x^{2}y + x^{2} - 2y) \implies B = 2x^{2}y - 2x^{2}y - x^{2} + 2y = -x^{2} + 2y. 
			\end{split}
		\end{align}
	Hence, our annihilator 1-form is 
		\begin{equation}
			\omega = -2xy\;dx -(x^{2} - 2y)\;dy + dz. 
		\end{equation}
	On integral surfaces, $\omega = 0$. Hence, we observe that 
		\begin{equation}
			dz = 2xy\;dx + (x^{2} - 2y)\;dy \implies \frac{\partial z}{\partial x} = 2xy \text{ and }\frac{\partial z}{\partial y} = x^{2} - 2y. 
		\end{equation}
	From the first differential equation, we find 
		\begin{equation}
			z = x^{2}y + f(y). 
		\end{equation}
	From the second equation, 
		\begin{equation}
			x^{2} + \frac{df}{dy} = x^{2} - 2y \implies \frac{df}{dy} = -2y \implies f(y) = -y^{2} + c. 
		\end{equation}
	Therefore, we obtain 
		\begin{equation}
			z = x^{2}y - y^{2} + c. 
		\end{equation}
	Plugging in the point $(0,0,z_{0})$, we obtain $c = z_{0}$. Hence, the explicit formula for the integral submanifold passing through the point $(0,0,z_{0})$ is given by 
		\begin{equation}
			x^{2}y - y^{2} -z = z_{0}. 
		\end{equation}
\end{solutions}
\begin{prb}{2019-J-11-6}
	Let $X$ and $Y$ be vector fields on $\mathbb{R}^{3}$, defined by 
		\begin{equation*}
			X = \frac{\partial}{\partial x} + x\frac{\partial}{\partial y} + y\frac{\partial}{\partial z} \qquad \text{and} \qquad Y = y\frac{\partial}{\partial x} + z\frac{\partial}{\partial y} + \frac{\partial}{\partial z}. 
		\end{equation*}
	Is there a coordinate chart $\varphi = (x_{1}, x_{2}, x_{3}): U \to \mathbb{R}^{3}$ of the neighborhood of the origin $0 \in \mathbb{R}^{3}$ such that 
		\begin{equation*}
			X|_{U} = \frac{\partial}{\partial x_{1}} \qquad \text{and} \qquad Y|_{U} = \frac{\partial}{\partial x_{2}}.
		\end{equation*}
\end{prb}
\begin{solutions}
	We claim that there does \textit{not} exist such a coordinate chart. Suppose to the contrary that there do exist such coordinates on a neighborhood $U$ of the origin $0 \in \mathbb{R}^{3}$. We observe that 
		\begin{equation}
			[X|_{U}, Y|_{U}] = \frac{\partial}{\partial x_{1}}\left(\frac{\partial}{\partial x_{2}}\right) - \frac{\partial}{\partial x_{2}}\left(\frac{\partial}{\partial x_{1}}\right) = 0. 
		\end{equation}
	I.e., the Lie Bracket of the vector fields in these coordinates vanish everywhere on $U$. On the other hand, computing the Lie Bracket of $X$ and $Y$ in the original coordinates, 
		\begin{align}
			\begin{split}
				[X, Y] &= (1, x, y) \cdot (y, z, 1)  - (y, z, 1) \cdot (1,x,y)\\
				&= (0,0,0) + (x,0,0) + (0,y,0) - (0,y,0) - (0,0,z) - (0,0,0) \\
				&= (x,0,-z) = x\frac{\partial}{\partial x} - z\frac{\partial}{\partial z}. 
			\end{split}
		\end{align}
	Since the Lie Bracket of $X$ and $Y$ is not identically zero on $U$, we have reached a contradiction. Therefore, by contradiction, we see that such a coordinate chart cannot exist. 
\end{solutions}
\begin{prb}{2019-A-I-4}
	Let $f: \mathbb{RP}^{3} \to \mathbb{T}^{3} = S^{1} \times S^{1} \times S^{1}$ be a smooth map. Show that $f$ is not an immersion. 
\end{prb}
\begin{solutions}
	Let $f: \mathbb{RP}^{3} \to \mathbb{T}^{3} = S^{1} \times S^{1} \times S^{1}$ be a smooth map. Assume to the contrary that $f$ is an immersion. First, we prove the \textit{comps lemma}, which we will use to develop our argument. 
	\begin{quote}
		(\textbf{Comps Lemma}) Let $M$ and $N$ be smooth connected $n$-manifolds, and $f: M \to N$ a (smooth) immersion. If $M$ is compact and nonempty, then $N$ is compact and $f$ is a (smooth) covering map. 
	\end{quote}
	\begin{proof}
		Let $M, N$ be smooth connected $n$-manifolds, $f: M \to N$ an immersion, and $M$ compact and nonempty. Since $f$ is an immersion, the map $df_{p}: T_{p}M \to T_{f(p)}N$ is injective at each $p \in M$ so that $f$ is a local diffeomorphism. This implies that $f(M)$ is open in $N$ (continuous image of an open set) and $f(M)$ is closed in $N$ (continuous image of a compact set is compact, and compact subsets of Hausdorff spaces are closed). Since $N$ is connected and $M$ is nonempty, $f(M) = N$. Therefore, $N$ is compact. 
		
		To show that $f$ is a covering map, it remains to be shown that $N$ is evenly covered. Let $q \in N$ be arbitrary but fixed. Since $f$ is a local diffeomorphism, $f^{-1}(q) \subset M$ is closed and discrete, which means $f^{-1}(q) = \{x_{1}, \ldots, x_{s}\}$ for some finite $s$ and $x_{j} \in M$. Since $M$ is Hausdorff, we may pick a collection $\brac*{U_{j}}_{j = 1}^{s}$ of open subsets of $M$ such that $x_{j} \in U_{j}$ for each $j$, and $U_{i} \cap U_{j} = \varnothing$ for each $i \neq j$; shrink each $U_{j}$ if needed so that $f|_{U_{j}} \to f(U_{j})$  is a diffeomorphism. Set $V = \bigcap_{j = 1}^{s}f(U_{j})$ so that $V$ is an evenly covered neighborhood of $q \in N$. 
	\end{proof}
	
	In our case, $\mathbb{RP}^{3}$ is a smooth connected compact nonempty 3-manifold, and $\mathbb{T}^{3}$ is a smooth connected 3-manifold. By the Comps Lemma, $\mathbb{T}^{3}$ is compact and $f$ is a smooth covering map. Consider the induced \textit{injective} homomorphism $f_{\ast}: \pi_{1}(\mathbb{RP}^{3}) \to \pi_{1}(\mathbb{T}^{3})$. Since $\pi_{1}(\mathbb{RP}^{3}) = \mathbb{Z}/2\mathbb{Z}$, $\pi_{1}(\mathbb{T}^{3}) = \mathbb{Z} \times \mathbb{Z} \times \mathbb{Z}$, and the former has torsion while the latter does not since each subgroup of $\mathbb{Z}^{3}$ is free abelian, there cannot exist such an injective homomorphism. Hence, by contradiction, we must have that $f$ cannot be a immersion. 
\end{solutions}
\begin{prb}{2017-J-II-1}
	Let $f: M \to \mathbb{R}$ be a smooth function on a smooth manifold $M$. In an arbitrary smooth local coordinate chart $\pi: U \to \mathbb{R}^{n}$ of $M$, define 
		\begin{equation}
			\mathscr{D}f \coloneqq \sum_{i = 1}^{n}\frac{\partial f}{\partial x^{i}}\frac{\partial}{\partial x^{i}}. 
		\end{equation}
	Does $\mathscr{D}f$ give a well-defined vector field on $M$? 
\end{prb}
\begin{solutions}
	No, $\mathscr{D}f$ does \textit{not} give a well-defined vector field on $M$. In fact, we claim that $\mathscr{D}f$ does not transform covariantly. Let $(U, (x^{i}))$ and $(V, (\ti{x}^{i}))$ be smooth local coordinate charts on $M$, and let $p \in U \cap V$. In the remainder of the proof, we shall use Einstein Summation Convention. We find that 
		\begin{align}
			\begin{split}
				\mathscr{D}F &= \frac{\partial f}{\partial x^{i}}(p)\frac{\partial}{\partial x^{i}}\bigg|_{p} \\
				&= \left(\frac{\partial f}{\partial \ti{x}^{j}}(\hat{p})\frac{\partial \ti{x}^{j}}{\partial x^{i}}\bigg|_{p}\frac{\partial \ti{x}^{k}}{\partial x^{i}}\right)\frac{\partial}{\partial \ti{x}^{k}}\bigg|_{\hat{p}} \\
				&\neq \frac{\partial f}{\partial \ti{x}^{k}}(\hat{p})\frac{\partial}{\partial \ti{x}^{k}}\bigg|_{\hat{p}} = \mathscr{D}f, 
			\end{split}
		\end{align}
	which is a contradiction. Therefore, $\mathscr{D}f$ does not give a well-defined vector field on $M$. 
\end{solutions}
\begin{prb}{2012-J-I-4}
	Let $x, y, z$ be the usual coordinates on $\mathbb{R}^{3}$. Consider the 1-form on $\mathbb{R}^{3}$ given by 
		\begin{equation*}
			\varphi = dx + ydz. 
		\end{equation*}	
	Is it possible to find smooth functions $u$ and $v$ on $\mathbb{R}^{3}$ such that $\varphi = udv$? Why? 
\end{prb}
\begin{solutions}
	Let $\varphi_{1} = dx + ydz$ and $\varphi_{2} = udv$ for some smooth functions $u$ and $v$ on $\mathbb{R}^{3}$. Then, we observe that 
		\begin{align}
			\begin{split}
				d\varphi_{1} &= d(dx + ydz) = d(dx) + d(ydz) \\
				&= dy \wedge dz. \\
				\varphi_{1}\wedge d\varphi_{1} &= (dx + ydz) \wedge (dy \wedge dz) = dx \wedge dy \wedge dz. \\
				d\varphi_{2} &= d(udv) = du \wedge dv. \\
				\varphi_{2}\wedge d\varphi_{2} &= udv \wedge (du \wedge dv) = 0. 
			\end{split}
		\end{align}
	We observe that $\varphi_{1}\wedge d\varphi_{1}$ is the volume form on $\mathbb{R}^{3}$ and hence is identically nonzero everywhere on $\mathbb{R}^{3}$. On the other hand, $\varphi_{2} \wedge d\varphi_{2}$ is zero everywhere on $\mathbb{R}^{3}$. This means that it is not possible to find smooth functions $u$ and $v$ on $\mathbb{R}^{3}$ such that $\varphi = udv$. 
\end{solutions}
\begin{prb}{2011-A-II-5}
	On $\mathbb{R}^{4}$, equipped with coordinates $(x, y, z,t)$, let $X$ and $Y$ be the vector fields given by 
		\begin{equation*}
			X = \frac{\partial}{\partial x} + z\frac{\partial}{\partial y} \qquad \text{and} \qquad Y = x \frac{\partial}{\partial z} + \frac{\partial}{\partial t}. 
		\end{equation*}
	If $f: \mathbb{R}^{4} \to \mathbb{R}$ is a smooth function satisfying $Xf = Yf = 0$, show that $f$ is constant. 
\end{prb}
\begin{solutions}
	Let $f: \mathbb{R}^{4} \to \mathbb{R}$ be a smooth function satisfying $Xf = Yf = 0$, where $X, Y$ are the vector fields described above. First, we observe that $[X, Y]f = 0$, where 
		\begin{align}
			\begin{split}
				[X, Y] &= \partial_{x}(x\partial_{z}) + \partial_{x}(\partial_{t}) + z\partial_{y}(x\partial_{z}) + z\partial_{y}(\partial_{t}) - \left(x\partial_{z}(\partial_{x}) + x\partial_{z}(z\partial_{y}) + \partial_{t}(\partial_{x}) + \partial_{t}(z\partial_{y})\right) \\
				&= \partial_{z} - x \partial_{y}. 
			\end{split}
		\end{align}
	Since $[X, Y]f = 0$, $[X, [X, Y]]f = 0$, where 
		\begin{align}
			\begin{split}
				[X, [X, Y]] &= -\partial_{y} - \partial_{y} = -2\partial_{y}.
			\end{split}
		\end{align}
	This immediately implies that $-2\partial_{y}f = 0 \implies \partial_{y}f = 0$ everywhere on $\mathbb{R}^{4}$. Hence, since $[X, Y]f = 0$ everywhere, $0 = \partial_{z}f - x\partial_{y}f = \partial_{z}f$ everywhere on $\mathbb{R}^{4}$. Likewise, $0 = Xf = \partial_{x}f + z\partial_{y}f = \partial_{x}f$. Then since $0 = Yf = x\partial_{z}f + \partial_{t}f = \partial_{t}f$, we see that $\partial_{t}f = 0$ everywhere. Hence, since all of the first order partial derivatives of $f$ are zero everywhere on $\mathbb{R}^{4}$, it follows that $f$ is constant. 
\end{solutions}
\begin{prb}{2010-J-I-2}
	Consider the differential 1-form $\varphi = dx^{1} + x^{2}dx^{3}$ on $\mathbb{R}^{3}$. Is it possible to find a smooth coordinate system $(y^{1}, y^{2}, y^{3})$ on a neighborhood of the origin such that $\varphi = f\;dy^{1}$ in these new coordinates, for some smooth function $f(y^{1}, y^{2}, y^{3})$? Support your answer with a proof. 
\end{prb}
\begin{solutions}
	Let $f = f(y^{1}, y^{2}, y^{3})$ be a smooth function on $\mathbb{R}^{3}$, where $(y^{1}, y^{2}, y^{3})$ is a smooth coordinate system on some neighborhood of the origin. Let $\varphi_{1} = f\;dy^{1}$. Then we observe that 
		\begin{align}
			\begin{split}
				d\varphi_{1} &= d(f\;dy^{1}) = df \wedge dy^{1} \\
				&= \left(\frac{\partial f}{\partial y^{1}}dy^{1} + \frac{\partial f}{\partial y^{2}}dy^{2} + \frac{\partial f}{\partial y^{3}}dy^{3}\right) \wedge dy^{1} \\
				&= -\left(\frac{\partial f}{\partial y^{2}} dy^{1}\wedge dy^{2} + \frac{\partial f}{\partial y^{3}}\;dy^{1} \wedge dy^{3}\right). \\
				\varphi_{1}\wedge d\varphi_{1} &= 0.\\
				d\varphi &= d(dx^{1} + x^{2}dx^{3}) = dx^{2} \wedge dx^{3}. \\
				\varphi \wedge d\varphi &= dx^{1} \wedge dx^{2} \wedge dx^{3}. 
			\end{split}
		\end{align}
	I.e., $\varphi \wedge d\varphi$ is the volume form on $\mathbb{R}^{3}$, and hence is nonzero everywhere. On the other hand, $\varphi_{1} \wedge d\varphi_{1}$ is zero \textit{everywhere}. Therefore, it is not possible to find a smooth coordinate system $(y^{1}, y^{2}, y^{3})$ on a neighborhood of the origin such that $\varphi = f\;dy^{1}$ for some smooth function $f = f(y^{1},y^{2}, y^{3})$. 
\end{solutions}
\begin{prb}{2010-J-I-3}
	Let $f: S^{1} \times S^{1} \to \mathbb{RP}^{2}$ be a smooth map from the 2-torus to the projective plane. Prove that $f$ cannot be an immersion. 
\end{prb}
\begin{solutions}
	Our strategy for this problem is to develop and prove the \textit{Comps Lemma}, which we will then use to prove the problem. 
		\begin{quote}
			\textbf{(Comps Lemma)} Let $M$ and $N$ be smooth connected $n$-manifolds, and $f: M \to N$ a (smooth) immersion. If $M$ is compact and nonempty, then $N$ is compact and $f$ is a (smooth) covering map. 
			
			\textbf{\textit{Proof}} Let $M$ and $N$ be smooth connected $n$-manifolds and let $f: M \to N$ be an immersion, and consider the map $df_{p}: T_{p}M \to T_{f(p)}N$, for $p \in M$. Since $\dim{T_{p}M} = \dim{T_{f(p)}N} = n$ for every $p \in M$ and $f$ is an immersion, it follows that $df_{p}$ has constant rank $n$. Therefore, the Inverse Function Theorem implies that $f$ is a local diffeomorphism. Since local diffeomorphisms are open maps, $f(M)$ is open in $N$. On the other hand, since the continuous image of compact sets is compact and compact subsets of Hausdorff spaces are closed, $f(M)$ is closed in $N$. Since $f(M)$ is nonempty and $N$ is connected, $f(M) = N$, which proves that $N$ is compact. 
			
			Now it remains to be shown that $N$ is evenly covered. Let $q \in N$ and consider $f^{-1}(q)$. Since $f$ is a local diffeomorphism, $f^{-1}(q)$ is closed in $M$. Moreover, for each $x \in f^{-1}(q)$, we can find a neighborhood $U_{x}$ of $x$ such that $f|_{U_{x}}$ is a diffeomorphism. Therefore, $U_{x}$ contains no other point in the preimage of $f$. This means that every point in the preimage of $q$ is isolated, which means that $f^{-1}(q)$ is discrete. Since $M$ is compact and discrete subsets of compact spaces are finite, we conclude that $f^{-1}(q)$ is a finite set $\{x_{1}, \ldots, x_{s}\}$. By the local diffeomorphism property of $f$, we can find open subsets $U_{1}, \ldots, U_{s}$ such that each $U_{j}$ contains $x_{j}$ and $f|U_{j}$ is a diffeomorphism. Since $M$ is Hausdorff, we may shrink these sets so that they are pairwise disjoint. Now set $V = \bigcap_{j = 1}^{s}f(U_{j})$, which is an evenly covered neighborhood of $q$ in $N$. Therefore, $N$ is evenly covered. 
		\end{quote}
	Now assume to the contrary that $f: S^{1} \times S^{1} \to \mathbb{RP}^{2}$ is a smooth immersion. Since $S^{1} \times S^{1}$ is a smooth, connected, compact, and nonempty 2-manifold, while $\mathbb{RP}^{2}$ is a smooth connected 2-manifold, it follows from the Comps Lemma that $f: M \to N$ is a smooth covering map. This implies that the induced homomorphism, $f_{\ast}: \pi_{1}(S^{1} \times S^{1}) \to \pi_{1}(\mathbb{RP}^{2})$ is injective. We recall the following facts about the fundamental group: 
		\begin{equation}
			\pi_{1}(S^{1} \times S^{1}) = \pi_{1}(S^{1}) \times \pi_{1}(S^{1}) \cong \mathbb{Z} \times \mathbb{Z} \quad \text{and} \quad \pi_{1}(\mathbb{RP}^{2}) \cong \mathbb{Z}/2\mathbb{Z}. 
		\end{equation}
	Here, we observe that the fundamental group of $S^{1} \times S^{1}$ is countably infinite, while the fundamental group of $\mathbb{RP}^{2}$ is finite. Therefore, the induced homomorphism cannot be injective, which is a contradiction. Hence, $f$ cannot be an immersion. 
\end{solutions}
\begin{prb}{2002-J-II-1 (Comps Lemma)}
	Let $M$ and $N$ be smooth, connected $n$-dimensional manifolds, and let $f: M \to N$ be an immersion. (That is, assume that the derivative of $f$ always sends nonzero tangent vectors to nonzero tangent vectors.) If $M$ is compact and nonempty, show that $N$ is compact, and that $f$ is a covering map. 
\end{prb}
\begin{solutions}
	Let $M, N$ be smooth connected $n$-manifolds, and $f: M \to N$ an immersion. Assume $M$ is compact and nonempty, and consider the differential $df_{p}: T_{p}M \to T_{f(p)}N$. Since $\dim{T_{p}M} = \dim{T_{f(p)}N} = n$ at every $p \in M$ and $f$ is an immersion, $df_{p}$ has constant rank $n$. Hence, the Inverse Function Theorem implies that $f$ is a local diffeomorphism. Since local diffeomorphisms are open maps, $f(M)$ is open in $N$. On the other hand, since the continuous image of compact sets is compact and compact subsets of Hausdorff spaces are closed, $f(M)$ is closed in $N$. Since $f(M)$ is nonempty and $N$ is connected, $f(M) = N$, which proves that $N$ is compact. 
	
	It remains to be shown that $N$ is evenly covered. Let $q \in N$, and consider $f^{-1}(q)$. Since $f$ is a local diffeomorphism, $N$ is Hausdorff, and singletons are closed in Hausdorff spaces, $f^{-1}(q)$ is closed, and hence, compact in $M$. Moreover, since $f$ is a local diffeomorphism, for each $x \in f^{-1}(q)$, there exists a neighborhood $U_{x}$ such that $f|_{U_{x}}$ is a diffeomorphism. By the Hausdorff property on $M$, we may shrink each neighborhood $U_{x}$ to some smaller neighborhood $U_{x}'$ so that $f|_{U_{x}'}$ is still a diffeomorphism and the neighborhoods are pairwise disjoint. This means that every $x \in f^{-1}(q)$ is isolated, so that the preimage of $q$ is discrete. Since discrete, compact subsets must necessarily be finite, $f^{-1}(q) = \{x_{1}, \ldots, x_{s}\}$ for finitely many $x_{j} \in M$. As suggested above, for each $j$, let $U_{j}$ be a neighborhood of $x_{j}$ in $M$ such that, after possibly shrinking each set, $U_{i} \cap U_{j} = \varnothing$ for all $i \neq j$ and $f|_{U_{j}}$ is a diffeomorphism. Set $V = \bigcap_{j = 1}^{s}f(U_{j})$, which is then seen to be an evenly covered neighborhood of $q \in N$. Hence, $N$ is evenly covered, concluding our proof that $f$ is a covering map. 
\end{solutions}
\begin{prb}{2025-A-II-2}
	Consider the plane distribution in $\mathbb{R}^{3}$ spanned by two vector fields 
		\begin{equation*}
			V = \partial_{x} + 2xy\partial_{z} \quad \text{and} \quad W = x\partial_{x} + \partial_{y} + (2x^{2}y + x^{2} - 2y)\partial_{z}. 
		\end{equation*}
	\begin{enumerate}[itemsep =-2pt,label = (\roman{*})]
		\item Show that this distribution is integrable.
		\item Does the pair of vector fields $V$ and $W$ generate a coordinate system on integral surfaces? If not, find a pair that can play this role for the local integral surfaces passing through points $(0,0,z_{0})$. 
	\end{enumerate}	
\end{prb}
\begin{solutions}
	$ $\newline\vspace{-0.65cm}
	\begin{enumerate}[itemsep =-2pt,label = (\roman{*})]
		\item Let $D$ be the plane distribution in $\mathbb{R}^{3}$ spanned by the two vector fields $V$ and $W$ described above. By the Frobenius theorem, $D$ is integrable if and only if $D$ is involutive. Therefore, we will verify that $D$ is involutive, for which it suffices to check that the Lie Bracket of $V$ and $W$ is a smooth section of $D$. For ease of notation, we shall write $V = (1, 0, 2xy)$ and $W = (x, 1, 2x^{2}y + x^{2} - 2y)$. Then we observe the Lie Bracket of the vector fields to be 
			\begin{align}
				\begin{split}
					[V, W] &= V(W) - W(V) \\
					&= (1,0,2xy) \cdot (x,1,2x^{2}y + x^{2} - 2y) - (x, 1, 2x^{2}y + x^{2} - 2x) \cdot (1,0,2xy) \\
					&= (1,0,4xy + 2x) + (0,0,0) + (0, 0, 0) - (0,0,2xy) - (0,0,2y) - (0,0,0) \\
					&= (1,0,2xy) = V. 
				\end{split}
			\end{align}
		Hence, we conclude that $D$ is involutive, which then proves that $D$ is integrable. 
		\item We claim that $V$ and $W$ does \textit{not} generate a coordinate system on integral surfaces. Assume to the contrary that $V$ and $W$ generates coordinates $(u, v)$ on an integral surface $\mathscr{S}$ such that $V|_{\mathscr{S}} = \partial/\partial u$ and $W|_{\mathscr{S}} = \partial/\partial v$. Then it is straightforward to see that $V|_{\mathscr{S}}$ and $W|_{\mathscr{S}}$ commute with each other, which means that their Lie Bracket is identically zero. However, since $\mathscr{S}$ is an integral surface spanned by the distribution $D$, $[V|_{\mathscr{S}}, W|_{\mathscr{S}}] = ([V,W])|_{\mathscr{S}}$; but this is a contradiction since $[V, W]$ is nonvanishing everywhere (from (i)). Hence, by contradiction, it follows that $V$ and $W$ do \textit{not} generate a coordinate system on integral surfaces. To see the second part, consider the vector fields $\ti{V} = V$ and $\ti{W} = W - xV$, which do commute with each other. 
	\end{enumerate}
\end{solutions}
\begin{prb}{2003-J-11-5}
	Is the unit three-sphere $S^{3}$ diffeomorphic to the product of two other smooth manifolds of dimensions $> 0$? 
\end{prb}
\begin{solutions}
	We claim that $S^{3}$ \textit{cannot} be diffeomorphic to the product of two other smooth manifolds of nonzero dimensions. Assume to the contrary: let $M_{1}, M_{2}$ be smooth manifolds of nonzero dimensions such that $S^{3} \cong_{f} M_{1} \times M_{2}$. Since diffeomorphisms preserve dimensions, $3 = \dim{S^{3}} = \dim{M_{1} \times M_{2}} = \dim{M_{1}} + \dim{M_{2}}$. Since neither of $M_{1}, M_{2}$ are 0-manifolds, without loss of generality, we must have $\dim{M_{1}} = 1$ and $\dim{M_{2}} = 2$. Next, since $S^{3}$ is compact and connected, it follows that $M_{1} \times M_{2}$ must be compact and connected, so that $M_{1}$ and $M_{2}$ are compact and connected. Third, since $f: S^{3} \to M_{1} \times M_{2}$ is a diffeomorphism, the induced homomorphism $f_{\ast}: \pi_{1}(S^{3}) \longmapsto \pi_{1}(M_{1} \times M_{2}) = \pi_{1}(M_{1}) \times \pi_{1}(M_{2})$ is an isomorphism. Since $S^{n}$ is simply connected for all $n \geq 2$, $\pi_{1}(S^{3}) = \{0\}$. On the other hand, since the only compact connected smooth 1-manifold, up to diffeomorphism, is the circle $S^{1}$, $\pi_{1}(M_{1}) \cong \pi_{1}(S^{1}) \cong \mathbb{Z}$. This implies that $\pi_{1}(M_{1} \times M_{2})$ is not a trivial group, which contradicts our claim that $f_{\ast}$ is an isomorphism. Hence, by contradiction, $S^{3}$ cannot be diffeomorphic to the product of two smooth manifolds of nonzero dimensions. 
\end{solutions}

\newpage 
\begin{prb}{2017-A-I-5}
	On $\mathbb{R}^{4}$, with standard coordinates $(x, y, z, t)$, consider the 1-form $$\theta = xdy - ydx + zdt -tdz.$$ Is there a smooth immersion $j: \mathbb{R}^{3} \looparrowright \mathbb{R}^{4}$ such that $j^{\ast}\theta = 0$ everywhere? 
\end{prb}
\begin{solutions}
	Suppose there exists a smooth immersion $j: \mathbb{R}^{3} \looparrowright \mathbb{R}^{4}$ such that $j^{\ast}\theta = 0$ everywhere; without loss of generality, we may replace $\mathbb{R}^{3}$ with an open ball inside $\mathbb{R}^{3}$, which is diffeomorphic to $\mathbb{R}^{3}$ so that $0$ is not contained in the image of $j$. 
	
	Since $j^{\ast}\theta = 0$ everywhere, for each $p\in \mathbb{R}^{3}$, $\theta_{j(p)}(dj_{p}(v)) = 0$ for all $v \in T_{p}(\mathbb{R}^{3})$. But this implies that for every $p \in \mathbb{R}^{3}$, $T_{j(p)}j(\mathbb{R}^{3}) \subset \ker{\theta_{j(p)}}$. Since $j$ is an immersion, $dj_{p}$ is injective for every $p \in \mathbb{R}^{3}$ so that $\dim{T_{j(p)}j(\mathbb{R}^{3})} = 3$. Since $\theta \neq 0$ except at the origin, $\ker{\theta_{j(p)}}$ is also 3-dimensional. Hence, $T_{j(p)}j(\mathbb{R}^{3}) = \ker{\theta_{j(p)}}$. Therefore, the image $j(\mathbb{R}^{3})$ is a 3-dimensional integral manifold of the distribution $D = \ker{\theta_{j(p)}}$. 
	
	Now, we investigate the distribution given by $\theta$. We observe that 
		\begin{align}
			\begin{split}
				d\theta &= d(xdy - ydx + zdt - tdz) = 2dx\wedge dy + 2dz \wedge dt.  \\
				\theta \wedge d\theta &= 2zdt\wedge dx \wedge dy -2t dz\wedge dx \wedge dy + 2x dy \wedge dz \wedge dt -2y dx \wedge dz \wedge dt. 
			\end{split}
		\end{align}
	Hence, we see that $\theta \wedge d\theta \neq 0$ for all $q \in \mathbb{R}^{4} \setminus \{0\}$, which means by the Frobenius theorem that $D$ is not integrable anywhere in $\mathbb{R}^{4}\setminus \{0\}$. Hence, there is no immersed 3-dimensional submanifold whose tangent spaces equals $D$ at any point $p \neq 0$. But this contradicts our previous claim that the image of $j$ is such a manifold. Therefore, no such immersion can exist unless its image lies entirely in $\{0\}$, which is impossible. 
\end{solutions}
\begin{prb}{2017-A-I-5 (Practice)}
	On $\mathbb{R}^{4}$, with standard coordinates $(x, y, z, t)$, consider the 1-form $$\alpha = zdx + xdz + ydt - tdy.$$ Is there a smooth immersion $i: \mathbb{R}^{3} \looparrowright \mathbb{R}^{4}$ such that $j^{\ast}\alpha = 0$?
\end{prb}
\begin{solutions}
	Suppose $i: \mathbb{R}^{3}\looparrowright \mathbb{R}^{4}$ is a smooth immersion such that $i^{\ast}\alpha= 0$ everywhere on $\mathbb{R}^{3}$ for the 1-form described above; without loss of generality, we may replace $\mathbb{R}^{3}$ with a unit ball contained in $\mathbb{R}^{3}$ so that the image of $i$ does not contain point of the form $(0,y,0,t)$ for $y, t \in \mathbb{R}$. Let $D$ be the smooth rank 3 distribution on $\mathbb{R}^{4}\setminus \{0\}$ defined by $D_{q} = \ker{\alpha_{q}}$ for all $q \in \mathbb{R}^{4}$ ($D$ is indeed a smooth rank 3 distribution since $\alpha_{q} \neq 0$ for all $q \neq 0$). 
	
	Since $i^{\ast}\alpha = 0$, for every $p \in \mathbb{R}^{3}$ and $v \in T_{p}\mathbb{R}^{3}$, 
		\begin{equation}
			\alpha_{i(p)}(di_{p}(v)) = 0. 
		\end{equation}
	This implies that $T_{i(p)}i(\mathbb{R}^{3}) \subset \ker{\alpha_{i(p)}} = D_{i(p)}$ for every $p \in \mathbb{R}^{3}$. Since $i$ is an immersion, $\dim{T_{i(p)}i(\mathbb{R}^{3})} = 3$. From this, we conclude that $T_{i(p)}i(\mathbb{R}^{3}) = D_{i(p)}$ for every $p \in \mathbb{R}^{3}$; therefore, the image $i(\mathbb{R}^{3})$ is a 3-dimensional integral manifold of the distribution $D$. Now we investigate $D$. We observe that 
		\begin{align}
			\begin{split}
				d\alpha &= d(zdx + xdz + ydt - tdy) = 2dy \wedge dt. \\
				\alpha \wedge d\alpha &= 2zdx\wedge dy \wedge dt + 2x dz \wedge dy \wedge dt.
			\end{split}
		\end{align}
	We note that $\alpha \wedge d\alpha \neq 0$ everywhere in $\mathbb{R}^{4}\setminus \{(0,y,0,t): y, t \in \mathbb{R}\}$. Therefore, by the Frobenius theorem, $D$ is nowhere integrable on $\mathbb{R}^{4}\setminus \{(0,y,0,t): y, t \in \mathbb{R}\}$. This means there is no immersed 3-dimensional submanifold whose tangent spaces are sections of $D$ at any point $p \notin \{(0,y,0,t): y, t \in\mathbb{R}\}$. But the image of $i$ must be such a submanifold according to our work above. Therefore, by contradiction, $i$ cannot be an immersion. 
\end{solutions}
\begin{prb}{2017-A-I-5 (Practice II)}
	On $\mathbb{R}^{3}$, with standard coordinates $(x, y, z)$, consider the 1-form $$\beta = dx + dy + dz.$$ Is there a smooth immersion $k: \mathbb{R}^{2} \looparrowright \mathbb{R}^{3}$ such that $k^{\ast}\beta = 0$?
\end{prb}
\begin{solutions}
	We claim that there exists a smooth immersion $k: \mathbb{R}^{2} \looparrowright \mathbb{R}^{3}$ such that $k^{\ast}\beta = 0$. First, note that $\beta \neq 0$ for all $q \in \mathbb{R}^{3}$, which means that $\beta$ defines a smooth rank 2 distribution $D$ on $\mathbb{R}^{3}$ as follows: $D_{q} = \ker{\beta_{q}}$ for all $q \in \mathbb{R}^{3}$. Moreover, since $d\beta = d(dx + dy + dz) = 0$ everywhere on $\mathbb{R}^{3}$, $\beta \wedge d\beta = 0$ everywhere. This implies that $D$ is involutive; by the Frobenius Theorem, $D$ is everywhere integrable. Hence, $D$ has 2-dimensional integral manifolds. Now since $\beta = dx + dy + dz = d(x + y + z)$, $x + y + z$ is constant along any integral manifold. Hence, the integral manifolds are exactly of the form $\brac*{(x, y, z) \in \mathbb{R}^{3}: x + y + z  = c}$. Consider, for concreteness, the integral manifold $M = \brac*{(x, y, z) \in \mathbb{R}^{3}: x + y + z = 0}$, and let $k: \mathbb{R}^{2} \to M \subset \mathbb{R}^{3}$ be the map $(x, y, z) = k(u, v) = (u, v, -u -v)$. It is straightforward to check that $k$ is a smooth immersion and that
		\begin{equation}
			dx = du, \qquad dy = dv, \qquad dz = -du - dv, 
		\end{equation}
	so that 
		\begin{equation}
			k^{\ast}\beta = du + dv - du -dv = 0. 
		\end{equation}
\end{solutions}
\begin{prb}{2017-A-I-5 (Practice III)}
	On $\mathbb{R}^{3}$, with standard coordinates $(x, y, z)$, consider the 1-form $$\gamma = xdx + dy + dz.$$ Is there a smooth immersion $K: \mathbb{R}^{2} \looparrowright$ such that $k^{\ast}\gamma = 0$?
\end{prb}
\begin{solutions}
	We claim there exists a smooth immersion $k: \mathbb{R}^{2} \looparrowright \mathbb{R}^{3}$ such that $k^{\ast}\gamma = 0$. First, we observe that $\gamma \neq 0$ for all $q \in \mathbb{R}^{3}$, so that $\gamma$ defines a smooth rank 2 distribution $D$ on $\mathbb{R}^{3}$ as follows: $D_{q} = \ker{\gamma_{q}}$ for all $q \in \mathbb{R}^{3}$. Next, since 
		\begin{equation}
			d\gamma = d(x dx + dy + dz) = dx \wedge dx = 0 \text{ everywhere on $\mathbb{R}^{3}$}, 
		\end{equation}
	$\gamma \wedge d\gamma = 0$ everywhere on $\mathbb{R}^{3}$. Hence, by the Frobenius Theorem, $D$ is involutive and hence everywhere integrable. In particular, this means that $D$ has 2-dimensional integral manifolds at every $q \in \mathbb{R}^{3}$. We note that $\gamma = xdx + dy + dz = d((1/2)x^{2} + y + z)$, so that $(1/2)x^{2} + y + z$ is constant along the integral manifolds. Therefore, the integral manifolds of $D$ are of the form: 
		\begin{equation}
			\brac*{(x, y, z) \in \mathbb{R}^{3}: \frac{1}{2}x^{2} + y + z = c}, 
		\end{equation}
	where $c$ is some constant. For concreteness, let $M = \brac*{(x, y, z) \in \mathbb{R}^{3}: \nicefrac{1}{2}x^{2} + y + z = 0}$, and define the map $k: \mathbb{R}^{2} \to \mathbb{R}^{3}$ by $k(u, v) = (u, v, -\nicefrac{1}{2}u^{2} - v)$. Since the coordinate functions of $k$ are polynomials, $k$ is smooth. Moreover, since 
		\begin{equation}
			dk = \begin{pmatrix} 
					1 & 0 \\
					0 & 1 \\ 
					-u & -1
				\end{pmatrix}, 
		\end{equation}
	we see that the columns of $dk$ are linearly independent for all $(u, v) \in \mathbb{R}^{2}$ so that $\operatorname{rank}{dk_{p}} = 2$ for all $p \in \mathbb{R}^{2}$. Hence, $dk_{p}$ is injective, which proves that $k$ is an immersion. Finally, since $dx = du$, $dy = dv$, and $dz = -u du - dv$, 
		\begin{equation}
			k^{\ast}\gamma = u du + dv - udu - dv = 0.
		\end{equation}
	Hence, $k$ is a smooth immersion that satisfies the desired conditions. 
\end{solutions}


\newpage 
\subsection{Algebra}
\begin{prb}{2019-J-I-1}
	Let $A$ and $B$ be $n \times n$ invertible matrices over complex numbers, satisfying $$AB = \lambda BA \qquad \text{ for some } \lambda \in \mathbb{C}.$$ Prove that $A^{n}$ and $B$ commute. 
\end{prb}
\begin{solutions}
	Let $A, B$ be $n \times n$ invertible matrices over the complex numbers, and $\lambda \in \mathbb{C}$ such that $AB = \lambda BA$. Multiplying both sides by $A^{-1}$, 
		\begin{equation}
			B = A^{-1}AB = \lambda A^{-1}BA \implies \det{B} = \lambda^{n}\det{A^{-1}}\det{B}\det{A}. 
		\end{equation}
	Since $\det{A^{-1}} = \det{A}^{-1}$, it follows that $\det{B} = \lambda^{n}\det{B}$. Since $B$ is invertible, $\det{B} \neq 0$, which implies that $\lambda^{n} = 1$. Now we claim that $A^{m}B = \lambda^{m}BA^{m}$ for all $m \in \mathbb{Z}_{+}$. By the hypothesis, the case for $m = 1$ is true. Suppose that for some $m\geq 1$, $A^{m}B = \lambda^{m}BA^{m}$. Then for $m +1$, 
		\begin{equation}
			A^{m + 1}B = A(A^{m}B) = A(\lambda^{m}BA^{m}) = \lambda^{m}(AB)A^{m} = \lambda^{m}(\lambda BA)A^{m} = \lambda^{m + 1}BA^{m + 1}, 
		\end{equation}
	which means that our claim holds by induction. Hence, we finally see that 
		\begin{equation}
			A^{n}B = \lambda^{n}BA^{n} = BA^{n}, 
		\end{equation}
	which means that $A^{n}$ and $B$ commute. 
\end{solutions}
\begin{prb}{2024-J-I-4}
	For each field $K$, prove that the polynomial ring $K[x, y]$ in two variables is not a principal ideal domain. 
\end{prb}
\begin{solutions}
	Let $K$ be a field, and consider the polynomial ring $K[x, y]$. To show that $K[x, y]$ is not a principal ideal domain, it suffices to demonstrate an ideal of the polynomial ring that is not principal. Consider the ideal $(x, y)$ generated by $x, y \in K[x, y]$. It is straightforward to see that $(x,y)$ is a \textit{proper} ideal of $K[x, y]$ since, for example, $K \ni 1 \not \in K[x, y]$. Assume to the contrary that $(x, y) = (f(x, y))$ is principal. Then since $x$ and $y$ are contained in $(f(x, y))$, $f(x,y)$ must divide both $x$ and $y$. Since $f$ divides $x$, either $f$ is a unit or $f$ must be an associate of $x$. Likewise, since $f$ divides $y$, either $f$ is a unit or $f$ is an associate of $y$. But since $x$ and $y$ are not associates of each other, $f$ is forced to be a unit. But this implies that the ideal generated by $f$ must be the whole polynomial ring; i.e., $(x, y)= (f(x, y)) = K[x, y]$, which is impossible since $(x, y)$ was shown to be proper. Therefore, by contradiction, $(x, y)$ cannot be a principal ideal, and so $K[x, y]$ cannot be a PID. 
\end{solutions}
\begin{prb}{2020-J-I-1}
	Let $G$ be a finite non-abelian group, and let $Z(G)$ denote its center. Prove that $|Z(G)| \leq \frac{1}{4}|G|$, and then give an example where equality holds. 
\end{prb}
\begin{solutions}
	Let $G$ be a finite non-abelian group, and let $Z(G)$ denote its center. Assume to the contrary that $|Z(G)| > \frac{1}{4}|G| \implies |G|/|Z(G)| < 4$. Since $Z(G)$ is a normal subgroup of $G$, 
		\begin{equation}
			|G/Z(G)| = |G|/|Z(G)| < 4. 
		\end{equation}
	Since the order of a group must necessarily be a positive integer, the above observation forces $|G/Z(G)|$ to be $1, 2$, or $3$. (i) If $|G/Z(G)| = 1$, then since $|G| = |Z(G)|$, $G$ is abelian, which contradicts our hypothesis. (ii) if $|G/Z(G)| = 2$, then $G/Z(G) \cong \mathbb{Z}_{2}$ is cyclic, forcing $G$ to be abelian, which is a contradiction of the hypothesis and the fact that $Z(G)$ is a proper subgroup of $G$ since $|Z(G)| = \frac{1}{2}|G|$. (iii) Likewise, if $|G/Z(G)| = 3$, then $G/Z(G) \cong \mathbb{Z}_{3}$ is cyclic, forcing $G$ to abelian, which is a contradiction. Hence, by contradiction, we must have $|G/Z(G)| = |G|/|Z(G)| \geq 4 \implies |Z(G)| \leq \frac{1}{4}|G|$. For an explicit example where equality holds, consider the quaternion group $Q_{8}$ consisting of eight elements, whose center consists of just the two elements, $\pm 1$. 
\end{solutions}
\begin{prb}{2018-J-II-5}
	Let $G$ denote a group of order 80. Prove that $G$ contains a nontrivial normal subgroup. 
\end{prb}
\begin{solutions}
	Let $G$ denote a group of order $80 = 2^{4} \cdot 5$. By the Sylow Theorems, 
		\begin{equation}
			n_{2} \in \{1, 5\} \qquad \text{ and } \qquad n_{5} \in \{1, 16\}. 
		\end{equation}
	If $n_{5} = 1$ or $n_{2} = 1$, then we are done since either $G$ has a nontrivial, proper, normal subgroup of order 5 or a nontrivial, proper, normal subgroup of order 16. So suppose $n_{5} = 16$ and $n_{2} = 5$. Since every element of $G$ of order 5 must be contained in a Sylow 5-subgroup, Sylow 5-subgroups intersect trivally (by Lagrange's Theorem), and each Sylow 5-subgroup contains 4 nonidentity elements, $G$ must contain 64 elements of order 5. Now choose two distinct Sylow 2-subgroups, and denote these by $P_{1}$ and $P_{2}$. Since $P_{1}$ and $P_{2}$ are distinct, $P_{2}$ contains at least one nonidentity element that is not contained in $P_{1}$. Since each Sylow 2-subgroup contains 15 nonidentity elements whose order is a power of 2, we see that $G$ must contain at least 16 nonidentity elements whose order is a power of 2 (15 from $P_{1}$, and at least one from $P_{2}$). But this means that $G$ contains at least $16 + 64 + 1 = 81$ elements, which is a contradiction. Hence, either $n_{2} = 1$ or $n_{5}= 1$. 
\end{solutions}
\begin{prb}{2016-J-I-1}
	Let $G$ be a finite group of order 80. Prove that $G$ is not simple. 
\end{prb}
\begin{solutions}
	See above. 
\end{solutions}
\begin{prb}{2017-J-II-6}
	Without using the classification of simple groups, show that every simple group of order 60 has exactly 10 subgroups of order 3. Hint: Consider the action of this group on subgroups of order 3 by conjugation. 
\end{prb}
\begin{solutions}
	Let $G$ be a group of order $60 = 2^{2} \cdot 3 \cdot 5$. By Sylow's Theorem: 
		\begin{align}
			\begin{split}
				n_{2} &\in \{1, 3, 5,15\}. \\ 
				n_{3} &\in \{1, 4, 10\}. \\
				n_{5} &\in \{1, 6\}. 
			\end{split}
		\end{align}
	Every subgroup of order 3 must necessarily be a Sylow 3-subgroup of $G$, so it suffices to show that $n_{3} = 10$. Since $G$ is a simple group, $n_{3} \neq 1$, because if $n_{3} = 1$, then $G$ has a nontrivial proper normal subgroup of order 3, which is a contradiction. So either $n_{3} = 4$ or $n_{3} = 10$. Suppose $n_{3} = 4$, and let $G$ act on $\operatorname{Syl}_{3}(G)$ by conjugation; this group action induces a homomorphism, $\varphi: G \to S_{4}$. Since $|G| = 60 \nmid 24 = |S_{4}|$, it follows that $\varphi$ is not injective. Hence, the kernel $K = \ker{\varphi}$ is a nontrivial normal subgroup of $G$. $K$ must be a proper subgroup of $G$ since otherwise the group action is trivial, which means that $G$ contains a single Sylow 3-subgroup, which is impossible because $n_{3} = 4$. However, if $K$ is a proper nontrivial normal subgroup of $G$, then $G$ cannot be simple, which is a contradiction. Hence, $n_{3}$ must be 10. 
\end{solutions}
\begin{prb}{2013-J-I-2}
	Prove that every group of order 72 is non-simple. 
\end{prb}
\begin{solutions}
	Let $G$ be a group of order $72 = 2^{3} \cdot 3^{2}$. By Sylow's Theorem, 
		\begin{equation}
			n_{2} \in \{1, 3, 9\} \qquad \text{ and } \qquad n_{3} \in \{1, 4\}. 
		\end{equation}
	If $n_{3} = 1$, then we are done since $G$ contains a nontrivial normal proper subgroup of order $9$. So suppose $n_{3} = 4$, and let $G$ act on $\operatorname{Syl}_{3}(G)$ by conjugation. This group action induces a homomorphism $\varphi: G \to S_{4}$. Since $|G| = 72 \nmid 24 = |S_{4}|$, $\varphi$ cannot be injective. This means that $K = \ker{\varphi}$ must be a nontrivial normal subgroup of $G$. If $K = G$, then the group action is trivial which means that for any $P \in \operatorname{Syl}_{3}(G)$ and $g \in G$, $gPg^{-1} = P$ so that $P$ is normal. But since every Sylow 3-subgroup of $G$ is a conjugate of each other, $G$ must contain a single Sylow 3-subgroup contradicting our hypothesis that $n_{3} = 4$. Hence, $K$ must be a nontrivial normal proper subgroup of $G$. This concludes the proof that $G$ is non-simple. 
\end{solutions}
\begin{prb}{2012-J-I-6}
	Let $G$ be a group of order 315 whose center contains a subgroup $N$ of order 9. Prove that $G$ is abelian.  
\end{prb}
\begin{solutions}
	Let $G$ be a group of order $315 = 3^{2} \cdot 5 \cdot 7$, whose center $Z = Z(G)$ contains a subgroup $N$ of order 9; immediately, we see by Sylow's Theorem that $N$ is a Sylow 3-subgroup of $G$. By Lagrange's Theorem, $|Z| = 9k$ for some positive integer $k$. Then since $N \leq Z \leq G$, we have 
		\begin{equation}
			35 = [G:N] = [G:Z][Z:N] = k \cdot [G:Z] \implies |G|/|Z| =|G/Z| = 35k^{-1}. 
		\end{equation}
	Since $G/Z$ is a positive integer, we must have $k = 1, 5, 7$, or $35$. We cannot have $k = 5$: if $k = 5$, then $|G/Z| = 7$ so that $G/Z$ is cyclic. Then $G$ is abelian, forcing $|Z| = |G| \implies |G|/|Z| = 1$, which is a contradiction. For the same reason, $k \neq 7$. Suppose $k = 1$ so that $|G/Z| = 35$. Using Sylow's Theorem again, we see that $G/Z \cong \mathbb{Z}_{35}$, and hence is cyclic. Therefore, $G$ is abelian, which contradicts our hypothesis that $Z$ is a proper group of order $315/35 = 9$. Hence, by process of elimination, we deduce that $k = 35 \implies |G|/|Z| = 1 \implies |G| = |Z|$. Hence, $G$ is abelian. 
\end{solutions}
\begin{prb}{2009-J-I-5}
	Let $G$ be a group of order 63. Prove that $G$ is not simple (i.e., that $G$ contains a nontrivial normal subgroup). 
\end{prb}
\begin{solutions}
	Let $G$ be a group of order $63 = 3^{2} \cdot 7$. By Sylow's Theorem, 
		\begin{equation}
			n_{3} \in \{1, 7\} \qquad \text{ and } \qquad n_{7} \in \{1\}. 
		\end{equation}
	Hence, $G$ contains a unique Sylow 7-subgroup $P$, which must then be normal in $G$. Since Sylow 7-subgroups of $G$ have order 7, $P \lneq G$, so that $G$ is not simple. 
\end{solutions}
\begin{prb}{2007-J-I-3}
	Let $A$ and $B$ be $n \times n$ invertible matrices with complex entries such that $$AB = \lambda BA$$ for some scalar $\lambda$. Prove that $\lambda$ is an $n^{\text{th}}$ root of unity. If $\lambda$ is a primitive $n^{\text{th}}$ root of unity, prove that the characteristic polynomials of $A$ and $B$ both have the form $\chi(t) = t^{n} + \text{constant}$.
\end{prb}
\begin{solutions}
	Since $A$ is invertible, left-multiplying both sides by $A^{-1}$: 
		\begin{equation}
			AB = \lambda BA \implies B = \lambda A^{-1}BA. 
		\end{equation}
	Taking the determinant, 
		\begin{equation}
			\det{B} = \det(\lambda A^{-1}BA) = \lambda^{n}\det{A^{-1}}\det{B}\det{A} = \lambda^{n}\det{A}^{-1}\det{B}\det{A} = \lambda^{n}\det{B}. 
		\end{equation}
	Since $B$ is invertible, $\det{B} \neq 0$ so that cancelling this term from both sides yields $\lambda^{n} = 1$. Hence, $\lambda$ is an $n^{\text{th}}$ root of unit. Suppose $\lambda$ is a primitive $n^{\text{th}}$ root of unity, which means that $n$ is the \textit{smallest} positive integer for which $\lambda^{n} = 1$. \textcolor{red}{[!! Complete Later !!]}
\end{solutions}

\begin{prb}{2022-J-I-3}
	Show that a group of order $1,000,000$ contains a proper normal subgroup (i.e., is not simple). 
\end{prb}
\begin{solutions}
	Let $G$ be a group of order $1,000,000 = 250 \cdot 4 \cdot 250 \cdot 4 = (5^{3} \cdot 2^{3})^{2} = 2^{6}5^{6}$. By Sylow's Theorem, 
		\begin{equation}
			n_{5} \in \{1, 2^{4}\} \qquad \text{ and } \qquad n_{2} \in \{1, 5,5^{2}, \ldots, 5^{6}\}. 
		\end{equation}
	If $n_{5} = 1$, we are done. So assume $n_{5} = 16$, and let $G$ act on $\operatorname{Syl}_{5}(G)$ by conjugation. This induces a homomorphism $\varphi: G \to S_{16}$. Suppose that this mapping is an embedding, which means that $\ker{\varphi}$ is trivial. Then $|G|$ must divide $16!$. But since $10^{6} \nmid 16!$, $\varphi$ cannot be an embedding. Hence, $\ker{\varphi}$ is a nontrivial proper normal subgroup of $G$ so that $G$ is not simple. 
\end{solutions}
\begin{prb}{2006-A-II-5}
	Show that a group of order 55 is not simple. 
\end{prb}
\begin{solutions}
	Let $G$ be a group of order $55 = 5 \cdot 11$. Then by Sylow's Theorem, $G$ contains a unique Sylow 11-subgroup of order 11, which must then be normal. I.e., $G$ contains a nontrivial normal proper subgroup of order 11 so that $G$ is not simple. 
\end{solutions}

\end{document}