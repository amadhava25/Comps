\documentclass{article}
\usepackage{../../header}
\usepackage{notomath}
\setcounter{secnumdepth}{0}
\title{Comps Practice}
\author{\me}
\date{January 8, 2026}

\begin{document}
	\maketitle
	\fullline 
	\tableofcontents
	\halfline 
	\newpage 

\subsection{Comps Lemma}
\begin{prb}{Comps Lemma}
	Let $M, N$ be smooth, connected, $n$-manifolds, and $f: M \to N$ a (smooth) immersion. If $M$ is compact and nonempty, then $N$ is compact and $f$ is a (smooth) covering map. 
\end{prb}
\begin{solutions}
	Let $M, N$ be smooth, connected $n$-manifolds and $f: M \to N$ an immersion. Assume that $M$ is compact and nonempty. Since $\operatorname{dim}{N} = n$ and $f$ is an immersion, $\operatorname{rank}{df_{p}} = n$ at every $p \in M$. Hence, by the Inverse Function Theorem, $f$ is a local diffeomorphism. Since local diffeomorphisms are open maps, $f(M)$ is open in $N$. On the other hand, since the continuous image of compact sets is compact, $f(M)$ is compact in $N$. Since $N$ is Hausdorff, $f(M)$ is closed in $N$. Since $N$ is connected, $f(M) = N$. Therefore, $N$ is compact. 
	
	Now, let $q \in N$, and consider $f^{-1}(q) \subset M$. For each $x \in f^{-1}(q)$, let $U_{x}$ be an open neighborhood of $M$ containing $x$. Since $M$ is Hausdorff, we can shrink each $U_{x}$ so that these neighborhoods are pairwise disjoint. This means that each $x \in f^{-1}(q)$ is isolated, and hence $f^{-1}(q)$ is discrete. Since $M$ is compact, we conclude that $f^{-1}(q)$ must be finite; let $f^{-1}(q) = \{x_{1}, \ldots, x_{s}\}$. As noted above, for each $j = 1, \ldots, s$, let $U_{j}$ be a neighborhood of $x_{j}$ such that $f|_{U_{j}}: U_{j} \to V_{j} \subset N$ is a diffeomorphism. Then by the Hausdorff condition on $M$, shrink each $U_{j}$ so that $U_{i}\cap U_{j} = \varnothing$ for all $i \neq j$; $f$ remains a diffeomorphism on these shrunken neighborhoods. Setting $V = \bigcap_{1}^{s}f(U_{j})$ and taking $\ti{U}_{j} = f^{-1}(V) \cap U_{j}$ gives us an evenly covered neighborhood of $q$ in $N$. 
\end{solutions}

\begin{prb}{(Comps Lemma - Local Homeomorphisms)}
	Let $M, N$ be smooth, connected $n$-manifolds and $f: M \to N$ a local homeomorphism. If $M$ is compact and nonempty, then $N$ is compact and $f$ is a covering map. 
\end{prb}

\begin{prb}{(Comps Lemma - Submersions)}
	Let $M, N$ be smooth, connected $n$-manifolds and $F: M \to N$ a submersion. If $M$ is compact and nonempty, then $N$ is compact and $F$ is a covering map. 
\end{prb}
\begin{solutions}
	Let $M, N$ be smooth, connected $n$-manifolds and $F:M \to N$ a submersion. Also assume $M$ is compact and nonempty. Since submersions are open maps, $f(M)$ is open in $N$. On the other hand, since $F$ is continuous, continuous images of compacts sets are compact, and compact subsets of Hausdorff spaces are closed, $F(M)$ is closed in $N$. Hence, since $N$ is connected and $F(M)$ is nonempty, $F(M) = N$. This proves that $N$ is compact. We also claim that $F$ is a local diffeomorphism. Since $F$ is a submersion, at every $p \in M$, $dF_{p}: T_{p}M \to T_{f(p)}N$ is surjective. Since $\dim{M}= \dim{N} = n$, it follows that $dF_{p}$ is bijective. Hence, by the Inverse Function Theorem, $F$ is a local diffeomorphism. 
	
	All that remains to be seen is that $F$ is a covering map. Let $q \in N$ and consider the closed subset $F^{-1}(q) \subset M$. Since $F$ is a local diffeomorphism, for each $x \in F^{-1}(q)$, there exists a neighborhood $U_{x}$ such that $F|_{U_{x}}$ is a local diffeomorphism. Since $M$ is Hausdorff, we may shrink these neighborhoods so that they are pairwise disjoint. This means that each $x \in F^{-1}(q)$ is isolated, and hence, $f^{-1}(q)$ is discrete. Since $M$ is compact, $f^{-1}(q)$ is finite; let $f^{-1}(q) = \{x_{1},\ldots, x_{s}\}$. For each $j = 1, \ldots, s$, let $U_{j}$ be a neighborhood of $x_{j}$ such that $F|_{U_{j}}$ is a diffeomorphism. Since $M$ is Hausdorff, we shrink these neighborhoods such that they are pairwise disjoint; $F$ remains a diffeomorphism on each shrunken $U_{j}$. Set $V = \bigcap_{1}^{s}f(U_{j})$, and let $\ti{U}_{j} = f^{-1}(V) \cap U_{j}$. Hence, $V$ is an evenly covered neighborhood of $q \in N$, which concludes the proof that $F$ is a covering map. 
\end{solutions}

\subsection{Steinhaus Theorem}
\begin{prb}{(Steinhaus Theorem)}
	Let $E$ be a Lebesgue measurable subset of $\mathbb{R}^{n}$ such that $m^{n}(E) > 0$, and let $v_{1}, \ldots, v_{N}$ be a finite collection of vectors in $\mathbb{R}^{n}$. Then there exists $R > 0$, depending on $E$, and $M = \max\{|v_{1}|, \ldots, |v_{N}|\}$ such that for all $0 < r < R$, there exists $p \in S$ so that the $(N + 1)$-points, $p, p + rv_{1}, \ldots, p + rv_{1} + \dotsm + rv_{n}\in S$. 
\end{prb}
\begin{solutions}
	Let $E$ be a measurable subset of $\mathbb{R}^{n}$ with positive Lebesgue measure. We recall that the Lebesgue measure is \textit{regular} (which means it is both \textit{inner} and \textit{outer} regular). By inner regularity, there exists a compact set $K_{1} \subset E$ such that $m^{n}(K_{1}) > 0$. Let $\beta < (2^{N} - 1)^{-1}$; by outer regularity, there exists an open set $U$ containing $K_{1}$ such that 
		\begin{equation}
			m^{n}(U) \leq (1 + \beta)m^{n}(K_{1}). 
		\end{equation}
	Since $K_{1}$ is compact, $d_{1} = d(K_{1}, U^{c}) > 0$. Let $R = d_{1}/M$, and choose an arbitrary $r$ such that $0 < r < R$. First, observe that the set $K_{1} + rv_{1}$ is contained in $U$, since otherwise, 
		\begin{equation}
			d(K_{1}, U^{c}) \leq |rv_{1}| \leq rM < d_{1}. 
		\end{equation}
	Therefore, $K_{1} \cup (K_{1} + rv_{1}) \subset U$, and so 
		\begin{equation}
			m^{n}(U) \geq m^{n}(K_{1}\cup (K_{1} + rv_{1})) = m^{n}(K_{1}) + m^{n}(K_{1} + rv_{1}) - m^{n}(K_{1} \cap (K_{1} + rv_{1})). 
		\end{equation}
	Since the Lebesgue measure is translation invariant, 
		\begin{equation}
			m^{n}(K_{1} \cap (K_{1} + rv_{1})) \geq 2m^{n}(K_{1}) - m^{n}(U) \geq 2m^{n}(K_{1}) - m^{n}(K_{1}) - \beta m^{n}(K_{1}) = (1 - \beta)m^{n}(K_{1}). 
		\end{equation}
	Since $\beta < 1$, it follows that $m^{n}(K_{1} \cap (K_{1} + rv_{1})) > 0$, and so $K_{1} \cap (K_{1} + rv_{1}) \neq \varnothing$. Now we proceed by induction. For each $i = 1, \ldots, N$, let $K_{i + 1} = K_{i} \cap (K_{i} + rv_{i})$. Each $K_{i} + rv_{i}$ must be contained in $U$ (by a generalization of the argument made above) and each $K_{i + 1} \subset K_{i} \subset U$. We claim that for each $i$, $m^{n}(K_{i + 1}) \geq (1- (2^{i} - 1)\beta)m^{n}(K_{1})$. We have already proven the base case $i = 1$. So assume the result holds for some $1 \leq m < N$. Then 
		\begin{equation}
			m^{n}(U) \geq m^{n}(K_{i} \cup (K_{i} + rv_{i})) = m^{n}(K_{i}) + m^{n}(K_{i} + rv_{i}) - m^{n}(K_{i} \cap (K_{i} + rv_{i})). 
		\end{equation}
	By translation invariance of the Lebesgue measure, 
		\begin{align}
			\begin{split}
			m^{n}(K_{i + 1}) = m^{n}(K_{i} + rv_{i}) &\geq 2m^{n}(K_{i}) - m^{n}(U) \geq 2(1 - (2^{i} - 1)\beta)m^{n}(K_{1}) - (1 + \beta)m^{n}(K_{1}) \\
			&= m^{n}(K_{1}) - 2^{i + 1}\beta m^{n}(K_{1}) + 2 \beta m^{n}(K_{1}) - \beta m^{n}(K_{1}) \\
			&= (1 - (2^{i + 1} - 1)\beta)m^{n}(K_{1}). 
			\end{split} 
		\end{align}
	Hence, since $\beta < (2^{N} - 1)^{-1}$, we obtain a nested sequence of compact subsets $\varnothing \neq K_{N + 1} \subset K_{N} \subset \dotsm\subset K_{1} \subset U$. Let $q \in K_{N + 1}$ be arbitrary. Since $K_{N + 1} = K_{N} \cap (K_{N} + rv_{N})$, the point $q - rv_{N}$ is contained in $K_{N}$. Then since $K_{N} = K_{N - 1} \cap (K_{N - 1} \cap rv_{N - 1})$, $q - rv_{N} - rv_{N - 1} \in K_{N - 1}$. Proceeding inductively, we obtain the sequence $\{q, q - rv_{N}, q - rv_{N} - rv_{N - 1}, \ldots, q - rv_{N} - \dotsm -rv_{1}\} \subset K_{1} \subset E$. Hence, the proof concludes. 
\end{solutions}
\begin{prb}{Fat Cantor Set}
	There exists a closed nowhere dense subset $E \subset \mathbb{R}$ of positive Lebesgue measure. 
\end{prb}
\begin{solutions}
	Consider the interval $[0,1] \subset \mathbb{R}$. Delete the open set fourth $(\frac{3}{8}, \frac{5}{8})$, which leaves the two line segments 
		\begin{equation}
			\left[0, \frac{3}{8}\right] \cup \left[\frac{5}{8}, 1\right]. 
		\end{equation}
	From each of these segments, remove the corresponding open middle fourths again, yielding the set 
		\begin{equation}
			\left[0,\frac{5}{32}\right] \cup \left[\frac{7}{32}, \frac{3}{8}\right] \cup \left[\frac{5}{8}, \frac{25}{32}\right] \cup \left[\frac{27}{32}, 1\right]. 
		\end{equation}
	We repeat this procedure inductively, removing an interval of width $4^{n}$ from each of the remaining $2^{n - 1}$ intervalsOverall, we remove intervals of total length 
		\begin{equation}
			\sum_{n = 0}^{\infty}\frac{2^{n}}{2^{2n + 2}} = \frac{1}{2}, 
		\end{equation}
	which means that the Lebesgue measure of the overall set is $1/2 > 0$. Moreover, the set is the intersection of a sequence of closed sets so that it is closed. Finally, the set does not contain any intervals so that it has empty interior. Therefore, the fat Cantor set is a closed nowhere dense subset of $\mathbb{R}$ of positive Lebesgue measure. 
\end{solutions}

\newpage 
\subsection{January 2025}
\begin{prb}{2025-J-I-1 (Algebra)}
Let $R$ be a UFD (unique factorization domain). Let $F$ be its quotient field. Let $p(x) = x^{n} + b_{n - 1}x^{n- 1} + \dotsm + b_{0} \in F[x]$ be a monic polynomial with coefficients in $R$ admitting a root $a \in F$. Prove that $a \in R$. 
\end{prb}
\begin{solutions}
	Let $R$ be a UFD, and $F$ its quotient field. Let $p(x) = x^{n} + b_{n- 1}x^{n - 1} + \dotsm + b_{0} \in F[x]$ be a monic polynomial with coefficients in $R$ admitting a root $a \in F$. Let $a = c/d$, where $c, d \in R\setminus \{0\}$ so that $\gcd(c, d) = 1$. By definition of a root, we must have 
		\begin{equation}
			0 = p(a) = \left(\frac{c}{d}\right)^{n} + b_{n - 1}\left(\frac{c}{d}\right)^{n - 1} + \dotsm + b_{0}. 
		\end{equation}
	Multiplying both sides by $d^{n}$, 
		\begin{equation}
			c^{n} + d(b_{n-1}c^{n-1} + b_{n - 2}c^{n - 2}d \dotsm + b_{0}d^{n-1}) = 0 \implies c^{n} = -d(b_{n - 1}c^{n - 1} + \dotsm + b_{0}d^{n-1}). 
		\end{equation}
	From this, we observe that $d \mid c^{n}$. If $d$ is not a unit in $R$, then every nonidentity irreducible divisor of $d$ is an irreducible divisor of $c^{n}$, and hence an irreducible divisor of $c$. But this contradicts our hypothesis that $\gcd(c, d) = 1$. Hence, $d$ has to be a unit of $R$. If $v \in R\setminus \{0\}$ such that $dv= vd = 1$, then 
		\begin{equation}
			a = \frac{c}{d} = \frac{c}{d}\cdot \frac{v}{v} = cv \in R. 
		\end{equation}
	Hence, this concludes the proof. 
\end{solutions}
\begin{prb}{2025-J-I-2 (Real Analysis)}
	Let $\{f_{n}\}_{n\geq 1}$ be a sequence of Lebesgue-measurable functions on $[0,1]$. Suppose that 
		\begin{equation*}
			\int_{0}^{1}f^{2}_{n}dm \leq \frac{1}{n^{2}} \quad \text{ for all $n\geq 1$}. 
		\end{equation*}
	Prove that $f_{n}$ converges to $0$ a.e. on $[0,1]$. 
\end{prb}
\begin{solutions}
	Let $\{f_{n}\}_{n\geq 1}$ be a sequence of Lebesgue-measurable functions on $[0,1]$ so that 
		\begin{equation}
			\int_{0}^{1}f_{n}^{2}dm \leq \frac{1}{n^{2}} \quad \text{ for all } n \geq 1. 
		\end{equation}
	Consider the sequence $\brac*{\sum_{1}^{m}f_{n}^{2}}$, which is increasing and converges a.e. to $\sum_{1}^{\infty}f_{n}^{2}$. Hence, by the Monotone Convergence Theorem, 
		\begin{equation}
			\sum_{1}^{\infty}\int_{0}^{1} f_{n}^{2} = \lim_{m \to \infty}\sum_{1}^{m}\int_{0}^{1} f_{n}^{2} = \lim_{m \to \infty}\int_{0}^{1} \sum_{1}^{m}f_{n}^{2} = \int_{0}^{1}\sum_{1}^{\infty}f_{n}^{2} \leq \int_{0}^{1}\sum_{1}^{\infty}\frac{1}{n^{2}} < \infty. 
		\end{equation}
	Therefore, $\sum_{1}^{\infty}f_{n}^{2} \in L^{1}(\mathbb{R})$, which means that $\sum_{1}^{\infty}f_{n}^{2} < \infty$ a.e. on $[0,1]$. Hence, $\sum_{n = 1}^{\infty}f_{n}^{2}$ converges a.e. on $[0,1]$. This implies that $f_{n}^{2} \to 0$ a.e. on $[0,1]$, and hence $f_{n} \to 0$ a.e. on $[0,1]$. 
\end{solutions}
\begin{prb}{2025-J-I-3 (Geometry/Topology)}
	Let $M$ be an orientable, connected, and compact smooth $n$-manifold with boundary. Show that there is no (smooth) retraction to the boundary, that is, there does not exist a smooth map $f: M \to \partial M$ such that $f(x) = x$ when $x \in \partial M$. 
\end{prb}
\begin{solutions}
	Let $M$ be an orientable, connected, and compact smooth $n$-manifold with boundary. Assume to the contrary that there exists a smooth map $f: M \to \partial M$ such that $f(x) = x$ when $x \in \partial M$. Let $\omega \in \Omega^{n- 1}(\partial M)$ be a volume form for the boundary of $M$. Since volume forms are closed (hence, $\omega$ is closed), we have by Stokes's theorem 
		\begin{equation}
			0 = \int_{M}f^{\ast}d\omega = \int_{M}d(f^{\ast}\omega) = \int_{\partial M}f^{\ast}\omega = \int_{\partial M}\omega > 0, 
		\end{equation}
	which is a contradiction. Hence, by contradiction, there cannot exist a smooth retraction to the boundary. 
\end{solutions}
\begin{prb}{2025-J-II-3 (Algebra)}
	Let $V$ be a vector space of dimension $n$ over $\mathbb{Q}$. Let $T: V \to V$ be a linear transformation with minimal polynomial $x^{4} - x^{2} - 2$ over $\mathbb{Q}$. Show that $n$ must be even. 
\end{prb}
\begin{solutions}
	Consider $V$ as a module over the ring $\mathbb{Q}[x]$ by letting a polynomial $f(x) \in \mathbb{Q}[x]$ act as the linear operator $f(T)$. Since $\dim{V} = n$, this module is finitely generated. By the structure theorem for finitely generated modules over principal ideal domains, $V$ is isomorphic to a direct sum of modules of the form $\mathbb{Q}[x]/(p(x))^{e}$, where $p(x) \in \mathbb{Q}[x]$ is irreducible. Moreover, each $p(x)$ must divide the minimal polynomial of $T$. We note that over $\mathbb{Q}$, 
		\begin{equation}
			x^{4} - x^{2} - 2 = (x^{2} - 2)(x^{2} + 1),
		\end{equation}
	where both factors are irreducible over $\mathbb{Q}$. Therefore, the only choices for $p(x)$ are $x^{2} - 2$ and $x^{2} + 1$. Therefore, $\mathbb{Q}[x]/(p(x))^{e}$ has dimension $\deg{p} \cdot e = 2e$ for each choice of $p$. Since $2$ divides these dimensions, we conclude that $2$ must divide $n$. Hence, $n$ is even. 
\end{solutions}
\begin{prb}{2025-J-II-4 (Topology)}
	Let $\Sigma_{2}$ be a compact oriented surface of genus 2. Is there a submersion $f: \Sigma_{2} \to S^{1} \times S^{1}$, where $S^{1}$ denotes the unit circle? 
\end{prb}
\begin{solutions}
	Assume to the contrary that there exists a submersion $f: \Sigma_{2} \to S^{1} \times S^{1}$, where $S^{1}$ denotes the unit circle. Since $\dim{\Sigma_{2}} = \dim{S^{1} \times S^{1}} = 2$, $df_{p}$ must have constant rank 2 at every $p \in \Sigma_{2}$. Hence, $f$ is a local diffeomorphism. Since $f$ is a local diffeomorphism, $f(\Sigma_{2})$ is compact in $S^{1} \times S^{1}$; since $S^{1} \times S^{1}$ is Hausdorff, $f(\Sigma_{2})$ must be closed in $S^{1} \times S^{1}$. On the other hand, since local diffeomorphisms are open maps, $f(\Sigma_{2})$ is open in $S^{1} \times S^{1}$. Therefore, since $S^{1} \times S^{1}$ is connected, $f(\Sigma_{2}) = S^{1} \times S^{1}$; i.e., $f$ is surjective. Therefore, $f$ is a covering map. This means that the induced homomorphism, $f_{\ast}: \pi_{1}(\Sigma_{2}) \to \pi_{1}(S^{1}\times S^{1})$ is injective, and so $f_{\ast}(\pi_{1}(\Sigma_{2})) \cong \operatorname{img}{f_{\ast}} \leq \pi_{1}(S^{1} \times S^{1})$. However, $\pi_{1}(S^{1} \times S^{1}) \cong \mathbb{Z} \times\mathbb{Z}$ is an abelian group and cannot have any nonabelian subgroups, whereas $\pi_{1}(\Sigma_{2})$ is nonabelian. Hence, by contradiction, $f$ cannot be a submersion. 
\end{solutions}
\begin{prb}{2025-J-II-5 (Analysis)}
	Let $V$ be a topological vector space whose topology is Hausdorff. Let $X_{1}$ and $X_{2}$ be two Banach spaces, and assume there exist continuous linear bijections $F_{1}: X_{1} \to V$ and $F_{2}: X_{2} \to V$. Show that there is a continuous linear bijection $G: X_{1} \to X_{2}$. 
\end{prb}
\begin{solutions}
	Assume the given hypotheses. Let $G = F_{2}^{-1} \circ F_{2}$. Since $F_{1}, F_{2}$ are bijections, we conclude that $G$ is a bijection. Likewise, since $F_{1}, F_{2}$ are linear, $G$ must also be linear. It suffices to prove that $G$ is continuous. By the Closed Graph Theorem, continuity of $G$ is equivalent to the graph of $G$ being a closed subspace of $X_{1} \times X_{2}$. Let $\{x_{n}\} \subset X_{1}$ be a sequence in $X_{1}$ such that $x_{n} \to x$ and $y_{n} = Gx_{n} \to y$. We need to show that $y = Gx$. By continuity of $F_{1}$, $F_{1}x_{n} \to Fx$. By continuity of $F_{2}$, 
		\begin{equation}
			F_{2}y = \lim F_{2}y_{n} = \lim F_{2}Gx_{n} = \lim F_{1}x_{n} = F_{1}x. 
		\end{equation}
	Since $F_{2}$ is bijective, $y = F_{2}^{-1}F_{1}x = Gx$. Hence, the graph of $G$ is closed, which implies that $G$ is continuous. 
\end{solutions}

\subsection{August 2025}
\begin{prb}{2025-A-I-1 (Geometry/Topology)}
	Let $S$ be a closed orientable surface of genus 4 and $C$ be an embedded circle that partitions $S$ into two subsurfaces of genus 2. Does $S$ retract to $C$? 
\end{prb}
\begin{solutions}
	We claim that the answer is no; assume to the contrary that there exists a retraction $r: S \to C$. Let $i: C \hookrightarrow S$ be the inclusion map so that $r \circ i = \operatorname{id}_{C}$. Now since $C$ is an embedded circle, $H_{1}(C)$ (i.e., the first homology) is isomorphic to $H_{1}(S^{1}) = \mathbb{Z}$. On the other hand, since $C$ is separating in $S$, its homology class in $H_{1}(S)$ is the zero element. Hence, the induced map $i_{\ast}: H_{1}(C) \to H_{1}(S)$ is the zero map. But this is impossible since if $i_{\ast}$ is the zero map,
		\begin{equation}
			0 = r_{\ast} \circ i_{\ast} = (r \circ i)_{\ast} = \operatorname{id}_{H_{1}}(C),
		\end{equation}
	which is a contradiction. Hence, no such retraction can exist. 
\end{solutions}

\begin{prb}{2025-A-I-6 (Algebra)}
	Let $f(x)$ be an irreducible polynomial of degree $n$ over a field $F$, and let $g(x)$ be any polynomial in $F[x]$. Prove that every irreducible factor of the composition $f(g(x))$ has degree divisible by $n$. 
\end{prb}
\begin{solutions}
	Let $h(x)$ be an irreducible factor of $f(g(x))$ in $F[x]$ and let $\alpha$ be the root of $h(x)$ in some algebraic closure of $F$. Since $h$ is irreducible and $\alpha$ is a root, the minimum polynomial of $\alpha$ over $F$ is $h$. Therefore, 
		\begin{equation}
			\deg{h} = [F(\alpha): F]. 
		\end{equation}
	Now since $\alpha$ is a root of $h(x)= f(g(x))$, $f(g(\alpha)) = 0$. In particular, $g(\alpha)$ is a root of $f$. Since $f$ is irreducible of degree $n$ over $F$, the minimal polynomial of $g(\alpha)$ over $\alpha$ is $f$. Hence, 
		\begin{equation}
			[F(g(\alpha)):F] = n. 
		\end{equation}
	Since $F \subset F(g(\alpha)) \subset F(\alpha)$, by the Tower Law, 
		\begin{equation}
			\deg{h} = [F:(\alpha):F] = [F(\alpha):F(g(\alpha))] \cdot [F(g(\alpha)): F] = n[F(\alpha):F(g(\alpha))],
		\end{equation}
	so that $n \mid \deg{h}$. Hence, this concludes the proof. 
\end{solutions}
\begin{prb}{2025-A-II-2 (Geometry/Topology)}
	Consider the plane distribution in $\mathbb{R}^{3}$ spanned by two vector fields 
		\begin{equation}
			V = \partial_{x} + 2xy\partial_{z}, \qquad W = x\partial_{x} + \partial_{y} + (2x^{2}y + x^{2} - 2y)\partial_{z}. 
		\end{equation}
	\begin{enumerate}[itemsep =-2pt,label = (\roman{*})]
		\item Show that this distribution is integrable. 
		\item Does the pair of vector fields $V$ and $W$ generate a coordinate system on integral surfaces? If not, find a pair that can play this role for the local integral surfaces passing through points $(0,0,z_{0})$. 
	\end{enumerate}
\end{prb}
\begin{solutions}
	$ $\newline \vspace{-0.65cm}
	\begin{enumerate}[itemsep =-2pt,label = (\roman{*})]
		\item Let $D$ be the plane distribution in $\mathbb{R}^{3}$ spanned by the two vector fields $V$ and $W$ given above. Then by the Frobenius Theorem, $D$ is integrable if and only if $D$ is involutive, which is true if and only if the Lie Bracket of $V$ and $W$ is a smooth section of $D$ at each $p \in \mathbb{R}^{3}$. We observe that: 
			\begin{align}
				\begin{split}
					V(W) &= \left(\partial_{x} + 2xy\partial_{z}\right)(x\partial_{x} + \partial_{y} + (2x^{2}y + x^{2} - 2y)\partial_{z}) \\
					&= \partial_{x} + (4xy + 2x)\partial_{z}. \\
					W(V) &= \left(x\partial_{x} + \partial_{y} + (2x^{2}y + x^{2} - 2y)\partial_{z}\right)(\partial_{x}+ 2xy\partial_{z}) \\
					&= 2xy\partial_{z} + 2x\partial_{z}. 
				\end{split}
			\end{align}
		Therefore, for any $p \in \mathbb{R}^{3}$, 
			\begin{equation}
				[V, W] = V(W) - W(V) = \partial_{x} + 2xy\partial_{z} = V. 
			\end{equation}
		Since $V$ is a smooth section of $D$, we conclude that $D$ is involutive, and hence integrable. 
		\item Let $\mathscr{S}$ be an integral surface, and assume there are coordinates $(u, v)$ on $\mathscr{S}$ such that $V|_{\mathscr{S}} = \partial_{u}$ and $W|_{\mathscr{S}} = \partial_{v}$. Then we observe that $[V|_{\mathscr{S}}, W|_{\mathscr{S}}] = \partial_{u}(\partial_{v}) - \partial_{v}(\partial_{u}) = 0$. On the other hand, 
			\begin{equation}
				[V|_{\mathscr{S}}, W|_{\mathscr{S}}] = ([V, W])|_{\mathscr{S}} = V|_{\mathscr{S}} \neq 0, 
			\end{equation}
		which is a contradiction. Therefore, $V$ and $W$ cannot generate a coordinate system on integral surfaces. However, consider the fields $\ti{V} = V$ and $\ti{W} = W - xV$ on $\mathbb{R}^{3}$. Then since
			\begin{equation}
				[\ti{V}, \ti{W}] = V(W - xV) - (W - xV)(V) = VW - xVV - W(V) + xVV = 0, 
			\end{equation} 
		and so this pair generates a coordinate system on all integral surfaces. 
	\end{enumerate}
\end{solutions}


\subsection{January 2024}
\begin{prb}{2024-J-I-1 (Algebra)}
	For distinct odd primes $p$ and $q$, prove that every finite group of order $2pq$ is a semidirect product of a normal subgroup of order $pq$ and a subgroup of order 2. 
\end{prb}
\begin{solutions}
	Let $G$ be a group of order $2pq$, where $p, q$ are distinct odd primes. Without loss of generality, assume $q > p$. By Sylow's Theorem, 
		\begin{equation}
			n_{q} \in \{1, 2, p, 2p\} \cap \{1, q + 1, \ldots\} = 1, 
		\end{equation}
	since $q > 2$ and $q > p$. Therefore, $G$ has a unique, normal, Sylow $q$-subgroup, which we denote as $Q$. Let $P$ be a Sylow $p$-subgroup of $G$. By the Second Isomorphism Theorem, we conclude that $N = PQ$ is a subgroup of $G$ of order $|P||Q| = pq$. Since $|G: N| = 2pq/(pq) = 2$, where $2$ is the smallest prime dividing $|G|$, we conclude that $N$ is a normal subgroup of $G$. Next, by Cauchy's Theorem, $G$ contains an element of order 2. Let $M$ be the subgroup generated by this element, which also must have order 2. By Lagrange's Theorem, $N \cap M = \{e\}$. Next, 
		\begin{equation}
			|NM| = \frac{|N||M|}{|N \cap M|} = |N||M| = 2pq = |G|,
		\end{equation}
	so that $G = NM$. Therefore, we conclude that $G = N \rtimes M$. 
\end{solutions}
\begin{prb}{2024-J-I-2 (Geometry/Topology)}
	Let $p: E \to B$ be a covering space map, with $B$ and $E$ path connected. Choose a point $e_{0} \in E$ and $b_{0} \in B$ such that $p(e_{0}) = b_{0}$. This gives us a subgroup $H = p_{\ast}\pi_{1}(E, e_{0})$ of the fundamental group $G = \pi_{1}(B, b_{0})$. 
	
	Construct a bijection between the fiber $p^{-1}(b_{0})$ and the set of right cosets of $H$ and prove that this is indeed a bijection. Prove that the number of sheets of $p$ equals the index $(G: H)$. 
\end{prb}
\begin{solutions}
	Assume all of the given hypotheses. Let $\phi: \pi_{1}(B, b_{0}) \to p^{-1}(b_{0})$ be the lifting correspondence induced by $p$ defined by $\phi([f]) = \tilde{f}(1)$, where $\tilde{f}$ is the lift of $f$, and let $\Phi: \pi_{1}(B, b_{0})/H \to p^{-1}(b_{0})$ be the map induced by $\phi$. It suffices to prove that $\Phi$ is a bijection. 
		\begin{enumerate}[itemsep =-2pt,label = (\roman{*})]
			\item Since $E$ is path connected and $p: E \to B$ is a covering map, the lifting correspondence $\phi$ must be surjective. Hence, since $\Phi$ is induced by $\phi$, it follows that $\Phi$ is also surjective. 
			\item Now we will show that $\Phi$ is injective. Let $f$ and $g$ be two paths in $B$, and $\ti{f}, \ti{g}$ their liftings to paths in $E$ that begin at $e_{0}$. We must show that $\ti{f}(1) = \ti{g}(1)$ iff $[f] \in H \ast [g]$. 
			
			($\Leftarrow$) Suppose $[f] = [h \ast g]$, where $h = p \circ \tilde{h}$ for some loop $\tilde{h}$ in $E$ based at $e_{0}$. Since $\tilde{g}$ is a path in $E$ that \textit{begins} at $e_{0}$, the product $\tilde{h} \ast \tilde{g}$ is well-defined. Since $[f] = [h \ast g]$, it follows that $\tilde{f}$ and $\tilde{h} \ast \tilde{g}$ must end at the same point. Hence, $\tilde{f}$ and $\tilde{g}$ end at the same point. Therefore, $\phi([f]) = \phi([g])$. 
			
			($\Rightarrow$) Suppose $\phi([f]) = \phi([g])$, which means that $\tilde{f}(1) = \tilde{g}(1)$. Then the product of $\tilde{f}$ with the reverse of $\tilde{g}$ is well-defined and is a loop $\tilde{h}$ in $E$ based at $e_{0}$. By direct computation, $[\tilde{h} \ast \tilde{g}] = [\tilde{f}]$. If $\tilde{F}$ is a path homotopy between $\tilde{h} \ast \tilde{g}$ and $\tilde{f}$, then $p \circ \tilde{F}$ is a path homotopy between $h \ast g$ and $f$, which means that $[f] \in H\ast [g]$. Hence, this concludes the proof that $\Phi$ is injective. 
		\end{enumerate}
	Hence, $|p^{-1}(b_{0})| = |G/H| = (G:H)$. 
\end{solutions}
\begin{prb}{2024-J-I-3 (Complex Analysis)}
	Suppose $f$ is continuous on the plane and holomorphic on $\mathbb{C}\setminus \mathbb{R}$. Prove that $f$ is holomorphic on the whole plane. 
\end{prb}
\begin{solutions}
		Let $f:\mathbb{C}\to\mathbb{C}$ be continuous on $\mathbb{C}$ and holomorphic on 
		$\mathbb{C}\setminus\mathbb{R}$. We show that $f$ is holomorphic on all of $\mathbb{C}$.
		
		By Morera's Theorem, it suffices to prove that
		
		
		\[
		\oint_{\gamma} f(z)\,dz = 0
		\]
		
		
		for every closed piecewise $C^{1}$ curve $\gamma \subset \mathbb{C}$.
		
		If $\gamma$ lies entirely in the upper or lower half-plane, then $f$ is holomorphic
		on a neighborhood of $\gamma$, and by the Cauchy--Goursat theorem,
		
		
		\[
		\oint_{\gamma} f(z)\,dz = 0.
		\]
		
		
		
		Now suppose that $\gamma$ intersects the real axis.  
		For $\varepsilon>0$, construct a closed piecewise $C^{1}$ curve $\gamma_{\varepsilon}$ 
		by modifying $\gamma$ so that it avoids the real axis by small detours of height 
		$\pm\varepsilon$. Then $\gamma_{\varepsilon}\subset \mathbb{C}\setminus\mathbb{R}$, 
		so $f$ is holomorphic on a neighborhood of $\gamma_{\varepsilon}$, and hence
		
		
		\[
		\oint_{\gamma_{\varepsilon}} f(z)\,dz = 0.
		\]
		
		
		
		Since $f$ is continuous on $\mathbb{C}$, it is uniformly continuous on compact sets, 
		and the total length of the detours tends to $0$ as $\varepsilon\to 0$. Therefore,
		
		
		\[
		\lim_{\varepsilon\to 0} \oint_{\gamma_{\varepsilon}} f(z)\,dz
		= \oint_{\gamma} f(z)\,dz.
		\]
		
		
		Thus $\oint_{\gamma} f(z)\,dz = 0$.
		
		Since this holds for every closed piecewise $C^{1}$ curve in $\mathbb{C}$, 
		Morera's Theorem implies that $f$ is holomorphic on all of $\mathbb{C}$.
\end{solutions}


\begin{prb}{2024-J-I-4 (Algebra)}
	For each field $K$, prove that the polynomial ring $K[x, y]$ in two variables is not a principal ideal domain. 
\end{prb}
\begin{solutions}
	Let $K$ be a field, and consider the polynomial ring $K[x, y]$. Let $(x, y)$ be the proper ideal of $K[x, y]$ generated by the monomials $x$ and $y$. Assume to the contrary that $(x, y) = (f(x,y))$ where $f(x, y) \in K[x, y]$ is not a unit of the polynomial ring. Since $x \in (f(x, y))$, $f(x, y) \mid x$. By our assumption that $f$ is not a unit, it follows that $f(x, y)$ is an associate of $x$. Likewise, $f(x, y)$ must be an associate of $y$. But this is impossible since $x$ and $y$ are not associates of each other. This forces $f(x, y)$ to be a unit, which means that $(f(x, y)) = K[x, y]$. But this contradicts the fact that $(x, y) = (f(x, y))$ is a proper ideal. Hence, by contradiction, $(x, y)$ is not a principal ideal, and so $K[x, y]$ is not a principal ideal domain. 
\end{solutions}
\begin{prb}{2024-J-I-5 (Geometry/Topology)}
	Let $\alpha$ be a closed 1-form on $\mathbb{RP}^{n}$, $n > 1$. Show that if $f:[0,1] \to \mathbb{RP}^{n}$ is a smooth function such that $f(0) = f(1)$, then 
		\begin{equation*}
			\int_{[0,1]}f^{\ast}\alpha = 0. 
		\end{equation*}
	Include all calculations that are relevant to your solution. 
\end{prb}
\begin{solutions}
	We recall that $H^{k}(\mathbb{RP}^{n}) = 0$ for all $0 < k < n$ so that $H^{1}(\mathbb{RP}^{n}) = 0$ if $n > 1$. In particular, this means that $\alpha$ is also an exact 1-form on $\mathbb{RP}^{n}$. Let $g$ be a smooth function on $\mathbb{RP}^{n}$ so that $\alpha = dg$. Then 
		\begin{equation}
			\int_{0}^{1}f^{\ast}\alpha = \int_{0}^{1}f^{\ast}dg = \int_{0}^{1}d(f^{\ast}g) = g(f(1)) - g(f(0)) = 0, 
		\end{equation}
	where the last equality follows from the fact that $f(1) = f(0)$. Hence, the proof concludes. 
\end{solutions}
\begin{prb}{2024-J-I-6 (Real Analysis)}
	Let $f$ and $g$ be Lebesgue-measurable functions on $\mathbb{R}$. Define the convolution 
		\begin{equation*}
			(f \ast g)(x) = \int_{\mathbb{R}}f(x - y)g(y)\;dy
		\end{equation*}
	for all $x$ such that the integral exists. Prove that if $f \in L^{p}(\mathbb{R})$ and $g \in L^{q}(\mathbb{R})$ with $p, q \in (1, \infty)$ satisfying $\frac{1}{p} + \frac{1}{q} = 1$, then $f \ast g$ is a bounded continuous function on $\mathbb{R}$. 
\end{prb}
\begin{solutions}
	Assume the given hypotheses. Then by H\"older's inequality, for any $x \in \mathbb{R}$, 
		\begin{equation}
			\abs{(f \ast g)(x)} \leq \int_{\mathbb{R}}\abs{f(x -y)g(y)}\;dy \leq \norm{f(x - \cdot)}_{p} \norm{g}_{q}. 
		\end{equation}
	Since $L^{p}$ norms are translation invariant, $\norm{f(x - \cdot)}_{p} = \norm{f}_{p}$. Hence, $|(f \ast g)(x)| \leq \norm{f}_{p}\norm{g}_{q} = M < \infty$ for all $x \in \mathbb{R}$. Hence, we conclude that $f \ast g$ is a bounded function on $\mathbb{R}$. Next, let $\tau_{z}$ be the translation operator defined by $\tau_{z}f = f(x - z)$. Since translation operators are continuous in the $L^{p}$ norms, $\norm{\tau_{z}f - f} \to 0$ as $z \to 0$, which implies that 
		\begin{align}
			\norm{\tau_{z}(f \ast g) - (f\ast g)}_{\infty} &= \norm{(\tau_{z}f - f)\ast g}_{\infty} \\
			&\leq \norm{\tau_{z}f - f}_{p}\norm{g}_{q} \longrightarrow 0 \text{ as $z \to 0$}. 
		\end{align}
	Hence, $f \ast g$ is uniformly continuous, and therefore continuous on $\mathbb{R}$. Note that the inequality used in the second line of the above equation comes from \textit{Young's convolution inequality}, which states the following: 
		\begin{quote}
			\textbf{(Young's Convolution Inequality)} Let $f \in L^{p}$, $g \in L^{q}$, and $p^{-1} + q^{-1} = r^{-1} + 1$. Then $\norm{f \ast g}_{r} \leq \norm{f}_{p} \norm{g}_{q}$.
		\end{quote}
	In our case, we had $r = \infty$ so that $r^{-1} = 0$. 
\end{solutions}
\begin{prb}{2024-J-II-2}
	Suppose $E \subset \mathbb{R}^{2}$ is a set of positive Lebesgue measure. Show that there are points $a, b, c$ in $E$ such that their connecting segments form a right angle, i.e., $a - b$ is perpendicular to $c - b$ (as vectors in $\mathbb{R}^{2}$). 
\end{prb}
\begin{solutions}
	Let $E \subset \mathbb{R}^{2}$ be a set of positive Lebesgue measure; let $m^{2}$ denote the Lebesgue measure on $\mathbb{R}^{2}$. Let $\{v_{1}, v_{2}, v_{3}\}$ be a collection of vectors in $\mathbb{R}^{2}$ such that $v_{1} \perp v_{2}$, and $v_{3} = -v_{1}$. Without loss of generality, assume that $\norm{v_{j}} = 1$ for all $j = 1, \ldots, 3$. By inner regularity of the Lebesgue measure, there exists a compact subset $K_{1} \subset E$ such that $m^{2}(K_{1}) > 0$. Taking $\beta < 1/7$, by outer regularity of the Lebesgue measure, there exists an open set $U$ containing $K_{1}$ such that $m^{2}(U) \leq (1 + \beta)m^{2}(K_{1})$.
	
	Since $K_{1}$ is compact, $d_{1} = d(K_{1}, U^{c}) > 0$. Hence, let $R = d_{1}$. Fix some $r \in (0, R)$ and consider the set $K_{1} + rv_{1}$. We claim that $K_{1} + rv_{1} \subset U$ since if otherwise, 
		\begin{equation}
			d(K_{1}, U^{c}) \leq |rv_{1}| = r < d_{1}, \text{ which is a contradiction.}
		\end{equation}
	Hence, $K_{1} \cup (K_{1} + rv_{1}) \subset U$, which means that 
		\begin{equation}
			m^{2}(U) \geq m^{2}(K_{1} \cup (K_{1} + rv_{1})) = m^{2}(K_{1}) + m^{2}(K_{1} + rv_{1}) - m^{2}(K_{1} \cap (K_{1} + rv_{1})). 
		\end{equation}
	By translation invariance of the Lebesgue measure, $m^{2}(K_{1}) + m^{2}(K_{1} + rv_{1}) = 2m^{2}(K_{1})$ so that 
		\begin{equation}
			m^{2}(K_{1} \cap (K_{1} + rv_{1})) = 2m^{2}(K_{1}) - m^{2}(U) \geq (1 - \beta)m^{2}(K_{1}). 
		\end{equation}
	Since $\beta < 1$, $m^{2}(K_{1} \cap (K_{1} + rv_{1})) > 0$ so that the set is nonempty. For $i = 1, \ldots, 3$, define $K_{i + 1} = K_{i} \cap  (K_{i} + rv_{i})$. Generalizing the argument from above shows that each $K_{i + 1} \subset U$. We claim that $m^{2}(K_{i + 1}) \geq (1 - (2^{i} - 1)\beta)m^{2}(K_{1})$ for each $i$; the above work establishes the result for $i = 1$. Now assume the result holds for some $1 \leq j < 3$. Then 
		\begin{equation}
			m^{2}(U) \geq m^{2}(K_{j} \cup (K_{j} + rv_{j})) = m^{2}(K_{j}) + m^{2}(K_{j} + rv_{j}) - m^{2}(K_{j} \cap (K_{j} + rv_{j})) = 2m^{2}(K_{j}) - m^{2}(K_{j} \cap (K_{j} + rv_{j})). 
		\end{equation} 
	Therefore, 
		\begin{align}
			\begin{split} 
			m^{2}(K_{j} \cap (K_{j} + rv_{j})) &= 2m^{2}(K_{j}) - m^{2}(U) \\
			&\geq 2m^{2}(K_{1}) -2^{j + 1}\beta m^{2}(K_{1}) + 2\beta m^{2}(K_{1}) - m^{2}(K_{1}) - \beta m^{2}(K_{1}) \\
			&= (1 - (2^{j + 1} - 1)\beta)m^{2}(K_{1}). 
			\end{split} 
		\end{align} 
	Since $\beta < (2^{3} - 1)^{-1} = 7^{-1}$, we conclude that each $K_{i}$ is nonempty. Hence, we obtain a nested sequence $\varnothing \neq K_{4} \subset \dotsm \subset K_{1} \subset E$. Let $q \in K_{4}$; since $K_{4} = K_{3} \cap (K_{3} + rv_{3})$, $q - rv_{3} \in K_{3}$. Following inductively, we obtain a sequence of points $\{p, p + rv_{1}, p + rv_{1} + rv_{2}, p + rv_{1}+ rv_{2} + rv_{3}\}\subset E$, with $p \in K_{1}$, and $p + rv_{j} \in K_{j}$ for $j= 1, 2, 3$ (note we have renamed $q - rv_{1} - \dotsm - rv_{3} = p$, and so on). Let $a = p$, $b = p + rv_{1}$, and $c = p + rv_{1} + rv_{2}$. Then $a - b = -rv_{1}$ and $c - b= rv_{2}$. By hypothesis on $v_{1}$ and $v_{2}$, $a - b$ is orthogonal to $c - b$. 
\end{solutions}
\begin{prb}{2024-J-II-3 (Geometry/Topology)}
	Let $\Sigma$ be a genus 2 surface embedded in $\mathbb{R}^{3}$ as shown in the picture. Let $M$ be the closure of the \textit{unbounded} component of $\mathbb{R}^{3}\setminus \Sigma$; in other words, $M$ is the part of $\mathbb{R}^{3}$ which is \textit{not} enclosed by $\Sigma$. \vspace{-0.25cm}
		\begin{enumerate}[itemsep =-2pt,label = (\alph{*})]
			\item Compute $\pi_{1}(M)$. 
			\item Is $\Sigma$ a retract of $M$? 
		\end{enumerate}
	\begin{center}
		\includegraphics[width = 0.3\linewidth]{"../Graphics/2024JII3"}
	\end{center}
\end{prb}
\begin{solutions}
	$ $\newline 
	\begin{enumerate}[itemsep =-2pt,label = (\alph{*})]
		\item 
	\end{enumerate}
\end{solutions}
\begin{prb}{2024-J-II-5 (Real Analysis)}
	Let $P$ be the vector space over $\mathbb{R}$ of (finite degree) polynomials in the variable $x \in (-\infty, \infty)$. Show that $P$ cannot be a Banach space with respect to any norm, that is, if $\norm{\empspace}$ is some norm on $P$, then $P$ is not complete under this norm. Hint: You may use the Baire Category Theorem. 
\end{prb}
\begin{solutions}
	We recall the Baire Category Theorem: 
		\begin{quote}
			\textbf{(Baire Category Theorem)} Let $X$ be a complete metric space. \vspace{-0.35cm}
				\begin{enumerate}[itemsep =-2pt,label = (\alph{*})]
					\item If $\{U_{n}\}_{1}^{\infty}$ is a sequence of open dense subsets of $X$, then $\bigcap_{1}^{\infty}U_{n}$ is dense in $X$. 
					\item $X$ is not a countable union of nowhere dense sets. 
				\end{enumerate}
		\end{quote}
	For each positive integer $n$, let $P_{n}$ be the vector space of all polynomials of degree $\leq n$ so that $P = \bigcup_{n \in \mathbb{N}}P_{n}$. Let $\norm{\empspace}$ be a norm on $P$ and assume to the contrary that $P$ is complete under this norm; this means that $P$ is a complete metric space. Since $X$ cannot be the countable union of nowhere dense sets, it follows that there exists some positive integer $n_{0}$ so that $P_{n_{0}}$ is not nowhere dense; i.e., the closure of $P_{n_{0}}$ has nonempty interior. Since any finite dimensional vector subspace of a normed vector space is closed, it follows that $P_{n_{0}}$ is closed in $P$; i.e., $\overline{P}_{n_{0}} = P_{n_{0}}$. Hence, by our hypothesis, $P_{n_{0}}$ has nonempty interior. Let $p \in P_{n_{0}}$ and $B(r, p)$ a ball of radius $r > 0$ centered at $p$ that is contained \textit{entirely within} $P_{n_{0}}$. Let $u \in P\setminus \{0\}$ be arbitrary, and set 
		\begin{equation}
			u' = p + \frac{r \cdot u}{2\norm{u}} \implies u' \in B(r, p) \subset P_{n_{0}}. 
		\end{equation}
	But since $P_{n_{0}}$ is a vector space, this implies that $u \in P_{n_{0}}$. Since $u$ was arbitrary in $P$, this means that $P_{n_{0}}= P$, which is a contradiction. Hence, every $P_{n}$ must have empty interior, which then contradicts the Baire Category Theorem. Hence, $P$ cannot be a Banach space with respect to any norm. 
\end{solutions}

\begin{prb}{2024-J-II-6 (Geometry/Topology)}
	Let $M$ be a smooth $n$-manifold, and let $\varphi$ be a differential $k$-form on $M$ which is closed, in the sense that $d\varphi = 0$. At each point $p \in M$, define 
		\begin{equation}
			D_{p}= \brac*{v \in T_{p}M: v\righthalfcup\varphi = 0}, 
		\end{equation}
	where $\righthalfcup$ denotes the interior product. Assume $\ell \coloneqq \dim{D_{p}}$, so that $D \subset TM$ is a rank-$\ell$ vector sub-bundle of the tangent bundle of $M$. Prove that $D$ is an integrable distribution of $\ell$-planes, in the sense of the Frobenius Theorem. 
\end{prb}
\begin{solutions}
	By the Frobenius Theorem, it suffices to prove that $D$ is involutive, which is to say that if $X, Y$ are smooth sections of $D$, then $[X, Y]$ is also a smooth section of $D$. Indeed, let $X, Y$ be smooth sections of $D$, which means that $X\righthalfcup \varphi, Y\righthalfcup \varphi = 0$. Observe that,
		\begin{equation}
			[X,Y]\righthalfcup \varphi = \mathscr{L}_{X}(Y\righthalfcup \varphi) - Y\righthalfcup(\mathscr{L}_{X}\varphi). 
		\end{equation}
	By hypothesis, $Y \righthalfcup \varphi = 0$ so that $\mathscr{L}_{X}(Y \righthalfcup \varphi) = 0$. On the other hand, by Cartan's Formula, 
		\begin{equation}
			\mathscr{L}_{X}\varphi = d(X\righthalfcup \varphi) + X\righthalfcup d\varphi = 0, 
		\end{equation}
	by the hypotheses. Hence, this shows that $[X, Y] \righthalfcup \varphi = 0$, and so $[X, Y]$ is a smooth section of $D$. Therefore, $D$ is involutive, which means that it is Frobenius integrable. 
\end{solutions}
\begin{prb}{2024-J-II-4 (Algebra)}
	Let $\alpha = \sqrt{2 + \sqrt{3}} \in \mathbb{C}$. Let $K$ be the smallest Galois extension of $\mathbb{Q}$ which contains $\alpha$. Describe the Galois group $\operatorname{Gal}(K/\mathbb{Q})$. 
\end{prb}
\begin{solutions}
	Let $\alpha = \sqrt{2 + \sqrt{3}} \in \mathbb{C}$, and $K$ the smallest Galois extension of $\mathbb{Q}$ that contains $\alpha$. We start by finding the minimal polynomial of $\alpha$. We observe that 
		\begin{equation}
			\alpha^{2} = 2 + \sqrt{3} \implies (\alpha^{2} - 2)^{2} - 3 = 0. 
		\end{equation}
	Simplifying, 
		\begin{equation}
			\alpha^{4} - 4\alpha^{2} + 1 = 0. 
		\end{equation}
	I.e., the polynomial $x^{4} - 4x^{2} + 1$ is the minimal polynomial of $\alpha$. Solving this polynomial over an algebraic closure of $\mathbb{Q}$, we obtain the four roots, $\pm\sqrt{2 + \sqrt{3}}, \pm \sqrt{2 - \sqrt{3}}$. Hence, the elements of the Galois group $\operatorname{Gal}(K/\mathbb{Q})$ are the identity permutation, the permutation $\sigma$ that fixes $\pm \sqrt{2 - \sqrt{3}}$ and permutes $\pm \sqrt{2 + \sqrt{3}}$, the permutation $\tau$ that fixes $\pm\sqrt{2 + \sqrt{3}}$ and permutes $\pm \sqrt{2 - \sqrt{3}}$, and the permutation $\sigma \tau$. Labeling these roots as $\alpha_{1}, \ldots, \alpha_{4}$, we see that $\operatorname{Gal}(K/\mathbb{Q}) \cong \{1, (1\;2), (3\;4), (1\;2)(3\;4)\} \cong V \subset S_{4}$, where $V$ is the Klein-4 subgroup. 
\end{solutions}


\subsection{August 2024}
\begin{prb}{2024-A-I-1 (Geometry/Topology)}
	Let $M$ be a smooth compact manifold without boundary, and let $\varphi$ be a smooth closed 1-form on $M$ that has the property that $\varphi \neq 0$ at every point of $M$. Prove that the first de Rham cohomology $\h{1}(M)$ of the given manifold is non-zero. 
\end{prb}
\begin{solutions}
	Let $M$ be a smooth compact manifold without boundary and let $\varphi$ be a smooth closed 1-form on $M$ that has the property that $\varphi \neq 0$ at every point of $M$. Suppose that $\varphi$ is exact; i.e., assume there exists a smooth function $f$ on $M$ such that $\varphi = df$. By the Extreme Value Theorem, since $M$ is compact, $f$ must have either a maximum or minimum value at some point $p \in M$. Since all of the first-order partial derivatives of $f$ must vanish at the point $p$ where $f$ attains its maximum/minimum value, $df|_{p} = 0$. This means that $\varphi$ must also vanish at $p$, which contradicts our hypothesis that $\varphi$ is nowhere vanishing. Hence, by contradiction, $\varphi$ cannot be an exact form. Since $\h{1}(M) \coloneqq \brac*{\text{closed 1-forms on $M$}}/\brac*{\text{exact 1-forms on $M$}}$ and we have shown the existence of a closed 1-form that is \textit{not} an exact 1-form, we conclude that $\h{1}(M)$ is non-zero. 
\end{solutions}
\begin{prb}{2024-A-I-2 (Geometry/Topology)}
	Suppose that $f: \Sigma_{2} \to \Sigma_{1}$ is a continuous map between a genus 2 closed orientable surface $\Sigma_{2}$ and a torus $\Sigma_{1}$. Prove that $f$ is not a local homeomorphism. In other words, show that there exists a point $x \in \Sigma_{2}$ which does not have an open neighborhood $U \subset \Sigma_{2}$ on which the restriction $f|_{U}$ is a homeomorphism between $U$ and $f(U)$. 
\end{prb}
\begin{solutions}
	Before presenting our argument, we will state and prove a quick technical lemma. 
		\begin{quote}
			(\textbf{Modified Comps Lemma}) Let $M$ and $N$ be smooth connected manifolds, and $f: M \to N$ a local homeomorphism. If $M$ is compact and nonempty, then $N$ is compact and $f$ is a covering map. 
			
			\begin{proof}
				Let $M$ and $N$ be smooth connected manifolds, and $f: M \to N$ a local homeomorphism. Since $f$ is an open map, $f(M)$ is open in $M$. Next since the continuous image of a compact set is compact and a compact subset of a Hausdorff space is closed, $f(M)$ is closed in $N$. Hence, since $N$ is connected, $f(M) = N$, which means $N$ is connected and $f$ is surjective. 
				
				Now let $q \in N$, and consider the closed subset $f^{-1}(q) \subset M$. For each $x \in f^{-1}(q)$, there exists a neighborhood $U_{x}$ such that $f|_{U_{x}}$ is a homeomorphism. Since $M$ is Hausdorff, we may shrink these neighborhoods so that they are pairwise disjoint. Hence, each $x \in f^{-1}(q)$ is isolated, which means $f^{-1}(q)$ is discrete. Since discrete subspaces of compact spaces is necessarily finite, $f^{-1}(q)$ is finite; let $\{x_{1}, \ldots, x_{s}\} = f^{-1}(q)$. As stated above, for each $j = 1, \ldots, s$, we may find a neighborhood $U'_{j}$ such that $f|_{U'_{j}}$ is a homeomorphism. Using Hausdorff-ness of $M$, we may shrink these neighborhoods to obtain the collection $\{\ti{U}_{j}\}_{1}^{s}$ of pairwise disjoint open neighborhoods. Set $V = \bigcap_{1}^{s}U_{j}$, which is then an evenly covered neighborhood of $q$. Therefore, $f$ is a covering map. 
			\end{proof}
		\end{quote} 
	Now assume to the contrary that $f: \Sigma_{2} \to \Sigma_{1}$ is a local homeomorphism; by the modified Comps Lemma, $f$ is a covering map. Moreover, $\Sigma_{2}$ must be a $k$-sheeted covering space for some finite positive integer $k$, which means that $\chi(\Sigma_{2}) = k \cdot \chi(\Sigma_{1})$. However, this is impossible since $\chi(\Sigma_{1}) = 0$, while $\chi(\Sigma_{2}) = 2 - 2(2) = 2 - 4 = -2$. Therefore, $f$ cannot be a local homeomorphism. 
\end{solutions}

\begin{prb}{2024-A-I-5 (Algebra)}
	Determine whether or not the complex number $i = \sqrt{-1}$ is in the field $\mathbb{Q}(\alpha)$, where $\alpha$ is any complex number subject to the relation $\alpha^{3} + \alpha + 1 = 0$. Justify your answer. 
\end{prb}
\begin{solutions}
	The polynomial $x^{3} + x + 1$ has no roots in $\mathbb{Q}$ (by the rational root test), and so is irreducible (since it is a cubic). This means that $\mathbb{Q}(\alpha)$ is an extension of degree 3 over $\mathbb{Q}$. Therefore, it cannot contain the field $\mathbb{Q}(i)$, which has degree $2$ over $\mathbb{Q}$ (since the minimal polynomial of $i$ is $x^{2} + 1$) since $2 \nmid 3$. 
\end{solutions}
\begin{prb}{2024-A-II-1 (Geometry/Topology)}
	Recall that $S^{n}$ denotes the unit sphere in $\mathbb{R}^{n + 1}$. Also recall that a smooth map is called a smooth submersion if its differential is everywhere surjective. Prove or disprove each of the following statements: \vspace{-0.25cm}
		\begin{enumerate}[itemsep =-2pt,label = (\alph{*})]
			\item There is a smooth submersion $F: S^{3} \to S^{1}$. 
			\item There is a smooth submersion $F: S^{3} \to S^{2}$. 
		\end{enumerate}
\end{prb}
\begin{solutions}
	$ $\newline \vspace{-1cm}
	\begin{enumerate}[itemsep =-2pt,label = (\alph{*})]
		\item \textcolor{red}{[!! Complete Later !!]}
	\end{enumerate}
\end{solutions}

\begin{prb}{2024-A-II-2 (Geometry/Topology)}
	On $\mathbb{R}^{5}$, equipped with standard coordinates \newline $(v, w, x, y,z)$, consider the 1-form 
		\begin{equation*}
			\theta = dz + v\;dx + w\;dy. 
		\end{equation*}
	Are there two smooth functions $f, g: \mathbb{R}^{5} \to \mathbb{R}$ such that $\theta = f\;dg$? Justify your answer by means of concrete solutions. 
\end{prb}
\begin{solutions}
	We claim that there do \textit{not} exist smooth functions $f, g: \mathbb{R}^{5} \to \mathbb{R}$ such that $\theta = f\;dg$. Assume to the contrary. First, we observe that if $\theta = fdg$, then 
		\begin{equation}
			d\theta = d(fdg) = df \wedge dg \implies \theta \wedge d\theta = f dg \wedge df \wedge dg = 0. 
		\end{equation}
	I.e., if $\theta = fdg$, then $\theta \wedge d\theta$ must be identically zero. However, since $\theta = dz + vdx + wdy$, we note that 	
		\begin{equation}
			d\theta = d^{2}z + d(vdx) + d(wdy) = dv \wedge dx + dw \wedge dy \implies \theta \wedge d\theta = dz \wedge dv \wedge dx + dz \wedge dw \wedge dy + vdx \wedge dw \wedge dy + w dy \wedge dv \wedge dx, 
		\end{equation}
	which is nowhere vanishing on $\mathbb{R}^{5}$. Hence, by contradiction, there cannot exist two smooth functions $f, g: \mathbb{R}^{5} \to \mathbb{R}$ such that $\theta = fdg$. 
\end{solutions}


\subsection{January 2023}
\begin{prb}{2023-J-II-4 (Geometry/Topology)}
	Prove that $S^{2} \times S^{2}$ is not diffeomorphic to $M_{1} \times M_{2} \times M_{3}$, where $M_{1}, M_{2}, M_{3}$ are smooth manifolds of nonzero dimension. 
\end{prb}
\begin{solutions}
	We begin with a technical lemma, that we will use to prove the desired result. 
		\begin{quote}
			(\textbf{Comps Lemma}) Let $M, N$ be smooth, connected $n$-manifolds and $f: M \to N$ a (smooth) immersion. If $M$ is compact and nonempty, then $N$ is compact and $f$ is a (smooth) covering map. 
			
			\begin{proof}
				Let $M, N$ be smooth connected $n$-manifolds, $f: M \to N$ an immersion, and $M$ compact and nonempty. Since $\dim{N} = n$ everywhere and $f$ is an immersion, $df_{p}: T_{p}M \to T_{f(p)}N$ has constant rank $n$ everywhere. Hence, by the Inverse Function Theorem, $f$ is a local diffeomorphism. Since local diffeomorphisms are open maps, $f(M)$ is open in $N$. Next since the continuous image of compact sets is compact, $f(M)$ is compact in $N$. Since $N$ is Hausdorff, $f(M)$ must be closed in $N$. Therefore, since $N$ is connected, we conclude that $f(M) = N$. This means that $N$ is compact and $f$ is surjective. All that remains is to show that $f$ is a covering map. 
				
				Let $q \in N$, and consider $f^{-1}(q)$, which is closed in $M$. For each $x \in f^{-1}(q)$, there exists a neighborhood $U_{x}$ of $x$ such that $f|_{U_{j}}$ is a diffeomorphism.  Since $M$ is Hausdorff, we may shrink these neighborhoods so that they are pairwise disjoint. This means that each $x \in f^{-1}(q)$ is isolated. Hence, $f^{-1}(q)$ is discrete in $M$. Since discrete subspaces of compact spaces must be finite, it follows that $f^{-1}(q)$ is finite; let $f^{-1}(q) = \{x_{1}, \ldots, x_{s}\}$. As stated above, for each $j = 1, \ldots, s$, we can find a neighborhood $U_{j}$ of $x_{j}$ such that $f|_{U_{j}}: U_{j} \to V_{j} \subset N$ is a diffeomorphism. Since $M$ is Hausdorff, we may shrink these neighborhoods so that $U_{i} \cap U_{j} = \varnothing$ for all $i \neq j$; $f$ restricted to each of these new $U_{j}$'s remains a diffeomorphism. Set $V = \bigcap_{1}^{s}f(U_{j})$, and define $\ti{U}_{j} = f^{-1}(V) \cap  U_{j}$. For each $j$, $f: \ti{U}_{j} \to V$ is a diffeomorphism and $V =  \bigsqcup_{1}^{s}f(U_{j})$. Hence, $V$ is an evenly covered neighborhood of $q$, so that $f$ is a covering map.  
			\end{proof}
		\end{quote} 
	Now, assume to the contrary that $f: S^{2} \times S^{2} \to M_{1} \times M_{2} \times M_{3}$ is a diffeomorphism; since diffeomorphisms preserve dimensions and $M_{1}, M_{2}, M_{3}$ have nonzero dimensions, it follows, without loss of generality,  that $M_{1}, M_{2}$ are 1-dimensional and $M_{3}$ is 2-dimensional. Since diffeomorphisms of manifolds are immersions, by the Comps Lemma, $M_{1} \times M_{2} \times M_{3}$ must be compact and connected; by projecting onto each manifold, $M_{1}, M_{2}, M_{3}$ must be compact and connected. Moreover, the induced group homomorphism $f_{\ast}: \pi_{1}(S^{2} \times S^{2}) \to \pi_{1}(M_{1} \times M_{2} \times M_{3}) = \pi_{1}(M_{1}) \times \pi_{1}(M_{2}) \times \pi_{1}(M_{3})$ must be an isomorphism. Since $S^{2}$ is simply connected, 
		\begin{equation}
			\pi_{1}(S^{2} \times S^{2}) = \pi_{1}(S^{2}) \times \pi_{1}(S^{2}) = \{0\}. 
		\end{equation}
	On the other hand, since the only compact connected 1-manifold, up to diffeomorphism, is the unit circle $S^{1}$, and $\pi_{1}(S^{1}) \cong \mathbb{Z}$ is not trivial, $\pi_{1}(M_{1} \times M_{2} \times M_{3})$ is not trivial. But this contradicts our claim that $f_{\ast}$ is an isomorphism. Hence, by contradiction, $f$ cannot be a diffeomorphism. 
\end{solutions}
\begin{prb}{2023-J-II-3 (Geometry/Topology)}
	Consider the form $\omega = (x^{2} + x + y)dy \wedge dz$ on $\mathbb{R}^{3}$. Let $S^{2} \subset \mathbb{R}^{3}$ be the unit sphere, and $i: S^{2} \to \mathbb{R}^{3}$ be the inclusion map. \vspace{-0.25cm}
		\begin{enumerate}[itemsep =-2pt,label = (\alph{*})]
			\item Calculate $\int_{S^{2}}i^{\ast}\omega$. 
			\item Construct a closed form $\alpha$ on $\mathbb{R}^{3}$ such that $i^{\ast}\alpha = i^{\ast}\omega$, or show that such a form $\alpha$ does not exist. 
		\end{enumerate}
\end{prb}
\begin{solutions}
	$ $\newline \vspace{-0.65cm}
	\begin{enumerate}[itemsep =-2pt,label = (\alph{*})]
		\item (\textbf{Method 1}) Consider the form $\omega = (x^{2} + x + y)dy\wedge dz$ on $\mathbb{R}^{3}$, and let $i: S^{2}\hookrightarrow \mathbb{R}^{3}$ be the inclusion map. Let $D = [0, \pi] \times [0, 2\pi]$, and $F: D \to S^{2}$ be the coordinate map defined by 
			\begin{equation}
				F(\varphi, \theta) = (\sin(\varphi)\cos(\theta), \sin(\varphi)\sin(\theta),\cos(\varphi)). 
			\end{equation}
		Taking $D_{1} = [0, \pi] \times [0, \pi]$ and $D_{2} = [0, \pi]\times [\pi, 2\pi]$, and letting $F_{1} = F|_{D_{1}}$ and $F_{2} = F|D_{2}$, we observe that 
			\begin{equation}
				\int_{S^{2}}i^{\ast}\omega = \int_{D_{1}}F_{1}^{\ast}i^{\ast}\omega + \int_{D_{1}}F_{2}^{\ast}\omega = \int_{D_{1}}(i \circ F_{1})^{\ast}\omega + \int_{D_{2}}(i \circ F_{2}^{\ast})\omega = \int_{D}F^{\ast}\omega, 
			\end{equation}
		where the last equality follows from the fact that $i \circ F_{1,2} = F_{1,2}$. We observe that 
			\begin{equation}
				F^{\ast}dy = \cos(\varphi)\sin(\theta)d\varphi + \sin(\varphi)\cos(\theta)d\theta \qquad \text{and} \qquad F^{\ast}dz = -\sin(\varphi)d\varphi. 
			\end{equation}
		Therefore, 
			\begin{equation}
				F^{\ast}\omega = \left[\sin^{2}(\varphi)\cos^{2}(\theta) + \sin(\varphi)\cos(\theta) + \sin(\varphi)\sin(\theta)\right]\sin^{2}(\varphi)\cos(\theta)d\varphi \wedge d\theta. 
			\end{equation}
		From this, we conclude that 
			\begin{equation}
				\int_{S^{2}}i^{\ast}\omega = \int_{0}^{2\pi}\int_{0}^{\pi}\left[\sin^{2}(\varphi)\cos^{2}(\theta) + \sin(\varphi)\cos(\theta) + \sin(\varphi)\sin(\theta)\right]\sin^{2}(\varphi)\cos(\theta)d\varphi d\theta = \frac{4\pi}{3}. 
			\end{equation}
		(\textbf{Method 2}) Using Stokes Theorem, 
			\begin{equation}
				\int_{S^{2}}i^{\ast}\omega = \int_{B^{3}}d\omega, 
			\end{equation}
		where $B^{3}$ indicates the $3$-ball (recall that $S^{1} = \partial B^{3}$). We compute, $d\omega = (2x  + 1)dx \wedge dy \wedge dz$ so that 
			\begin{equation}
				\int_{S^{2}}i^{\ast}\omega = \int_{B^{3}}d\omega = \int_{B^{3}}2xdxdydz + \int_{B^{3}}dxdydz = \int_{B^{3}}dxdydz = \frac{4\pi}{3}, 
			\end{equation}
		where the first integral after the second inequality is zero due to symmetry. 
		\item Suppose there exists a closed form $\alpha$ on $\mathbb{R}^{3}$ such that $i^{\ast}\alpha = i^{\ast}\omega$. Since $\alpha$ is closed, $d\alpha = 0$. Hence, 
			\begin{equation}
				\int_{S^{2}}i^{\ast}\alpha = \int_{B^{3}}d(i^{\ast}\alpha) = \int_{B^{3}}i^{\ast}d\alpha = 0 \neq \frac{4\pi}{3} =\int_{S^{2}}i^{\ast}\omega, 
			\end{equation}
		which is a contradiction. Hence, such a closed form cannot exist. 
	\end{enumerate}
\end{solutions}
\begin{prb}{2023-J-I-5 (Algebra)}
	Consider the following irreducible polynomial over $\mathbb{Q}$: $p(x) = x^{4} - 3x^{2} - 1$. \vspace{-0.25cm}
		\begin{enumerate}[itemsep =-2pt,label = (\alph{*})]
			\item Describe the splitting field of $p(x)$. 
			\item Consider the Galois group of $p(x)$. Compute its order and determine if it is abelian. 
		\end{enumerate}
\end{prb}
\begin{solutions}
	$ $\newline \vspace{-1cm}
	\begin{enumerate}[itemsep =-2pt,label = (\alph{*})]
		\item Let $p(x) = x^{4} - 3x^{2} - 1$. By the rational root test, $p(x)$ has no roots over $\mathbb{Q}$. Moreover, it is straightforward to check that $p(x)$ is not the product of irreducible quadratics with rational coefficients. Hence, $p(x)$ is irreducible over $\mathbb{Q}$. We start by finding the roots of $p(x)$; let $u = x^{2}$. Then 
			\begin{equation}
				u^{2} - 3u - 1 = 0 \implies u = \frac{3 \pm \sqrt{13}}{2} \implies x = \pm \sqrt{\frac{3 \pm \sqrt{13}}{2}}. 
			\end{equation}
		Let 
			\begin{equation}
				\alpha = \sqrt{\frac{3 + \sqrt{13}}{2}}, \qquad \beta = \sqrt{\frac{3 - \sqrt{13}}{2}}. 
			\end{equation}
		Observe that $\alpha^{2}\beta^{2} = -1$ so that $\beta = \pm \frac{i}{\alpha}$. Therefore, the splitting field of $p(x)$ is 
			\begin{equation}
				\mathbb{Q}(\alpha, i). 
			\end{equation}
		Observe that the minimal polynomial of $i$ is $x^{2} + 1$, which is irreducible over $\mathbb{Q}(\alpha)$ so that $[\mathbb{Q}(\alpha, i):\mathbb{Q}(\alpha)] = 2$. On the other hand, the minimal polynomial of $\alpha$ is a degree 4 polynomial so that $[\mathbb{Q}(\alpha): \mathbb{Q}] = 4$. Hence, by the tower law, $[\mathbb{Q}(\alpha, i): \mathbb{Q}] = 8$. 
		\item By the last work in (a), the order of the Galois group of $p(x)$ is $8$. Now, we will determine the Galois group of $p(x)$. Recall that elements of $\operatorname{Gal}(\mathbb{Q}(\alpha, i)/\mathbb{Q})$ are automorphisms $\varphi$ of the field $\mathbb{Q}(\alpha, i)$ with the constraints that: (1) $\varphi$ fixes $\mathbb{Q}$, (2) $\varphi(\alpha)$ must be another root of the minimal polynomial of $\alpha$ over $\mathbb{Q}$, and (3) $\varphi(i)$ must be another root of $x^{2} + 1$. We will explicitly work through each of the elements. 
			\begin{enumerate}[itemsep =-2pt,label =(\roman{*})]
				\item $\sigma: i \mapsto -i, \alpha \mapsto \alpha$. This permutation has order 2 since $\sigma^{2}(\alpha)  = \sigma(\alpha) = \alpha$ and $\sigma^{2}(i) = \sigma(-i) = i$. 
				\item $\tau: i \mapsto i, \alpha \mapsto -\alpha$. Once again, this permutation has order 2. 
				\item $\rho: i \mapsto -i, \alpha \mapsto \beta = \frac{i}{\alpha}$. To compute the order of this permutation, observe that 
					\begin{equation}
						\rho^{2}(\alpha) = \rho(i\alpha^{-1}) = (-i) \cdot \frac{1}{i/\alpha} = -\alpha \implies \rho^{4}(\alpha) = \rho^{2}(-\alpha) = \alpha. 
					\end{equation}
				Likewise, $\rho^{4}(i) = \rho^{2}(i) = i$. Hence, $\rho$ has order $4$. 
			\end{enumerate}
		Now, consider the three elements given above. We compute 
			\begin{equation}
				\sigma\rho\sigma(i) = \sigma\rho(-i) = \sigma(i) = -i = \rho^{-1}(i). 
			\end{equation}
		Likewise, 
			\begin{equation}
				\sigma\rho\sigma(\alpha) = \sigma\rho(\alpha) = \sigma(i)\sigma(\alpha)^{-1} = -\frac{i}{\alpha} = \rho^{-1}(\alpha). 
			\end{equation}
		Therefore, $\sigma \rho \sigma = \rho^{-1}$. Hence, 
			\begin{equation}
				\operatorname{Gal}(\mathbb{Q}(\alpha,i)/\mathbb{Q}) = \{1, \sigma, \rho, \rho^{2}, \rho^{3}, \sigma\rho, \sigma\rho^{2},  \sigma\rho^{3}\} \cong D_{8}. 
			\end{equation}
		Since the dihedral group is not abelian, we conclude that the Galois group for $p(x)$ is non-abelian. 
	\end{enumerate}
\end{solutions}
\begin{prb}{2023-J-I-5 (Algebra I)}
	Determine the Galois group of $x^{3} - x^{2} - 4$. 
\end{prb}
\begin{solutions}
	Let $p(x) = x^{3} - x^{2} - 4$. We start by finding the roots of $p(x)$ over some algebraic closure of $\mathbb{Q}$. Observe that $2$ is a solution. Using polnomial long division, 
		\begin{equation}
			p(x) = (x - 2)(x^{2} + x + 2) \implies x = 2, \frac{-1 \pm \sqrt{-7}}{2}. 
		\end{equation}
	Hence, the splitting field of $p(x)$ is $\mathbb{Q}(\sqrt{7}i)$. Now since $\operatorname{Gal}(\mathbb{Q}(\sqrt{7}i)/\mathbb{Q})$ is the group of automorphisms of the splitting field $\mathbb{Q}(\sqrt{7}i)$ that preserve $\mathbb{Q}$. Since there are exactly two automorphisms (namely, the identity permutation fixing $\sqrt{7}i$ and the conjugation map $\sqrt{7}i \mapsto -\sqrt{7}i$), we conclude that $\operatorname{Gal}(\mathbb{Q}(\sqrt{7}i)/\mathbb{Q}) \cong \mathbb{Z}_{2}$. 
\end{solutions}
\begin{prb}{2023-J-I-5 (Algebra II)}
	Determine the Galois group of $x^{3} - 2x + 4$. 
\end{prb}
\begin{solutions}
	Let $p(x) = x^{3} - 2x + 4$. We start by finding the roots of $p(x)$ over some algebraic closure of $\mathbb{Q}$. Clearly $-2$ is a root of $p(x)$. Using polynomial long division, 
		\begin{equation}
			p(x) = (x+ 2)(x^{2} - 2x + 2) \implies x = -2, 1 \pm \sqrt{-1}. 
		\end{equation}
	Hence, the splitting field of $p(x)$ is $\mathbb{Q}(i)$, which is a quadratic extension of $\mathbb{Q}$. Now since $\operatorname{Gal}(\mathbb{Q}(i)/\mathbb{Q})$ is the group of automorphisms of the splitting field $\mathbb{Q}(i)$ that preserve $\mathbb{Q}$, and there exactly two such automorphisms (namely, the identity fixing $i$, and the conjugation map $i \mapsto -i$), we conclude that $\operatorname{Gal}(\mathbb{Q}(i)/\mathbb{Q}) \cong \mathbb{Z}/2\mathbb{Z}$. 
\end{solutions}
\begin{prb}{2023-J-I-5 (Algebra III)}
	Determine the Galois group of $x^{3} - x + 1$. 
\end{prb}
\begin{solutions}
	Let $p(x) = x^{3} - x + 1$. We start by finding the roots of $x$ over some algebraic closure of $\mathbb{Q}$. Since the only possible rational roots of $p$ over $\mathbb{Q}$ are $\pm 1$ by the Rational Root Test, and neither of these are actually roots of $p$, we conclude that $p$ is irreducible. Hence, a root of $f(x)$ generates an extension of degree 3 so that the degree of the splitting field of $F$ is divisible by 3. Since the Galois group is a subgroup of $S_{3}$, either $\operatorname{Gal}(\mathbb{Q}(\alpha_{1}, \alpha_{2}, \alpha_{3})/\mathbb{Q}) \cong A_{3}$ or $\operatorname{Gal}(\mathbb{Q}(\alpha_{1}, \alpha_{2}, \alpha_{3})/\mathbb{Q}) \cong S_{3}$. Since $p$ is already a depressed cubic, we calculate its discriminant to be $-4(-1)^{3} - 27(1)^{2} = -23$. Since the discriminant is not a perfect square in $\mathbb{Q}$, we conclude that $\operatorname{Gal}(\mathbb{Q}(\alpha_{1}, \alpha_{2},\alpha_{3})/\mathbb{Q}) \cong S_{3}$. 
\end{solutions}
\begin{prb}{2023-J-I-4 (Geometry/Topology)}
	Let $\omega$ be a smooth nowhere vanishing 1-form on a smooth 3-manifold $M^{3}$. \vspace{-0.25cm}
		\begin{enumerate}[itemsep =-2pt,label = (\alph{*})]
			\item Show that the distribution defined at each point $p \in M$ by 
				\begin{equation}
					\ker{\omega_{p}} = \brac*{v \in T_{p}M^{3}: \omega_{p}(v) = 0}
				\end{equation}
			is integrable if and only if $\omega \wedge d\omega = 0$. 
			\item Give an example of a codimension one distribution on $\mathbb{R}^{3}$ that is not integrable. 
		\end{enumerate}
\end{prb}
\begin{solutions}
	$ $\newline \vspace{-1cm}
	\begin{enumerate}[itemsep=-2pt,label = (\alph{*})]
		\item We recall that a distribution $D$ is Frobenius integrable if and only if given two smooth sections $X, Y$ of $D$, the Lie Bracket $[X, Y]$ is also a smooth section of $D$. Therefore, let $X, Y$ be smooth sections of $D$, which means that $\omega(X), \omega(Y) = 0$ by definition of $D$. We recall that 
			\begin{equation}
				d\omega(X, Y) = X(\omega(Y)) - Y(\omega(X)) - \omega([X, Y]) = -\omega([X, Y]), 
			\end{equation}
		where the first two terms are identically zero by our hypothesis. Therefore, $D$ is integrable if and only if $[X, Y]$ is a smooth section of $D$ if and only if $\omega([X, Y]) = 0$. Now, if $D$ were integrable, then for any field $Z$ on $\mathbb{R}^{3}$, 
			\begin{equation}
				\omega \wedge d\omega(X, Y,Z) = \omega(Z)d\omega(X, Y) = 0, 
			\end{equation}
		where the other terms vanish by assumption on $X$ and $Y$. Hence, since $X,Y \in \ker{\omega}$ were arbitrary and $Z$ was arbitrary, $\omega \wedge d\omega = 0$. On the other hand, if $\omega \wedge d\omega = 0$, let $p\in M$, $Z_{p} \in T_{p}M$ with $\omega_{p}(Z_{p}) \neq 0$ and $X_{p}, Y_{p} \in \ker{\omega_{p}}$. Then 
			\begin{equation}
				0 = (\omega \wedge d\omega)_{p}(X_{p}, Y_{p}, Z_{p}) = \omega_{p}(Z_{p})d\omega_{p}(X_{p}, Y_{p}). 
			\end{equation}
		Hece, $d\omega_{p}(X_{p}, Y_{p}) = 0$. This means that for smooth sections $X, Y$ of $\ker{\omega}$, $d\omega(X, Y) = 0$, and so $D$ is integrable. 
		\item Consider the smooth nowhere vanishing 1-form $\omega = ydx + dy + dz$ on $\mathbb{R}^{3}$, and let $D$ be the distribution on $\mathbb{R}^{3}$ defined at each point $p \in M$ by $D_{p} = \ker{\omega_{p}}$. By the rank-nullity theorem, $\operatorname{dim}{D} = \operatorname{dim}{T_{p}\mathbb{R}^{3}} - \operatorname{rank}{\omega} = 3 - 1 = 2$. Hence, $\operatorname{codim}{D} = 3 - 2 = 1$. Next, we observe that $d\omega = dy \wedge dx$, which is identically not zero. Then $\omega \wedge d\omega = dz \wedge dy \wedge dx$, which is also not identically zero. Hence, by the conclusion in (a), $D$ is not integrable. 
	\end{enumerate}
\end{solutions}
\begin{prb}{2023-J-I-1 (Real Analysis)}
	Give (with proof) an example of a Banach space $X$ and a norm closed set $E \subset X$ that is not weakly closed. 
\end{prb}
\begin{solutions}
	Let $X = \mathscr{H}$, where $\mathscr{H}$ is any infinite-dimensional Hilbert space, and let $E = \{v_{n}\}$ be an infinite orthonormal set in $\mathscr{H}$. Since $\norm{x_{i} - x_{j}} \geq 1$ for any two distinct vectors $x_{i}, x_{j}$, it follows that if $\norm{x_{n} - x} \to 0$, then the sequence must eventually be constant, which means $x \in E$. However, $E$ is not weakly closed: fix an arbitrary element $y \in \mathscr{H}$. Then by Bessel's inequality, 
		\begin{equation}
			\sum_{1}^{\infty}\abs{\braket{y, v_{n}}}^{2} \leq \norm{y}^{2}, 
		\end{equation}
	so that the sequence of inner products $a_{n} = \braket{y, v_{n}}$ is square summable, so $a_{n} \to 0$. This means that $v_{n} \to 0$ weakly. However, $0 \notin E$ since $\norm{0} = 0\neq 1$. Hence, $E$ is not weakly closed. 
\end{solutions}
\begin{prb}{2023-J-I-2 (Complex Analysis)}
	Set $\mathbb{D} = \brac*{z \in \mathbb{C}: \norm{z} < 1}$, $f: \mathbb{D} \to \brac*{w \in \mathbb{C}: e^{-\pi/2} < \abs{\omega} < e^{\pi/2}}$ be a holomorphic map satisfying $f(0) = 1$. Show that $\abs{f'(0)} \leq 2$. 
\end{prb}
\begin{solutions}
	\textcolor{red}{[!! Complete Later]}
\end{solutions}
\begin{prb}{2023-J-II-1 (Real Analysis)}
	Suppose that $f: \mathbb{R} \to \mathbb{R}$ is continuous. Show that $f^{-1}(y) = \{x \in \mathbb{R}: f(x) = y\}$ has Lebesgue measure zero for Lebesgue almost all $y$. 
\end{prb}
\begin{solutions}
	Let $f: \mathbb{R} \to \mathbb{R}$ be continuous, and define the following: 
		\begin{align}
			\begin{split}
				\Gamma &\coloneqq \brac*{(x, y) \in \mathbb{R}^{2}: f(x) = y}. \\
				\Gamma_{x} &\coloneqq \{y \in \mathbb{R}: f(x) = y\} = \{f(x)\}. \\
				\Gamma^{y} &\coloneqq \{x \in \mathbb{R}: f(x) = y\} = f^{-1}(y). \\
				g&: \mathbb{R} \to [0, \infty) \quad \text{ defined by } \quad g(y) = m(f^{-1}(y)).
			\end{split}
		\end{align}
	Since $\Gamma_{x}$ is a singleton for every $x \in \mathbb{R}$, we conclude that $m(\Gamma_{x}) = 0$ for all $x$. By Fubini-Tonelli, we recall that 
		\begin{equation}
			m^{2}(\Gamma) = \int_{\mathbb{R}^{2}}\chi_{\Gamma}dxdy = \int_{\mathbb{R}}\left(\int_{\mathbb{R}}\chi_{\Gamma}dx\right)dy = \int_{\mathbb{R}}\left(\int_{\mathbb{R}}\chi_{\Gamma}dy\right)dx. 
		\end{equation}
	However, 
		\begin{equation}
			\int_{\mathbb{R}}\chi_{\Gamma}dy = m(\Gamma_{x}) = 0. 
		\end{equation}
	This means that 
		\begin{equation}
			\int_{\mathbb{R}}\left(\int_{\mathbb{R}}\chi_{\Gamma}dx\right)dy = 0 \implies \int_{\mathbb{R}}\chi_{\Gamma}dx = 0 \text{ a.e.}
		\end{equation}
	Now, we observe that 
		\begin{equation}
			\int_{\mathbb{R}}\chi_{\Gamma}dx = g(y) = m(f^{-1}(y)).
		\end{equation}
	Hence, this means that $m(f^{-1}(y)) = 0$ a.e., which implies that $f^{-1}(y)$ has Lebesgue measure zero for Lebesgue almost all $y$. 
\end{solutions}
\begin{prb}{2023-J-II-2 (Real Analysis)}
	Suppose that $f$ is continuous on $[0,1]$ and $\int_{0}^{1}f(x)x^{k}dx = 0$ for $k =0, \ldots, n$. Prove that either $f$ is either identically zero, or $f$ must change sign at least $n + 1$ times. We say that $f$ changes sign $n$ times if there are points $x_{1} < \dotsm < x_{n + 1}$ so that $f(x_{j})f(x_{j + 1}) < 0$ for $j = 1, \ldots, n$. 
\end{prb}
\begin{solutions}
	Suppose that $f$ is continuous on $[0,1]$ and $\int_{0}^{1}f(x)x^{k}dx = 0$ for $k =0, \ldots, n$. If $f$ is identically zero, then the claim is trivial and we are done. So assume that $f \nequiv 0$. Suppose $f$ changes sign only $n$ times. By the definition provided above, we can find $n$ points $x_{1}, \ldots, x_{n}$ such that $f(x_{j}) = 0$ for each $j = 1, \ldots, n$. Consider the function $g(x) = \pm f(x) \cdot \prod_{1}^{n}(x - x_{j})$, which must be continuous on $[0,1]$ since it is the product of finitely many continuous functions. For some choice of $\pm$, $g(x) \geq 0$ for all $x \in [0,1]$. Since $\prod_{1}^{n}(x - x_{j})$ is a polynomial of degree $n$, we conclude by the hypothesis that 
		\begin{equation}
			\int_{0}^{1}g(x)dx = 0. 
		\end{equation}
	Since $g(x) \geq 0$, this forces $g(x) =0$ and so $f$ has to be identically zero, which contradicts our hypothesis. Hence, by contradiction, $f$ has to change at least $n + 1$ times. 
\end{solutions}
\subsection{August 2023}
\begin{prb}{2023-A-I-1 (Algebra)}
	Let $V$ be an $n$-dimensional vector space over a field $F$. An element $A \in \operatorname{End}{V}$ is called \textit{nilpotent} if $A^{k} = 0$ for some $k > 1$. Prove that $A$ is nilpotent if and only if $$\operatorname{Tr}(\Lambda^{i}A) = 0, \quad i = 1, \ldots, n$$ where $\Lambda^{i}A$ denotes the induced action of $A$ on the wedge product $\Lambda^{i}V$ for each $i$. 
\end{prb}

\begin{prb}{2023-A-I-5 (Geometry/Topology)}
	Let $T$ be the 2-torus $S^{1} \times S^{1}$ with an open 2-disk removed: 
		\begin{center}
			\includegraphics[width = 0.2\linewidth]{"../Graphics/2023A15.png"}
		\end{center} 
	Show that there is no continuous retraction $r$ onto its boundary (i.e., no continuous map $r: T \to \partial T$ satisfying $r^{2} = r$). 
\end{prb}
\begin{solutions}
	Let $T$ be the 2-torus $S^{1} \times S^{1}$ with an open 2-disk removed, $\iota: \partial T \to T$ the inclusion map, and assume to the contrary that $r: T \to \partial T$ is a continuous retraction. Then the composition $r_{\ast} \circ \iota_{\ast}: \pi_{1}(\partial T) \to \pi_{1}(\partial T)$ must be the identity map. Since $\partial T \cong S^{1}$, $\pi_{1}(\partial T) = \mathbb{Z}$, and is generated by the element $1$. By a direct computation, since $\partial_{1}(T) = \mathbb{Z} \ast \mathbb{Z}$ is the free product on two generators $a$ and $b$  $\iota_{\ast}$ maps $1$ to the element $aba^{-1}b^{-1}$. But then $r_{\ast}$ maps the commutator into the abelian group $\mathbb{Z}$, where the commutator must be zero. This contradicts our claim that $r_{\ast} \circ \iota_{\ast}$ is the identity map. Hence, by contradiction, there cannot be any continuous retraction of $T$ onto its boundary. 
\end{solutions}
\begin{prb}{2023-A-I-6 (Complex Analysis)}
	Let $\mathbb{D} \subset \mathbb{C}$ be the open unit disk. Is there a holomorphic function $f$ with $f(\mathbb{D}) = \mathbb{D}$, $f(0) = f'(0) = 2/3$? If so, give a formula. If not, prove that it cannot exist. 
\end{prb}
\begin{solutions}
	The problem lends itself nicely to an application of the Schwarz-Pick Theorem: 
		\begin{quote}
			(\textbf{Schwarz-Pick Theorem}) Let $f: \mathbb{D} \to \mathbb{D}$ be holomorphic. If $\abs{f(z)} \leq 1$ for all $z$, and $f(a) = b$ for some $a, b \in \mathbb{D}$, then 
				\begin{equation*}
					\abs{f'(a)} \leq \frac{1 - |b|^{2}}{1 - |a|^{2}}.
				\end{equation*}
		\end{quote}
	Now assume that a holomorphic function $f$ with $f(\mathbb{D}) = \mathbb{D}$, $f(0) = f'(0) = 2/3$ exists. Then by the Schwarz-Pick Lemma, 
		\begin{equation}
			\frac{2}{3} \leq \frac{1 - \nicefrac{4}{9}}{1 - 0} = \frac{5}{9} < \frac{2}{3}, 
		\end{equation}
	which is a contradiction. Hence, no such holomorphic function can exist. 
\end{solutions}
\begin{prb}{2023-A-I-2 (Geometry/Topology)}
	Let $f: T^{2} \to S^{2}$ be a smooth map from the 2-torus to the 2-sphere. Can $f$ be an immersion? If the answer is yes, give an explicit example. If the answer is no, then give a proof. 
\end{prb}
\begin{solutions}
	We begin by stating and proving a technical lemma, which we will then use in our argument. 
		\begin{quote}
			(\textbf{Comps Lemma}) Let $M$ and $N$ be smooth connected $n$-manifolds, and $f: M \to N$ a (smooth) immersion. If $M$ is compact and nonempty, then $N$ is compact and $f$ is a (smooth) covering map. 
			
			\begin{proof}
				Let $M$ and $N$ be smooth connected $n$-manifolds, and $f:M \to N$ an immersion. Since $\dim{M} = \dim{N} = n$, and $f$ is an immersion, the map $df_{p}: T_{p}M \to T_{f(p)}N$ has constant rank $n$ at every $p \in M$. Hence, by the Inverse Function Theorem, $f$ is a local diffeomorphism. Since local diffeomorphisms are open maps, $f(M)$ is open in $N$. On the other hand, since continuous images of compact sets are compact, $f(M)$ is compact in $N$; since $N$ is Hausdorff, $f(M)$ is closed in $N$. Since $N$ is connected, it follows that $f(M) = N$. Therefore, $N$ is compact. All that remains is to show is that $f$ is a covering map. 
				
				Let $q \in N$; by continuity of $f$, $f^{-1}(q)$ is a closed subset of $M$. For each $x \in f^{-1}(q)$, there exists an open neighborhood $U_{x}$ of $x$ such that $f|_{U_{x}}$ is a diffeomorphism. Since $M$ is Hausdorff, we can shrink these neighborhoods so that they are pairwise disjoint. This means that each $x \in f^{-1}(q)$ is isolated, implying that $f^{-1}(q)$ is discrete. Since $M$ is compact, it follows that $f^{-1}(q)$ is finite; let $f^{-1}(q) = \{x_{1}, \ldots, x_{s}\}$. As stated above, for each $j= 1, \ldots, s$, we may find an open neighborhood $U'_{j}$ so that $f|U'_{j}$ is a diffeomorphism. Moreover, we can shrink these neighborhoods to obtain a pairwise disjoint collection $\{\ti{U}_{j}\}_{1}^{s}$ of neighborhoods. Set $V = \bigcap_{1}^{s}f(\ti{U}_{j})$. Then taking $U_{j} = f^{-1}(V) \cap \ti{U}_{j}$, $V$ is an evenly covered neighborhood of $p$, so that $f$ is a covering map. 
			\end{proof}
		\end{quote}
	Now assume to the contrary that there exists an immersion $f: T^{2} \to S^{2}$. By the Comps Lemma, $f$ must be a covering map. Hence, the induced homomorphism of groups $f_{\ast}: \pi_{1}(T^{2}) \to \pi_{1}(S^{2})$ must be injective. Since $S^{2}$ is simply connected, $\pi_{1}(S^{2}) \cong \{0\}$. However, $\pi_{1}(T^{2})$ is not a trivial group (in fact, $\pi_{1}(T^{2}) \cong \mathbb{Z} \times \mathbb{Z}$). This means that $f_{\ast}$ cannot be injective. Therefore, by contradiction, $f$ cannot be an immersion. Hence, there exist no immersions from $T^{2}$ to $S^{2}$. 
\end{solutions}
\begin{prb}{2023-A-II-1 (Algebra)}
	A field extension $K/L$ is called algebraic, if every element in $K$ satisfies a polynomial equation with coefficients in $L$. Let $F, K, L$ be fields such that $F \supset K \supset L$, and $F/K$ and $K/L$ are algebraic extensions. Prove that $F/L$ is also an algebraic extension. 
\end{prb}
\begin{solutions}
	Since subfields of subfields is a subfield, $L$ is a subfield of $F$. Hence, it suffices to show that every element in $F$ satisfies a polynomial equation with coefficients in $L$. Let $a \in F$, and let 
	\begin{equation}
		k(x) = k_{n}x^{n} + k_{n - 1}x^{n-1} + \dotsm + k_{0} \in K[x]
	\end{equation}
	such that $k(a) = 0$; this follows since $F/K$ is an algebraic extension. Each $k_{j} \in K$, and hence is algebraic over $L$. Therefore, $L' = L(k_{0}, \ldots, k_{n})$ is a finite extension of $L$. Since $k(a) = 0$ and $k(x)$ now has its coefficients in $L'$, it follows that $a$ is algebraic over $L'$ so that $L'(a)$ is a finite extension of $L$. Then since 
	\begin{equation}
		[L(a):L] = [L(a):L'][L':L], 
	\end{equation}
	it follows that $L(a)$ is a finite extension of $L$. Therefore, $a$ is algebraic over $L$. Since $a$ was arbitrary, $F/L$ is an algebraic extension. 
\end{solutions}
\begin{prb}{2023-A-I-2 (Geometry/Topology)}
	Let $f: T^{2} \to S^{2}$ be a smooth map from the 2-torus to the 2-sphere. Can $f$ be an immersion? If the answer is yes, given an explicit example. If the answer is no, then give a proof. 
\end{prb}
\begin{solutions}
	There cannot be an immersion $f: T^{2} \to S^{2}$. To prove our answer, we will state and proof a technical lemma. 
	\begin{quote}
		(\textbf{Comps Lemma}) Let $M, N$ be smooth, connected, $n$-manifolds and $f: M \to N$ a (smooth) immersion. If $M$ is compact and nonempty, then $f$ is a (smooth) covering map. 
		\begin{proof}
			Let $M, N$ be smooth connected $n$-manifolds, $M$ compact, and $f: M \to N$ an immersion. Since $\dim{N} = n$ everywhere and $f$ is an immersion, $df_{p}: T_{p}M \to T_{f(p)}N$ has constant rank $n$ everywhere. Hence, by the Inverse Function Theorem, $f$ is a local diffeomorphism. Let $q \in N$ so that $f^{-1}(q) \subset M$ is closed. For each $x \in f^{-1}(q)$, there exists a neighborhood $U_{x}$ such that $f|_{U_{x}}: U_{x} \to V_{x} \subset N$ is a diffeomorphism. Since $M$ is Hausdorff, we can shrink these neighborhoods so that they are pairwise disjoint. Since every $x \in f^{-1}(q)$ is now isolated, it follows that $f^{-1}(q)$ is discrete. Since $M$ is compact, we conclude that $f^{-1}(q)$ must be finite; let $f^{-1}(q) = \{x_{1}, \ldots, x_{s}\}$. As stated above, for each $j = 1, \ldots, s$, we can find a neighborhood $U_{j}$ of $x_{j}$ so that $f|_{U_{j}}: U_{j} \to V_{j} \subset N$ is a diffeomorphism. Again, since $M$ is Hausdorff, we can shrink these neigborhoods so that $U_{i} \cap U_{j} = \varnothing$ for all $i \neq j$; $f$ restricted to each of these shrunken neighborhoods remains a diffeomorphism. Now set $V = \bigcap_{1}^{s}f(U_{j})$, and define $\ti{U}_{j} \subset M$ by $\ti{U}_{j} = f^{-1}(V) \cap U_{j}$ for each $j = 1, \ldots, s$. Hence, $V$ is an evenly covered neighborhood of $q \in N$, which means $f$ is a covering map. That $f$ is surjective comes from recognizing that $f(M) = N$ due to connectedness of $N$. 
		\end{proof}
	\end{quote}
	
	Now, assume $f: T^{2} \to S^{2}$ is an immersion. Since $T^{2}, S^{2}$ are smooth, connected 2-manifolds, and $T^{2}$ is compact and nonempty, by the Comps Lemma, $f$ is a covering map. Hence, the induced homomorphism $f_{\ast}: \pi_{1}(T^{2})\to \pi_{1}(S^{2})$ is injective. Since $S^{2}$ is simply connected, $\pi_{1}(S^{2}) \cong \{0\}$. On the other hand, $\pi_{1}(T^{2}) \cong \mathbb{Z} \times \mathbb{Z}$. Since the order of $\pi_{1}(T^{2})$ is more than one, $f_{\ast}$ cannot be injective. Hence, $f$ cannot be an immersion. 
\end{solutions}

\begin{prb}{2023-A-II-5 (Geometry/Topology)}
Let $(t, x, y, z)$ be the standard coordinate system on $\mathbb{R}^{4}$, and let $\phi$ be the non-zero smooth 1-form on $\mathbb{R}^{4}$ defined by 
	\begin{equation*}
		\phi = dt + ydx + zdy. 
	\end{equation*}
Let $D$ be the 3-plane field on $\mathbb{R}^{4}$ that consists of tangent vectors $V$ such that $\phi(V) = 0$. Is $D$ Frobenius integrable? Support your answer with a proof. 
\end{prb}
\begin{solutions}
	Let $D$ be the 3-plane field on $\mathbb{R}^{4}$  defined as follows: for each $p \in \mathbb{R}^{4}$, 
		\begin{equation}
			D_{p} = \brac*{v \in T_{p}\mathbb{R}^{4}: \phi(v) = 0} \eqqcolon \ker{\phi_{p}}. 
		\end{equation}
	Hence, by the Frobenius Theorem, $D$ is Frobenius integrable if and only if $\phi \wedge d\phi = 0$. We compute: 
		\begin{equation}
			d\phi = d(dt + y\;dx + z\;dy) = d^{2}t + dy \wedge dx + dz \wedge dy = dy \wedge dx + dz \wedge dy. 
		\end{equation}
	Therefore, 
		\begin{equation}
			\phi \wedge d\phi = dt \wedge dy \wedge dx + dt \wedge dz \wedge dy + y dx \wedge dz \wedge dy. 
		\end{equation}
	Since $\phi \wedge d\phi$ is nowhere vanishing on $\mathbb{R}^{4}$, $D$ is not Frobenius integrable. 
\end{solutions}
\begin{prb}{2023-A-I-1 (Algebra)}
	Let $V$ be a $n$-dimensional vector space over a field $F$. An element $A \in \operatorname{End}{V}$ is called \textit{nilpotent}, if $A^{k} = 0$ for some $k > 1$. Prove that $A$ is nilpotent if and only if 
		\begin{equation}
			\operatorname{Tr}(\Lambda^{i}A) = 0, \quad i = 1, \ldots, n,
		\end{equation} 
	where $\Lambda^{i}A$ denotes the induced action of $A$ on the wedge product $\Lambda^{i}V$ for each $i$. 
\end{prb}
\begin{solutions}
	Let $V$ be a $n$-dimensional vector space over a field $F$, and let $A \in \operatorname{End}{V}$. Recall that $\Lambda^{i}A$, the induced action of $A$ on the wedge product $\Lambda^{i}V$, is defined to be 
		\begin{equation}
			(\Lambda^{i}A)(v_{1} \wedge \dotsm \wedge v_{i}) = Av_{1} \wedge \dotsm \wedge Av_{i}, \qquad v_{j} \in V \text{ for all } j =1, \ldots, i. 
		\end{equation}
	Over an algebraic closure of $F$, $A$ has eigenvalues $\lambda_{1}, \ldots, \lambda_{n}$. Suppose $A$ is diagonalizable, with the set of eigenvectors given by $\{v_{1}, \ldots, v_{n}\}$. Then for each $i = 1, \ldots, n$, since the collection 
		\begin{equation*}
			\brac*{v_{j_{1}} \wedge \dotsm\wedge v_{j_{i}}: 1 \leq j_{1} < \dotsm < j_{i} \leq n}
		\end{equation*}
	is a basis of $\Lambda^{i}V$, and for each $i$-tuple, $\Lambda^{i}A(v_{j_{1}} \wedge \dotsm \wedge v_{j_{i}}) = Av_{j_{1}} \wedge \dotsm \wedge Av_{j_{i}} = (\lambda_{j_{1}}\dotsm\lambda_{j_{i}})(v_{j_{1}} \wedge \dotsm \wedge v_{j_{i}})$, it follows that the eigenvalues of $\Lambda^{i}A$ are the set of all products of the form $\lambda_{j_{1}} \dotsm \lambda_{j_{i}}$ for $1 \leq j_{1} < \dotsm < j_{i} \leq n$, counting for multiplicity. Hence,
		\begin{equation}
			\operatorname{Tr}(\Lambda^{i}A) = \sum_{1 \leq j_{1} < \dotsm < j_{i} \leq n}\lambda_{j_{1}}\dotsm\lambda_{j_{i}}. 
		\end{equation}
	If $A$ is not diagonalizable, since the eigenvalues of $\Lambda^{i}A$ depend only on the eigenvalues of $A$, we may assume $A$ is in Jordan normal form. Indeed, if $A = PJP^{-1}$, then 
		\begin{equation}
			\Lambda^{i}(A) = \Lambda^{i}(PJP^{-1}) = \Lambda^{i}(P)\Lambda^{i}(J)\Lambda^{i}(P^{-1}), 
		\end{equation}
	so $\Lambda^{i}A$ and $\Lambda^{i}J$ are similar and therefore have the same eigenvalues. Thus it suffices to compute the eigenvalues of $\Lambda^{i}J$, which are exactly the products $\lambda_{j_{1}}\dotsm\lambda_{j_{i}}$ of the eigenvalues of $A$. 
	
	If $A$ is nilpotent so that $A^{k} = 0$ for some $k > 1$, then since $0 = A^{k}v = \lambda^{k}v$ for all eigenvectors $v$ of $A$, it follows that every eigenvalue of $A$ is zero. Therefore, the above expression implies that $\operatorname{Tr}(\Lambda^{i}A) = 0$ for all $i = 1, \ldots, n$. On the other hand, expanding the characteristic polynomial for $A$ is given by:
		\begin{equation}
			p_{A}(t) = \det(tI - A) = t^{n} - \operatorname{Tr}(\Lambda^{1}A)t^{n - 1} + \dotsm + (-1)^{n}\operatorname{Tr}(\Lambda^{n}A). 
		\end{equation}
	If $\operatorname{Tr}(\Lambda^{i}A)=0$ for all $i = 1, \ldots, n$, then we conclude that the characteristic polynomial of $A$ is precisely $t^{n}$. Therefore, $A$'s eigenvalues are all zero. Hence, the minimal polynomial of $A$ is of the form $t^{k}$ for some $k \leq n$. This implies that $A^{k} = 0$, and so $A$ is nilpotent. 
\end{solutions}
\begin{prb}{2023-A-II-6 (Complex Analysis)}
	Find the number of solutions (counting multiplicity) to $z^{8} - 5z^{6} + 2z^{3} - z - 1 = 0$ that lie inside the unit disk. 
\end{prb}
\begin{solutions}
	Recall Rouch{\'e}'s Formula, which states that 
		\begin{quote}
			For any two complex-valued functions $f$ and $g$ holomorphic inside some region $K$ with closed and simple contour $\partial K$, if $|g(z)| < |f(z)|$ on $\partial K$, then $f$ and $f +g$ have the same number of zeros inside $K$, where each zero is counted as many times as its multiplicity. 
		\end{quote} 
	Pick $f(z) = 5z^{6}$ and set $h(z) = z^{8} + 2z^{3} - z - 1$ so that $p(z) = z^{8} - 5z^{6} + 2z^{3} - z - 1 = h(z) - f(z)$. On the unit disk $\partial S^{1}$, we observe that 
		\begin{align}
			\begin{split}
				\abs{f(z)} &= \abs{5z^{6}} = 5 \\
				&= 1 + 2 + 1 + 1 \\
				&= |z^{8}| + 2|z^{3}| + |z| + |1| \\
				&\geq |h(z)|. 
			\end{split}
		\end{align}
	Hence, $p(z) = h(z) - f(z)$ has the same number of zeros, counting multiplicity, as $f(z)$. Since $f(z)$ has six zeros in the unit disk, we conclude that $p(z)$ must also have six zeros inside the unit disk. 
\end{solutions}


\begin{prb}{2023-A-II-4 (Real Analysis)}
	Let $\mu$ be a (positive) Borel probability measure on $[0,1]$, such that for all $t \in [0,1]$ we have $\mu(\{t\}) = 0$. Let $\mu_{n}$ be a (positive) Borel probability measure on $[0,1]$ for $n = 1, 2, \ldots$. Suppose $\mu_{n} \to \mu$ in the weak$^{\ast}$ topology. Let $F(t)=  \mu([0,t])$ and $F_{n}(t)  = \mu_{n}([0,t])$. Prove that $F_{n} \to F$ uniformly. 
\end{prb}

\subsection{January 2022}
\begin{prb}{2022-J-I-1 (Complex Analysis)}
	Suppose $f: \mathbb{D} \to \mathbb{C}$ is a holomorphic function on the unit disk so that 
		\begin{equation*}
			f\left(\frac{1}{n}\right) = \frac{5}{n^{2}}, \qquad \forall n \in \mathbb{Z}, n \geq 2. 
		\end{equation*}
	Find $f''(0)$. 
\end{prb}
\begin{solutions}
	This problem requires the notion of an analytic continuation: 
		\begin{quote}
			(\textbf{Analytic Continuation}) Let $f$ be an analytic function defined on a non-empty open subset $U$ of the complex plane $\mathbb{C}$. If $V$ is a larger open subset of $\mathbb{C}$ containing $U$, and $F$ is an analytic function defined on $V$ such that $F(z) = f(z)$ for all $z \in U$, then $F$ is called an \textit{analytic continuation of $f$}. I.e., $F|_{U} = f$. 
		\end{quote}
	For our problem, since the function $F(z) = 5z^{2}$ agrees with $f(z)$ at a sequence of points converging in $\mathbb{D}$ and $\mathbb{D}$ is connected, $f(z) = 5z^{2}$. Hence, $f''(0) = 10$. 
\end{solutions} 
\begin{prb}{2022-J-I-3 (Algebra)}
	Show that a group of order 1,000,000 contains a proper normal subgroup (i.e., is not simple). 
\end{prb}
\begin{solutions}
	Let $G$ be a group of order 1,000,000 $=10^{6} = 2^{6} \cdot 5^{6}$. By Sylow's Theorem, 
	\begin{align}
		\begin{split}
			n_{2} &\in \{1, 5, 5^{2}, 5^{3}, 5^{4}, 5^{5}, 5^{6}\} \cap \{2k + 1: k \in \mathbb{N}\}, \\
			n_{5} &\in \{1, 2, 4, 8, 16, 32, 64\} \cap \{5k + 1: k \in \mathbb{N}\} = \{1, 16\}. 
		\end{split}
	\end{align}
	If $n_{5} = 1$, then we are done since the unique Sylow 5-subgroup must necessarily be normal. So suppose $n_{5} = 16$, and let $G$ act on $\operatorname{Syl}_{5}(G)$ by conjugation. This induces a homomorphism $\varphi: G \to \varphi(G) \leq S_{16}$. However, $|G| = 10^{6} \nmid 16! = |S_{16}|$. This means that $\varphi$ cannot be an injective homomorphism since if otherwise, $|\varphi(G)| = |G|$, but this is impossible since $|G| \nmid |S_{16}|$. Therefore, $\ker{\varphi}$ is a nontrivial normal subgroup of $G$. If $\ker{\varphi} = G$, then every Sylow 5-subgroup of $G$ is normal and is, in fact, unique, which contradicts our hypothesis that $n_{5} = 16$. Hence, $\ker{\varphi}$ is a proper nontrivial normal subgroup of $G$, which means that $G$ cannot be simple. 
\end{solutions}
\begin{prb}{2022-J-II-3 (Real Analysis)}
	Prove or give a counterexample: if $E \subset \mathbb{R}$ is a Lebesgue measurable subset of positive Lebesgue measure, then some countable union of translates of $E$ covers $\mathbb{R}$.
\end{prb}
\begin{solutions}
	The statement is not necessarily true. Consider the \textit{fat Cantor} set, which is a subset of $\mathbb{R}$ that is nowhere dense and has positive Lebesgue measure $1/2$; call this set $E$. Assume to the contrary that $\mathbb{R}$ is the countable union of translates of $E$. Since $E$ is nowhere dense, each translate of $E$ must also be nowhere dense. Then $\mathbb{R}$ is the countable union of nowhere dense sets, which violates the Baire Category Theorem. Hence, the statement is not necessarily true. 
\end{solutions}
\begin{prb}{2022-J-II-4 (Complex Analysis)}
	Let $U \subset \mathbb{C}$ be an open subset. Suppose $f_{i}: U \to \mathbb{C}$ is a sequence of holomorphic functions converging uniformly on compact subsets to a function $f: U \to \mathbb{C}$. Show that $f$ is also holomorphic. Justify each step clearly. 
\end{prb}
\begin{solutions}
	It is sufficient to show that $f$ is holomorphic on any disk $\mathbb{D}\subset \mathbb{C}$ with $\overline{\mathbb{D}} \subset \mathbb{C}$. Since $f_{i} \to f$ uniformly on $\mathbb{D}$, $f$ is continuous on $\mathbb{D}$. We will use Morera's Theorem, which states the following: 
		\begin{quote}
			\textbf{(Morera's Theorem)} A continuous, complex-valued function $f$ defined on an open set $D$ in the complex plane that satisfies 
				\begin{equation*}
					\oint_{\gamma}f(z)\;dz = 0
				\end{equation*}
			for every closed piecewise $C^{1}$ curve $\gamma$ in $D$ must be holomorphic on $D$. 
		\end{quote}
	Let $\gamma$ be a closed piecewise $C^{1}$ curve in $\mathbb{D}$. Then 
		\begin{equation}
			\oint_{\gamma}f(z)\;dz = \oint_{\gamma}\lim_{i \to \infty}f_{i}(z)\;dz = \lim_{i \to \infty}\oint_{\gamma}f_{i}(z)\;dz  = \lim_{i \to \infty}(0) = 0. 
		\end{equation}
	The first equality follows from the hypothesis that $f_{i} \to f$; the second equality follows from the fact that this convergence is uniform; and the third equality follows from the fact that each $f_{i}$ is holomorphic. Therefore, $f$ is also holomorphic. 	
\end{solutions}

\subsection{August 2022}
\begin{prb}{2022-A-I-3 (Real Analysis)}
	Show that $L^{4}(X, \mu) \not\subset L^{5}(X, \mu)$ if and only if there are subsets of arbitrarily small measure. 
\end{prb}
\begin{solutions}
	($\Rightarrow$) Suppose $X$ contains subsets of arbitrarily small positive measure. This means that for each positive integer $n$, there exists a measurable subset $\ti{E}_{n}$ of $X$ such that $\mu(\ti{E}_{n}) = 2^{-n}$. From this collection, we can obtain a sequence of disjoint sets $\{E_{n}\}$ such that for each $n$, $0 < \mu(E_{n}) < 2^{-n}$. Define the function 
		\begin{equation}
			f = \sum_{1}^{\infty}\mu(E_{n})^{-1/5}\chi_{E_{n}}. 
		\end{equation}
	We claim that $f \in L^{4}(X, \mu)$: 
		\begin{align}
			\begin{split}
				\norm{f}_{4}^{4} &= \int_{X}\abs{f}^{4} = \int_{X}\sum_{1}^{\infty}\abs{\mu(E_{n})}^{-4/5}\chi_{E_{n}} \\
				&= \sum_{1}^{\infty}\mu(E_{n})^{-4/5}\int_{X}\chi_{E_{n}} \\
				&= \sum_{1}^{\infty}\mu(E_{n})^{1/5} = \sum_{1}^{\infty}2^{-n/5} = \frac{2^{4/5}}{2^{4/5} - 2} < \infty. 
			\end{split}
		\end{align}
	On the other hand, we claim that $f \notin L^{5}(X, \mu)$: 
		\begin{align}
			\begin{split}
				\norm{f}_{5}^{5} &= \int_{X}\abs{f}^{5} = \int_{X}\sum_{1}^{\infty}\mu(E_{n})^{-1}\chi_{E_{n}} \\
				&= \sum_{1}^{\infty}\mu(E_{n})^{-1}\int_{X}\chi_{E_{n}} \\
				&= \sum_{1}^{\infty}1 = \infty. 
			\end{split}
		\end{align}
	Hence, $L^{4}(X, \mu) \not\subset L^{5}(X, \mu)$. 
	
	($\Leftarrow$) Now suppose that $L^{4}(X, \mu) \not\subset L^{5}(X, \mu)$. Let $f \in L^{4}\setminus L^{5}$, and for each positive integer $n$, define $E_{n} = \brac*{x \in X: \abs{f(x)} > n}$. We observe that 
		\begin{equation}
			\infty > \norm{f}_{4}^{4} = \int_{X}\abs{f}^{4} \geq \int_{E_{n}}\abs{f}^{4} \geq \int_{E_{n}}n^{4} = n^{4}\mu(E_{n}) \implies \mu(E_{n}) \leq \frac{\norm{f}_{4}^{4}}{n^{4}}
		\end{equation}
	I.e., we observe that $\mu(E_{n}) \to 0$ as $n \to \infty$. It suffices to show that each $E_{j}$ has positive measure (combined with the previous observation, this yields the result that $X$ has arbitrarily small \textit{positively} measured subsets). Assume to the contrary. Suppose there exists some $j_{0}$ for which $\mu(E_{j_{0}}) = 0$. Let $F_{j_{0}} = E_{j_{0}}^{c}$. Then we observe that since $\abs{f} \leq n$ on $F_{j_{0}}$, 
	\begin{align}
		\begin{split}
			\norm{f}_{q}^{q} &= \int \abs{f}^{q} = \int \abs{f}^{q}\chi_{F_{j_{0}}} \\
			&= \int \abs{f}^{q- p}\abs{f}^{p}\chi_{F_{j_{0}}} \leq n^{q - p}\int \abs{f}^{p}\chi_{F_{j_0}} \\
			&\eqqcolon n^{q - p}\norm{f}_{p}^{p} < \infty. 
		\end{split}
	\end{align}
	This shows that $f \in L^{p}$, which contradicts our hypothesis that $f \in L^{p}\setminus L^{q}$. Therefore, by contradiction, $\mu(E_{n}) \to 0$ and $\mu(E_{n}) > 0$ for all $n$. 
\end{solutions}


\begin{prb}{2022-A-I-4 (Geometry/Topology)}
	Find the fundamental group of the space of unordered pairs of distinct points of $z^{n}$. 
\end{prb}




\begin{prb}{2022-A-I-5 (Complex Analysis)}
	If $\Omega \subset \mathbb{C}$ is simply connected and $f: \Omega \to \mathbb{C}\setminus \{0\}$ is a holomorphic function, is there a holomorphic function $g: \Omega \to \mathbb{C}$ such that $f = \exp(g)$ on $\Omega$? Prove or give a counterexample. 
\end{prb}
\begin{solutions}
	Yes, there exists a function $g: \Omega \to \mathbb{C}$ such that $f = \exp(g)$. Consider the function $f'/f: \Omega \to \mathbb{C}$, which is holomorphic since $f$ is zero-free. Fix $z_{0} \in \Omega$, and define 
		\begin{equation}
			h(z) = \int_{\gamma|_{Z}}\frac{f'(\zeta)}{f(\zeta)}d\zeta, \qquad z \in \Omega, 
		\end{equation}
	where $\gamma|_{z}$ is a smooth curve connecting $z_{0}$ to $z$; since $\Omega$ is simply connected, Cauchy's Theorem tells us that $h(z)$ is independent of the choice of curve $\gamma|_{z}$. Then, we note that 
		\begin{equation}
			h'(z) = \frac{f'(z)}{f(z)}. 
		\end{equation}
	So consider the function $f\exp(-h)$. Then 
		\begin{equation}
			\frac{d}{dz}f\exp(-h) = \exp(-h)(f' - h'f) = 0, 
		\end{equation}
	so that $f\exp(-h)$ is a non-zero constant $c$. So we must have $f = c\exp(h)$. Let $g = h + C$ where $e^{C} = c$. Then 
		\begin{equation}
			\exp(g) = \exp(h + C) = e^{C}\exp(h) = c\exp(h) = f. 
		\end{equation}
	The proof concludes. The main ideas are simple connectedness and Cauchy's Theorem, and the fact that for every holomorphic function $f$, there exists a primitive holomorphic function $g$ such that $g'(z) = f(z)$. But the existence of a primitive function holds only on simply connected regions. 
\end{solutions}

\begin{prb}{2022-A-II-I (Real Analysis)}
	Suppose $E \subset \mathbb{R}^{2}$ has positive Lebesgue area. Show that $E$ contains 3 points that form the vertices of an equilateral triangle. 
\end{prb}
\begin{solutions}
	Let $E \subset \mathbb{R}^{2}$ be a set of positive Lebesgue measure (we will denote by $m^{2}$ the Lebesgue measure on $\mathbb{R}^{2}$). Let $\{v_{1}, v_{2}\}$ be a collection of unit vectors in $\mathbb{R}^{2}$ so that the angle between $v_{1}$ and $v_{2}$ is $120^{\circ}$, and let $\beta < 1/3$. By inner regularity of the Lebesgue measure, there exists a compact set $K_{1} \subset E$ so that $m^{2}(K_{1}) > 0$. Then by \textit{outer} regularity of the Lebesgue measure, there exists an open set $U$ containing $K_{1}$ such that $m^{2}(U) \leq (1 + \beta)m^{2}(K_{1})$. 
	
	Since $K_{1}$ is compact, $d_{1} = d(K_{1}, U^{c})$ is positive; so let $R = d_{1}$, pick an arbitrary $r \in (0, R)$, and consider the set $K_{1} + rv_{1}$. $K_{1} + rv_{1}$ has to be contained within $U$ since otherwise, 
		\begin{equation}
			d(K_{1}, U^{c}) < |rv_{1}| = r < d_{1}, \text{ which is a contradiction.}
		\end{equation}
	Hence, $K_{1} \cup (K_{1} + rv_{1}) \subset U$, which means 
		\begin{equation}
			m^{2}(U) \geq m^{2}(K_{1}\cup (K_{1} + rv_{1})) = m^{2}(K_{1}) + m^{2}(K_{1} + rv_{1}) - m^{2}(K_{1} \cap (K_{1} + rv_{1})) = 2m^{2}(K_{1}) - m^{2}(K_{1} \cap (K_{1} + rv_{1})), 
		\end{equation}
	where the last equality follows from translation invariance of the Lebesgue measure. Hence, $m^{2}(K_{1} \cap (K_{1} + rv_{1})) = 2m^{2}(K_{1}) - m^{2}(U) \geq (1 -\beta)m^{2}(K_{1}) > 0$. Therefore, $K_{2} \coloneqq K_{1}\cap (K_{1} + rv_{1})$ is nonempty. Now define $K_{3} = K_{2} \cap (K_{2} + rv_{2})$. Using the same reasoning as above, we observe that $K_{3}\neq \varnothing$ and $K_{3} \subset K_{2}$. Hence, we obtain a nested sequence of sets $\varnothing \neq K_{3} \subset K_{2} \subset K_{1} \subset E$. Let $M \in K_{3}$. Since $K_{3} = K_{2} \cap (K_{2} + rv_{1})$, $N = q - rv_{2} \in K_{2}$. Likewise, $O = q - rv_{2} - rv_{1} \in K_{1}$. These three points form the vertices of a triangle. Then since 
		\begin{align}
			\begin{split}
				\norm{M - N} = r, \qquad \norm{N - O} = r, \qquad \norm{M - O} = \norm{r(v_{2} + v_{1})} = r \norm{v_{2} + v_{1}} = r. 
			\end{split}
		\end{align}
\end{solutions}
\begin{prb}{2022-A-II-4 (Algebra)}
	Let $G$ be a finite group in which $(ab)^{p} = a^{p}b^{p}$ for every $a, b \in G$, where $p$ is a prime dividing $|G|$. Prove that the Sylow $p$-subgroup of $G$ is normal in $G$ (and is in fact unique). 
\end{prb}
\begin{solutions}
	Let $G$ be a finite group in which $(ab)^{p} = a^{p}b^{p}$ for  every $a, b \in G$, where $p$ is a prime dividing $|G|$. Consider the map $\varphi: G \to G$ defined by $\varphi(g)= g^{p}$. This map is a homomorphism since for any $g, h \in G$, 
		\begin{equation}
			\varphi(gh) = (gh)^{p} = g^{p}h^{p} = \varphi(g)\varphi(h), 
		\end{equation}
	where the second equality follows from the hypothesis. Consider the map 
		\begin{equation}
			\varphi^{k} \coloneqq \underbrace{\varphi \circ \dotsm \circ \varphi}_{k \text{ copies}}, 
		\end{equation}
	which must also be a homomorphism since the composition of homomorphisms is a homomorphism. The kernel of $\varphi^{k}$ consists exactly of those elements $x \in G$ whose order is a power of $p$ (i.e., $x^{p^{r}} = 1$ for some positive integer $r$) since 
		\begin{equation}
			\varphi^{k}(x) = x^{p^{k}} = x^{p^{r + (k - r)}} = \left(x^{p^{r}}\right)^{p^{k - r}} = 1^{p^{k - r}} = 1. 
		\end{equation}
	Hence, since every element with order equal to some order of $p$ belongs in a Sylow $p$-subgroup of $G$, 
		\begin{equation}
			\ker{\varphi^{k}} = \bigcup_{P\in \operatorname{Syl}_{p}(G)}P. 
		\end{equation}
	Moreover, $\ker{\varphi^{k}}$ must be a $p$-subgroup of $G$ since if not, there exists a prime $p' \neq p$ dividing $|\ker{\varphi^{k}}|$, which means by Cauchy's Theorem that $\ker{\varphi^{k}}$ contains an element of order $p'$ (which is impossible). Hence, since $\ker{\varphi^{k}}$ is a $p$-subgroup of $G$ containing a Sylow $p$-subgroup, by maximality of Sylow $p$-subgroups, $\ker{\varphi^{k}}$ must be a Sylow $p$-subgroup of $G$. Hence, $G$ has a unique Sylow $p$-subgroup. And since kernels of homomorphisms are normal subgroups, this Sylow $p$-subgroup must be normal. 
\end{solutions}
\begin{prb}{2022-A-II-5}
	If $f: [-1,2] \to \mathbb{R}$ is continuous and increasing, show that the set of $x \in [0,1]$ where 
		\begin{equation*}
			\int_{0}^{1}\frac{f(x + t) - f(x - t)}{t}\;dt = \infty, 
		\end{equation*}
	has Lebesgue zero measure. 
\end{prb}
\begin{solutions}
	Since $f$ is continuous and increasing on $[0,1]$, we must have 
		\begin{equation}
			f(x) = a + \int_{0}^{x}d\mu(t),
		\end{equation}
	where $\mu$ is a non-atomic measure. Therefore, $f'(x)$ exists and is finite for Lebesgue almost every $x$. Therefore, for almost every $x$, there exists a finite $M < \infty$ so that 
		\begin{equation}
			\abs{f(x + t) - f(x)} \leq M\abs{t}, 
		\end{equation}
	which means 
		\begin{equation}
			\abs{\int_{0}^{1}\frac{f(x + t) - f(x - t)}{t}\;dt} \leq \int_{0}^{1}\frac{\abs{f(x + t) - f(x - t)}}{t}dt \leq \int_{0}^{1}M\;dt < \infty
		\end{equation}
	for Lebesgue almost $x$. 
\end{solutions}
\begin{prb}{2021-A-II-6 (Real Analysis)}
	Suppose $\mu$ is a finite positive measure of compact support and 
		\begin{equation*}
			\int_{\mathbb{R}}x^{n}d\mu(x) =0
		\end{equation*}
	for every $n \in \{0,1,2,\ldots\}$. Show that $\mu$ is the zero measure. 
\end{prb}
\begin{solutions}
	Suppose $\mu$ is a finite positive measure of compact support and that 
		\begin{equation}
			\int_{\mathbb{R}}x^{n}\;d\mu(x) =0, \qquad \forall n \in \{0,1,2,\ldots\}. 
		\end{equation}
	 Let $E$ be the support of $\mu$. Our strategy is to show that $\int_{\mathbb{R}}f\;d\mu(x) =0$ for all continuous functions $f$ on $\mathbb{R}$, for which we shall use the Stone-Weierstra{\ss} theorem: 
	 	\begin{quote}
	 		\textbf{(Stone-Weierstra{\ss} Theorem)} Let $X$ be a compact Hausdorff space, $C(X,\mathbb{R})$ the space of all continuous functions on $X$. Suppose $\mathscr{B}$ is a subalgebra of $C(X,\mathbb{R})$ that separates points. If there exists $x_{0} \in X$ such that $f(x_{0}) = 0$ for all $f \in \mathscr{B}$, then $\mathscr{B}$ is dense in $\{f \in C(X,  \mathbb{R}): f(x_{0}) = 0\}$. Otherwise, $\mathscr{B}$ is dense in $C(X, \mathbb{R})$. 
	 	\end{quote} 
	 Since $\mathbb{R}$ is Hausdorff, $E$ is Hausdorff; by hypothesis, $E$ is compact. Hence, the Stone-Weierstrass theorem is applicable for our case. Let $\mathscr{B}$ be the subalgebra of $C(E,\mathbb{R})$ that separates points. Then by the theorem, $\mathscr{B}$ is dense in $C(E, \mathbb{R})$, which means that any continuous function can be uniformly approximated by a sequence of polynomials in $\mathscr{B}$. Let $f \in C(E, \mathbb{R})$ be arbitrary, and consider a sequence $\{p_{j}(x)\}_{1}^{\infty}$ that uniformly converges to $f$. Then 
	 	\begin{equation}
	 		\int_{\mathbb{R}}f(x)\;d\mu(x) = \int_{E}f(x)\;d\mu(x) = \int_{E}\lim_{j \to \infty}p_{j}(x)\;d\mu(x) = \lim_{j \to \infty}\int_{E}p_{j}(x)\;d\mu(x) = \lim_{j \to \infty}\left[\sum_{k = 1}^{\deg{p_{j}}}a_{k}\int_{E}x^{k}\;d\mu(x)\right] = 0, 
	 	\end{equation}
	 where the last equality follows from the hypothesis. Hence, since for \textit{every} continuous function $f$, the integral over $\mathbb{R}$ with respect to $\mu$ is zero, we conclude that $\mu$ has to be the zero measure. 
\end{solutions}

\subsection{January 2021}
\begin{prb}{2021-J-1-3 (Real Analysis)}
	If $E \subset \mathbb{R}$ is Lebesgue measurable and $f: \mathbb{R} \to \mathbb{R}$ is Lipschitz, then show that $f(E)$ is also Lebesgue measurable. 
\end{prb}

\subsection{August 2021}
\begin{prb}{2021-A-I-6 (Geometry/Topology)}
	What connected spaces can be finitely-sheeted covering spaces of a sphere with three handles? 
\end{prb}
\begin{solutions}
	We claim that the finitely-sheeted covering spaces of a sphere with three handles are exactly the closed orientable connected surfaces of genus of the form $2k + 1$ for some positive integer $k$. Let $M$ be a $k$-sheeted covering space of a sphere with three handles. If $M$ were nonorientable, then since covering maps are local diffeomorphisms and local diffeomorphisms preserve orientability, the sphere with three handles must also be nonorientable, which is a contradiction. Hence, $M$ has to be orientable. Next, since $M$ is a $k$-sheeted covering space of the sphere with three handles, which has Euler characteristic $2 - 2(3) = -4$, we must have 
		\begin{equation}
			2 - 2g_{M} = \chi(M) = -4k \implies g_{M} -1 = 2k \implies g_{M} = 2k + 1. 
		\end{equation}
\end{solutions}
\begin{prb}{2021-A-II-1 (Geometry/Topology)}
	Let $M$ be a compact manifold (without boundary) and $\pi: M \to S^{1}$ a submersion onto the circle. Show that the de Rham group $\h{1}(M) \neq 0$. 
\end{prb}
\begin{solutions}
	Let $M$ be a compact manifold (without boundary) and $\pi: M \to S^{1}$ a submersion onto the circle. Assume to the contrary that $\h{1}(M) \neq 0$ which means that every closed form on $M$ is an exact form. Since $\h{1}(S^{1}) \cong \mathbb{R}$, let $[\omega]$ be a generator of this cohomology group, where $\omega$ is a nowhere vanishing closed 1-form on $S^{1}$. Since $\pi$ is a submersion, the 1-form $\pi^{\ast}\omega$ must also be a nowhere vanishing closed form on $M$. By our hypothesis on the de Rham cohomology group in degree one of $M$, $\pi^{\ast}\omega$ is exact, which means there exists a smooth function $f$ such that $\pi^{\ast}\omega = df$. Since $M$ is compact and $f$ is smooth, $f$ must attain either a maximum or minimum value at some $p_{0} \in M$. This means that $df_{p_{0}} = 0$. But this contradicts our claim that $\pi^{\ast}\omega$ is nowhere vanishing. Hence, by contradiction, $\h{1}(M) \neq 0$. 
\end{solutions}


\subsection{January 2020}
\begin{prb}{2020-J-I-1 (Algebra)}
	Let $G$ be a finite non-abelian group, and let $Z(G)$ denote its center. Prove that $\abs{Z(G)} \leq \frac{1}{4}\abs{G}$, and then give an example where equality holds. 
\end{prb}
\begin{solutions}
	Let $G$ be a finite non-abelian group, and let $Z(G)$ denote its center. Assume to the contrary that $|Z(G)| > \frac{1}{4}|G| \implies |G|/|Z(G)| < 4$. Since $|Z(G)| \mid |G|$, $|G|/|Z(G)|$ is a positive integer. Therefore, one of the three must necessarily be true: (1) $|G|/|Z(G)| = 1$, (2) $|G|/|Z(G)| = 2$, (3) $|G|/|Z(G)| = 3$. If (1) were true, then since $|Z(G)| = |G|$, $G$ has to be abelian, which contradicts our hypothesis. If (2) were true, then $G/Z \cong \mathbb{Z}/2\mathbb{Z}$ which is cyclic. Hence, $G$ would have to be abelian, which is a contradiction. Finally, if (3) were true, then $G/Z \cong \mathbb{Z}/3\mathbb{Z}$ which is cyclic. Hence, $G$ would have to be abelian, which ia a contradiction. Hence, $|Z(G)| \not> \frac{1}{4}|G|$, which means $|Z(G)| \leq \frac{1}{4}|G|$. 
\end{solutions}
\begin{prb}{2020-J-I-4 (Geometry/Topology)}
	Let $\theta$ be a closed smooth 1-form on a compact $C^{\infty}$ manifold $M$ with empty boundary, and let $v$ be a smooth vector field on $M$. Prove that the Lie derivative $\mathscr{L}_{v}\theta$ vanishes at some point of $M$. 
\end{prb}
\begin{solutions}
	Let $\theta$ be a closed smooth 1-form on a compact $C^{\infty}$ manifold $M$ with empty boundary, and let $v$ be a smooth vector field on $M$. By Cartan's Formula for the Lie derivative, 
		\begin{equation}
			\mathscr{L}_{v}\theta = i_{v}(d\theta) + d(i_{v}\theta), 
		\end{equation}
	where $i_{v}(\cdot)$ denotes the interior product. Since $\theta$ is a closed 1-form, $d\theta = 0$. So $\mathscr{L}_{v}\theta = d(i_{v}\theta)$. Since $\theta$ is a 1-form, $i_{v}\theta$ is a 0-form on $M$, i.e., a smooth function on $M$. Since $M$ is compact, $i_{v}\theta$ must attain a extrema at some point in $M$, which means that its differential $d(i_{v}\theta)$ must vanish where it achieves its maximum or minimum. This then implies that $\mathscr{L}_{v}\theta$ vanishes at this point. 
\end{solutions}


\subsection{August 2020}
\begin{prb}{2020-A-II-1 (Complex Analysis)}
	How many roots (counted with multiplicity) does the function $$g(z) = 6z^{3} + e^{z} + 1$$ have in the unit disk $|z| < 1$?
\end{prb}
\begin{solutions}
	Let $g(z) = 6z^{3} + e^{z} + 1$, which is holomorphic. Let $f(z) = 6z^{3}$ and $h(z) = e^{z} + 1$. Then on the unit circle $|z| = 1$, 
		\begin{align}
			\begin{split}
				|h(z)| &\leq |e^{z}| + 1 \leq e^{|z|} + 1 \\
				&\leq e + 1 \\
				&< 6 = 6|z|^{3} = |f(z)|. 
			\end{split}
		\end{align} 
	Hence, by Rouch{\'e}'s Formula, $g(z)$ has the same number of zeros as $f(z)$. Counting multiplicity, $f(z)$ has three solutions in the unit disk, which means that $g(z)$ also has three solutions in the unit disk. 
\end{solutions}
\begin{prb}{2020-A-II-4 (Geometry/Topology)}
	Let $M$ and $N$ be compact connected orientable smooth manifolds and let $f: M \to N$ be a smooth mapping. Recall the degree of $f$ is the integral 
		\begin{equation*}
			\deg(f) = \int_{M}f^{\ast}\omega
		\end{equation*}
	over $M$ of the pullback $f^{\ast}\omega$ of any top-degree smooth form $\omega$ on $N$ whose integral over $N$ is one. Recall the degree is an integer, denote it by $\deg(f)$. Now consider the map 
		\begin{equation*}
			f_{\#}: \pi_{1}(M) \to \pi_{1}(N)
		\end{equation*}
	on fundamental groups induced by $f$. Suppose that the image of $f_{\#}$ has finite index, $\operatorname{ind}(f)$. Prove that $\operatorname{ind}(f)$ divides $\deg(f)$. 
\end{prb}
\begin{solutions}
	Let $M, N$ be compact connected orientable smooth manifolds and let $f: M \to N$ be a smooth mapping. Suppose that $H \coloneqq f_{\#}(\pi_{1}(M))$ is a subgroup of $\pi_{1}(N)$ of finite index $k$. This means there exists a $k$-sheeted covering $p: \ti{N} \to N$ so that $p_{\#}(\pi_{1}(\ti{N})) = H$. By the lifting criterion for coverings, $f$ lifts to a smooth map 
		\begin{equation}
			\ti{f}: M \to \ti{N}
		\end{equation}
	such that $f = p \circ \ti{f}$. Let $\omega$ be a top-degree smooth form on $N$ whose integral over $N$ is one. Since $p: \ti{N} \to N$ is a $k$-sheeted covering of orientable manifolds, we must have $\deg(p) = k$. Therefore, 
		\begin{equation}
			\deg(f) = \deg(p \circ \ti{f}) = \deg(p)\deg(\ti{f}) = \operatorname{ind}(f) \cdot \deg(\ti{f}). 
		\end{equation}
	Since $\deg(\ti{f})$ is an integer, we conclude that $\operatorname{ind}(f) \mid \deg(f)$. 
\end{solutions}
\begin{prb}{2020-J-I-2 (Geometry/Topology)}
	Let $M$ and $N$ be smooth compact connected oriented $n$-manifolds without boundary. Suppose that $\pi_{1}(M)$ is finite, but that $\pi_{1}(M)$ is infinite. Prove that every smooth map $\Psi: M \to N$ has degree zero. 
\end{prb}


\newpage 
\subsection{January 2019}
\begin{prb}{2019-J-I-1 (Algebra)}
	Let $A$ and $B$ be $n\times n$ invertible matrices over complex numbers, satisfying 
		\begin{equation*}
			AB = \lambda BA \text{ for some $\lambda \in \mathbb{C}$}.
		\end{equation*}
	Prove that $A^{n}$ and $B$ commute. 
\end{prb}
\begin{solutions}
	Let $A$ and $B$ be $n\times n$ invertible matrices over complex numbers so that $AB = \lambda BA$ for some $\lambda \in \mathbb{C}$. Since $A$ is invertible, left-multiplying both sides by $A^{-1}$ yields, 
		\begin{equation}
			B = \lambda A^{-1}BA. 
		\end{equation}
	So taking the determinant, we obtain: 
		\begin{equation}
			\det{B} = \lambda^{n}\det{A^{-1}}\det{B}\det{A} = \lambda^{n}\det{A}^{-1}\det{B}\det{A} = \lambda^{n}\det{B}. 
		\end{equation}
	Since $B$ is invertible, $\det{B} \neq 0$, which means that $\lambda^{n} = 1$ (i.e., $\lambda$ is an $n\textsuperscript{th}$ root of unity). Now, we claim that for any $m \in \mathbb{N}$, $A^{m}B = \lambda^{m}BA^{m}$. By hypothesis, this claim is true for the base case $m = 1$. Suppose the claim is true for some $m\geq 1$. Then 
		\begin{equation}
			A^{m + 1}B = A(A^{m}B) = \lambda^{m}(ABA^{m}) = \lambda^{m}(\lambda BA)A^{m} = \lambda^{m + 1}BA^{m + 1}. 
		\end{equation}
	Therefore, the claim is true by induction. This implies that 
		\begin{equation}
			A^{n}B = \lambda^{n}BA^{n} = BA^{n}, 
		\end{equation}
	so that $A^{n}$ and $B$ commute. 
\end{solutions}


\begin{prb}{2019-J-II-5}
	Let $G$ be a finite group, and let $H$ be a non-normal subgroup of $G$ of index $n$. Show that if $|H|$ is divisible by a prime $p \geq n$, then $G$ is not simple. 
\end{prb}
\begin{solutions}
	Let $G$ be a finite group, $H$ a non-normal subgroup of $G$ of index $n$ such that $|H|$ is divisible by a prime $p \geq n$. Let $G$ act on the set of left cosets of $H$; this induces a group homomorphism $\varphi: G \to S_{n}$. Consider the kernel of this group action, $K = \ker{\varphi}$. \textit{If} $K = G$, then for every $g \in G$, $gHg^{-1} = H$, which implies that $H$ is a normal subgroup of $G$ -- a contradiction. Hence, $\ker{\varphi}$ is a proper normal subgroup of $G$. Likewise, $\ker{\varphi} \neq H$ since this equality also forces $H$ to be normal. All that remains is to show that $\ker{\varphi}$ is not trivial. Since $p\mid |H|$, let $P$ be a Sylow $p$-subgroup of $H$. \textcolor{red}{[!! Complete Later !!]}
\end{solutions}



\subsection{August 2018}
\begin{prb}{2018-A-II-3 (Analysis)}
	Suppose $E, F$ are two measurable subsets of the real numbers that both have positive measure. Prove that $E + F = \{x + y: x \in E, y \in F\}$ contains an interval. 
\end{prb}


\begin{prb}{2018-A-I-3 (Complex Analysis)}
	Show that if $c > 1$, then the function 
		\begin{equation*}
			f(z) = ze^{c - z} -1
		\end{equation*}
	has precisely one root in $\Delta = \{\abs{z} < 1\}$, and this root is real and positive. 
\end{prb}
\begin{solutions}
	Let $c > 1$, and $f(z) = ze^{c - z} - 1$. Let $g(z) = ze^{c - z}$ and $h(z) =1$. On $\partial \Delta$, 
		\begin{align}
			\begin{split}
				|h(z)| &= 1 < e^{c - 1} = |g(z)|,
			\end{split}
		\end{align}
	so that by Rouch{\'e}'s Theorem, $f(z)$ has the same number of roots as $g(z)$. Since the exponential has no roots but the function $z$ has one root inside the unit disk, we conclude that $f(z)$ has precisely one root in $\Delta$. Now consider the real-valued function $\ti{f}(x) = xe^{c - z} - 1$ obtained by restricting $f(z)$ to the real line. We observe that 
		\begin{equation}
			f(0) = -1 < 0 \qquad \text{ and } \qquad f(1) = e^{c - 1} - 1 > 0. 
		\end{equation}
	Since $f(z)$ is continuous, it follows from the intermediate value theorem that $f(x)$ must have a root inside the interval $(0,1)$. Such a root must necessarily be real and positive. Hence, the proof concludes. 
\end{solutions}





\subsection{January 2017}
\begin{prb}{2017-J-I-1 (Geometry/Topology)}
	Let $\Sigma_{1}$ be a torus and let $\Sigma_{2}$ be a genus-2 surface. Show that there is no submersion from $\Sigma_{2}$ to $\Sigma_{1}$. 		
\end{prb}
\begin{solutions}
	Let $\Sigma_{1}$ be a torus and $\Sigma_{2}$ be a genus-2 surface. We begin with a second modification to the Comps Lemma. Assume to the contrary that $F$ is a submersion from $\Sigma_{2}$ to $\Sigma_{1}$. By the second modification to the Comps Lemma, $F: \Sigma_{2} \to \Sigma_{1}$ must be a $k$-sheeted covering map for some finite $k > 0$. This implies that $\chi(\Sigma_{2}) = k \cdot \chi(\Sigma_{1})$, where $\chi(\cdot) = 2 - 2g$ denotes the Euler characteristic of a closed surface of genus $g$. But this is impossible since $\chi(\Sigma_{2}) = -2 < 0 = k \cdot 0 = k \cdot \chi(\Sigma_{1})$. Hence, by contradiction, there cannot be any submersions from $\Sigma_{2}$ to $\Sigma_{1}$. 
\end{solutions}
\begin{prb}{2017-J-I-6 (Geometry/Topology)}
	Let $M$ be a smooth 4-manifold, let $\phi$ be a 3-form on $M$, and let $U\subset M$ be the open set of points where $\varphi \neq 0$. Show that $\varphi$ is closed if and only if, near any $p\in U$, one can find a smooth coordinate system $(x^{1}, x^{2}, x^{3}, x^{4})$ in which $$\varphi = dx^{1} \wedge dx^{2} \wedge dx^{3}.$$
\end{prb}
\begin{solutions}
	Assume the hypotheses of the problem. Recall that $\varphi$ is closed if and only if $d\varphi$ is identically zero. Let $p \in U$ and suppose that we can find a smooth coordinate system $(x^{1}, x^{2}, x^{3}, x^{4})$ in some neighborhood of $p$ in $U$ so that  $\varphi = dx^{1} \wedge dx^{2} \wedge dx^{3}$. Then $d\varphi_{p} = d^{2}x^{1} \wedge dx^{2} \wedge dx^{3} + \dotsm + dx^{1} \wedge dx^{2} \wedge d^{2}x^{3} = 0$. Since this is true for all $p \in U$, we conclude that $d\varphi$ is identically zero on $M$, and hence $\varphi$ is closed. 
	
	Now assume that $\varphi$ is closed, which means that $\varphi \wedge d\varphi$ is identically zero. At each point $p \in U$, define $$D_{p} = \ker{\varphi_{p}},$$ which is Frobenius integrable by our previous observation. In particular, $D_{p}$ is a 1-dimensional distribution. Since $L$ is integrable, we can find smooth coordinates $(x^{1}, \ldots, x^{4})$ near $p$ such that $D_{p} = \operatorname{span}\left\{\partial_{x^{4}}\right\}$. Since $\varphi$ annihilates $\partial_{x^{4}}$, it must be a linear combination of $dx^{1}, dx^{2}$, and $dx^{3}$. Suppose $\varphi = f dx^{1} \wedge dx^{2} \wedge dx^{3}$. Then 
		\begin{equation}
			0 = d\varphi = f_{x^{1}}dx^{1} \wedge dx^{1} \wedge \dotsm \wedge dx^{3} + f_{x^{2}}dx^{2} \wedge dx^{1} \wedge \dotsm \wedge dx^{3} + \dotsm +  f_{x^{4}} \wedge dx^{1} \wedge \dotsm \wedge dx^{4}. 
		\end{equation}
	The first  three terms are all zero. The last term is zero iff $f_{x^{4}} = 0$, which means $f = f(x^{1}, x^{2}, x^{3})$. \textcolor{red}{[!! Complete Later !!]} 
\end{solutions}
\begin{prb}{2017-J-II-1}
	Let $f: M \to \mathbb{R}$ be a smooth function on a smooth manifold $M$. In an arbitrary smooth local coordinate chart $x: U \to \mathbb{R}^{n}$ of $M$, define 
		\begin{equation*}
			\mathscr{D}f \coloneqq \frac{\partial f}{\partial x^{i}}\frac{\partial}{\partial x^{i}}. 
		\end{equation*}
	Does $\mathscr{D}f$ give a well-defined vector field on $M$? 
\end{prb}
\begin{solutions}
	We claim that $\mathscr{D}f$ does not give a well-defined vector field on $M$. Let $(U, (x^{i}))$ and $(V, (\ti{x}^{i}))$ denote two overlapping smooth local coordinate charts on $M$, and let $p \in U \cap V$. Then 
		\begin{align}
			\begin{split}
				\mathscr{D}f &= \frac{\partial f}{\partial x^{i}}(p)\frac{\partial}{\partial x^{i}}\bigg|_{p} \\
				&= \frac{\partial f}{\partial \ti{x}^{j}}\frac{\partial \ti{x}^{j}}{\partial x^{i}}\bigg|_{p}\frac{\partial \ti{x}^{k}}{\partial x^{i}}\bigg|_{p}\frac{\partial}{\partial \ti{x}^{k}}\bigg|_{\hat{p}}, 
			\end{split}
		\end{align}
	which is identically not equal to $(\partial_{\ti{x^{k}}}f)\partial_{\ti{x}^{k}}$, which is the expression for $\mathscr{D}f$ in the smooth coordinate chart $(V, (\ti{x}^{j}))$.  
\end{solutions}
\begin{prb}{2017-J-II-2 (Real Analysis)}
	Suppose $f: [0,1] \to \mathbb{R}$ is measurable. Suppose further that for all $g \in L^{2}([0,1])$, we have that $fg \in L^{2}([0,1])$. Show that $f$ is in $L^{\infty}([0,1])$. 
\end{prb}
\begin{solutions}
	Let $f: [0,1] \to \mathbb{R}$ be measurable, and suppose that for all $g \in L^{2}([0,1])$, $fg \in L^{2}([0,1])$. Assume to the contrary that $f \notin L^{\infty}([0,1])$, which means that for every positive integer $n$, the set 
		\begin{equation}
			E_{n} = \{x: |f_{n}(x)| \geq n\}
		\end{equation}
	has positive measure. Consider the simple function 
		\begin{equation}
			g = \sum_{1}^{\infty}\frac{1}{n\sqrt{m(E_{n})}}\chi_{E_{n}}
		\end{equation}
	so that 
		\begin{equation}
			\norm{g}_{2}^{2} = \int_{0}^{1} \sum_{1}^{\infty}\frac{1}{n^{2}m(E_{n})}\chi_{E_{n}} = \sum_{1}^{\infty}\frac{1}{n^{2}} < \infty. 
		\end{equation}
	On the other hand 
		\begin{eqnarray}
			\norm{fg}_{2}^{2} = \int_{0}^{1}|f|^{2}\sum_{1}^{\infty}\frac{1}{n^{2}m(E_{n})}\chi_{E_{n}} \geq \sum_{1}^{\infty}\int_{E_{n}}\frac{1}{m(E_{n})} = \sum_{1}^{\infty}1 >\infty, 
		\end{eqnarray}
	which means $fg \notin L^{2}$. This is a contradiction. Hence, by contradiction, $f \in L^{\infty}([0,1])$. 
\end{solutions}

\subsection{August 2017}
\begin{prb}{2017-A-I-1 (Geometry/Topology)}
	Let $M$ be a smooth compact connected $n$-manifold (without boundary), and let $F: M \to \mathbb{R}^{n}$ be a smooth map. Does $F$ necessarily have a critical point? 
\end{prb}
\begin{solutions}
	Let $M$ be a smooth compact connected $n$-manifold (without boundary), and let $F: M \to \mathbb{R}^{n}$ be a smooth map. Suppose $F$ has no critical points, which means that $dF_{p}$ is surjective at every $p \in M$. I.e., $\operatorname{rank}{dF_{p}} = n$ for every $p \in M$. Let $F = (f_{1}, \ldots, f_{n})$, where each $f_{j}: M \to \mathbb{R}$ is a component function of $F$. Fix some $f_{j}$; since $M$ is compact, $f_{j}$ must attain a maximum or minimum at some point $p \in M$. This means that $df_{j}(p) =0$. But since $dF_{p} = (df_{1}(p), \ldots, df_{j}(p), \ldots, df_{n}(p))$, $\operatorname{rank}{dF_{p}} \neq n$, which is a contradiction. Hence, $F$ must have a critical point. 
\end{solutions}
\begin{prb}{2017-A-II-3 (Algebra)}
	Let $K$ denote the splitting field of $f(x) = x^{4} + x^{2} + 1$ over $\mathbb{Q}$. Compute the Galois group $\operatorname{Gal}(K/\mathbb{Q})$. 
\end{prb}
\begin{solutions}
	Let $f(x) = x^{4} + x^{2} + 1$; by the rational root test, $f(x)$ has no rational roots. However,
		\begin{equation}
			f(x) = x^{4} + x + 1 = (x^{2} + x + 1)(x^{2} - x+ 1), 
		\end{equation}
	where each quadratic factor is irreducible by the rational root test. The roots of these quadratic factors are 
		\begin{equation}
			x = \pm\sqrt{\frac{-1 \pm \sqrt{-3}}{2}}. 
		\end{equation}
	Let $\alpha = \sqrt{\frac{-1 + \sqrt{-3}}{2}}$ and $\beta = \sqrt{\frac{-1 -\sqrt{-3}}{2}}$. We observe then that $\alpha^{2}\beta^{2} = 1 \implies \beta = \pm \frac{1}{\alpha}$. On the other hand, 
		\begin{equation}
			\alpha^{2} = -\frac{1}{2} + \frac{\sqrt{-3}}{2},
		\end{equation}
	so that $\alpha \in \mathbb{Q}(\sqrt{-3})$. Hence, we conclude that the splitting field of $f(x)$ over $\mathbb{Q}$ is $K = \mathbb{Q}(\sqrt{-3})$. Since the minimal polynomial of $\sqrt{-3}$ over $\mathbb{Q}$ has degree 2, $[\mathbb{Q}(\sqrt{3}): \mathbb{Q}] = 2$. Hence, the Galois group $\operatorname{Gal}(K/\mathbb{Q})$ has order 2, which means $\operatorname{Gal}(K/\mathbb{Q}) \cong \mathbb{Z}/2\mathbb{Z}$. 
\end{solutions}
\subsection{January 2013}
\begin{prb}{2013-J-II-6 (Geometry/Topology)}
	Let $M$ be a smooth compact manifold, and suppose that there is a smooth map $F: M \to S^{1}$ whose derivative is non-zero at every point. Prove that the de Rham cohomology space $\h{1}(M)$ is non-zero. 
\end{prb}
\begin{solutions}
	Let $M$ be a smooth compact manifold, and $F: M \to S^{1}$ a smooth map whose derivative is non-zero at every point. Assume to the contrary that the de Rham cohomology space $\h{1}(M) =0$, which means that every closed 1-form on $M$ is exact. Since $\h{1}(S^{1}) \cong \mathbb{R}$, there exists a nowhere vanishing closed 1-form $\omega$ on $S^{1}$ such that its equivalence class generates $\h{1}(S^{1})$. Then since $F$ is a smooth map, $F^{\ast}\omega$ is a closed 1-form on $M$. Since $\h{1}(M) = 0$, $F^{\ast}\omega$ is an exact form; i.e., there exists a smooth function: $f: M \to \mathbb{R}$ such that $F^{\ast}\omega = df$. Since $f$ is smooth and $M$ is compact, $f$ must have a maximum or minimum at some point $p \in M$, which implies that $df_{p} = 0$ at $p \in M$. Therefore, $0 = (F^{\ast}\omega)_{p} = \omega_{F(p)} \circ dF_{p}$. Since $\omega$ is nowhere vanishing, we conclude that $dF_{p} =0$. But this contradicts our assumption that $dF$ is non-zero at every point. Hence, by contradiction, $\h{1}(M) \neq 0$. 
\end{solutions}


\newpage 
\subsection{August 2013}
\begin{prb}{2013-A-II-4 (Geometry/Topology)}
	Let $\theta$ be a smooth 1-form on a manifold $M$ such that $\theta \neq 0$ everywhere. Let $D \subset TM$ be the vector subbundle defined by 
		\begin{equation*}
			D = \ker{\theta} = \brac*{v \in TM: \theta(v) =0}. 
		\end{equation*}
	Prove that $D$ is Frobenius integrable if and only if $\theta \wedge d\theta = 0$ everywhere. 
\end{prb}
\begin{solutions}
	Assume the hypotheses of the problem. We recall that $D$ is Frobenius integrable if and only if for any pair of smooth sections $X, Y$ of $D$, $[X, Y]$ is a smooth section of $D$. So let $X, Y$ be smooth sections of $D$, which means that $\theta(X) = \theta(Y) = 0$ everywhere. Suppose that $D$ is Frobenius integrable so that $\theta([X, Y]) = 0$. Since $\theta$ is not identically zero, for any $p \in M$, there exists a vector $R_{p}$ with $\theta_{p}(R_{p}) = 1$. This means that locally one can choose a smooth vector field $R$ with $\theta(R) = 1$. On this neighborhood, we have $T_{p}M= RR_{p} \oplus D_{p}$. Now, we note that 
		\begin{equation}
			\theta \wedge d\theta(X, Y, R) = \theta(X)d\theta(Y,R) + \theta(Y)d\theta(R,X) + \theta(R)d\theta(X, Y). 
		\end{equation}
	The first two terms are identically zero by our hypothesis. For the latter, we note that
		\begin{equation}
			d\theta(X, Y) = X(\theta(Y)) - Y(\theta(X)) - \theta([X, Y]), 
		\end{equation}
	which is identically zero. Hence, this means that $\theta \wedge d\theta(R, X, Y)$ is zero. This means that $(\theta \wedge d\theta)_{p} = 0$ for all $p \in M$. Hence, $\theta \wedge d\theta$ is identically zero. Now suppose $\theta \wedge d\theta$ is identically zero. Let $X, Y$ be smooth sections of $D$ and pick a local vector field $R$ such that $\theta(R) = 1$. We recover once again that 
		\begin{equation}
			0 = \theta \wedge d\theta(X, Y, R) = -\theta(R)\theta([X,Y]) \implies \theta([X, Y]) = 0. 
		\end{equation}
	Hence, $[X, Y] \in \Gamma(D)$, which means that $D$ is Frobenius integrable. 
\end{solutions}
\begin{prb}{2013-J-II-5 (Real Analysis)}
	Let $E \subset [0,1]$ be a measurable set. Assume $E$ has positive Lebesgue measure. Show that there are $\alpha$ and $\beta$ such that all three numbers $\alpha, \alpha + \beta$, $\alpha + 2\beta \in E$. 
\end{prb}
\begin{solutions}
	Let $E \subset [0,1]$ be  measurable set with positive Lebesgue measure, and $\epsilon < 1/3$. By inner regularity of the Lebesgue measure, there exists a compact set $K_{1} \subset E$ so that $m(K_{1}) > 0$. By outer regularity of the Lebesgue measure, there exists an open set $U \supset K_{1}$ so that 
		\begin{equation}
			m(U) \leq (1 + \epsilon)m(K_{1}). 
		\end{equation}
	Since $K_{1}$ is compact, the quantity $D = d(K_{1}, U^{c}) > 0$. So let $R = D/2$, and pick an arbitrary $\beta \in (0, D/2)$. We first claim that $K_{1} + \beta \subset U$, since if not, then 
		\begin{equation}
			d(K_{1}, U^{c}) < \beta = \frac{D}{2} < D, 
		\end{equation}
	which is a contradiction. In particular, this means that $K_{1} \cup (K_{1} + \beta) \subset U$ so that 
		\begin{equation}
			m(U) \geq m(K_{1} \cup (K_{1} + \beta)) = m(K_{1}) + m(K_{1} + \beta) - m(K_{1} \cap (K_{1} + \beta)). 
		\end{equation}
	By translation invariance of the Lebesgue measure, $m(K_{1}) = m(K_{1} + \beta)$ so that 
		\begin{equation}
			m(K_{1} \cap (K_{1} + \beta)) \geq 2m(K_{1}) - m(U) \geq 2m(K_{1}) - (1 + \epsilon)m(K_{1}) = (1 - \epsilon)m(K_{1}). 
		\end{equation}
	Since $\epsilon < 1$, we conclude that $m(K_{1} \cap (K_{1} + \beta)) > 0$ so that $K_{1} \cap (K_{1} + \beta) \neq \varnothing$. Now for $j = 1, 2$, define $K_{j + 1} = K_{j} \cap (K_{j} + \beta)$. Generalizing the arguments from above, we see that $K_{j} + \beta \subset U$ for $j = 1, 2$ and $m(K_{j + 1}) \geq (1 -\epsilon(2^{j} - 1))m(K_{1}) > 0$ so that $K_{1}, K_{2}, K_{3}$ are nonempty. Hence, this produces a nested sequence of nonempty sets $\varnothing \neq K_{3} \subset K_{2} \subset K_{1} \subset E$. Let $q \in K_{3}$ be arbitrary; since $K_{3} = K_{2} \cap (K_{2} + \beta)$, $q - \beta \in K_{2}$. And since $K_{2} = K_{1} \cap (K_{1} + \beta)$, $q - \beta -\beta = q -2\beta \in K_{1}$, Let $\alpha = q - 2\beta$. This proves that $\{\alpha, \alpha + \beta, \alpha + 2\beta\} \subset E$, concluding the proof. 
\end{solutions}


\newpage 
\subsection{Textbook Problems}
\begin{prb}{Lee-7-5}
	Let $M$ be a smooth compact manifold. Show that there is no submersion $F: M \to \mathbb{R}^{k}$ for any $k > 0$. 
\end{prb}
\begin{solutions}
	Let $M$ be a smooth compact manifold, and assume to the contrary that there exists a submersion $F: M \to \mathbb{R}^{k}$ for some $k > 0$. Since $M$ is compact, $F$ must attain either a maximum or minimum at some point $p \in M$, which means that $dF_{p} = 0$. But this is impossible since $F$ is a submersion, which means that $\operatorname{rank}{dF_{p}} = \operatorname{dim}{\mathbb{R}^{k}} = k > 0$. Hence, by contradiction, $F$ cannot be a submersion. 
\end{solutions}


\begin{prb}{D\&F-14.6.2}
	Determine the Galois groups of the following polynomials: \vspace{-0.25cm}
		\begin{enumerate}[itemsep =-2pt,label = (\roman{*})]
			\item $x^{3} - x^{2} - 4$ 
			\item $x^{3} - 2x + 4$
			\item $x^{3} - x + 1$ 
			\item $x^{3} + x^{2} - 2x - 1$.
		\end{enumerate}
\end{prb}
\begin{solutions}
	$ $\newline \vspace{-1cm}
	\begin{enumerate}[itemsep =-2pt,label = (\alph{*})]
		\item Let $f(x) = x^{3} - x^{2} - 4$. We note that $f$ has a rational root $x = 2$ since $2^{3} - 2^{2} - 4 = 8 - 4 - 4= 0$. Using polynomial long division, we find that $f(x)$ is reducible over $\mathbb{Q}$ as the product 
			\begin{equation}
				f(x) = (x - 2)(x^{2} + x + 2). 
			\end{equation}
		By the rational root test, the quadratic factor is irreducible and has complex roots 
			\begin{equation}
				x_{1,2} = \frac{-1 \pm \sqrt{-7}}{2}. 
			\end{equation}
		Therefore, the splitting field of $f(x)$ is $\mathbb{Q}(\sqrt{-7})$, which has degree 2 since the minimal polynomial of $\sqrt{-7}$ is $x^{2} + 7$. Therefore, the Galois group $\operatorname{Gal}(\mathbb{Q}(\sqrt{-7})/\mathbb{Q})$ has order 2; hence the Galois group is $\mathbb{Z}/2\mathbb{Z}$. 
		\item Let $f(x)= x^{3} - 2x + 4$. We note that $f(x)$ has a rational root $x = -2$ since $(-2)^{3} - 2(-2) + 4 = -8 + 4 + 4 = 0$. Hence using polynomial long division, 
			\begin{equation}
				f(x) = (x + 2)(x^{2} - 2x + 2). 
			\end{equation}
		By the rational root test, $x^{2} - 2x + 2$ is irreducible over $\mathbb{Q}$ with complex roots $1 \pm i$. Therefore, the splitting field of $f(x)$ is $\mathbb{Q}(i)$, which has degree 2 since the minimal polynomial of $i$ is $x^{2} + 1$. Therefore, the Galois group $\operatorname{Gal}(\mathbb{Q}(i)/\mathbb{Q})$ has order 2; hence the Galois group is $\mathbb{Z}/2\mathbb{Z}$. 
		\item Let $f(x) = x^{3} - x + 1$; by the rational root test $f(x)$ is irreducible over $\mathbb{Q}$. However, since $f$ is already a depressed cubic, we note that its discrimant is $-4p^{3} - 27q^{2} = 4 - 27 = -23$. Since $-23$ is not a perfect square, we conclude that the Galois group is $S_{3}$. In fact, the splitting field for this cubic is $\mathbb{Q}(\alpha, \sqrt{-23})$, where $\alpha$ is a root of $x^{3} - x + 1$. 
		\item Let $f(x) = x^{3} + x^{2} - 2x - 1$; by the rational root test $f(x)$ is irreducible over $\mathbb{Q}$. Therefore, we will now depress the cubic. Let $x = y - 1/3$. Then 
			\begin{align}
				\begin{split}
					x^{3} + x^{2} - 2x - 1 &= y^{3} - \frac{7}{3}y - \frac{7}{27}. 
				\end{split}
			\end{align}
		The discriminant of the depressed cubic is, 
			\begin{equation}
				D = -4p^{3} - 27q^{2} = 4\left(\frac{7^{3}}{27}\right) - 27\left(\frac{7^{2}}{27^{2}}\right) = \frac{7^{2}}{27}\left(4 \cdot 7 - 1\right) = 7^{2}. 
			\end{equation}
		Since the discriminant is a square, we see that the Galois group of the polynomial is $A_{3}$. 
	\end{enumerate}
\end{solutions}
\begin{prb}{D\&F-14.6.4}
	Determine the Galois group of $x^{4} - 25$. 
\end{prb}
\begin{solutions}
	Let $f(x) = x^{4} - 25$. The roots of $f(x)$ are $\zeta_{4}^{0}\sqrt[4]{25}, \zeta_{4}^{1}\sqrt[4]{25}, \zeta_{4}^{2}\sqrt[4]{25}$, and $\zeta_{4}^{3}\sqrt[4]{25}$, where $\zeta_{4}$ is the primitive 4th root of unity. Here, we recall that the automorphisms in the Galois group of $f$ act transitively on the roots of $f(x)$. Hence, the Galois group of $f(x)$ must contain the automorphism that maps $\sqrt[4]{25} \mapsto -\sqrt[4]{25}$ (i.e., a reflection) and $\sqrt[4]{25} \mapsto \zeta_{4}^{j}\sqrt[4]{25}$ (i.e., a rotation). Hence, the Galois group is $D_{8}$. 
\end{solutions}

\begin{prb}{D\&F-14.6.5}
	Determine the Galois group of $x^{4} + 4$. 
\end{prb}
\begin{solutions}
	Let $f(x) = x^{4} + 4$, which is irreducible over $\mathbb{Q}$. However, the four roots of $f(x)$ are $\pm 1 \pm i$. This means that the splitting field of $f(x)$ is $\mathbb{Q}(i)$, which is a degree 2 extension over $\mathbb{Q}$. Hence, the Galois group is of order 2, which implies that the Galois group is the cyclic group $\mathbb{Z}/2\mathbb{Z}$. 
\end{solutions}
\begin{prb}{MAT532-F-4}
	Suppose $E \subset \mathbb{R}^{2}$ is Lebesgue measurable. For a square $Q$, let $C_{Q}$ be the \textit{white} squares of a $(8 \times 8)$ checkerboard fitted exactly in $Q$ (so a white square has sidelength 1/8 the sidelength of $Q$). Suppose that for almost any $x \in E$, and any square $Q_{x}$ with $x$ in its lower left corner, we have that $E \cap C_{Q_{x}} = \varnothing$, i.e., $E$ does not intersect the white squares of a checkerboard fitted to $Q_{x}$. Show $m(E) = 0$, where $m$ is Lebesgue measure. 
\end{prb}
\begin{solutions}
	Let $E \subset \mathbb{R}^{2}$ be Lebesgue measurable, and set $A = \{x \in E: E \cap C_{Q_{x}} = \varnothing \text{ for any square $Q_{x}$}\}$; by hypothesis, $A$ consists of almost every $x \in E$. Assume to the contrary that $m(E) \neq 0$ and pick $x \in A$. For this $x$, construct a family of sets $\{E_{r}\}_{r > 0}$ as follows: for each $r$, let $E_{r}$ be a square of sidelength $r/\sqrt{2}$ with $x$ in its lower left corner. It is straightforward to see that for every $r > 0$, $E_{r} \subset B(x, r)$ and $m(E_{r}) = 2\pi^{-1}m(B(r, x))$. Hence, $\{E_{r}\}$ shrinks nicely to $x$. Now, by hypothesis, $m(E \cap E_{r}) \leq \frac{1}{2}m(E_{r})$ for every $r$ since $E$ intersects at most half of $E_{r}$. This means that 
	\begin{equation}
		\limsup_{r \to 0}\frac{m(E \cap E_{r})}{m(E_{r})} \leq \frac{1}{2}. 
	\end{equation}
	I.e., for almost every $x \in E$, the Lebesgue density is at most $1/2$, which contradicts the Lebesgue Density Theorem. Therefore, by contradiction, $m(E) = 0$. 
\end{solutions}
\begin{prb}{MAT532-7-4}
	Suppose a set $E \subset \mathbb{R}^{3}$ satisfies that for every $x \in\mathbb{R}^{3}$ and $r > 0$, there exists a point $z \in B(x, r)$ such that $E \cap B(z, r/2) \cap B(x, 2r) = \varnothing$. Show that $m(E) = 0$, where $m$ is the Lebesgue measure on $\mathbb{R}^{3}$. 
\end{prb}
\textcolor{red}{[!! Complete Later !!]}

\begin{prb}{(Algebra-Classification-I)}
	Classify all groups of order 2026. 
\end{prb}
\begin{solutions}
	Let $G$ be a group of order $2026 = 2 \cdot 1013$. By Sylow's Theorem, $G$ must contain a normal Sylow 5-subgroup, which we denote by $H$. Let $K$ be a Sylow 2-subgroup of $G$; note $K \cong \mathbb{Z}_{2}$. By Lagrange's Theorem, $H$ and $K$ must intersect trivially. Moreover, $|HK| = |H||K|/|H \cap K| = |H||K| = |G|$, so that $G = HK$. Hence, by the recognition theorem for semidirect products, $G \cong H \rtimes_{\varphi} \mathbb{Z}_{2}$, where $\varphi \in \operatorname{Aut}{H} = \mathbb{Z}_{1013}^{\ast} \cong \mathbb{Z}_{1012}$. So we look for homomorphisms $\varphi: \mathbb{Z}_{2} \to \mathbb{Z}_{1012}$; each homomorphism is completely determined by where the generator $1$ is mapped to. 
		\begin{enumerate}[itemsep =-2pt,label = (\roman{*})]
			\item Consider the map $1 \mapsto 0$, which corresponds to the trivial homomorphism. Then the semidirect product is just the direct product, and so $G \cong \mathbb{Z}_{1013} \times \mathbb{Z}_{2}$. 
			\item Consider the map $\varphi: 1 \mapsto 506$, where $506$ is the unique element of $\mathbb{Z}_{1012}$ with order 2. This is a non-trivial homomorphism with kernel $\{0\}$. Hence, this gives a non-abelian group $\mathbb{Z}_{1013} \rtimes_{\varphi} \mathbb{Z}_{2}$. 
		\end{enumerate}
	Hence, up to isomorphism, there are only two groups of order 2026. 
\end{solutions} 
\begin{prb}{(Algebra-Classification-II)}
	Classify all groups of order 1969. 
\end{prb}
\begin{solutions}
	Let $G$ be a group of order $1969 = 11 \cdot 179$. By Sylow's Theorem, $G$ must contain a normal Sylow 179-subgroup, which we denote by $H$. Let $K$ be a Sylow 11-subgroup of $G$; note $K \cong \mathbb{Z}_{11}$. By Langrage's Theorem, $H$ and $K$ must intersect trivially and $G = HK$. Therefore, $G = H \rtimes_{\varphi} K$ for some $\varphi \in \operatorname{Aut}{H} = \mathbb{Z}_{179}^{\ast}\cong \mathbb{Z}_{178}$. So we look for homomorphisms $\varphi: \mathbb{Z}_{11} \to \mathbb{Z}_{178}$; each homomorphism is completely determine by where the generator 1 is mapped to. \vspace{-0.35cm}
		\begin{enumerate}[itemsep =-2pt,label= (\roman{*})]
			\item Consider the map $1 \mapsto 0$. This corresponds to the trivial homomorphism so that the semidirect product is just the direct product. Therefore, $G \cong \mathbb{Z}_{179} \times \mathbb{Z}_{11} \cong \mathbb{Z}_{1969}$ (by the Chinese Remainder Theorem). 
			\item Since $1$ has order 11, $1$ must map to some nonzero element of $\mathbb{Z}_{178}$ of order 11; but since $11$ and $178$ are relatively prime, there exists no such element. 
		\end{enumerate}
	Hence, we conclude that there is exactly one group of order 1969, which is precisely $\mathbb{Z}_{1969}$. 
\end{solutions}
\begin{prb}{2008-J-I-3 (Algebra)}
	Classify all groups of order 28. 
\end{prb}
\begin{solutions}
	Let $G$ be a group of rder $28 = 2^{2} \cdot 7$. By Sylow's Theorem, $G$ contains a normal Sylow 7-subgroup, which we denote by $H$. Let $K$ be a Sylow 2-subgroup, which has order 4. By Lagrange's Theorem, $H$ and $K$ must intersect trivially and $G = HK$. Hence, by the recognition theorem for semidirect products, $G = H \rtimes_{\varphi} K$ for some $\varphi \in \operatorname{Aut}(H) = \mathbb{Z}_{7}^{\ast} \cong \mathbb{Z}_{6}$. So we look for homomorphisms $\varphi: K \to \mathbb{Z}_{6}$, where $K$ is a group of order 4. Up to isomorphism, there are precisely two groups of order 4: (1) $\mathbb{Z}_{4}$, and (2) $\mathbb{Z}_{2} \times \mathbb{Z}_{2}$. We consider each case separately: 
		\begin{enumerate}[itemsep =-2pt,label = (\Roman{*})]
			\item Consider the case $K = \mathbb{Z}_{4}$, which has two generators: $1$ and $3$. Each homomorphism $\varphi: \mathbb{Z}_{4} \to \mathbb{Z}_{6}$ is determined by where $\varphi$ sends a generator with the constrain that $1$ may be sent to only those elements of $\mathbb{Z}_{6}$ whose order divides 4 (namely $0,3$). 
				\begin{enumerate}[itemsep =-2pt,label = (\roman{*})]
					\item Suppose $\varphi_{1}: 1 \mapsto 0$. Then since $\varphi(3) = 3 \cdot \varphi(1) = 0$, $\varphi$ is the trivial homomorphism. In this case, the semidirect product is the direct product and $G$ is isomorphic to the abelian group $\mathbb{Z}_{7} \times \mathbb{Z}_{4}$. 
					\item Suppose $\varphi_{2}: 1 \mapsto 3$. Then this is a nontrivial homomorphism with image consisting of $\{0,3\}$ and kernel consisting of $\{0,2\}$. Hence, this produces a non-abelian group $\mathbb{Z}_{7} \rtimes_{\varphi_{2}} \mathbb{Z}_{4}$. 
				\end{enumerate}
			\item Now consider the case $K = \mathbb{Z}_{2} \times \mathbb{Z}_{2} = \braket{a} \times \braket{b}$. $\psi: \mathbb{Z}_{2} \times \mathbb{Z}_{2} \to \mathbb{Z}_{6}$ is determined uniquely by $\psi(a)$ and $\psi(b)$ provided that its order divides 2. This means $\psi(a),\psi(b) \in \{0,3\}$. 
				\begin{enumerate}[itemsep =-2pt,label = (\roman{*})]
					\item Suppose $\psi_{1}(a) = \psi_{1}(b) = 0$. The semidirect is then a direct product and so $G \cong \mathbb{Z}_{7} \times \mathbb{Z}_{2} \times \mathbb{Z}_{2} \cong \mathbb{Z}_{14} \times \mathbb{Z}_{2}$. 
					\item Suppose $\psi_{2}(a) = 0$ and $\psi_{2}(b) = 3$. This is a nontrivial homomorphism so that $G \cong \mathbb{Z}_{7} \rtimes_{\psi_{2}}\mathbb{Z}_{2}^{2}$ is non-abelian. 
					\item Suppose $\psi_{3}(a) = 3$ and $\psi_{3}(b) = 0$. Then $\psi_{3} = \psi_{2} \circ \theta$ where $\theta$ is the automorphism of $\mathbb{Z}_{2} \times \mathbb{Z}_{2}$ given by $\theta(a) = b$ and $\theta(b) = a$. Hence, this semidirect product gives the same group as in case (ii). 
					\item Suppose $\psi_{4}(a) = \psi_{4}(b) = 3$. Then $\psi_{4} = \psi_{3} \circ \theta$ where $\theta$ is the automorphism of $\mathbb{Z}_{2} \times \mathbb{Z}_{2}$ given by $\theta(a) = a$ and $\theta(b) = ab$. Hence, this semidirect product gives the same group as in case (iii). 
				\end{enumerate}
		\end{enumerate}
	Altogether, we conclude that there are exactly four isomorphism classes of groups of order 28, namely $\mathbb{Z}_{7} \times \mathbb{Z}_{4}$, $\mathbb{Z}_{7} \rtimes_{\varphi_{2}}\mathbb{Z}_{4}$, $\mathbb{Z}_{14} \times \mathbb{Z}_{2}$, and $\mathbb{Z}_{7} \rtimes_{\psi_{2}}\mathbb{Z}_{2}^{2}$, of which exactly two are abelian. 
\end{solutions}
\begin{prb}{2010-J-II-5 (Algebra)}
	Classify (up to isomorphism) all groups of order 45. 
\end{prb}
\begin{solutions}
	Let $G$ be a group of order $45 = 3^{2} \cdot 5$. By Sylow's Theorem, $G$ has a normal Sylow 5-subgroup, which we denote by $H$. Let $K$ denote a Sylow 3-subgroup of $G$, which has order 9. By Lagrange's Theorem, $H, K$ intersect trivially and $|G| = |H||K|$ so that $G = HK$. Hence, $G \cong H \rtimes_{\varphi} K$ for some $\varphi \in \operatorname{Aut}(H) \cong \mathbb{Z}_{5}^{\ast} \cong \mathbb{Z}_{4}$. Hence, we look at homomorphisms $\varphi: K \to \mathbb{Z}_{4}$. There are exactly two groups of order 9, up to isomorphism; namely, these are $\mathbb{Z}_{9}$ and $\mathbb{Z}_{3} \times \mathbb{Z}_{3}$. Hence, we consider each separately. 
		\begin{enumerate}[itemsep=-2pt,label = (\Roman{*})]
			\item Let $K = \mathbb{Z}_{9}$, which has generators 1, 2, 4, 5, 7, and 8. Each homomorphism $\varphi: K \to \mathbb{Z}_{4}$ is determined uniquely by where $\varphi$ sends a generator with the constraint that they may only be sent to those elements of $\mathbb{Z}_{4}$ whose order divides $9$. There is only one such element, namely 0. Hence, the only group we get is the direct product $\mathbb{Z}_{9} \times \mathbb{Z}_{5} \cong \mathbb{Z}_{45}$, which is abelian. 
			\item Let $K = \mathbb{Z}_{3} \times \mathbb{Z}_{3}$. Each $\psi: \mathbb{Z}_{3} \times\mathbb{Z}_{3} = \braket{a} \times \braket{b} \to \mathbb{Z}_{4}$ is uniquely determined by $\psi(a)$ and $\psi(b)$ provided they divide $3$. But there is only one such element in $\mathbb{Z}_{4}$, which is zero. Hence, we only get the direct product $\mathbb{Z}_{3} \times \mathbb{Z}_{3} \times \mathbb{Z}_{5} \cong \mathbb{Z}_{3} \times \mathbb{Z}_{15}$, which is abelian. 
		\end{enumerate}
	Therefore, we find that (1) there are exactly two groups, up to isomorphism, of order 45; and (2) both groups are abelian. 
\end{solutions}
\begin{prb}{2003-J-I-6 (Algebra)}
	$ $\newline \vspace{-0.65cm}
	\begin{enumerate}[itemsep =-2pt,label = (\alph{*})]
		\item Prove that a group of order $p^{2}$, where $p$ is a prime number, is abelian. 
		\item Classify groups of order $p^{2}$ up to isomorphism. 
	\end{enumerate}
\end{prb}
\begin{solutions}
	$ $\newline \vspace{-1cm}
	\begin{enumerate}[itemsep =-2pt,label = (\alph{*})]
		\item Let $G$ be a group of order $p^{2}$, and let $Z(G)$ be its center. By Lagrange's Theorem, $|Z(G)| \in \{1, p, p^{2}\}$. If $|Z(G)| = p^{2}$ and so $G = Z(G)$, which means $G$ is abelian. $|Z(G)| \neq p$ since otherwise $|G/Z(G)| = p$ forcing $G/Z$ to be cyclic and $G$ to be abelian (which contradicts $Z(G)$ being a proper subgroup of $G$). Finally $|Z(G)|$ cannot be one, since the center of any $p$-group must necessarily be nontrivial (by the class equation). Hence, $Z(G) = G$, which means $G$ is abelian. 
		\item Since every group of order $p^{2}$ must necessarily be abelian, up to isomorphism, there must be exactly two groups, namely $\mathbb{Z}_{p}\times \mathbb{Z}_{p}$ and $\mathbb{Z}_{p^{2}}$. 
	\end{enumerate}
\end{solutions}
\begin{prb}{2010-J-I-5 (Algebra)}
	Consider the following irreducible polynomial over $\mathbb{Q}$: $p(x) = x^{4} - 3x^{2} - 1$. \vspace{-0.35cm}
	\begin{enumerate}[itemsep =-2pt,label = (\alph{*})]
		\item Describe the splitting field of $p(x)$. 
		\item Consider the Galois group of $p(x)$. Compute its order and determine if it is abelian. 
	\end{enumerate}
\end{prb}
\begin{solutions}
	\vspace{-0.35cm}
	\begin{enumerate}[itemsep =-2pt,label = (\alph{*})]
			\item Let $p(x) = x^{4} - 3x^{2} - 1$. By the rational root test, $p(x)$ has no roots over $\mathbb{Q}$. Moreover, it is straightforward to check that $p(x)$ is not the product of irreducible quadratics with rational coefficients. Hence, $p(x)$ is irreducible over $\mathbb{Q}$. We start by finding the roots of $p(x)$; let $u = x^{2}$. Then 
			\begin{equation}
				u^{2} - 3u - 1 = 0 \implies u = \frac{3 \pm \sqrt{13}}{2} \implies x = \pm \sqrt{\frac{3 \pm \sqrt{13}}{2}}. 
			\end{equation}
			Let 
			\begin{equation}
				\alpha = \sqrt{\frac{3 + \sqrt{13}}{2}}, \qquad \beta = \sqrt{\frac{3 - \sqrt{13}}{2}}. 
			\end{equation}
			Observe that $\alpha^{2}\beta^{2} = -1$ so that $\beta = \pm \frac{i}{\alpha}$. Therefore, the splitting field of $p(x)$ is 
			\begin{equation}
				\mathbb{Q}(\alpha, i). 
			\end{equation}
			Observe that the minimal polynomial of $i$ is $x^{2} + 1$, which is irreducible over $\mathbb{Q}(\alpha)$ so that $[\mathbb{Q}(\alpha, i):\mathbb{Q}(\alpha)] = 2$. On the other hand, the minimal polynomial of $\alpha$ is a degree 4 polynomial so that $[\mathbb{Q}(\alpha): \mathbb{Q}] = 4$. Hence, by the tower law, $[\mathbb{Q}(\alpha, i): \mathbb{Q}] = 8$. 
			\item By the last work in (a), the order of the Galois group of $p(x)$ is $8$. Now, we will determine the Galois group of $p(x)$. Recall that elements of $\operatorname{Gal}(\mathbb{Q}(\alpha, i)/\mathbb{Q})$ are automorphisms $\varphi$ of the field $\mathbb{Q}(\alpha, i)$ with the constraints that: (1) $\varphi$ fixes $\mathbb{Q}$, (2) $\varphi(\alpha)$ must be another root of the minimal polynomial of $\alpha$ over $\mathbb{Q}$, and (3) $\varphi(i)$ must be another root of $x^{2} + 1$. We will explicitly work through each of the elements. 
			\begin{enumerate}[itemsep =-2pt,label =(\roman{*})]
				\item $\sigma: i \mapsto -i, \alpha \mapsto \alpha$. This permutation has order 2 since $\sigma^{2}(\alpha)  = \sigma(\alpha) = \alpha$ and $\sigma^{2}(i) = \sigma(-i) = i$. 
				\item $\tau: i \mapsto i, \alpha \mapsto -\alpha$. Once again, this permutation has order 2. 
				\item $\rho: i \mapsto -i, \alpha \mapsto \beta = \frac{i}{\alpha}$. To compute the order of this permutation, observe that 
				\begin{equation}
					\rho^{2}(\alpha) = \rho(i\alpha^{-1}) = (-i) \cdot \frac{1}{i/\alpha} = -\alpha \implies \rho^{4}(\alpha) = \rho^{2}(-\alpha) = \alpha. 
				\end{equation}
				Likewise, $\rho^{4}(i) = \rho^{2}(i) = i$. Hence, $\rho$ has order $4$. 
			\end{enumerate}
			Now, consider the three elements given above. We compute 
			\begin{equation}
				\sigma\rho\sigma(i) = \sigma\rho(-i) = \sigma(i) = -i = \rho^{-1}(i). 
			\end{equation}
			Likewise, 
			\begin{equation}
				\sigma\rho\sigma(\alpha) = \sigma\rho(\alpha) = \sigma(i)\sigma(\alpha)^{-1} = -\frac{i}{\alpha} = \rho^{-1}(\alpha). 
			\end{equation}
			Therefore, $\sigma \rho \sigma = \rho^{-1}$. Hence, 
			\begin{equation}
				\operatorname{Gal}(\mathbb{Q}(\alpha,i)/\mathbb{Q}) = \{1, \sigma, \rho, \rho^{2}, \rho^{3}, \sigma\rho, \sigma\rho^{2},  \sigma\rho^{3}\} \cong D_{8}. 
			\end{equation}
			Since the dihedral group is not abelian, we conclude that the Galois group for $p(x)$ is non-abelian. 
	\end{enumerate}
\end{solutions}
\begin{prb}{2015-A-II-5 (Algebra)}
	Find the splitting field and the Galois group of the polynomial $x^{4} - 5x^{2} + 5$ over $\mathbb{Q}$. 
\end{prb}
\begin{solutions}
	Let $p(x) = x^{4} - 5x^{2} + 5$. By the rational root test, $p(x)$ has no rational roots. Moreover, it is straightforward to see that $p(x)$ is not expressible as the product of irreducible quadratics. Hence, $p(x)$ is irreducible over $\mathbb{Q}$. We find its four complex roots as follows. Let $u = x^{2}$. Then 
		\begin{equation}
			u^{2} - 5u + 5 = 0 \implies u = \frac{5 \pm \sqrt{25 - 20}}{2} = \frac{5 \pm \sqrt{5}}{2} \implies x = \pm \sqrt{\frac{5 \pm \sqrt{5}}{2}}. 
		\end{equation}
	Let
		\begin{equation}
			\alpha \coloneqq \sqrt{\frac{5 + \sqrt{5}}{2}}, \qquad \beta \coloneqq \sqrt{\frac{5 - \sqrt{5}}{2}}. 
		\end{equation}
	We observe that 
		\begin{equation}
			\alpha^{2} = \frac{5}{2} + \frac{\sqrt{5}}{2} \qquad \text{ and } \qquad \alpha^{2}\beta^{2} = 5 \implies \beta = \pm \frac{5}{\alpha}. 
		\end{equation}
	Therefore, the splitting field is $\mathbb{Q}(\sqrt{5}, \alpha)$. Since the minimal polynomial of $\sqrt{5}$ over $\mathbb{Q}$ is $x^{2} - 5$, which has degree 2, $[\mathbb{Q}(\sqrt{5}):\mathbb{Q}] = 2$. On the other hand, the minimal polynomial of $\alpha$ over $\mathbb{Q}(\sqrt{5})$ is 
		\begin{equation}
			x^{2} - \frac{5 + \sqrt{5}}{2},
		\end{equation}
	so that $[\mathbb{Q}(\sqrt{5},\alpha):\mathbb{Q}] = 2$. Hence, by the Tower Law, $[\mathbb{Q}(\sqrt{5},\alpha): \mathbb{Q}] = 4$, which means that the corresponding Galois group has order 4; there are two groups, up to isomorphism, of order 4 (namely $\mathbb{Z}_{2} \times \mathbb{Z}_{2}$ and $\mathbb{Z}_{4}$). The elements of $\operatorname{Gal}(\mathbb{Q}(\sqrt{5}, \alpha)/\mathbb{Q})$ are precisely the automorphisms on $\mathbb{Q}(\sqrt{5}, \alpha)$ that fix $\mathbb{Q}$ such that the automorphism group acts transitively on the roots. Consider the permutation $\rho: \alpha \mapsto -\beta = -\frac{5}{\alpha}$ and $\rho: \sqrt{5} \mapsto -\sqrt{5}$. We observe that 
		\begin{align}
			\begin{split}
				\rho^{2}(\sqrt{5}) &= \rho(-\sqrt{5}) = \sqrt{5}. \\
				\rho^{2}(\alpha) &= \rho\left(-\frac{5}{\alpha}\right) = -5\rho(\alpha)^{-1} = \alpha  \\
				\implies \rho^{3}(\alpha) &= \rho(\alpha) = -5\alpha^{-1} \\
				\implies \rho^{4}(\alpha) &= -5\rho(\alpha)^{-1} = -5 \cdot \left(-\frac{\alpha}{5}\right) = \alpha. 
			\end{split}
		\end{align}
	I.e., $\rho$ is an element of order 4. Therefore, since only $\mathbb{Z}_{4}$ has an element of order 4, we conclude that $\operatorname{Gal}(\sqrt{5}, \alpha)/\mathbb{Q}) \cong \mathbb{Z}/4\mathbb{Z}$. 
\end{solutions}
\begin{prb}{2003-A-II-4 (Algebra)}
	Let $E$ be a splitting field of $f(x) = x^{3} + x^{2} - 2x - 1$ over the field of rational numbers $\mathbb{Q}$. Find the Galois group of $E/\mathbb{Q}$. (Hint: first prove that $f(x):f(x^{2} - 2)$.) \textcolor{orange}{This was the exact notation used in the problem...} 
\end{prb}
\begin{solutions}
	Let $f(x) = x^{3} + x^{2} - 2x - 1$. By the rational root test, $f(x)$ has no rational roots and hence is irreducible over $\mathbb{Q}$ (being a polynomial of degree 3). Consider the substitution $x = y - 1/3$: 
		\begin{equation}
			f(y) = y^{3} - \frac{7}{3}y - \frac{7}{27}. 
		\end{equation}
	The discriminant of this depressed cubic is 
		\begin{equation}
			D = -4p^{3} - 27q^{2} =4\left(\frac{7^{3}}{27}\right) - 27\left(\frac{7^{2}}{27^{2}}\right) = 7^{2}\left(\frac{28}{27} -\frac{1}{27}\right) = 7^{2}. 
		\end{equation}
	Since the discriminant is a perfect square, we conclude that the Galois group is $A_{3}$. 
\end{solutions}
\begin{prb}{2014-J-I-5 (Algebra)}
	Let $K$ denote the splitting field for $(x^{5} - 1)(x^{3} - 2)$ over the rational numbers $\mathbb{Q}$. Compute the cardinality of the Galois group $G$ for the extension $\mathbb{Q} \subset K$, and show that $G$ is not abelian. 
\end{prb}
\begin{solutions}
	Let $K$ denote the splitting field for $(x^{5} - 1)(x^{3} - 2)$. We note that the splitting field for $x^{3} - 2$ is $\mathbb{Q}(\sqrt[3](2), \zeta_{3})$, where $\zeta_{3}$ is the primitive 3rd root of unity, and the splitting field for $x^{5} - 1$ is $\mathbb{Q}(\zeta_{5})$, where $\zeta_{5}$ is the primitive 5th root of unity. Now, since 3 and 5 are relatively prime, the 3rd primitive roots of unity cannot be expressed as a linear combination of 5th roots of unity. Likewise, $\sqrt[3]{2} \notin \mathbb{Q}(\zeta_{5})$. Hence, $\mathbb{Q}(\sqrt[3]{2}, \zeta_{3}) \cap \mathbb{Q}(\zeta_{5}) = \mathbb{Q}$, which means that 
		\begin{equation}
			\operatorname{Gal}(K/\mathbb{Q}) \cong \operatorname{Gal}(\mathbb{Q}(\sqrt[3]{2},\zeta_{3})/\mathbb{Q}) \times \operatorname{Gal}(\mathbb{Q}(\zeta_{5})/\mathbb{Q}). 
		\end{equation}
	From this, we see that the order of $G$ is 24. Now consider $\operatorname{Gal}(\mathbb{Q}(\sqrt[3]{2},\zeta_{3})/\mathbb{Q})$. The corresponding minimal polynomial is $x^{3} - 2$, which is a depressed cubic. Since its discriminant is $-108$, which is not a square, we conclude that $\operatorname{Gal}(\mathbb{Q}(\sqrt[3]{2}, \zeta_{3})/\mathbb{Q}) \cong S_{3}$, which is not abelian. Hence, we conclude that $G$ is not abelian.  
\end{solutions}
\begin{prb}{2003-J-I-5 (Algebra)}
	Let $f(x) = x^{5} - 2$. Find generators and relations for the Galois group $G \coloneqq \operatorname{Gal}(F/\mathbb{Q})$ of the splitting field $F$ of $f(x)$ over the rational numbers $\mathbb{Q}$. 
\end{prb}
\begin{solutions}
	Let $f(x) = x^{5} - 2$, which has no roots in $\mathbb{Q}$ by the rational root test. It is also straightforward to check that $x^{5} - 2$ cannot be written as the product of an irreducible cubic and irreducible quadratic so that $f(x)$ is indeed irreducible over $\mathbb{Q}$. The roots of this polynomial are $\sqrt[5]{2}, \zeta_{5}\sqrt[5]{2}, \ldots, \zeta_{5}^{4}\sqrt[5]{2}$, where $\zeta_{5}$ is the primitive 5th root of unity. Therefore, the splitting field $F$ of $f(x)$ must contain the field $\mathbb{Q}(\sqrt[5]{2}, \zeta_{5})$. On the other hand, each of the roots mentioned above lie in this field so that $F = \mathbb{Q}(\sqrt[5]{2}, \zeta_{5})$. Moreover, it follows that $[F: \mathbb{Q}] = 5 \cdot 4 = 20$ so that $G$ is a group of order 20. \textcolor{red}{[!! Complete Later !!]}
\end{solutions}
\begin{prb}{2014-J-II-2 (Algebra)}
	Let $H$ denote a normal subgroup of the finite group $G$. If $P$ denotes a Sylow $p$-subgroup of $H$, then prove that $G= N(P;G)H$ (where $N(P;G)$ denotes the normalizer of $P$ in $G$). 
\end{prb}
\begin{solutions}
	Let $H$ denote a normal subgroup of the finite group $G$, and let $P$ be a Sylow $p$-subgroup of $H$. Let $N_{G}(P) \coloneqq N(P;G)$ denote the normalizer of $P$ in $G$. Since $H$ is normal, $KH \leq G$ for any $K \leq G$. In particular, $N_{G}(P)H \leq G$. Hence, it suffices to show that $G \leq N_{G}(P)H$. Since $P \leq H$ and $H$ is a normal subgroup of $G$, for any $g \in G$, $gPg^{-1} \leq gHg^{-1}= H$ so that $gPg^{-1}$ is another Sylow $p$-subgroup of $H$. On the other hand, we also know that all Sylow $p$-subgroups of $H$ are conjugate by elements of $H$. Hence, for each $g \in G$, there exists a corresponing $h \in H$ such that 
		\begin{equation}
			hPh^{-1} = gPg^{-1} \implies (g^{-1}h)P(g^{-1}h)^{-1} = P. 
		\end{equation}
	I.e, $g^{-1}h \in N_{G}(P)$, which means $g \in HN_{G}(P) = N_{G}(P)H$, where the equality stems from $N_{G}(P)H$ being a subgroup of $G$. Hence, since $g$ was arbitrary, $G \subseteq N_{G}(P)H$, which concludes the proof. 
\end{solutions}
\begin{prb}{2012-J-II-6 (Real Analysis)}
	Let $(V, \norm{\empspace})$ be a normed vector space. Assume that for every sequence $\{x_{n}\}_{1}^{\infty}$ in $V$ with $\sum_{1}^{\infty}\norm{x_{n}} < \infty$, the sequence of partial sums $\brac*{\sum_{1}^{N}x_{n}}_{1}^{\infty}$ is convergent in $V$. Prove that $V$ is complete. 
\end{prb}
\begin{solutions}
	Let $(V, \norm{\empspace})$ be a normed vector space so that every absolutely convergent series is convergent. Let $\{x_{n}\}_{1}^{\infty}$ be a Cauchy sequence in $V$. This means that we can find an increasing sequence of positive integers $n_{1} < n_{2} < \dotsm$ such that for each $j$ and $n,m > n_{j}$, 
		\begin{equation}
			\norm{x_{n} - x_{m}} < 2^{-j}. 
		\end{equation}
	Let $y_{1} = x_{n_{1}}$, and $y_{j} = x_{n_{j}} - x_{n_{j -1}}$ for each $j > 1$. Then $\sum_{1}^{k}y_{n_{j}} = x_{n_{k}}$. Then we observe that 
		\begin{equation}
			\sum_{1}^{\infty}\norm{y_{j}} \leq \norm{y_{1}} + \sum_{2}^{\infty}\norm{y_{j}} \leq \norm{y_{1}} + \frac{1}{2} < \infty. 
		\end{equation}
	Hence, by the hypothesis on $V$, $\sum_{1}^{N}y_{n}$ converges to $\sum_{1}^{\infty}y_{n}$. But this means that $\{x_{n_{j}}\}$ converges in $V$. Since $\{x_{n}\}$ is Cauchy, it follows that the sequence converges to the same limit. Hence, $V$ is complete. 
\end{solutions}
\begin{prb}{2006-A-II-4 (Complex Analysis)}
	Show that $z^{7} - 4z^{3} + z - 1$ has 3 zeros inside the unit circle (counted with multiplicity.)
\end{prb}
\begin{solutions}
	Let $f(z) = -4z^{3}$, and $g(z) = z^{7} + z - 1$, both of which are holomorphic functions. For all $\abs{z} = 1$, we observe that 
		\begin{align}
			\begin{split} 
				\abs{g(z)} &\leq |z|^{7} + |z| + 1 \\
				&= 1 + 1 + 1 = 3 \\
				&\leq 4 = 4|z|^{3} = |f(z)|. 
			\end{split}
		\end{align}
	Therefore, by Rouch{\'e}'s Theorem, $f(z)$ and $f(z) + g(z)$ have the same number of zeros inside the unit circle (counted with multiplicity). $f(z)$ has 3 zeros (being a degree 3 polynomial), and $f(z) + g(z) = z^{7} - 4z^{3} + z - 1$. Hence, this concludes the claim. 
\end{solutions}
\begin{prb}{2012-A-I-4 (Complex Analysis)}
	Find, with proof, the precise number of zeros of the complex polynomial $p(z) = z^{9} - 2z^{6} + z^{2} - 8z + 2$ inside the annulus $1 < \abs{z} < 2$. 
\end{prb}
\begin{solutions}
	Let $p(z) = z^{9} - 2z^{6} + z^{2} - 8z + 2$. The number of solutions to $p(z)$ inside the disk $\mathbb{D}_{2}$ of radius 2 must be the sum of the number of the number of solutions inside the unit disk $\mathbb{D}$ and the number of solutions in the annulus $1 < |z| < 2$. First, we compute the number of solutions inside $\mathbb{D}_{2}$. 
	
	Let $f(z) = z^{9}$, and $g(z) = -2z^{6} + z^{2} - 8z + 2$. Then on $\partial \mathbb{D}_{2}$, 
		\begin{align}
			\begin{split}
				|g(z)| &\leq 2|z|^{6} + |z|^{2} + 8|z| + 2 \\
				&= 2^{7} + 2^{2} + 16 + 2 = 2(3 + 8 + 2^{6}) \\
				&= 2(11 + 64) = 2(75) = 150 < 2^{9} = 8(64) = 512 = |f(z)|. 
			\end{split}
		\end{align}
	Hence, by Rouch{\'e}'s Theorem, since $p(z) = f(z) + g(z)$ and $f(z)$ has nine roots, $p(z)$ has nine roots inside $\mathbb{D}_{2}$. Now let $f(z) = -8z$ and $g(z) = z^{9} - 2z^{6} + z^{2} + 2$. On $\partial \mathbb{D}$, 
		\begin{align}
			\begin{split}
				\abs{g(z)} &\leq 1 + 2 + 1 + 2 = 6 \\
				&< 8 = |f(z)|. 
			\end{split}
		\end{align}
	Hence, by Rouch{\'e}'s Theorem, $p(z)$ has the same number of roots as $f(z)$ in $\mathbb{D}$, which is one. Therefore, we conclude that $p(z)$ has a total of eight solutions inside the annulus $1 < |z| < 2$. 
\end{solutions}
\begin{prb}{2015-A-I-3 (Complex Analysis)}
	Let $f$ be a holomorphic function defined on a neighborhood of the closed unit disk $\overline{\mathbb{D}} \coloneqq \brac*{z \in \mathbb{C}: |z| \leq 1}$, and assume that $|f(z)| < 1$ for all $|z| = 1$. Determine the number of fixed points of $f$ in $\overline{\mathbb{D}}$. 
\end{prb}
\begin{solutions}
	Let $f$ be a holomorphic function defined on a neighborhood of the closed unit disk $\overline{D}$ such that $|f(z)| < 1$ for all $|z| = 1$. If $z$ is a fixed point of $f$ in $\overline{D}$, then $z$ is a solution to the function $f(z) - z$. Hence, the number of fixed points of $f$ in $\overline{\mathbb{D}}$ is equal to the number of solutions to $f(z) - z$ in $\overline{\mathbb{D}}$. Since $|f(z)| < |z|$ for all $z \in \partial\overline{\mathbb{D}}$ and $z$ has exactly one root in $\overline{\mathbb{D}}$, we conclude that $f$ has exactly one fixed point in $\overline{\mathbb{D}}$.
\end{solutions}

\newpage 
\subsection{Other Qualifying Exams}
\begin{prb}{RUT-2023-A-I-1 (Algebra)}
	Classify the groups of order $2023 = 7 \cdot 17^{2}$ up to isomorphism. (You may use without proof the well-known result that if $p$ is a prime, then every group of order $p^{2}$ is abelian.)
\end{prb}
\begin{solutions}
	Let $G$ be a group of order $2023 = 7 \cdot 17^{2}$. By Sylow's Theorem, $G$ contains a normal Sylow 7-subgroup and a normal Sylow 17-subgroup. Let $H \cong \mathbb{Z}_{7}$ denote the Sylow 7-subgroup and $K$ denote the Sylow 17-subgroup; note that either $K \cong \mathbb{Z}_{17^{2}}$ or $K \cong \mathbb{Z}_{17} \times \mathbb{Z}_{17}$. Hence, $G \cong H \rtimes_{\varphi} K$, where $\varphi \in \operatorname{Aut}(H) \cong \mathbb{Z}_{7}^{\times} \cong \mathbb{Z}_{6}$. We consider various cases. 
		\begin{enumerate}[itemsep =-2pt,label = (\Roman{*})]
			\item Suppose $K = \mathbb{Z}_{17^{2}}$, which has a single generator, $1$. Each homomorphism $\varphi: K \to \mathbb{Z}_{6}$ is uniquely determined by where the generator $1$ is mapped to, with the constraint that $\varphi(1)$ is an element that divides the order of $1$, namely $17^{2}$. Since the only such element is $0$, $\varphi$ is the trivial homomorphism, which means that the semidirect product is just the direct product, and so $G \cong \mathbb{Z}_{7} \times \mathbb{Z}_{17^{2}} \cong \mathbb{Z}_{2023}$; this is an abelian group. 
			\item Suppose $K = \mathbb{Z}_{17} \times \mathbb{Z}_{17} = \braket{a} \times \braket{b}$. Each homomorphism $\psi: \mathbb{Z}_{17} \times \mathbb{Z}_{17} \to \mathbb{Z}_{6}$ is uniquely determined by $\psi(a)$ and $\psi(b)$ with the constraint that these elements divide the order of $a$ and $b$ in $\mathbb{Z}_{17}$, which is 17. Since there is only one such element, namely 0, we find that the semidirect is just the direct product, and $G \cong \mathbb{Z}_{7} \times \mathbb{Z}_{17} \times \mathbb{Z}_{17} \cong \mathbb{Z}_{289} \times \mathbb{Z}_{7}$; which is abelian. 
		\end{enumerate}
	Hence, up to isomorphism, there are exactly two groups of order 2023, both of which are abelian. 
\end{solutions}


\newpage 
\subsection{Classification of Finite Groups}
Some facts we will use to classify groups are: 
\begin{itemize}
	\item Every group of order $p^{2}$, where $p$ is abelian, is abelian. 
	\item Every group of order $p$, where $p$ is prime, is isomorphic to $\mathbb{Z}_{p}$. 
\end{itemize}
\begin{enumerate}[itemsep =-2pt,label = (\arabic{*})]
	\item $|G| = 1$: This is the trivial group $\{1\}$. 
	\item $|G| = 2$: There is exactly one group, up to ismorphism, which is $\mathbb{Z}/2\mathbb{Z}$. This follows from Cauchy's Theorem which states that if $p$ divides $|G|$, where $p$ is prime, then $G$ contains an element of order $p$. 
	\item $|G| = 3$: There is exactly one group, up to isomorphism, which is $\mathbb{Z}/3\mathbb{Z}$. This follows from Cauchy's Theorem, which states that if $p$ divides $|G|$, where $p$ is prime, then $G$ contains an element of order $p$. 
	\item $|G| = 4 = 2^{2}$: $\mathbb{Z}_{4}$ and $\mathbb{Z}_{2} \times \mathbb{Z}_{2}$ are both groups of order 4. Now let $G$ be an arbitrary group of order 4; by Lagrange's Theorem, each element of $G$ can have order 1, 2, or 4. Suppose $G$ contains an element $x$ of order 4. Then $G = \braket{g}$; let $\varphi: \mathbb{Z}_{4} \to G$ be the map given by $\varphi(n) \mapsto g^{n}$; this is easily seen to be a group isomorphism. Now suppose $G$ has no element of order 4. Since the only element of $G$ with order 1 is the identity (by uniqueness of group identities), the three nontrivial elements of $G$ must have order 2. Consider the map $\varphi: \mathbb{Z}_{2} \times \mathbb{Z}_{2} \to G$ defined as follows: 
		\begin{equation}
			\begin{matrix}
				\varphi(0,0) = 1_{G}, & \varphi(1, 0) = a, \\
				\varphi(0,1) = b, & \varphi(1,1) = c, 
			\end{matrix} 
		\end{equation}
	where $a, b,$ and $c$ are the three nonidentity elements of $G$; $\varphi$ is easily seen to be an isomorphism. Hence, $G \cong \mathbb{Z}_{2} \times \mathbb{Z}_{2}$. 
	\item $|G| = 5$: There is exactly one group, up to isomorphism, which is $\mathbb{Z}/5\mathbb{Z}$. 
	\item $|G| = 6 = 2 \cdot 3$. By Sylow's Theorem, there exists a normal Sylow 3-subgroup, which we denote by $H$. Let $K$ be a Sylow 2-subgroup. By Lagrange's Theorem, $H$ and $K$ intersect trivially and $|HK| = |H||K|/|H \cap K| = |H||K| = 6 = |G|$ so that $G = HK$. Hence, by the recognition theorem for semidirect products, $G \cong H \rtimes_{\varphi} K$, where $\varphi \in \operatorname{Aut}(H) \cong \mathbb{Z}_{3}^{\ast} \cong \mathbb{Z}_{2}$. Hence, we look for homomorphisms $\varphi: K \to \mathbb{Z}_{2}$. Since $K$ is a group of order 2, $K\cong \mathbb{Z}_{2}$. Homomorphisms $\varphi: \mathbb{Z}_{2} \to \mathbb{Z}_{2}$ are determined uniquely by where the generator $1$ is sent to with the constraint that $\varphi(1)$ divides the order of $1$, which is $2$. Hence, either $\varphi_{1}(1) = 0$ (in which case, the homomorphism is trivial, the semidirect is just the direct product, and $G$ is the abelian group $\mathbb{Z}_{3} \times \mathbb{Z}_{2} \cong \mathbb{Z}_{6}$), or $\varphi_{2}(1) = 1$ (in which case, the homomorphism is \textit{non}trivial, and $G$ is the nonabelian group $\mathbb{Z}_{3} \rtimes_{\varphi_{2}}\mathbb{Z}_{2}$). Hence, up to isomorphism, there are exactly two groups of order 6, one abelian and the other nonabelian. 
	\item $|G| = 7$: There is exactly one group, up to isomorphism, which is $\mathbb{Z}/7\mathbb{Z}$. 
	\item $|G| = 8 = 2^{3}$: \textcolor{red}{[!! Complete Later !!]}
	\item $|G| = 9 = 3^{2}$: Every group of order $p^{2}$ abelian. So by the Fundamental Theorem for Finitely Generated Abelian Groups, $G \cong\mathbb{Z}_{9}$ or $\mathbb{G} \cong\mathbb{Z}_{3} \times \mathbb{Z}_{3}$. 
	\item $|G| = 10 = 2 \cdot 5$: By Sylow's Theorem, $G$ contains a normal Sylow 5-subgroup, which we denote by $H$. Let $K$ be a Sylow 10-subgroup. Then by Lagrange's Theorem, $H \cap K = \{e\}$ and $G = HK$. Therefore, $G \cong H \rtimes_{\varphi} K$, where $\varphi \in \operatorname{Aut}{H} \cong \mathbb{Z}_{5}^{\ast} \cong \mathbb{Z}_{4}$. Hence, we look for homomorphisms $\varphi: K \to \mathbb{Z}_{4}$. Since $K$ is a group of order 2, $K \cong \mathbb{Z}_{2}$. Homomorphisms $\varphi: \mathbb{Z}_{2} \to \mathbb{Z}_{4}$ are determined uniquely by where the generator $a$ is set to with the constraint that $\varphi(a)$ divides the order of $a$, which is $2$. The only such elements in $\mathbb{Z}_{4}$ are $0$ and $2$. If $\varphi_{1}: 1 \mapsto 0$, then $\varphi_{1}$ is just the trivial homomorphism which means that the semidirect is just a direct product and $G$ is isomorphic to the abelian group $\mathbb{Z}_{2} \times \mathbb{Z}_{10} \cong \mathbb{Z}_{10}$. If $\varphi_{2}: 1 \mapsto 2$, then $\varphi$ is a nontrivial homorphism, which means that $G$ is the nonabelian group $\mathbb{Z}_{5} \rtimes_{\varphi_{2}} \mathbb{Z}_{2}$. Hence, up to isomorphism, there are exactly 2 groups of order 10, only one of which is abelian. 
	\item $|G| = 11$: There is exactly one group of order 11, namely $\mathbb{Z}/11\mathbb{Z}$. 
	\item $|G| = 12 = 2^{2} \cdot 3$: By Sylow's Theorem, 
		\begin{align}
			\begin{split}
				n_{3} &\in \{1, 2, 4\} \cap \{1, 4, \ldots\} = \{1, 4\}. \\
				n_{2} &\in \{1, 3\} \cap \{1, 3, \ldots\} = \{1, 3\}. 
			\end{split}
		\end{align}
	Suppose $n_{3} = $. 
\end{enumerate}


\newpage 
\subsection{Essential Review Notes}
\subsubsection{Topological Vector Spaces}
\begin{itemize}
	\item \textbf{Def. (Topological Vector Space)} A vector space $\mathscr{X}$ over a field $K$ such that vector addition in $\mathscr{X}$ and scalar multiplication are continuous maps from $\mathscr{X} \times \mathscr{X}$ and $K \times \mathscr{X}$, respectively, to $\mathscr{X}$. 
	\item \textbf{Def. (Weak Convergence)} A sequence $\{x_{n}\}$ in a normed linear space $\mathscr{X}$ \textit{converges weakly} to $x \in X$ if the sequence of scalars $\{f(x_{n})\}$ converges to $f(x)$ for all $f \in \mathscr{X}^{\ast}$. 
	\item \textbf{Def. (Weak$^{\ast}$ Convergence)} Let $\mathscr{X}$ be a normed linear space. A sequence $\{f_{n}\} \subseteq \mathscr{X}^{\ast}$ is \textit{weak$^{\ast}$ convergent} to $f \in \mathscr{X}^{\ast}$ if $\{f_{n}(x)\}$ converges to $f(x)$ for all $x \in \mathscr{X}$. Note, all this really says that the sequence of scalars $\{\hat{x}(f_{n})\} = \{f_{n}(x)\}$ converges to $\hat{x}(f) = f(x)$ for all $\hat{x} \in \mathscr{X}^{\ast\ast}$ (read $x \in \mathscr{X}$). 
\end{itemize}

\end{document}