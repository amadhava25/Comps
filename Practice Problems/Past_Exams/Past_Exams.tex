\documentclass{article}
\usepackage{../../header}
\usepackage{notomath}
\setcounter{secnumdepth}{0}
\title{Comps Practice}
\author{\me}
\date{January 8, 2026}

\begin{document}
	\maketitle
	\fullline 
	\tableofcontents
	\halfline 
	\newpage 

\subsection{Comps Lemma}
\begin{prb}{Comps Lemma}
	Let $M, N$ be smooth, connected, $n$-manifolds, and $f: M \to N$ a (smooth) immersion. If $M$ is compact and nonempty, then $N$ is compact and $f$ is a (smooth) covering map. 
\end{prb}
\begin{solutions}
	Let $M, N$ be smooth, connected $n$-manifolds and $f: M \to N$ an immersion. Assume that $M$ is compact and nonempty. Since $\operatorname{dim}{N} = n$ and $f$ is an immersion, $\operatorname{rank}{df_{p}} = n$ at every $p \in M$. Hence, by the Inverse Function Theorem, $f$ is a local diffeomorphism. Since local diffeomorphisms are open maps, $f(M)$ is open in $N$. On the other hand, since the continuous image of compact sets is compact, $f(M)$ is compact in $N$. Since $N$ is Hausdorff, $f(M)$ is closed in $N$. Since $N$ is connected, $f(M) = N$. Therefore, $N$ is compact. 
	
	Now, let $q \in N$, and consider $f^{-1}(q) \subset M$. For each $x \in f^{-1}(q)$, let $U_{x}$ be an open neighborhood of $M$ containing $x$. Since $M$ is Hausdorff, we can shrink each $U_{x}$ so that these neighborhoods are pairwise disjoint. This means that each $x \in f^{-1}(q)$ is isolated, and hence $f^{-1}(q)$ is discrete. Since $M$ is compact, we conclude that $f^{-1}(q)$ must be finite; let $f^{-1}(q) = \{x_{1}, \ldots, x_{s}\}$. As noted above, for each $j = 1, \ldots, s$, let $U_{j}$ be a neighborhood of $x_{j}$ such that $f|_{U_{j}}: U_{j} \to V_{j} \subset N$ is a diffeomorphism. Then by the Hausdorff condition on $M$, shrink each $U_{j}$ so that $U_{i}\cap U_{j} = \varnothing$ for all $i \neq j$; $f$ remains a diffeomorphism on these shrunken neighborhoods. Setting $V = \bigcap_{1}^{s}f(U_{j})$ and taking $\ti{U}_{j} = f^{-1}(V) \cap U_{j}$ gives us an evenly covered neighborhood of $q$ in $N$. 
\end{solutions}

\begin{prb}{(Comps Lemma - Local Homeomorphisms)}
	Let $M, N$ be smooth, connected $n$-manifolds and $f: M \to N$ a local homeomorphism. If $M$ is compact and nonempty, then $N$ is compact and $f$ is a covering map. 
\end{prb}

\begin{prb}{(Comps Lemma - Submersions)}
	Let $M, N$ be smooth, connected $n$-manifolds and $F: M \to N$ a submersion. If $M$ is compact and nonempty, then $N$ is compact and $F$ is a covering map. 
\end{prb}
\begin{solutions}
	Let $M, N$ be smooth, connected $n$-manifolds and $F:M \to N$ a submersion. Also assume $M$ is compact and nonempty. Since submersions are open maps, $f(M)$ is open in $N$. On the other hand, since $F$ is continuous, continuous images of compacts sets are compact, and compact subsets of Hausdorff spaces are closed, $F(M)$ is closed in $N$. Hence, since $N$ is connected and $F(M)$ is nonempty, $F(M) = N$. This proves that $N$ is compact. We also claim that $F$ is a local diffeomorphism. Since $F$ is a submersion, at every $p \in M$, $dF_{p}: T_{p}M \to T_{f(p)}N$ is surjective. Since $\dim{M}= \dim{N} = n$, it follows that $dF_{p}$ is bijective. Hence, by the Inverse Function Theorem, $F$ is a local diffeomorphism. 
	
	All that remains to be seen is that $F$ is a covering map. Let $q \in N$ and consider the closed subset $F^{-1}(q) \subset M$. Since $F$ is a local diffeomorphism, for each $x \in F^{-1}(q)$, there exists a neighborhood $U_{x}$ such that $F|_{U_{x}}$ is a local diffeomorphism. Since $M$ is Hausdorff, we may shrink these neighborhoods so that they are pairwise disjoint. This means that each $x \in F^{-1}(q)$ is isolated, and hence, $f^{-1}(q)$ is discrete. Since $M$ is compact, $f^{-1}(q)$ is finite; let $f^{-1}(q) = \{x_{1},\ldots, x_{s}\}$. For each $j = 1, \ldots, s$, let $U_{j}$ be a neighborhood of $x_{j}$ such that $F|_{U_{j}}$ is a diffeomorphism. Since $M$ is Hausdorff, we shrink these neighborhoods such that they are pairwise disjoint; $F$ remains a diffeomorphism on each shrunken $U_{j}$. Set $V = \bigcap_{1}^{s}f(U_{j})$, and let $\ti{U}_{j} = f^{-1}(V) \cap U_{j}$. Hence, $V$ is an evenly covered neighborhood of $q \in N$, which concludes the proof that $F$ is a covering map. 
\end{solutions}

\subsection{Steinhaus Theorem}
\begin{prb}{(Steinhaus Theorem)}
	Let $E$ be a Lebesgue measurable subset of $\mathbb{R}^{n}$ such that $m^{n}(E) > 0$, and let $v_{1}, \ldots, v_{N}$ be a finite collection of vectors in $\mathbb{R}^{n}$. Then there exists $R > 0$, depending on $E$, and $M = \max\{|v_{1}|, \ldots, |v_{N}|\}$ such that for all $0 < r < R$, there exists $p \in S$ so that the $(N + 1)$-points, $p, p + rv_{1}, \ldots, p + rv_{1} + \dotsm + rv_{n}\in S$. 
\end{prb}
\begin{solutions}
	Let $E$ be a measurable subset of $\mathbb{R}^{n}$ with positive Lebesgue measure. We recall that the Lebesgue measure is \textit{regular} (which means it is both \textit{inner} and \textit{outer} regular). By inner regularity, there exists a compact set $K_{1} \subset E$ such that $m^{n}(K_{1}) > 0$. Let $\beta < (2^{N} - 1)^{-1}$; by outer regularity, there exists an open set $U$ containing $K_{1}$ such that 
		\begin{equation}
			m^{n}(U) \leq (1 + \beta)m^{n}(K_{1}). 
		\end{equation}
	Since $K_{1}$ is compact, $d_{1} = d(K_{1}, U^{c}) > 0$. Let $R = d_{1}/M$, and choose an arbitrary $r$ such that $0 < r < R$. First, observe that the set $K_{1} + rv_{1}$ is contained in $U$, since otherwise, 
		\begin{equation}
			d(K_{1}, U^{c}) \leq |rv_{1}| \leq rM < d_{1}. 
		\end{equation}
	Therefore, $K_{1} \cup (K_{1} + rv_{1}) \subset U$, and so 
		\begin{equation}
			m^{n}(U) \geq m^{n}(K_{1}\cup (K_{1} + rv_{1})) = m^{n}(K_{1}) + m^{n}(K_{1} + rv_{1}) - m^{n}(K_{1} \cap (K_{1} + rv_{1})). 
		\end{equation}
	Since the Lebesgue measure is translation invariant, 
		\begin{equation}
			m^{n}(K_{1} \cap (K_{1} + rv_{1})) \geq 2m^{n}(K_{1}) - m^{n}(U) \geq 2m^{n}(K_{1}) - m^{n}(K_{1}) - \beta m^{n}(K_{1}) = (1 - \beta)m^{n}(K_{1}). 
		\end{equation}
	Since $\beta < 1$, it follows that $m^{n}(K_{1} \cap (K_{1} + rv_{1})) > 0$, and so $K_{1} \cap (K_{1} + rv_{1}) \neq \varnothing$. Now we proceed by induction. For each $i = 1, \ldots, N$, let $K_{i + 1} = K_{i} \cap (K_{i} + rv_{i})$. Each $K_{i} + rv_{i}$ must be contained in $U$ (by a generalization of the argument made above) and each $K_{i + 1} \subset K_{i} \subset U$. We claim that for each $i$, $m^{n}(K_{i + 1}) \geq (1- (2^{i} - 1)\beta)m^{n}(K_{1})$. We have already proven the base case $i = 1$. So assume the result holds for some $1 \leq m < N$. Then 
		\begin{equation}
			m^{n}(U) \geq m^{n}(K_{i} \cup (K_{i} + rv_{i})) = m^{n}(K_{i}) + m^{n}(K_{i} + rv_{i}) - m^{n}(K_{i} \cap (K_{i} + rv_{i})). 
		\end{equation}
	By translation invariance of the Lebesgue measure, 
		\begin{align}
			\begin{split}
			m^{n}(K_{i + 1}) = m^{n}(K_{i} + rv_{i}) &\geq 2m^{n}(K_{i}) - m^{n}(U) \geq 2(1 - (2^{i} - 1)\beta)m^{n}(K_{1}) - (1 + \beta)m^{n}(K_{1}) \\
			&= m^{n}(K_{1}) - 2^{i + 1}\beta m^{n}(K_{1}) + 2 \beta m^{n}(K_{1}) - \beta m^{n}(K_{1}) \\
			&= (1 - (2^{i + 1} - 1)\beta)m^{n}(K_{1}). 
			\end{split} 
		\end{align}
	Hence, since $\beta < (2^{N} - 1)^{-1}$, we obtain a nested sequence of compact subsets $\varnothing \neq K_{N + 1} \subset K_{N} \subset \dotsm\subset K_{1} \subset U$. Let $q \in K_{N + 1}$ be arbitrary. Since $K_{N + 1} = K_{N} \cap (K_{N} + rv_{N})$, the point $q - rv_{N}$ is contained in $K_{N}$. Then since $K_{N} = K_{N - 1} \cap (K_{N - 1} \cap rv_{N - 1})$, $q - rv_{N} - rv_{N - 1} \in K_{N - 1}$. Proceeding inductively, we obtain the sequence $\{q, q - rv_{N}, q - rv_{N} - rv_{N - 1}, \ldots, q - rv_{N} - \dotsm -rv_{1}\} \subset K_{1} \subset E$. Hence, the proof concludes. 
	
\end{solutions}

\newpage 
\subsection{January 2025}
\begin{prb}{2025-J-I-1 (Algebra)}
Let $R$ be a UFD (unique factorization domain). Let $F$ be its quotient field. Let $p(x) = x^{n} + b_{n - 1}x^{n- 1} + \dotsm + b_{0} \in F[x]$ be a monic polynomial with coefficients in $R$ admitting a root $a \in F$. Prove that $a \in R$. 
\end{prb}
\begin{solutions}
	Let $R$ be a UFD, and $F$ its quotient field. Let $p(x) = x^{n} + b_{n- 1}x^{n - 1} + \dotsm + b_{0} \in F[x]$ be a monic polynomial with coefficients in $R$ admitting a root $a \in F$. Let $a = c/d$, where $c, d \in R\setminus \{0\}$ so that $\gcd(c, d) = 1$. By definition of a root, we must have 
		\begin{equation}
			0 = p(a) = \left(\frac{c}{d}\right)^{n} + b_{n - 1}\left(\frac{c}{d}\right)^{n - 1} + \dotsm + b_{0}. 
		\end{equation}
	Multiplying both sides by $d^{n}$, 
		\begin{equation}
			c^{n} + d(b_{n-1}c^{n-1} + b_{n - 2}c^{n - 2}d \dotsm + b_{0}d^{n-1}) = 0 \implies c^{n} = -d(b_{n - 1}c^{n - 1} + \dotsm + b_{0}d^{n-1}). 
		\end{equation}
	From this, we observe that $d \mid c^{n}$. If $d$ is not a unit in $R$, then every nonidentity irreducible divisor of $d$ is an irreducible divisor of $c^{n}$, and hence an irreducible divisor of $c$. But this contradicts our hypothesis that $\gcd(c, d) = 1$. Hence, $d$ has to be a unit of $R$. If $v \in R\setminus \{0\}$ such that $dv= vd = 1$, then 
		\begin{equation}
			a = \frac{c}{d} = \frac{c}{d}\cdot \frac{v}{v} = cv \in R. 
		\end{equation}
	Hence, this concludes the proof. 
\end{solutions}
\begin{prb}{2025-J-I-2 (Real Analysis)}
	Let $\{f_{n}\}_{n\geq 1}$ be a sequence of Lebesgue-measurable functions on $[0,1]$. Suppose that 
		\begin{equation*}
			\int_{0}^{1}f^{2}_{n}dm \leq \frac{1}{n^{2}} \quad \text{ for all $n\geq 1$}. 
		\end{equation*}
	Prove that $f_{n}$ converges to $0$ a.e. on $[0,1]$. 
\end{prb}
\begin{solutions}
	Let $\{f_{n}\}_{n\geq 1}$ be a sequence of Lebesgue-measurable functions on $[0,1]$ so that 
		\begin{equation}
			\int_{0}^{1}f_{n}^{2}dm \leq \frac{1}{n^{2}} \quad \text{ for all } n \geq 1. 
		\end{equation}
	Consider the sequence $\brac*{\sum_{1}^{m}f_{n}^{2}}$, which is increasing and converges a.e. to $\sum_{1}^{\infty}f_{n}^{2}$. Hence, by the Monotone Convergence Theorem, 
		\begin{equation}
			\sum_{1}^{\infty}\int_{0}^{1} f_{n}^{2} = \lim_{m \to \infty}\sum_{1}^{m}\int_{0}^{1} f_{n}^{2} = \lim_{m \to \infty}\int_{0}^{1} \sum_{1}^{m}f_{n}^{2} = \int_{0}^{1}\sum_{1}^{\infty}f_{n}^{2} \leq \int_{0}^{1}\sum_{1}^{\infty}\frac{1}{n^{2}} < \infty. 
		\end{equation}
	Therefore, $\sum_{1}^{\infty}f_{n}^{2} \in L^{1}(\mathbb{R})$, which means that $\sum_{1}^{\infty}f_{n}^{2} < \infty$ a.e. on $[0,1]$. Hence, $\sum_{n = 1}^{\infty}f_{n}^{2}$ converges a.e. on $[0,1]$. This implies that $f_{n}^{2} \to 0$ a.e. on $[0,1]$, and hence $f_{n} \to 0$ a.e. on $[0,1]$. 
\end{solutions}
\begin{prb}{2025-J-I-3 (Geometry/Topology)}
	Let $M$ be an orientable, connected, and compact smooth $n$-manifold with boundary. Show that there is no (smooth) retraction to the boundary, that is, there does not exist a smooth map $f: M \to \partial M$ such that $f(x) = x$ when $x \in \partial M$. 
\end{prb}
\begin{solutions}
	Let $M$ be an orientable, connected, and compact smooth $n$-manifold with boundary. Assume to the contrary that there exists a smooth map $f: M \to \partial M$ such that $f(x) = x$ when $x \in \partial M$. Let $\omega \in \Omega^{n- 1}(\partial M)$ be a volume form for the boundary of $M$. Since volume forms are closed (hence, $\omega$ is closed), we have by Stokes's theorem 
		\begin{equation}
			0 = \int_{M}f^{\ast}d\omega = \int_{M}d(f^{\ast}\omega) = \int_{\partial M}f^{\ast}\omega = \int_{\partial M}\omega > 0, 
		\end{equation}
	which is a contradiction. Hence, by contradiction, there cannot exist a smooth retraction to the boundary. 
\end{solutions}
\begin{prb}{2025-J-II-3 (Algebra)}
	Let $V$ be a vector space of dimension $n$ over $\mathbb{Q}$. Let $T: V \to V$ be a linear transformation with minimal polynomial $x^{4} - x^{2} - 2$ over $\mathbb{Q}$. Show that $n$ must be even. 
\end{prb}
\begin{solutions}
	Consider $V$ as a module over the ring $\mathbb{Q}[x]$ by letting a polynomial $f(x) \in \mathbb{Q}[x]$ act as the linear operator $f(T)$. Since $\dim{V} = n$, this module is finitely generated. By the structure theorem for finitely generated modules over principal ideal domains, $V$ is isomorphic to a direct sum of modules of the form $\mathbb{Q}[x]/(p(x))^{e}$, where $p(x) \in \mathbb{Q}[x]$ is irreducible. Moreover, each $p(x)$ must divide the minimal polynomial of $T$. We note that over $\mathbb{Q}$, 
		\begin{equation}
			x^{4} - x^{2} - 2 = (x^{2} - 2)(x^{2} + 1),
		\end{equation}
	where both factors are irreducible over $\mathbb{Q}$. Therefore, the only choices for $p(x)$ are $x^{2} - 2$ and $x^{2} + 1$. Therefore, $\mathbb{Q}[x]/(p(x))^{e}$ has dimension $\deg{p} \cdot e = 2e$ for each choice of $p$. Since $2$ divides these dimensions, we conclude that $2$ must divide $n$. Hence, $n$ is even. 
\end{solutions}
\begin{prb}{2025-J-II-4 (Topology)}
	Let $\Sigma_{2}$ be a compact oriented surface of genus 2. Is there a submersion $f: \Sigma_{2} \to S^{1} \times S^{1}$, where $S^{1}$ denotes the unit circle? 
\end{prb}
\begin{solutions}
	Assume to the contrary that there exists a submersion $f: \Sigma_{2} \to S^{1} \times S^{1}$, where $S^{1}$ denotes the unit circle. Since $\dim{\Sigma_{2}} = \dim{S^{1} \times S^{1}} = 2$, $df_{p}$ must have constant rank 2 at every $p \in \Sigma_{2}$. Hence, $f$ is a local diffeomorphism. Since $f$ is a local diffeomorphism, $f(\Sigma_{2})$ is compact in $S^{1} \times S^{1}$; since $S^{1} \times S^{1}$ is Hausdorff, $f(\Sigma_{2})$ must be closed in $S^{1} \times S^{1}$. On the other hand, since local diffeomorphisms are open maps, $f(\Sigma_{2})$ is open in $S^{1} \times S^{1}$. Therefore, since $S^{1} \times S^{1}$ is connected, $f(\Sigma_{2}) = S^{1} \times S^{1}$; i.e., $f$ is surjective. Therefore, $f$ is a covering map. This means that the induced homomorphism, $f_{\ast}: \pi_{1}(\Sigma_{2}) \to \pi_{1}(S^{1}\times S^{1})$ is injective, and so $f_{\ast}(\pi_{1}(\Sigma_{2})) \cong \operatorname{img}{f_{\ast}} \leq \pi_{1}(S^{1} \times S^{1})$. However, $\pi_{1}(S^{1} \times S^{1}) \cong \mathbb{Z} \times\mathbb{Z}$ is an abelian group and cannot have any nonabelian subgroups, whereas $\pi_{1}(\Sigma_{2})$ is nonabelian. Hence, by contradiction, $f$ cannot be a submersion. 
\end{solutions}
\begin{prb}{2025-J-II-5 (Analysis)}
	Let $V$ be a topological vector space whose topology is Hausdorff. Let $X_{1}$ and $X_{2}$ be two Banach spaces, and assume there exist continuous linear bijections $F_{1}: X_{1} \to V$ and $F_{2}: X_{2} \to V$. Show that there is a continuous linear bijection $G: X_{1} \to X_{2}$. 
\end{prb}
\begin{solutions}
	Assume the given hypotheses. Let $G = F_{2}^{-1} \circ F_{2}$. Since $F_{1}, F_{2}$ are bijections, we conclude that $G$ is a bijection. Likewise, since $F_{1}, F_{2}$ are linear, $G$ must also be linear. It suffices to prove that $G$ is continuous. By the Closed Graph Theorem, continuity of $G$ is equivalent to the graph of $G$ being a closed subspace of $X_{1} \times X_{2}$. Let $\{x_{n}\} \subset X_{1}$ be a sequence in $X_{1}$ such that $x_{n} \to x$ and $y_{n} = Gx_{n} \to y$. We need to show that $y = Gx$. By continuity of $F_{1}$, $F_{1}x_{n} \to Fx$. By continuity of $F_{2}$, 
		\begin{equation}
			F_{2}y = \lim F_{2}y_{n} = \lim F_{2}Gx_{n} = \lim F_{1}x_{n} = F_{1}x. 
		\end{equation}
	Since $F_{2}$ is bijective, $y = F_{2}^{-1}F_{1}x = Gx$. Hence, the graph of $G$ is closed, which implies that $G$ is continuous. 
\end{solutions}

\subsection{August 2025}
\begin{prb}{2025-A-I-1 (Geometry/Topology)}
	Let $S$ be a closed orientable surface of genus 4 and $C$ be an embedded circle that partitions $S$ into two subsurfaces of genus 2. Does $S$ retract to $C$? 
\end{prb}
\begin{solutions}
	We claim that the answer is no; assume to the contrary that there exists a retraction $r: S \to C$. Let $i: C \hookrightarrow S$ be the inclusion map so that $r \circ i = \operatorname{id}_{C}$. Now since $C$ is an embedded circle, $H_{1}(C)$ (i.e., the first homology) is isomorphic to $H_{1}(S^{1}) = \mathbb{Z}$. On the other hand, since $C$ is separating in $S$, its homology class in $H_{1}(S)$ is the zero element. Hence, the induced map $i_{\ast}: H_{1}(C) \to H_{1}(S)$ is the zero map. But this is impossible since if $i_{\ast}$ is the zero map,
		\begin{equation}
			0 = r_{\ast} \circ i_{\ast} = (r \circ i)_{\ast} = \operatorname{id}_{H_{1}}(C),
		\end{equation}
	which is a contradiction. Hence, no such retraction can exist. 
\end{solutions}

\begin{prb}{2025-A-I-6 (Algebra)}
	Let $f(x)$ be an irreducible polynomial of degree $n$ over a field $F$, and let $g(x)$ be any polynomial in $F[x]$. Prove that every irreducible factor of the composition $f(g(x))$ has degree divisible by $n$. 
\end{prb}
\begin{solutions}
	Let $h(x)$ be an irreducible factor of $f(g(x))$ in $F[x]$ and let $\alpha$ be the root of $h(x)$ in some algebraic closure of $F$. Since $h$ is irreducible and $\alpha$ is a root, the minimum polynomial of $\alpha$ over $F$ is $h$. Therefore, 
		\begin{equation}
			\deg{h} = [F(\alpha): F]. 
		\end{equation}
	Now since $\alpha$ is a root of $h(x)= f(g(x))$, $f(g(\alpha)) = 0$. In particular, $g(\alpha)$ is a root of $f$. Since $f$ is irreducible of degree $n$ over $F$, the minimal polynomial of $g(\alpha)$ over $\alpha$ is $f$. Hence, 
		\begin{equation}
			[F(g(\alpha)):F] = n. 
		\end{equation}
	Since $F \subset F(g(\alpha)) \subset F(\alpha)$, by the Tower Law, 
		\begin{equation}
			\deg{h} = [F:(\alpha):F] = [F(\alpha):F(g(\alpha))] \cdot [F(g(\alpha)): F] = n[F(\alpha):F(g(\alpha))],
		\end{equation}
	so that $n \mid \deg{h}$. Hence, this concludes the proof. 
\end{solutions}
\begin{prb}{2025-A-II-2 (Geometry/Topology)}
	Consider the plane distribution in $\mathbb{R}^{3}$ spanned by two vector fields 
		\begin{equation}
			V = \partial_{x} + 2xy\partial_{z}, \qquad W = x\partial_{x} + \partial_{y} + (2x^{2}y + x^{2} - 2y)\partial_{z}. 
		\end{equation}
	\begin{enumerate}[itemsep =-2pt,label = (\roman{*})]
		\item Show that this distribution is integrable. 
		\item Does the pair of vector fields $V$ and $W$ generate a coordinate system on integral surfaces? If not, find a pair that can play this role for the local integral surfaces passing through points $(0,0,z_{0})$. 
	\end{enumerate}
\end{prb}
\begin{solutions}
	$ $\newline \vspace{-0.65cm}
	\begin{enumerate}[itemsep =-2pt,label = (\roman{*})]
		\item Let $D$ be the plane distribution in $\mathbb{R}^{3}$ spanned by the two vector fields $V$ and $W$ given above. Then by the Frobenius Theorem, $D$ is integrable if and only if $D$ is involutive, which is true if and only if the Lie Bracket of $V$ and $W$ is a smooth section of $D$ at each $p \in \mathbb{R}^{3}$. We observe that: 
			\begin{align}
				\begin{split}
					V(W) &= \left(\partial_{x} + 2xy\partial_{z}\right)(x\partial_{x} + \partial_{y} + (2x^{2}y + x^{2} - 2y)\partial_{z}) \\
					&= \partial_{x} + (4xy + 2x)\partial_{z}. \\
					W(V) &= \left(x\partial_{x} + \partial_{y} + (2x^{2}y + x^{2} - 2y)\partial_{z}\right)(\partial_{x}+ 2xy\partial_{z}) \\
					&= 2xy\partial_{z} + 2x\partial_{z}. 
				\end{split}
			\end{align}
		Therefore, for any $p \in \mathbb{R}^{3}$, 
			\begin{equation}
				[V, W] = V(W) - W(V) = \partial_{x} + 2xy\partial_{z} = V. 
			\end{equation}
		Since $V$ is a smooth section of $D$, we conclude that $D$ is involutive, and hence integrable. 
		\item Let $\mathscr{S}$ be an integral surface, and assume there are coordinates $(u, v)$ on $\mathscr{S}$ such that $V|_{\mathscr{S}} = \partial_{u}$ and $W|_{\mathscr{S}} = \partial_{v}$. Then we observe that $[V|_{\mathscr{S}}, W|_{\mathscr{S}}] = \partial_{u}(\partial_{v}) - \partial_{v}(\partial_{u}) = 0$. On the other hand, 
			\begin{equation}
				[V|_{\mathscr{S}}, W|_{\mathscr{S}}] = ([V, W])|_{\mathscr{S}} = V|_{\mathscr{S}} \neq 0, 
			\end{equation}
		which is a contradiction. Therefore, $V$ and $W$ cannot generate a coordinate system on integral surfaces. However, consider the fields $\ti{V} = V$ and $\ti{W} = W - xV$ on $\mathbb{R}^{3}$. Then since
			\begin{equation}
				[\ti{V}, \ti{W}] = V(W - xV) - (W - xV)(V) = VW - xVV - W(V) + xVV = 0, 
			\end{equation} 
		and so this pair generates a coordinate system on all integral surfaces. 
	\end{enumerate}
\end{solutions}


\subsection{January 2024}
\begin{prb}{2024-J-I-1 (Algebra)}
	For distinct odd primes $p$ and $q$, prove that every finite group of order $2pq$ is a semidirect product of a normal subgroup of order $pq$ and a subgroup of order 2. 
\end{prb}
\begin{solutions}
	Let $G$ be a group of order $2pq$, where $p, q$ are distinct odd primes. Without loss of generality, assume $q > p$. By Sylow's Theorem, 
		\begin{equation}
			n_{q} \in \{1, 2, p, 2p\} \cap \{1, q + 1, \ldots\} = 1, 
		\end{equation}
	since $q > 2$ and $q > p$. Therefore, $G$ has a unique, normal, Sylow $q$-subgroup, which we denote as $Q$. Let $P$ be a Sylow $p$-subgroup of $G$. By the Second Isomorphism Theorem, we conclude that $N = PQ$ is a subgroup of $G$ of order $|P||Q| = pq$. Since $|G: N| = 2pq/(pq) = 2$, where $2$ is the smallest prime dividing $|G|$, we conclude that $N$ is a normal subgroup of $G$. Next, by Cauchy's Theorem, $G$ contains an element of order 2. Let $M$ be the subgroup generated by this element, which also must have order 2. By Lagrange's Theorem, $N \cap M = \{e\}$. Next, 
		\begin{equation}
			|NM| = \frac{|N||M|}{|N \cap M|} = |N||M| = 2pq = |G|,
		\end{equation}
	so that $G = NM$. Therefore, we conclude that $G = N \rtimes M$. 
\end{solutions}
\begin{prb}{2024-J-I-2 (Geometry/Topology)}
	Let $p: E \to B$ be a covering space map, with $B$ and $E$ path connected. Choose a point $e_{0} \in E$ and $b_{0} \in B$ such that $p(e_{0}) = b_{0}$. This gives us a subgroup $H = p_{\ast}\pi_{1}(E, e_{0})$ of the fundamental group $G = \pi_{1}(B, b_{0})$. 
	
	Construct a bijection between the fiber $p^{-1}(b_{0})$ and the set of right cosets of $H$ and prove that this is indeed a bijection. Prove that the number of sheets of $p$ equals the index $(G: H)$. 
\end{prb}
\begin{solutions}
	Assume all of the given hypotheses. Let $\phi: \pi_{1}(B, b_{0}) \to p^{-1}(b_{0})$ be the lifting correspondence induced by $p$ defined by $\phi([f]) = \tilde{f}(1)$, where $\tilde{f}$ is the lift of $f$, and let $\Phi: \pi_{1}(B, b_{0})/H \to p^{-1}(b_{0})$ be the map induced by $\phi$. It suffices to prove that $\Phi$ is a bijection. 
		\begin{enumerate}[itemsep =-2pt,label = (\roman{*})]
			\item Since $E$ is path connected and $p: E \to B$ is a covering map, the lifting correspondence $\phi$ must be surjective. Hence, since $\Phi$ is induced by $\phi$, it follows that $\Phi$ is also surjective. 
			\item Now we will show that $\Phi$ is injective. Let $f$ and $g$ be two paths in $B$, and $\ti{f}, \ti{g}$ their liftings to paths in $E$ that begin at $e_{0}$. We must show that $\ti{f}(1) = \ti{g}(1)$ iff $[f] \in H \ast [g]$. 
			
			($\Leftarrow$) Suppose $[f] = [h \ast g]$, where $h = p \circ \tilde{h}$ for some loop $\tilde{h}$ in $E$ based at $e_{0}$. Since $\tilde{g}$ is a path in $E$ that \textit{begins} at $e_{0}$, the product $\tilde{h} \ast \tilde{g}$ is well-defined. Since $[f] = [h \ast g]$, it follows that $\tilde{f}$ and $\tilde{h} \ast \tilde{g}$ must end at the same point. Hence, $\tilde{f}$ and $\tilde{g}$ end at the same point. Therefore, $\phi([f]) = \phi([g])$. 
			
			($\Rightarrow$) Suppose $\phi([f]) = \phi([g])$, which means that $\tilde{f}(1) = \tilde{g}(1)$. Then the product of $\tilde{f}$ with the reverse of $\tilde{g}$ is well-defined and is a loop $\tilde{h}$ in $E$ based at $e_{0}$. By direct computation, $[\tilde{h} \ast \tilde{g}] = [\tilde{f}]$. If $\tilde{F}$ is a path homotopy between $\tilde{h} \ast \tilde{g}$ and $\tilde{f}$, then $p \circ \tilde{F}$ is a path homotopy between $h \ast g$ and $f$, which means that $[f] \in H\ast [g]$. Hence, this concludes the proof that $\Phi$ is injective. 
		\end{enumerate}
	Hence, $|p^{-1}(b_{0})| = |G/H| = (G:H)$. 
\end{solutions}
\begin{prb}{2024-J-I-3 (Complex Analysis)}
	Suppose $f$ is continuous on the plane and holomorphic on $\mathbb{C}\setminus \mathbb{R}$. Prove that $f$ is holomorphic on the whole plane. 
\end{prb}
\begin{solutions}
		Let $f:\mathbb{C}\to\mathbb{C}$ be continuous on $\mathbb{C}$ and holomorphic on 
		$\mathbb{C}\setminus\mathbb{R}$. We show that $f$ is holomorphic on all of $\mathbb{C}$.
		
		By Morera's Theorem, it suffices to prove that
		
		
		\[
		\oint_{\gamma} f(z)\,dz = 0
		\]
		
		
		for every closed piecewise $C^{1}$ curve $\gamma \subset \mathbb{C}$.
		
		If $\gamma$ lies entirely in the upper or lower half-plane, then $f$ is holomorphic
		on a neighborhood of $\gamma$, and by the Cauchy--Goursat theorem,
		
		
		\[
		\oint_{\gamma} f(z)\,dz = 0.
		\]
		
		
		
		Now suppose that $\gamma$ intersects the real axis.  
		For $\varepsilon>0$, construct a closed piecewise $C^{1}$ curve $\gamma_{\varepsilon}$ 
		by modifying $\gamma$ so that it avoids the real axis by small detours of height 
		$\pm\varepsilon$. Then $\gamma_{\varepsilon}\subset \mathbb{C}\setminus\mathbb{R}$, 
		so $f$ is holomorphic on a neighborhood of $\gamma_{\varepsilon}$, and hence
		
		
		\[
		\oint_{\gamma_{\varepsilon}} f(z)\,dz = 0.
		\]
		
		
		
		Since $f$ is continuous on $\mathbb{C}$, it is uniformly continuous on compact sets, 
		and the total length of the detours tends to $0$ as $\varepsilon\to 0$. Therefore,
		
		
		\[
		\lim_{\varepsilon\to 0} \oint_{\gamma_{\varepsilon}} f(z)\,dz
		= \oint_{\gamma} f(z)\,dz.
		\]
		
		
		Thus $\oint_{\gamma} f(z)\,dz = 0$.
		
		Since this holds for every closed piecewise $C^{1}$ curve in $\mathbb{C}$, 
		Morera's Theorem implies that $f$ is holomorphic on all of $\mathbb{C}$.
\end{solutions}


\begin{prb}{2024-J-I-4 (Algebra)}
	For each field $K$, prove that the polynomial ring $K[x, y]$ in two variables is not a principal ideal domain. 
\end{prb}
\begin{solutions}
	Let $K$ be a field, and consider the polynomial ring $K[x, y]$. Let $(x, y)$ be the proper ideal of $K[x, y]$ generated by the monomials $x$ and $y$. Assume to the contrary that $(x, y) = (f(x,y))$ where $f(x, y) \in K[x, y]$ is not a unit of the polynomial ring. Since $x \in (f(x, y))$, $f(x, y) \mid x$. By our assumption that $f$ is not a unit, it follows that $f(x, y)$ is an associate of $x$. Likewise, $f(x, y)$ must be an associate of $y$. But this is impossible since $x$ and $y$ are not associates of each other. This forces $f(x, y)$ to be a unit, which means that $(f(x, y)) = K[x, y]$. But this contradicts the fact that $(x, y) = (f(x, y))$ is a proper ideal. Hence, by contradiction, $(x, y)$ is not a principal ideal, and so $K[x, y]$ is not a principal ideal domain. 
\end{solutions}
\begin{prb}{2024-J-I-5 (Geometry/Topology)}
	Let $\alpha$ be a closed 1-form on $\mathbb{RP}^{n}$, $n > 1$. Show that if $f:[0,1] \to \mathbb{RP}^{n}$ is a smooth function such that $f(0) = f(1)$, then 
		\begin{equation*}
			\int_{[0,1]}f^{\ast}\alpha = 0. 
		\end{equation*}
	Include all calculations that are relevant to your solution. 
\end{prb}
\begin{solutions}
	We recall that $H^{k}(\mathbb{RP}^{n}) = 0$ for all $0 < k < n$ so that $H^{1}(\mathbb{RP}^{n}) = 0$ if $n > 1$. In particular, this means that $\alpha$ is also an exact 1-form on $\mathbb{RP}^{n}$. Let $g$ be a smooth function on $\mathbb{RP}^{n}$ so that $\alpha = dg$. Then 
		\begin{equation}
			\int_{0}^{1}f^{\ast}\alpha = \int_{0}^{1}f^{\ast}dg = \int_{0}^{1}d(f^{\ast}g) = g(f(1)) - g(f(0)) = 0, 
		\end{equation}
	where the last equality follows from the fact that $f(1) = f(0)$. Hence, the proof concludes. 
\end{solutions}
\begin{prb}{2024-J-I-6 (Real Analysis)}
	Let $f$ and $g$ be Lebesgue-measurable functions on $\mathbb{R}$. Define the convolution 
		\begin{equation*}
			(f \ast g)(x) = \int_{\mathbb{R}}f(x - y)g(y)\;dy
		\end{equation*}
	for all $x$ such that the integral exists. Prove that if $f \in L^{p}(\mathbb{R})$ and $g \in L^{q}(\mathbb{R})$ with $p, q \in (1, \infty)$ satisfying $\frac{1}{p} + \frac{1}{q} = 1$, then $f \ast g$ is a bounded continuous function on $\mathbb{R}$. 
\end{prb}
\begin{solutions}
	Assume the given hypotheses. Then by H\"older's inequality, for any $x \in \mathbb{R}$, 
		\begin{equation}
			\abs{(f \ast g)(x)} \leq \int_{\mathbb{R}}\abs{f(x -y)g(y)}\;dy \leq \norm{f(x - \cdot)}_{p} \norm{g}_{q}. 
		\end{equation}
	Since $L^{p}$ norms are translation invariant, $\norm{f(x - \cdot)}_{p} = \norm{f}_{p}$. Hence, $|(f \ast g)(x)| \leq \norm{f}_{p}\norm{g}_{q} = M < \infty$ for all $x \in \mathbb{R}$. Hence, we conclude that $f \ast g$ is a bounded function on $\mathbb{R}$. Next, let $\tau_{z}$ be the translation operator defined by $\tau_{z}f = f(x - z)$. Since translation operators are continuous in the $L^{p}$ norms, $\norm{\tau_{z}f - f} \to 0$ as $z \to 0$, which implies that 
		\begin{align}
			\norm{\tau_{z}(f \ast g) - (f\ast g)}_{\infty} &= \norm{(\tau_{z}f - f)\ast g}_{\infty} \\
			&\leq \norm{\tau_{z}f - f}_{p}\norm{g}_{q} \longrightarrow 0 \text{ as $z \to 0$}. 
		\end{align}
	Hence, $f \ast g$ is uniformly continuous, and therefore continuous on $\mathbb{R}$. Note that the inequality used in the second line of the above equation comes from \textit{Young's convolution inequality}, which states the following: 
		\begin{quote}
			\textbf{(Young's Convolution Inequality)} Let $f \in L^{p}$, $g \in L^{q}$, and $p^{-1} + q^{-1} = r^{-1} + 1$. Then $\norm{f \ast g}_{r} \leq \norm{f}_{p} \norm{g}_{q}$.
		\end{quote}
	In our case, we had $r = \infty$ so that $r^{-1} = 0$. 
\end{solutions}
\begin{prb}{2024-J-II-2}
	Suppose $E \subset \mathbb{R}^{2}$ is a set of positive Lebesgue measure. Show that there are points $a, b, c$ in $E$ such that their connecting segments form a right angle, i.e., $a - b$ is perpendicular to $c - b$ (as vectors in $\mathbb{R}^{2}$). 
\end{prb}
\begin{solutions}
	Let $E \subset \mathbb{R}^{2}$ be a set of positive Lebesgue measure; let $m^{2}$ denote the Lebesgue measure on $\mathbb{R}^{2}$. Let $\{v_{1}, v_{2}, v_{3}\}$ be a collection of vectors in $\mathbb{R}^{2}$ such that $v_{1} \perp v_{2}$, and $v_{3} = -v_{1}$. Without loss of generality, assume that $\norm{v_{j}} = 1$ for all $j = 1, \ldots, 3$. By inner regularity of the Lebesgue measure, there exists a compact subset $K_{1} \subset E$ such that $m^{2}(K_{1}) > 0$. Taking $\beta < 1/7$, by outer regularity of the Lebesgue measure, there exists an open set $U$ containing $K_{1}$ such that $m^{2}(U) \leq (1 + \beta)m^{2}(K_{1})$.
	
	Since $K_{1}$ is compact, $d_{1} = d(K_{1}, U^{c}) > 0$. Hence, let $R = d_{1}$. Fix some $r \in (0, R)$ and consider the set $K_{1} + rv_{1}$. We claim that $K_{1} + rv_{1} \subset U$ since if otherwise, 
		\begin{equation}
			d(K_{1}, U^{c}) \leq |rv_{1}| = r < d_{1}, \text{ which is a contradiction.}
		\end{equation}
	Hence, $K_{1} \cup (K_{1} + rv_{1}) \subset U$, which means that 
		\begin{equation}
			m^{2}(U) \geq m^{2}(K_{1} \cup (K_{1} + rv_{1})) = m^{2}(K_{1}) + m^{2}(K_{1} + rv_{1}) - m^{2}(K_{1} \cap (K_{1} + rv_{1})). 
		\end{equation}
	By translation invariance of the Lebesgue measure, $m^{2}(K_{1}) + m^{2}(K_{1} + rv_{1}) = 2m^{2}(K_{1})$ so that 
		\begin{equation}
			m^{2}(K_{1} \cap (K_{1} + rv_{1})) = 2m^{2}(K_{1}) - m^{2}(U) \geq (1 - \beta)m^{2}(K_{1}). 
		\end{equation}
	Since $\beta < 1$, $m^{2}(K_{1} \cap (K_{1} + rv_{1})) > 0$ so that the set is nonempty. For $i = 1, \ldots, 3$, define $K_{i + 1} = K_{i} \cap  (K_{i} + rv_{i})$. Generalizing the argument from above shows that each $K_{i + 1} \subset U$. We claim that $m^{2}(K_{i + 1}) \geq (1 - (2^{i} - 1)\beta)m^{2}(K_{1})$ for each $i$; the above work establishes the result for $i = 1$. Now assume the result holds for some $1 \leq j < 3$. Then 
		\begin{equation}
			m^{2}(U) \geq m^{2}(K_{j} \cup (K_{j} + rv_{j})) = m^{2}(K_{j}) + m^{2}(K_{j} + rv_{j}) - m^{2}(K_{j} \cap (K_{j} + rv_{j})) = 2m^{2}(K_{j}) - m^{2}(K_{j} \cap (K_{j} + rv_{j})). 
		\end{equation} 
	Therefore, 
		\begin{align}
			\begin{split} 
			m^{2}(K_{j} \cap (K_{j} + rv_{j})) &= 2m^{2}(K_{j}) - m^{2}(U) \\
			&\geq 2m^{2}(K_{1}) -2^{j + 1}\beta m^{2}(K_{1}) + 2\beta m^{2}(K_{1}) - m^{2}(K_{1}) - \beta m^{2}(K_{1}) \\
			&= (1 - (2^{j + 1} - 1)\beta)m^{2}(K_{1}). 
			\end{split} 
		\end{align} 
	Since $\beta < (2^{3} - 1)^{-1} = 7^{-1}$, we conclude that each $K_{i}$ is nonempty. Hence, we obtain a nested sequence $\varnothing \neq K_{4} \subset \dotsm \subset K_{1} \subset E$. Let $q \in K_{4}$; since $K_{4} = K_{3} \cap (K_{3} + rv_{3})$, $q - rv_{3} \in K_{3}$. Following inductively, we obtain a sequence of points $\{p, p + rv_{1}, p + rv_{1} + rv_{2}, p + rv_{1}+ rv_{2} + rv_{3}\}\subset E$, with $p \in K_{1}$, and $p + rv_{j} \in K_{j}$ for $j= 1, 2, 3$ (note we have renamed $q - rv_{1} - \dotsm - rv_{3} = p$, and so on). Let $a = p$, $b = p + rv_{1}$, and $c = p + rv_{1} + rv_{2}$. Then $a - b = -rv_{1}$ and $c - b= rv_{2}$. By hypothesis on $v_{1}$ and $v_{2}$, $a - b$ is orthogonal to $c - b$. 
\end{solutions}
\begin{prb}{2024-J-II-3 (Geometry/Topology)}
	Let $\Sigma$ be a genus 2 surface embedded in $\mathbb{R}^{3}$ as shown in the picture. Let $M$ be the closure of the \textit{unbounded} component of $\mathbb{R}^{3}\setminus \Sigma$; in other words, $M$ is the part of $\mathbb{R}^{3}$ which is \textit{not} enclosed by $\Sigma$. \vspace{-0.25cm}
		\begin{enumerate}[itemsep =-2pt,label = (\alph{*})]
			\item Compute $\pi_{1}(M)$. 
			\item Is $\Sigma$ a retract of $M$? 
		\end{enumerate}
	\begin{center}
		\includegraphics[width = 0.3\linewidth]{"../Graphics/2024JII3"}
	\end{center}
\end{prb}
\begin{solutions}
	$ $\newline 
	\begin{enumerate}[itemsep =-2pt,label = (\alph{*})]
		\item 
	\end{enumerate}
\end{solutions}


\begin{prb}{2024-J-II-6 (Geometry/Topology)}
	Let $M$ be a smooth $n$-manifold, and let $\varphi$ be a differential $k$-form on $M$ which is closed, in the sense that $d\varphi = 0$. At each point $p \in M$, define 
		\begin{equation}
			D_{p}= \brac*{v \in T_{p}M: v\righthalfcup\varphi = 0}, 
		\end{equation}
	where $\righthalfcup$ denotes the interior product. Assume $\ell \coloneqq \dim{D_{p}}$, so that $D \subset TM$ is a rank-$\ell$ vector sub-bundle of the tangent bundle of $M$. Prove that $D$ is an integrable distribution of $\ell$-planes, in the sense of the Frobenius Theorem. 
\end{prb}
\begin{solutions}
	By the Frobenius Theorem, it suffices to prove that $D$ is involutive, which is to say that if $X, Y$ are smooth sections of $D$, then $[X, Y]$ is also a smooth section of $D$. Indeed, let $X, Y$ be smooth sections of $D$, which means that $X\righthalfcup \varphi, Y\righthalfcup \varphi = 0$. Observe that,
		\begin{equation}
			[X,Y]\righthalfcup \varphi = \mathscr{L}_{X}(Y\righthalfcup \varphi) - Y\righthalfcup(\mathscr{L}_{X}\varphi). 
		\end{equation}
	By hypothesis, $Y \righthalfcup \varphi = 0$ so that $\mathscr{L}_{X}(Y \righthalfcup \varphi) = 0$. On the other hand, by Cartan's Formula, 
		\begin{equation}
			\mathscr{L}_{X}\varphi = d(X\righthalfcup \varphi) + X\righthalfcup d\varphi = 0, 
		\end{equation}
	by the hypotheses. Hence, this shows that $[X, Y] \righthalfcup \varphi = 0$, and so $[X, Y]$ is a smooth section of $D$. Therefore, $D$ is involutive, which means that it is Frobenius integrable. 
\end{solutions}
\begin{prb}{2024-J-II-4 (Algebra)}
	Let $\alpha = \sqrt{2 + \sqrt{3}} \in \mathbb{C}$. Let $K$ be the smallest Galois extension of $\mathbb{Q}$ which contains $\alpha$. Describe the Galois group $\operatorname{Gal}(K/\mathbb{Q})$. 
\end{prb}
\begin{solutions}
	Let $\alpha = \sqrt{2 + \sqrt{3}} \in \mathbb{C}$, and $K$ the smallest Galois extension of $\mathbb{Q}$ that contains $\alpha$. We start by finding the minimal polynomial of $\alpha$. We observe that 
		\begin{equation}
			\alpha^{2} = 2 + \sqrt{3} \implies (\alpha^{2} - 2)^{2} - 3 = 0. 
		\end{equation}
	Simplifying, 
		\begin{equation}
			\alpha^{4} - 4\alpha^{2} + 1 = 0. 
		\end{equation}
	I.e., the polynomial $x^{4} - 4x^{2} + 1$ is the minimal polynomial of $\alpha$. Solving this polynomial over an algebraic closure of $\mathbb{Q}$, we obtain the four roots, $\pm\sqrt{2 + \sqrt{3}}, \pm \sqrt{2 - \sqrt{3}}$. Hence, the elements of the Galois group $\operatorname{Gal}(K/\mathbb{Q})$ are the identity permutation, the permutation $\sigma$ that fixes $\pm \sqrt{2 - \sqrt{3}}$ and permutes $\pm \sqrt{2 + \sqrt{3}}$, the permutation $\tau$ that fixes $\pm\sqrt{2 + \sqrt{3}}$ and permutes $\pm \sqrt{2 - \sqrt{3}}$, and the permutation $\sigma \tau$. Labeling these roots as $\alpha_{1}, \ldots, \alpha_{4}$, we see that $\operatorname{Gal}(K/\mathbb{Q}) \cong \{1, (1\;2), (3\;4), (1\;2)(3\;4)\} \cong V \subset S_{4}$, where $V$ is the Klein-4 subgroup. 
\end{solutions}


\subsection{August 2024}
\begin{prb}{2024-A-I-1 (Geometry/Topology)}
	Let $M$ be a smooth compact manifold without boundary, and let $\varphi$ be a smooth closed 1-form on $M$ that has the property that $\varphi \neq 0$ at every point of $M$. Prove that the first de Rham cohomology $\h{1}(M)$ of the given manifold is non-zero. 
\end{prb}
\begin{solutions}
	Let $M$ be a smooth compact manifold without boundary and let $\varphi$ be a smooth closed 1-form on $M$ that has the property that $\varphi \neq 0$ at every point of $M$. Suppose that $\varphi$ is exact; i.e., assume there exists a smooth function $f$ on $M$ such that $\varphi = df$. By the Extreme Value Theorem, since $M$ is compact, $f$ must have either a maximum or minimum value at some point $p \in M$. Since all of the first-order partial derivatives of $f$ must vanish at the point $p$ where $f$ attains its maximum/minimum value, $df|_{p} = 0$. This means that $\varphi$ must also vanish at $p$, which contradicts our hypothesis that $\varphi$ is nowhere vanishing. Hence, by contradiction, $\varphi$ cannot be an exact form. Since $\h{1}(M) \coloneqq \brac*{\text{closed 1-forms on $M$}}/\brac*{\text{exact 1-forms on $M$}}$ and we have shown the existence of a closed 1-form that is \textit{not} an exact 1-form, we conclude that $\h{1}(M)$ is non-zero. 
\end{solutions}
\begin{prb}{2024-A-I-2 (Geometry/Topology)}
	Suppose that $f: \Sigma_{2} \to \Sigma_{1}$ is a continuous map between a genus 2 closed orientable surface $\Sigma_{2}$ and a torus $\Sigma_{1}$. Prove that $f$ is not a local homeomorphism. In other words, show that there exists a point $x \in \Sigma_{2}$ which does not have an open neighborhood $U \subset \Sigma_{2}$ on which the restriction $f|_{U}$ is a homeomorphism between $U$ and $f(U)$. 
\end{prb}
\begin{solutions}
	Before presenting our argument, we will state and prove a quick technical lemma. 
		\begin{quote}
			(\textbf{Modified Comps Lemma}) Let $M$ and $N$ be smooth connected manifolds, and $f: M \to N$ a local homeomorphism. If $M$ is compact and nonempty, then $N$ is compact and $f$ is a covering map. 
			
			\begin{proof}
				Let $M$ and $N$ be smooth connected manifolds, and $f: M \to N$ a local homeomorphism. Since $f$ is an open map, $f(M)$ is open in $M$. Next since the continuous image of a compact set is compact and a compact subset of a Hausdorff space is closed, $f(M)$ is closed in $N$. Hence, since $N$ is connected, $f(M) = N$, which means $N$ is connected and $f$ is surjective. 
				
				Now let $q \in N$, and consider the closed subset $f^{-1}(q) \subset M$. For each $x \in f^{-1}(q)$, there exists a neighborhood $U_{x}$ such that $f|_{U_{x}}$ is a homeomorphism. Since $M$ is Hausdorff, we may shrink these neighborhoods so that they are pairwise disjoint. Hence, each $x \in f^{-1}(q)$ is isolated, which means $f^{-1}(q)$ is discrete. Since discrete subspaces of compact spaces is necessarily finite, $f^{-1}(q)$ is finite; let $\{x_{1}, \ldots, x_{s}\} = f^{-1}(q)$. As stated above, for each $j = 1, \ldots, s$, we may find a neighborhood $U'_{j}$ such that $f|_{U'_{j}}$ is a homeomorphism. Using Hausdorff-ness of $M$, we may shrink these neighborhoods to obtain the collection $\{\ti{U}_{j}\}_{1}^{s}$ of pairwise disjoint open neighborhoods. Set $V = \bigcap_{1}^{s}U_{j}$, which is then an evenly covered neighborhood of $q$. Therefore, $f$ is a covering map. 
			\end{proof}
		\end{quote} 
	Now assume to the contrary that $f: \Sigma_{2} \to \Sigma_{1}$ is a local homeomorphism; by the modified Comps Lemma, $f$ is a covering map. Moreover, $\Sigma_{2}$ must be a $k$-sheeted covering space for some finite positive integer $k$, which means that $\chi(\Sigma_{2}) = k \cdot \chi(\Sigma_{1})$. However, this is impossible since $\chi(\Sigma_{1}) = 0$, while $\chi(\Sigma_{2}) = 2 - 2(2) = 2 - 4 = -2$. Therefore, $f$ cannot be a local homeomorphism. 
\end{solutions}

\begin{prb}{2024-A-I-5 (Algebra)}
	Determine whether or not the complex number $i = \sqrt{-1}$ is in the field $\mathbb{Q}(\alpha)$, where $\alpha$ is any complex number subject to the relation $\alpha^{3} + \alpha + 1 = 0$. Justify your answer. 
\end{prb}
\begin{solutions}
	The polynomial $x^{3} + x + 1$ has no roots in $\mathbb{Q}$ (by the rational root test), and so is irreducible (since it is a cubic). This means that $\mathbb{Q}(\alpha)$ is an extension of degree 3 over $\mathbb{Q}$. Therefore, it cannot contain the field $\mathbb{Q}(i)$, which has degree $2$ over $\mathbb{Q}$ (since the minimal polynomial of $i$ is $x^{2} + 1$) since $2 \nmid 3$. 
\end{solutions}
\begin{prb}{2024-A-II-1 (Geometry/Topology)}
	Recall that $S^{n}$ denotes the unit sphere in $\mathbb{R}^{n + 1}$. Also recall that a smooth map is called a smooth submersion if its differential is everywhere surjective. Prove or disprove each of the following statements: \vspace{-0.25cm}
		\begin{enumerate}[itemsep =-2pt,label = (\alph{*})]
			\item There is a smooth submersion $F: S^{3} \to S^{1}$. 
			\item There is a smooth submersion $F: S^{3} \to S^{2}$. 
		\end{enumerate}
\end{prb}
\begin{solutions}
	$ $\newline \vspace{-1cm}
	\begin{enumerate}[itemsep =-2pt,label = (\alph{*})]
		\item \textcolor{red}{[!! Complete Later !!]}
	\end{enumerate}
\end{solutions}

\begin{prb}{2024-A-II-2 (Geometry/Topology)}
	On $\mathbb{R}^{5}$, equipped with standard coordinates \newline $(v, w, x, y,z)$, consider the 1-form 
		\begin{equation*}
			\theta = dz + v\;dx + w\;dy. 
		\end{equation*}
	Are there two smooth functions $f, g: \mathbb{R}^{5} \to \mathbb{R}$ such that $\theta = f\;dg$? Justify your answer by means of concrete solutions. 
\end{prb}
\begin{solutions}
	We claim that there do \textit{not} exist smooth functions $f, g: \mathbb{R}^{5} \to \mathbb{R}$ such that $\theta = f\;dg$. Assume to the contrary. First, we observe that if $\theta = fdg$, then 
		\begin{equation}
			d\theta = d(fdg) = df \wedge dg \implies \theta \wedge d\theta = f dg \wedge df \wedge dg = 0. 
		\end{equation}
	I.e., if $\theta = fdg$, then $\theta \wedge d\theta$ must be identically zero. However, since $\theta = dz + vdx + wdy$, we note that 	
		\begin{equation}
			d\theta = d^{2}z + d(vdx) + d(wdy) = dv \wedge dx + dw \wedge dy \implies \theta \wedge d\theta = dz \wedge dv \wedge dx + dz \wedge dw \wedge dy + vdx \wedge dw \wedge dy + w dy \wedge dv \wedge dx, 
		\end{equation}
	which is nowhere vanishing on $\mathbb{R}^{5}$. Hence, by contradiction, there cannot exist two smooth functions $f, g: \mathbb{R}^{5} \to \mathbb{R}$ such that $\theta = fdg$. 
\end{solutions}


\subsection{January 2023}
\begin{prb}{2023-J-II-4 (Geometry/Topology)}
	Prove that $S^{2} \times S^{2}$ is not diffeomorphic to $M_{1} \times M_{2} \times M_{3}$, where $M_{1}, M_{2}, M_{3}$ are smooth manifolds of nonzero dimension. 
\end{prb}
\begin{solutions}
	We begin with a technical lemma, that we will use to prove the desired result. 
		\begin{quote}
			(\textbf{Comps Lemma}) Let $M, N$ be smooth, connected $n$-manifolds and $f: M \to N$ a (smooth) immersion. If $M$ is compact and nonempty, then $N$ is compact and $f$ is a (smooth) covering map. 
			
			\begin{proof}
				Let $M, N$ be smooth connected $n$-manifolds, $f: M \to N$ an immersion, and $M$ compact and nonempty. Since $\dim{N} = n$ everywhere and $f$ is an immersion, $df_{p}: T_{p}M \to T_{f(p)}N$ has constant rank $n$ everywhere. Hence, by the Inverse Function Theorem, $f$ is a local diffeomorphism. Since local diffeomorphisms are open maps, $f(M)$ is open in $N$. Next since the continuous image of compact sets is compact, $f(M)$ is compact in $N$. Since $N$ is Hausdorff, $f(M)$ must be closed in $N$. Therefore, since $N$ is connected, we conclude that $f(M) = N$. This means that $N$ is compact and $f$ is surjective. All that remains is to show that $f$ is a covering map. 
				
				Let $q \in N$, and consider $f^{-1}(q)$, which is closed in $M$. For each $x \in f^{-1}(q)$, there exists a neighborhood $U_{x}$ of $x$ such that $f|_{U_{j}}$ is a diffeomorphism.  Since $M$ is Hausdorff, we may shrink these neighborhoods so that they are pairwise disjoint. This means that each $x \in f^{-1}(q)$ is isolated. Hence, $f^{-1}(q)$ is discrete in $M$. Since discrete subspaces of compact spaces must be finite, it follows that $f^{-1}(q)$ is finite; let $f^{-1}(q) = \{x_{1}, \ldots, x_{s}\}$. As stated above, for each $j = 1, \ldots, s$, we can find a neighborhood $U_{j}$ of $x_{j}$ such that $f|_{U_{j}}: U_{j} \to V_{j} \subset N$ is a diffeomorphism. Since $M$ is Hausdorff, we may shrink these neighborhoods so that $U_{i} \cap U_{j} = \varnothing$ for all $i \neq j$; $f$ restricted to each of these new $U_{j}$'s remains a diffeomorphism. Set $V = \bigcap_{1}^{s}f(U_{j})$, and define $\ti{U}_{j} = f^{-1}(V) \cap  U_{j}$. For each $j$, $f: \ti{U}_{j} \to V$ is a diffeomorphism and $V =  \bigsqcup_{1}^{s}f(U_{j})$. Hence, $V$ is an evenly covered neighborhood of $q$, so that $f$ is a covering map.  
			\end{proof}
		\end{quote} 
	Now, assume to the contrary that $f: S^{2} \times S^{2} \to M_{1} \times M_{2} \times M_{3}$ is a diffeomorphism; since diffeomorphisms preserve dimensions and $M_{1}, M_{2}, M_{3}$ have nonzero dimensions, it follows, without loss of generality,  that $M_{1}, M_{2}$ are 1-dimensional and $M_{3}$ is 2-dimensional. Since diffeomorphisms of manifolds are immersions, by the Comps Lemma, $M_{1} \times M_{2} \times M_{3}$ must be compact and connected; by projecting onto each manifold, $M_{1}, M_{2}, M_{3}$ must be compact and connected. Moreover, the induced group homomorphism $f_{\ast}: \pi_{1}(S^{2} \times S^{2}) \to \pi_{1}(M_{1} \times M_{2} \times M_{3}) = \pi_{1}(M_{1}) \times \pi_{1}(M_{2}) \times \pi_{1}(M_{3})$ must be an isomorphism. Since $S^{2}$ is simply connected, 
		\begin{equation}
			\pi_{1}(S^{2} \times S^{2}) = \pi_{1}(S^{2}) \times \pi_{1}(S^{2}) = \{0\}. 
		\end{equation}
	On the other hand, since the only compact connected 1-manifold, up to diffeomorphism, is the unit circle $S^{1}$, and $\pi_{1}(S^{1}) \cong \mathbb{Z}$ is not trivial, $\pi_{1}(M_{1} \times M_{2} \times M_{3})$ is not trivial. But this contradicts our claim that $f_{\ast}$ is an isomorphism. Hence, by contradiction, $f$ cannot be a diffeomorphism. 
\end{solutions}
\begin{prb}{2023-J-II-3 (Geometry/Topology)}
	Consider the form $\omega = (x^{2} + x + y)dy \wedge dz$ on $\mathbb{R}^{3}$. Let $S^{2} \subset \mathbb{R}^{3}$ be the unit sphere, and $i: S^{2} \to \mathbb{R}^{3}$ be the inclusion map. \vspace{-0.25cm}
		\begin{enumerate}[itemsep =-2pt,label = (\alph{*})]
			\item Calculate $\int_{S^{2}}i^{\ast}\omega$. 
			\item Construct a closed form $\alpha$ on $\mathbb{R}^{3}$ such that $i^{\ast}\alpha = i^{\ast}\omega$, or show that such a form $\alpha$ does not exist. 
		\end{enumerate}
\end{prb}
\begin{solutions}
	$ $\newline \vspace{-0.65cm}
	\begin{enumerate}[itemsep =-2pt,label = (\alph{*})]
		\item (\textbf{Method 1}) Consider the form $\omega = (x^{2} + x + y)dy\wedge dz$ on $\mathbb{R}^{3}$, and let $i: S^{2}\hookrightarrow \mathbb{R}^{3}$ be the inclusion map. Let $D = [0, \pi] \times [0, 2\pi]$, and $F: D \to S^{2}$ be the coordinate map defined by 
			\begin{equation}
				F(\varphi, \theta) = (\sin(\varphi)\cos(\theta), \sin(\varphi)\sin(\theta),\cos(\varphi)). 
			\end{equation}
		Taking $D_{1} = [0, \pi] \times [0, \pi]$ and $D_{2} = [0, \pi]\times [\pi, 2\pi]$, and letting $F_{1} = F|_{D_{1}}$ and $F_{2} = F|D_{2}$, we observe that 
			\begin{equation}
				\int_{S^{2}}i^{\ast}\omega = \int_{D_{1}}F_{1}^{\ast}i^{\ast}\omega + \int_{D_{1}}F_{2}^{\ast}\omega = \int_{D_{1}}(i \circ F_{1})^{\ast}\omega + \int_{D_{2}}(i \circ F_{2}^{\ast})\omega = \int_{D}F^{\ast}\omega, 
			\end{equation}
		where the last equality follows from the fact that $i \circ F_{1,2} = F_{1,2}$. We observe that 
			\begin{equation}
				F^{\ast}dy = \cos(\varphi)\sin(\theta)d\varphi + \sin(\varphi)\cos(\theta)d\theta \qquad \text{and} \qquad F^{\ast}dz = -\sin(\varphi)d\varphi. 
			\end{equation}
		Therefore, 
			\begin{equation}
				F^{\ast}\omega = \left[\sin^{2}(\varphi)\cos^{2}(\theta) + \sin(\varphi)\cos(\theta) + \sin(\varphi)\sin(\theta)\right]\sin^{2}(\varphi)\cos(\theta)d\varphi \wedge d\theta. 
			\end{equation}
		From this, we conclude that 
			\begin{equation}
				\int_{S^{2}}i^{\ast}\omega = \int_{0}^{2\pi}\int_{0}^{\pi}\left[\sin^{2}(\varphi)\cos^{2}(\theta) + \sin(\varphi)\cos(\theta) + \sin(\varphi)\sin(\theta)\right]\sin^{2}(\varphi)\cos(\theta)d\varphi d\theta = \frac{4\pi}{3}. 
			\end{equation}
		(\textbf{Method 2}) Using Stokes Theorem, 
			\begin{equation}
				\int_{S^{2}}i^{\ast}\omega = \int_{B^{3}}d\omega, 
			\end{equation}
		where $B^{3}$ indicates the $3$-ball (recall that $S^{1} = \partial B^{3}$). We compute, $d\omega = (2x  + 1)dx \wedge dy \wedge dz$ so that 
			\begin{equation}
				\int_{S^{2}}i^{\ast}\omega = \int_{B^{3}}d\omega = \int_{B^{3}}2xdxdydz + \int_{B^{3}}dxdydz = \int_{B^{3}}dxdydz = \frac{4\pi}{3}, 
			\end{equation}
		where the first integral after the second inequality is zero due to symmetry. 
		\item Suppose there exists a closed form $\alpha$ on $\mathbb{R}^{3}$ such that $i^{\ast}\alpha = i^{\ast}\omega$. Since $\alpha$ is closed, $d\alpha = 0$. Hence, 
			\begin{equation}
				\int_{S^{2}}i^{\ast}\alpha = \int_{B^{3}}d(i^{\ast}\alpha) = \int_{B^{3}}i^{\ast}d\alpha = 0 \neq \frac{4\pi}{3} =\int_{S^{2}}i^{\ast}\omega, 
			\end{equation}
		which is a contradiction. Hence, such a closed form cannot exist. 
	\end{enumerate}
\end{solutions}
\begin{prb}{2023-J-I-5 (Algebra)}
	Consider the following irreducible polynomial over $\mathbb{Q}$: $p(x) = x^{4} - 3x^{2} - 1$. \vspace{-0.25cm}
		\begin{enumerate}[itemsep =-2pt,label = (\alph{*})]
			\item Describe the splitting field of $p(x)$. 
			\item Consider the Galois group of $p(x)$. Compute its order and determine if it is abelian. 
		\end{enumerate}
\end{prb}
\begin{solutions}
	$ $\newline \vspace{-1cm}
	\begin{enumerate}[itemsep =-2pt,label = (\alph{*})]
		\item To determine the splitting field of $p(x)$, we must begin by finding the roots of $p(x)$ over some algebraic closure of $\mathbb{Q}$. Let $z = x^{2}$. Then 
			\begin{align}
				\begin{split}
					p(z) = 0 &\iff z^{2} - 3z - 1 = 0 \\
					&\iff z = \frac{3 \pm \sqrt{13}}{2} \\
					&\iff x = \pm\sqrt{\frac{3 + \sqrt{13}}{2}}, \pm \sqrt{\frac{3 - \sqrt{13}}{2}}. 
				\end{split}
			\end{align}
		Therefore, the splitting field of $p(x)$ is 
			\begin{equation}
				\mathbb{Q}\left(\sqrt{\frac{3 + \sqrt{13}}{2}}, \sqrt{\frac{3 -\sqrt{13}}{2}}\right).
			\end{equation}
		\item Label the roots as $\alpha_{1} = ((3 +\sqrt{13})/2)^{1/2}, \alpha_{2} = -((3 +\sqrt{13})/2)^{1/2}$, $\alpha_{3} = ((3 -\sqrt{13})/2)^{1/2}$, and $\alpha_{4} = -((3 -\sqrt{13})/2)^{1/2}$. The elements of the Galois group are the permutations $\{1, \sigma, \tau, \sigma\tau\}$, where $\sigma: \alpha_{1} \to \alpha_{2}$ and fixes $\alpha_{3}$ and $\tau:\alpha_{3} \to \alpha_{4}$ and fixes $\alpha_{1}$; i.e., $\sigma = (1\;2)$ and $\tau = (3\;4)$. Hence, 
			\begin{equation}
				\operatorname{Gal}(\mathbb{Q}(\alpha_{1},\alpha_{3})/\mathbb{Q}) \cong \{1, (1\;2), (3\;4), (1\;2)(3\;4)\} \subset S^{4}.
			\end{equation}
		In particular, we see that the Galois Group is isomorphic to the Klein-4 subgroup of $S_{4}$. Therefore, the Galois group of $p(x)$ has order 4 and is abelian. 
	\end{enumerate}
\end{solutions}
\begin{prb}{2023-J-I-5 (Algebra I)}
	Determine the Galois group of $x^{3} - x^{2} - 4$. 
\end{prb}
\begin{solutions}
	Let $p(x) = x^{3} - x^{2} - 4$. We start by finding the roots of $p(x)$ over some algebraic closure of $\mathbb{Q}$. Observe that $2$ is a solution. Using polnomial long division, 
		\begin{equation}
			p(x) = (x - 2)(x^{2} + x + 2) \implies x = 2, \frac{-1 \pm \sqrt{-7}}{2}. 
		\end{equation}
	Hence, the splitting field of $p(x)$ is $\mathbb{Q}(\sqrt{7}i)$. Now since $\operatorname{Gal}(\mathbb{Q}(\sqrt{7}i)/\mathbb{Q})$ is the group of automorphisms of the splitting field $\mathbb{Q}(\sqrt{7}i)$ that preserve $\mathbb{Q}$. Since there are exactly two automorphisms (namely, the identity permutation fixing $\sqrt{7}i$ and the conjugation map $\sqrt{7}i \mapsto -\sqrt{7}i$), we conclude that $\operatorname{Gal}(\mathbb{Q}(\sqrt{7}i)/\mathbb{Q}) \cong \mathbb{Z}_{2}$. 
\end{solutions}
\begin{prb}{2023-J-I-5 (Algebra II)}
	Determine the Galois group of $x^{3} - 2x + 4$. 
\end{prb}
\begin{solutions}
	Let $p(x) = x^{3} - 2x + 4$. We start by finding the roots of $p(x)$ over some algebraic closure of $\mathbb{Q}$. Clearly $-2$ is a root of $p(x)$. Using polynomial long division, 
		\begin{equation}
			p(x) = (x+ 2)(x^{2} - 2x + 2) \implies x = -2, 1 \pm \sqrt{-1}. 
		\end{equation}
	Hence, the splitting field of $p(x)$ is $\mathbb{Q}(i)$, which is a quadratic extension of $\mathbb{Q}$. Now since $\operatorname{Gal}(\mathbb{Q}(i)/\mathbb{Q})$ is the group of automorphisms of the splitting field $\mathbb{Q}(i)$ that preserve $\mathbb{Q}$, and there exactly two such automorphisms (namely, the identity fixing $i$, and the conjugation map $i \mapsto -i$), we conclude that $\operatorname{Gal}(\mathbb{Q}(i)/\mathbb{Q}) \cong \mathbb{Z}/2\mathbb{Z}$. 
\end{solutions}
\begin{prb}{2023-J-I-5 (Algebra III)}
	Determine the Galois group of $x^{3} - x + 1$. 
\end{prb}
\begin{solutions}
	Let $p(x) = x^{3} - x + 1$. We start by finding the roots of $x$ over some algebraic closure of $\mathbb{Q}$. Since the only possible rational roots of $p$ over $\mathbb{Q}$ are $\pm 1$ by the Rational Root Test, and neither of these are actually roots of $p$, we conclude that $p$ is irreducible. Hence, a root of $f(x)$ generates an extension of degree 3 so that the degree of the splitting field of $F$ is divisible by 3. Since the Galois group is a subgroup of $S_{3}$, either $\operatorname{Gal}(\mathbb{Q}(\alpha_{1}, \alpha_{2}, \alpha_{3})/\mathbb{Q}) \cong A_{3}$ or $\operatorname{Gal}(\mathbb{Q}(\alpha_{1}, \alpha_{2}, \alpha_{3})/\mathbb{Q}) \cong S_{3}$. Since $p$ is already a depressed cubic, we calculate its discriminant to be $-4(-1)^{3} - 27(1)^{2} = -23$. Since the discriminant is not a perfect square in $\mathbb{Q}$, we conclude that $\operatorname{Gal}(\mathbb{Q}(\alpha_{1}, \alpha_{2},\alpha_{3})/\mathbb{Q}) \cong S_{3}$. 
\end{solutions}
\begin{prb}{2023-J-I-4 (Geometry/Topology)}
	Let $\omega$ be a smooth nowhere vanishing 1-form on a smooth 3-manifold $M^{3}$. \vspace{-0.25cm}
		\begin{enumerate}[itemsep =-2pt,label = (\alph{*})]
			\item Show that the distribution defined at each point $p \in M$ by 
				\begin{equation}
					\ker{\omega_{p}} = \brac*{v \in T_{p}M^{3}: \omega_{p}(v) = 0}
				\end{equation}
			is integrable if and only if $\omega \wedge d\omega = 0$. 
			\item Give an example of a codimension one distribution on $\mathbb{R}^{3}$ that is not integrable. 
		\end{enumerate}
\end{prb}
\begin{solutions}
	$ $\newline \vspace{-1cm}
	\begin{enumerate}[itemsep=-2pt,label = (\alph{*})]
		\item We recall that a distribution $D$ is Frobenius integrable if and only if given two smooth sections $X, Y$ of $D$, the Lie Bracket $[X, Y]$ is also a smooth section of $D$. Therefore, let $X, Y$ be smooth sections of $D$, which means that $\omega(X), \omega(Y) = 0$ by definition of $D$. We recall that 
			\begin{equation}
				d\omega(X, Y) = X(\omega(Y)) - Y(\omega(X)) - \omega([X, Y]) = -\omega([X, Y]), 
			\end{equation}
		where the first two terms are identically zero by our hypothesis. Therefore, $D$ is integrable if and only if $[X, Y]$ is a smooth section of $D$ if and only if $\omega([X, Y]) = 0$. Now, if $D$ were integrable, then for any field $Z$ on $\mathbb{R}^{3}$, 
			\begin{equation}
				\omega \wedge d\omega(X, Y,Z) = \omega(Z)d\omega(X, Y) = 0, 
			\end{equation}
		where the other terms vanish by assumption on $X$ and $Y$. Hence, since $X,Y \in \ker{\omega}$ were arbitrary and $Z$ was arbitrary, $\omega \wedge d\omega = 0$. On the other hand, if $\omega \wedge d\omega = 0$, let $p\in M$, $Z_{p} \in T_{p}M$ with $\omega_{p}(Z_{p}) \neq 0$ and $X_{p}, Y_{p} \in \ker{\omega_{p}}$. Then 
			\begin{equation}
				0 = (\omega \wedge d\omega)_{p}(X_{p}, Y_{p}, Z_{p}) = \omega_{p}(Z_{p})d\omega_{p}(X_{p}, Y_{p}). 
			\end{equation}
		Hece, $d\omega_{p}(X_{p}, Y_{p}) = 0$. This means that for smooth sections $X, Y$ of $\ker{\omega}$, $d\omega(X, Y) = 0$, and so $D$ is integrable. 
		\item Consider the smooth nowhere vanishing 1-form $\omega = ydx + dy + dz$ on $\mathbb{R}^{3}$, and let $D$ be the distribution on $\mathbb{R}^{3}$ defined at each point $p \in M$ by $D_{p} = \ker{\omega_{p}}$. By the rank-nullity theorem, $\operatorname{dim}{D} = \operatorname{dim}{T_{p}\mathbb{R}^{3}} - \operatorname{rank}{\omega} = 3 - 1 = 2$. Hence, $\operatorname{codim}{D} = 3 - 2 = 1$. Next, we observe that $d\omega = dy \wedge dx$, which is identically not zero. Then $\omega \wedge d\omega = dz \wedge dy \wedge dx$, which is also not identically zero. Hence, by the conclusion in (a), $D$ is not integrable. 
	\end{enumerate}
\end{solutions}

\subsection{August 2023}
\begin{prb}{2023-A-I-1 (Algebra)}
	Let $V$ be an $n$-dimensional vector space over a field $F$. An element $A \in \operatorname{End}{V}$ is called \textit{nilpotent} if $A^{k} = 0$ for some $k > 1$. Prove that $A$ is nilpotent if and only if $$\operatorname{Tr}(\Lambda^{i}A) = 0, \quad i = 1, \ldots, n$$ where $\Lambda^{i}A$ denotes the induced action of $A$ on the wedge product $\Lambda^{i}V$ for each $i$. 
\end{prb}

\begin{prb}{2023-A-I-5 (Geometry/Topology)}
	Let $T$ be the 2-torus $S^{1} \times S^{1}$ with an open 2-disk removed: 
		\begin{center}
			\includegraphics[width = 0.2\linewidth]{"../Graphics/2023A15.png"}
		\end{center} 
	Show that there is no continuous retraction $r$ onto its boundary (i.e., no continuous map $r: T \to \partial T$ satisfying $r^{2} = r$). 
\end{prb}
\begin{solutions}
	Let $T$ be the 2-torus $S^{1} \times S^{1}$ with an open 2-disk removed, $\iota: \partial T \to T$ the inclusion map, and assume to the contrary that $r: T \to \partial T$ is a continuous retraction. Then the composition $r_{\ast} \circ \iota_{\ast}: \pi_{1}(\partial T) \to \pi_{1}(\partial T)$ must be the identity map. Since $\partial T \cong S^{1}$, $\pi_{1}(\partial T) = \mathbb{Z}$, and is generated by the element $1$. By a direct computation, since $\partial_{1}(T) = \mathbb{Z} \ast \mathbb{Z}$ is the free product on two generators $a$ and $b$  $\iota_{\ast}$ maps $1$ to the element $aba^{-1}b^{-1}$. But then $r_{\ast}$ maps the commutator into the abelian group $\mathbb{Z}$, where the commutator must be zero. This contradicts our claim that $r_{\ast} \circ \iota_{\ast}$ is the identity map. Hence, by contradiction, there cannot be any continuous retraction of $T$ onto its boundary. 
\end{solutions}
\begin{prb}{2023-A-I-6 (Complex Analysis)}
	Let $\mathbb{D} \subset \mathbb{C}$ be the open unit disk. Is there a holomorphic function $f$ with $f(\mathbb{D}) = \mathbb{D}$, $f(0) = f'(0) = 2/3$? If so, give a formula. If not, prove that it cannot exist. 
\end{prb}
\begin{solutions}
	The problem lends itself nicely to an application of the Schwarz-Pick Theorem: 
		\begin{quote}
			(\textbf{Schwarz-Pick Theorem}) Let $f: \mathbb{D} \to \mathbb{D}$ be holomorphic. If $\abs{f(z)} \leq 1$ for all $z$, and $f(a) = b$ for some $a, b \in \mathbb{D}$, then 
				\begin{equation*}
					\abs{f'(a)} \leq \frac{1 - |b|^{2}}{1 - |a|^{2}}.
				\end{equation*}
		\end{quote}
	Now assume that a holomorphic function $f$ with $f(\mathbb{D}) = \mathbb{D}$, $f(0) = f'(0) = 2/3$ exists. Then by the Schwarz-Pick Lemma, 
		\begin{equation}
			\frac{2}{3} \leq \frac{1 - \nicefrac{4}{9}}{1 - 0} = \frac{5}{9} < \frac{2}{3}, 
		\end{equation}
	which is a contradiction. Hence, no such holomorphic function can exist. 
\end{solutions}
\begin{prb}{2023-A-I-2 (Geometry/Topology)}
	Let $f: T^{2} \to S^{2}$ be a smooth map from the 2-torus to the 2-sphere. Can $f$ be an immersion? If the answer is yes, give an explicit example. If the answer is no, then give a proof. 
\end{prb}
\begin{solutions}
	We begin by stating and proving a technical lemma, which we will then use in our argument. 
		\begin{quote}
			(\textbf{Comps Lemma}) Let $M$ and $N$ be smooth connected $n$-manifolds, and $f: M \to N$ a (smooth) immersion. If $M$ is compact and nonempty, then $N$ is compact and $f$ is a (smooth) covering map. 
			
			\begin{proof}
				Let $M$ and $N$ be smooth connected $n$-manifolds, and $f:M \to N$ an immersion. Since $\dim{M} = \dim{N} = n$, and $f$ is an immersion, the map $df_{p}: T_{p}M \to T_{f(p)}N$ has constant rank $n$ at every $p \in M$. Hence, by the Inverse Function Theorem, $f$ is a local diffeomorphism. Since local diffeomorphisms are open maps, $f(M)$ is open in $N$. On the other hand, since continuous images of compact sets are compact, $f(M)$ is compact in $N$; since $N$ is Hausdorff, $f(M)$ is closed in $N$. Since $N$ is connected, it follows that $f(M) = N$. Therefore, $N$ is compact. All that remains is to show is that $f$ is a covering map. 
				
				Let $q \in N$; by continuity of $f$, $f^{-1}(q)$ is a closed subset of $M$. For each $x \in f^{-1}(q)$, there exists an open neighborhood $U_{x}$ of $x$ such that $f|_{U_{x}}$ is a diffeomorphism. Since $M$ is Hausdorff, we can shrink these neighborhoods so that they are pairwise disjoint. This means that each $x \in f^{-1}(q)$ is isolated, implying that $f^{-1}(q)$ is discrete. Since $M$ is compact, it follows that $f^{-1}(q)$ is finite; let $f^{-1}(q) = \{x_{1}, \ldots, x_{s}\}$. As stated above, for each $j= 1, \ldots, s$, we may find an open neighborhood $U'_{j}$ so that $f|U'_{j}$ is a diffeomorphism. Moreover, we can shrink these neighborhoods to obtain a pairwise disjoint collection $\{\ti{U}_{j}\}_{1}^{s}$ of neighborhoods. Set $V = \bigcap_{1}^{s}f(\ti{U}_{j})$. Then taking $U_{j} = f^{-1}(V) \cap \ti{U}_{j}$, $V$ is an evenly covered neighborhood of $p$, so that $f$ is a covering map. 
			\end{proof}
		\end{quote}
	Now assume to the contrary that there exists an immersion $f: T^{2} \to S^{2}$. By the Comps Lemma, $f$ must be a covering map. Hence, the induced homomorphism of groups $f_{\ast}: \pi_{1}(T^{2}) \to \pi_{1}(S^{2})$ must be injective. Since $S^{2}$ is simply connected, $\pi_{1}(S^{2}) \cong \{0\}$. However, $\pi_{1}(T^{2})$ is not a trivial group (in fact, $\pi_{1}(T^{2}) \cong \mathbb{Z} \times \mathbb{Z}$). This means that $f_{\ast}$ cannot be injective. Therefore, by contradiction, $f$ cannot be an immersion. Hence, there exist no immersions from $T^{2}$ to $S^{2}$. 
\end{solutions}
\begin{prb}{2023-A-II-1 (Algebra)}
	A field extension $K/L$ is called algebraic, if every element in $K$ satisfies a polynomial equation with coefficients in $L$. Let $F, K, L$ be fields such that $F \supset K \supset L$, and $F/K$ and $K/L$ are algebraic extensions. Prove that $F/L$ is also an algebraic extension. 
\end{prb}
\begin{solutions}
	Since subfields of subfields is a subfield, $L$ is a subfield of $F$. Hence, it suffices to show that every element in $F$ satisfies a polynomial equation with coefficients in $L$. Let $a \in F$, and let 
	\begin{equation}
		k(x) = k_{n}x^{n} + k_{n - 1}x^{n-1} + \dotsm + k_{0} \in K[x]
	\end{equation}
	such that $k(a) = 0$; this follows since $F/K$ is an algebraic extension. Each $k_{j} \in K$, and hence is algebraic over $L$. Therefore, $L' = L(k_{0}, \ldots, k_{n})$ is a finite extension of $L$. Since $k(a) = 0$ and $k(x)$ now has its coefficients in $L'$, it follows that $a$ is algebraic over $L'$ so that $L'(a)$ is a finite extension of $L$. Then since 
	\begin{equation}
		[L(a):L] = [L(a):L'][L':L], 
	\end{equation}
	it follows that $L(a)$ is a finite extension of $L$. Therefore, $a$ is algebraic over $L$. Since $a$ was arbitrary, $F/L$ is an algebraic extension. 
\end{solutions}
\begin{prb}{2023-A-I-2 (Geometry/Topology)}
	Let $f: T^{2} \to S^{2}$ be a smooth map from the 2-torus to the 2-sphere. Can $f$ be an immersion? If the answer is yes, given an explicit example. If the answer is no, then give a proof. 
\end{prb}
\begin{solutions}
	There cannot be an immersion $f: T^{2} \to S^{2}$. To prove our answer, we will state and proof a technical lemma. 
	\begin{quote}
		(\textbf{Comps Lemma}) Let $M, N$ be smooth, connected, $n$-manifolds and $f: M \to N$ a (smooth) immersion. If $M$ is compact and nonempty, then $f$ is a (smooth) covering map. 
		\begin{proof}
			Let $M, N$ be smooth connected $n$-manifolds, $M$ compact, and $f: M \to N$ an immersion. Since $\dim{N} = n$ everywhere and $f$ is an immersion, $df_{p}: T_{p}M \to T_{f(p)}N$ has constant rank $n$ everywhere. Hence, by the Inverse Function Theorem, $f$ is a local diffeomorphism. Let $q \in N$ so that $f^{-1}(q) \subset M$ is closed. For each $x \in f^{-1}(q)$, there exists a neighborhood $U_{x}$ such that $f|_{U_{x}}: U_{x} \to V_{x} \subset N$ is a diffeomorphism. Since $M$ is Hausdorff, we can shrink these neighborhoods so that they are pairwise disjoint. Since every $x \in f^{-1}(q)$ is now isolated, it follows that $f^{-1}(q)$ is discrete. Since $M$ is compact, we conclude that $f^{-1}(q)$ must be finite; let $f^{-1}(q) = \{x_{1}, \ldots, x_{s}\}$. As stated above, for each $j = 1, \ldots, s$, we can find a neighborhood $U_{j}$ of $x_{j}$ so that $f|_{U_{j}}: U_{j} \to V_{j} \subset N$ is a diffeomorphism. Again, since $M$ is Hausdorff, we can shrink these neigborhoods so that $U_{i} \cap U_{j} = \varnothing$ for all $i \neq j$; $f$ restricted to each of these shrunken neighborhoods remains a diffeomorphism. Now set $V = \bigcap_{1}^{s}f(U_{j})$, and define $\ti{U}_{j} \subset M$ by $\ti{U}_{j} = f^{-1}(V) \cap U_{j}$ for each $j = 1, \ldots, s$. Hence, $V$ is an evenly covered neighborhood of $q \in N$, which means $f$ is a covering map. That $f$ is surjective comes from recognizing that $f(M) = N$ due to connectedness of $N$. 
		\end{proof}
	\end{quote}
	
	Now, assume $f: T^{2} \to S^{2}$ is an immersion. Since $T^{2}, S^{2}$ are smooth, connected 2-manifolds, and $T^{2}$ is compact and nonempty, by the Comps Lemma, $f$ is a covering map. Hence, the induced homomorphism $f_{\ast}: \pi_{1}(T^{2})\to \pi_{1}(S^{2})$ is injective. Since $S^{2}$ is simply connected, $\pi_{1}(S^{2}) \cong \{0\}$. On the other hand, $\pi_{1}(T^{2}) \cong \mathbb{Z} \times \mathbb{Z}$. Since the order of $\pi_{1}(T^{2})$ is more than one, $f_{\ast}$ cannot be injective. Hence, $f$ cannot be an immersion. 
\end{solutions}

\begin{prb}{2023-A-II-5 (Geometry/Topology)}
Let $(t, x, y, z)$ be the standard coordinate system on $\mathbb{R}^{4}$, and let $\phi$ be the non-zero smooth 1-form on $\mathbb{R}^{4}$ defined by 
	\begin{equation*}
		\phi = dt + ydx + zdy. 
	\end{equation*}
Let $D$ be the 3-plane field on $\mathbb{R}^{4}$ that consists of tangent vectors $V$ such that $\phi(V) = 0$. Is $D$ Frobenius integrable? Support your answer with a proof. 
\end{prb}
\begin{solutions}
	Let $D$ be the 3-plane field on $\mathbb{R}^{4}$  defined as follows: for each $p \in \mathbb{R}^{4}$, 
		\begin{equation}
			D_{p} = \brac*{v \in T_{p}\mathbb{R}^{4}: \phi(v) = 0} \eqqcolon \ker{\phi_{p}}. 
		\end{equation}
	Hence, by the Frobenius Theorem, $D$ is Frobenius integrable if and only if $\phi \wedge d\phi = 0$. We compute: 
		\begin{equation}
			d\phi = d(dt + y\;dx + z\;dy) = d^{2}t + dy \wedge dx + dz \wedge dy = dy \wedge dx + dz \wedge dy. 
		\end{equation}
	Therefore, 
		\begin{equation}
			\phi \wedge d\phi = dt \wedge dy \wedge dx + dt \wedge dz \wedge dy + y dx \wedge dz \wedge dy. 
		\end{equation}
	Since $\phi \wedge d\phi$ is nowhere vanishing on $\mathbb{R}^{4}$, $D$ is not Frobenius integrable. 
\end{solutions}
\begin{prb}{2023-A-I-1 (Algebra)}
	Let $V$ be a $n$-dimensional vector space over a field $F$. An element $A \in \operatorname{End}{V}$ is called \textit{nilpotent}, if $A^{k} = 0$ for some $k > 1$. Prove that $A$ is nilpotent if and only if 
		\begin{equation}
			\operatorname{Tr}(\Lambda^{i}A) = 0, \quad i = 1, \ldots, n,
		\end{equation} 
	where $\Lambda^{i}A$ denotes the induced action of $A$ on the wedge product $\Lambda^{i}V$ for each $i$. 
\end{prb}
\begin{solutions}
	Let $V$ be a $n$-dimensional vector space over a field $F$, and let $A \in \operatorname{End}{V}$. Recall that $\Lambda^{i}A$, the induced action of $A$ on the wedge product $\Lambda^{i}V$, is defined to be 
		\begin{equation}
			(\Lambda^{i}A)(v_{1} \wedge \dotsm \wedge v_{i}) = Av_{1} \wedge \dotsm \wedge Av_{i}, \qquad v_{j} \in V \text{ for all } j =1, \ldots, i. 
		\end{equation}
	Over an algebraic closure of $F$, $A$ has eigenvalues $\lambda_{1}, \ldots, \lambda_{n}$. Suppose $A$ is diagonalizable, with the set of eigenvectors given by $\{v_{1}, \ldots, v_{n}\}$. Then for each $i = 1, \ldots, n$, since the collection 
		\begin{equation*}
			\brac*{v_{j_{1}} \wedge \dotsm\wedge v_{j_{i}}: 1 \leq j_{1} < \dotsm < j_{i} \leq n}
		\end{equation*}
	is a basis of $\Lambda^{i}V$, and for each $i$-tuple, $\Lambda^{i}A(v_{j_{1}} \wedge \dotsm \wedge v_{j_{i}}) = Av_{j_{1}} \wedge \dotsm \wedge Av_{j_{i}} = (\lambda_{j_{1}}\dotsm\lambda_{j_{i}})(v_{j_{1}} \wedge \dotsm \wedge v_{j_{i}})$, it follows that the eigenvalues of $\Lambda^{i}A$ are the set of all products of the form $\lambda_{j_{1}} \dotsm \lambda_{j_{i}}$ for $1 \leq j_{1} < \dotsm < j_{i} \leq n$, counting for multiplicity. Hence,
		\begin{equation}
			\operatorname{Tr}(\Lambda^{i}A) = \sum_{1 \leq j_{1} < \dotsm < j_{i} \leq n}\lambda_{j_{1}}\dotsm\lambda_{j_{i}}. 
		\end{equation}
	If $A$ is not diagonalizable, since the eigenvalues of $\Lambda^{i}A$ depend only on the eigenvalues of $A$, we may assume $A$ is in Jordan normal form. Indeed, if $A = PJP^{-1}$, then 
		\begin{equation}
			\Lambda^{i}(A) = \Lambda^{i}(PJP^{-1}) = \Lambda^{i}(P)\Lambda^{i}(J)\Lambda^{i}(P^{-1}), 
		\end{equation}
	so $\Lambda^{i}A$ and $\Lambda^{i}J$ are similar and therefore have the same eigenvalues. Thus it suffices to compute the eigenvalues of $\Lambda^{i}J$, which are exactly the products $\lambda_{j_{1}}\dotsm\lambda_{j_{i}}$ of the eigenvalues of $A$. 
	
	If $A$ is nilpotent so that $A^{k} = 0$ for some $k > 1$, then since $0 = A^{k}v = \lambda^{k}v$ for all eigenvectors $v$ of $A$, it follows that every eigenvalue of $A$ is zero. Therefore, the above expression implies that $\operatorname{Tr}(\Lambda^{i}A) = 0$ for all $i = 1, \ldots, n$. On the other hand, expanding the characteristic polynomial for $A$ is given by:
		\begin{equation}
			p_{A}(t) = \det(tI - A) = t^{n} - \operatorname{Tr}(\Lambda^{1}A)t^{n - 1} + \dotsm + (-1)^{n}\operatorname{Tr}(\Lambda^{n}A). 
		\end{equation}
	If $\operatorname{Tr}(\Lambda^{i}A)=0$ for all $i = 1, \ldots, n$, then we conclude that the characteristic polynomial of $A$ is precisely $t^{n}$. Therefore, $A$'s eigenvalues are all zero. Hence, the minimal polynomial of $A$ is of the form $t^{k}$ for some $k \leq n$. This implies that $A^{k} = 0$, and so $A$ is nilpotent. 
\end{solutions}
\begin{prb}{2023-A-II-6 (Complex Analysis)}
	Find the number of solutions (counting multiplicity) to $z^{8} - 5z^{6} + 2z^{3} - z - 1 = 0$ that lie inside the unit disk. 
\end{prb}
\begin{solutions}
	Recall Rouch{\'e}'s Formula, which states that 
		\begin{quote}
			For any two complex-valued functions $f$ and $g$ holomorphic inside some region $K$ with closed and simple contour $\partial K$, if $|g(z)| < |f(z)|$ on $\partial K$, then $f$ and $f +g$ have the same number of zeros inside $K$, where each zero is counted as many times as its multiplicity. 
		\end{quote} 
	Pick $f(z) = 5z^{6}$ and set $h(z) = z^{8} + 2z^{3} - z - 1$ so that $p(z) = z^{8} - 5z^{6} + 2z^{3} - z - 1 = h(z) - f(z)$. On the unit disk $\partial S^{1}$, we observe that 
		\begin{align}
			\begin{split}
				\abs{f(z)} &= \abs{5z^{6}} = 5 \\
				&= 1 + 2 + 1 + 1 \\
				&= |z^{8}| + 2|z^{3}| + |z| + |1| \\
				&\geq |h(z)|. 
			\end{split}
		\end{align}
	Hence, $p(z) = h(z) - f(z)$ has the same number of zeros, counting multiplicity, as $f(z)$. Since $f(z)$ has six zeros in the unit disk, we conclude that $p(z)$ must also have six zeros inside the unit disk. 
\end{solutions}





\subsection{August 2022}
\begin{prb}{A-II-I (Real Analysis)}
	Suppose $E \subset \mathbb{R}^{2}$ has positive Lebesgue area. Show that $E$ contains 3 points that form the vertices of an equilateral triangle. 
\end{prb}
\begin{solutions}
	Let $E \subset \mathbb{R}^{2}$ be a set of positive Lebesgue measure (we will denote by $m^{2}$ the Lebesgue measure on $\mathbb{R}^{2}$). Let $\{v_{1}, v_{2}\}$ be a collection of unit vectors in $\mathbb{R}^{2}$ so that the angle between $v_{1}$ and $v_{2}$ is $120^{\circ}$, and let $\beta < 1/3$. By inner regularity of the Lebesgue measure, there exists a compact set $K_{1} \subset E$ so that $m^{2}(K_{1}) > 0$. Then by \textit{outer} regularity of the Lebesgue measure, there exists an open set $U$ containing $K_{1}$ such that $m^{2}(U) \leq (1 + \beta)m^{2}(K_{1})$. 
	
	Since $K_{1}$ is compact, $d_{1} = d(K_{1}, U^{c})$ is positive; so let $R = d_{1}$, pick an arbitrary $r \in (0, R)$, and consider the set $K_{1} + rv_{1}$. $K_{1} + rv_{1}$ has to be contained within $U$ since otherwise, 
		\begin{equation}
			d(K_{1}, U^{c}) < |rv_{1}| = r < d_{1}, \text{ which is a contradiction.}
		\end{equation}
	Hence, $K_{1} \cup (K_{1} + rv_{1}) \subset U$, which means 
		\begin{equation}
			m^{2}(U) \geq m^{2}(K_{1}\cup (K_{1} + rv_{1})) = m^{2}(K_{1}) + m^{2}(K_{1} + rv_{1}) - m^{2}(K_{1} \cap (K_{1} + rv_{1})) = 2m^{2}(K_{1}) - m^{2}(K_{1} \cap (K_{1} + rv_{1})), 
		\end{equation}
	where the last equality follows from translation invariance of the Lebesgue measure. Hence, $m^{2}(K_{1} \cap (K_{1} + rv_{1})) = 2m^{2}(K_{1}) - m^{2}(U) \geq (1 -\beta)m^{2}(K_{1}) > 0$. Therefore, $K_{2} \coloneqq K_{1}\cap (K_{1} + rv_{1})$ is nonempty. Now define $K_{3} = K_{2} \cap (K_{2} + rv_{2})$. Using the same reasoning as above, we observe that $K_{3}\neq \varnothing$ and $K_{3} \subset K_{2}$. Hence, we obtain a nested sequence of sets $\varnothing \neq K_{3} \subset K_{2} \subset K_{1} \subset E$. Let $M \in K_{3}$. Since $K_{3} = K_{2} \cap (K_{2} + rv_{1})$, $N = q - rv_{2} \in K_{2}$. Likewise, $O = q - rv_{2} - rv_{1} \in K_{1}$. These three points form the vertices of a triangle. Then since 
		\begin{align}
			\begin{split}
				\norm{M - N} = r, \qquad \norm{N - O} = r, \qquad \norm{M - O} = \norm{r(v_{2} + v_{1})} = r \norm{v_{2} + v_{1}} = r. 
			\end{split}
		\end{align}
\end{solutions}




\subsection{August 2020}
\begin{prb}{2020-A-II-1 (Complex Analysis)}
	How many roots (counted with multiplicity) does the function $$g(z) = 6z^{3} + e^{z} + 1$$ have in the unit disk $|z| < 1$?
\end{prb}
\begin{solutions}
	Let $g(z) = 6z^{3} + e^{z} + 1$, which is holomorphic. Let $f(z) = 6z^{3}$ and $h(z) = e^{z} + 1$. Then on the unit circle $|z| = 1$, 
		\begin{align}
			\begin{split}
				|h(z)| &\leq |e^{z}| + 1 \leq e^{|z|} + 1 \\
				&\leq e + 1 \\
				&< 6 = 6|z|^{3} = |f(z)|. 
			\end{split}
		\end{align} 
	Hence, by Rouch{\'e}'s Formula, $g(z)$ has the same number of zeros as $f(z)$. Counting multiplicity, $f(z)$ has three solutions in the unit disk, which means that $g(z)$ also has three solutions in the unit disk. 
\end{solutions}

\newpage 
\subsection{January 2019}
\begin{prb}{2019-J-I-1 (Algebra)}
	Let $A$ and $B$ be $n\times n$ invertible matrices over complex numbers, satisfying 
		\begin{equation*}
			AB = \lambda BA \text{ for some $\lambda \in \mathbb{C}$}.
		\end{equation*}
	Prove that $A^{n}$ and $B$ commute. 
\end{prb}
\begin{solutions}
	Let $A$ and $B$ be $n\times n$ invertible matrices over complex numbers so that $AB = \lambda BA$ for some $\lambda \in \mathbb{C}$. Since $A$ is invertible, left-multiplying both sides by $A^{-1}$ yields, 
		\begin{equation}
			B = \lambda A^{-1}BA. 
		\end{equation}
	So taking the determinant, we obtain: 
		\begin{equation}
			\det{B} = \lambda^{n}\det{A^{-1}}\det{B}\det{A} = \lambda^{n}\det{A}^{-1}\det{B}\det{A} = \lambda^{n}\det{B}. 
		\end{equation}
	Since $B$ is invertible, $\det{B} \neq 0$, which means that $\lambda^{n} = 1$ (i.e., $\lambda$ is an $n\textsuperscript{th}$ root of unity). Now, we claim that for any $m \in \mathbb{N}$, $A^{m}B = \lambda^{m}BA^{m}$. By hypothesis, this claim is true for the base case $m = 1$. Suppose the claim is true for some $m\geq 1$. Then 
		\begin{equation}
			A^{m + 1}B = A(A^{m}B) = \lambda^{m}(ABA^{m}) = \lambda^{m}(\lambda BA)A^{m} = \lambda^{m + 1}BA^{m + 1}. 
		\end{equation}
	Therefore, the claim is true by induction. This implies that 
		\begin{equation}
			A^{n}B = \lambda^{n}BA^{n} = BA^{n}, 
		\end{equation}
	so that $A^{n}$ and $B$ commute. 
\end{solutions}


\begin{prb}{2019-J-II-5}
	Let $G$ be a finite group, and let $H$ be a non-normal subgroup of $G$ of index $n$. Show that if $|H|$ is divisible by a prime $p \geq n$, then $G$ is not simple. 
\end{prb}
\begin{solutions}
	Let $G$ be a finite group, $H$ a non-normal subgroup of $G$ of index $n$ such that $|H|$ is divisible by a prime $p \geq n$. Let $G$ act on the set of left cosets of $H$; this induces a group homomorphism $\varphi: G \to S_{n}$. Consider the kernel of this group action, $K = \ker{\varphi}$. \textit{If} $K = G$, then for every $g \in G$, $gHg^{-1} = H$, which implies that $H$ is a normal subgroup of $G$ -- a contradiction. Hence, $\ker{\varphi}$ is a proper normal subgroup of $G$. Likewise, $\ker{\varphi} \neq H$ since this equality also forces $H$ to be normal. All that remains is to show that $\ker{\varphi}$ is not trivial. Since $p\mid |H|$, let $P$ be a Sylow $p$-subgroup of $H$. \textcolor{red}{[!! Complete Later !!]}
\end{solutions}

\subsection{January 2017}
\begin{prb}{2017-J-I-1 (Geometry/Topology)}
	Let $\Sigma_{1}$ be a torus and let $\Sigma_{2}$ be a genus-2 surface. Show that there is no submersion from $\Sigma_{2}$ to $\Sigma_{1}$. 		
\end{prb}
\begin{solutions}
	Let $\Sigma_{1}$ be a torus and $\Sigma_{2}$ be a genus-2 surface. We begin with a second modification to the Comps Lemma. Assume to the contrary that $F$ is a submersion from $\Sigma_{2}$ to $\Sigma_{1}$. By the second modification to the Comps Lemma, $F: \Sigma_{2} \to \Sigma_{1}$ must be a $k$-sheeted covering map for some finite $k > 0$. This implies that $\chi(\Sigma_{2}) = k \cdot \chi(\Sigma_{1})$, where $\chi(\cdot) = 2 - 2g$ denotes the Euler characteristic of a closed surface of genus $g$. But this is impossible since $\chi(\Sigma_{2}) = -2 < 0 = k \cdot 0 = k \cdot \chi(\Sigma_{1})$. Hence, by contradiction, there cannot be any submersions from $\Sigma_{2}$ to $\Sigma_{1}$. 
\end{solutions}
\begin{prb}{2017-J-I-6 (Geometry/Topology)}
	Let $M$ be a smooth 4-manifold, let $\phi$ be a 3-form on $M$, and let $U\subset M$ be the open set of points where $\varphi \neq 0$. Show that $\varphi$ is closed if and only if, near any $p\in U$, one can find a smooth coordinate system $(x^{1}, x^{2}, x^{3}, x^{4})$ in which $$\varphi = dx^{1} \wedge dx^{2} \wedge dx^{3}.$$
\end{prb}
\begin{solutions}
	Assume the hypotheses of the problem. Recall that $\varphi$ is closed if and only if $d\varphi$ is identically zero. Let $p \in U$ and suppose that we can find a smooth coordinate system $(x^{1}, x^{2}, x^{3}, x^{4})$ in some neighborhood of $p$ in $U$ so that  $\varphi = dx^{1} \wedge dx^{2} \wedge dx^{3}$. Then $d\varphi_{p} = d^{2}x^{1} \wedge dx^{2} \wedge dx^{3} + \dotsm + dx^{1} \wedge dx^{2} \wedge d^{2}x^{3} = 0$. Since this is true for all $p \in U$, we conclude that $d\varphi$ is identically zero on $M$, and hence $\varphi$ is closed. 
	
	Now assume that $\varphi$ is closed, which means that $\varphi \wedge d\varphi$ is identically zero. At each point $p \in U$, define $$D_{p} = \ker{\varphi_{p}},$$ which is Frobenius integrable by our previous observation. In particular, $D_{p}$ is a 1-dimensional distribution. Since $L$ is integrable, we can find smooth coordinates $(x^{1}, \ldots, x^{4})$ near $p$ such that $D_{p} = \operatorname{span}\left\{\partial_{x^{4}}\right\}$. Since $\varphi$ annihilates $\partial_{x^{4}}$, it must be a linear combination of $dx^{1}, dx^{2}$, and $dx^{3}$. Suppose $\varphi = f dx^{1} \wedge dx^{2} \wedge dx^{3}$. Then 
		\begin{equation}
			0 = d\varphi = f_{x^{1}}dx^{1} \wedge dx^{1} \wedge \dotsm \wedge dx^{3} + f_{x^{2}}dx^{2} \wedge dx^{1} \wedge \dotsm \wedge dx^{3} + \dotsm +  f_{x^{4}} \wedge dx^{1} \wedge \dotsm \wedge dx^{4}. 
		\end{equation}
	The first  three terms are all zero. The last term is zero iff $f_{x^{4}} = 0$, which means $f = f(x^{1}, x^{2}, x^{3})$. \textcolor{red}{[!! Complete Later !!]} 
\end{solutions}


\subsection{August 2017}
\begin{prb}{2017-A-I-1 (Geometry/Topology)}
	Let $M$ be a smooth compact connected $n$-manifold (without boundary), and let $F: M \to \mathbb{R}^{n}$ be a smooth map. Does $F$ necessarily have a critical point? 
\end{prb}
\begin{solutions}
	Let $M$ be a smooth compact connected $n$-manifold (without boundary), and let $F: M \to \mathbb{R}^{n}$ be a smooth map. Suppose $F$ has no critical points, which means that $dF_{p}$ is surjective at every $p \in M$. I.e., $\operatorname{rank}{dF_{p}} = n$ for every $p \in M$. Let $F = (f_{1}, \ldots, f_{n})$, where each $f_{j}: M \to \mathbb{R}$ is a component function of $F$. Fix some $f_{j}$; since $M$ is compact, $f_{j}$ must attain a maximum or minimum at some point $p \in M$. This means that $df_{j}(p) =0$. But since $dF_{p} = (df_{1}(p), \ldots, df_{j}(p), \ldots, df_{n}(p))$, $\operatorname{rank}{dF_{p}} \neq n$, which is a contradiction. Hence, $F$ must have a critical point. 
\end{solutions}
\begin{prb}{2017-A-II-3 (Algebra)}
	Let $K$ denote the splitting field of $f(x) = x^{4} + x^{2} + 1$ over $\mathbb{Q}$. Compute the Galois group $\operatorname{Gal}(K/\mathbb{Q})$. 
\end{prb}
\begin{solutions}
	We begin by finding the splitting field of $f(x)$ over $\mathbb{Q}$. By The rational root test, we observe that $f(x)$ has no roots in $\mathbb{Q}$. Let $z = x^{2}$ so that 
	\begin{equation}
		z^{2} + z + 1 = 0 \implies z = \frac{-1 \pm \sqrt{1  - 4}}{2} = \frac{-1 \pm \sqrt{-3}}{2}. 
	\end{equation}
	Therefore, the roots of $f$ are 
		\begin{equation}
			\alpha_{1} = \sqrt{\frac{-1 + \sqrt{-3}}{2}}, \alpha_{2} = -\sqrt{\frac{-1 + \sqrt{-3}}{2}}, \alpha_{3} = \sqrt{\frac{-1 - \sqrt{-3}}{2}}, \alpha_{4} = -\sqrt{\frac{-1 - \sqrt{-3}}{2}}. 
		\end{equation}
	Here, we observe that 
		\begin{equation}
			\alpha_{1}^{2} + \beta_{1}^{2} = -1 \qquad \text{ and } \qquad \alpha_{1}^{2} - \beta_{1}^{2} = \sqrt{-3}. 
		\end{equation}
	This means that $K = \mathbb{Q}(\sqrt{-3})$ is the splitting field of $f(x)$ over $\mathbb{Q}$. Since the minimal polynomial of $\sqrt{-3}$ is of degree 2, it follows that $[\mathbb{Q}(\sqrt{-3}): \mathbb{Q}] = 2$. Hence, $|\operatorname{Gal}(K/\mathbb{Q})| = 2$. Therefore, we conclude that $\operatorname{Gal}(K/\mathbb{Q}) \cong \mathbb{Z}/2\mathbb{Z}$.  
\end{solutions}




\newpage 
\subsection{Textbook Problems}
\begin{prb}{Lee-7-5}
	Let $M$ be a smooth compact manifold. Show that there is no submersion $F: M \to \mathbb{R}^{k}$ for any $k > 0$. 
\end{prb}
\begin{solutions}
	Let $M$ be a smooth compact manifold, and assume to the contrary that there exists a submersion $F: M \to \mathbb{R}^{k}$ for some $k > 0$. Since $M$ is compact, $F$ must attain either a maximum or minimum at some point $p \in M$, which means that $dF_{p} = 0$. But this is impossible since $F$ is a submersion, which means that $\operatorname{rank}{dF_{p}} = \operatorname{dim}{\mathbb{R}^{k}} = k > 0$. Hence, by contradiction, $F$ cannot be a submersion. 
\end{solutions}


\begin{prb}{D\&F-14.6.2}
	Determine the Galois groups of the following polynomials: \vspace{-0.25cm}
		\begin{enumerate}[itemsep =-2pt,label = (\roman{*})]
			\item $x^{3} - x^{2} - 4$ 
			\item $x^{3} - 2x + 4$
			\item $x^{3} - x + 1$ 
			\item $x^{3} + x^{2} - 2x - 1$.
		\end{enumerate}
\end{prb}
\begin{solutions}
	$ $\newline \vspace{-1cm}
	\begin{enumerate}[itemsep =-2pt,label = (\alph{*})]
		\item Let $f(x) = x^{3} - x^{2} - 4$. We note that $f$ has a rational root $x = 2$ since $2^{3} - 2^{2} - 4 = 8 - 4 - 4= 0$. Using polynomial long division, we find that $f(x)$ is reducible over $\mathbb{Q}$ as the product 
			\begin{equation}
				f(x) = (x - 2)(x^{2} + x + 2). 
			\end{equation}
		By the rational root test, the quadratic factor is irreducible and has complex roots 
			\begin{equation}
				x_{1,2} = \frac{-1 \pm \sqrt{-7}}{2}. 
			\end{equation}
		Therefore, the splitting field of $f(x)$ is $\mathbb{Q}(\sqrt{-7})$, which has degree 2 since the minimal polynomial of $\sqrt{-7}$ is $x^{2} + 7$. Therefore, the Galois group $\operatorname{Gal}(\mathbb{Q}(\sqrt{-7})/\mathbb{Q})$ has order 2; hence the Galois group is $\mathbb{Z}/2\mathbb{Z}$. 
		\item Let $f(x)= x^{3} - 2x + 4$. We note that $f(x)$ has a rational root $x = -2$ since $(-2)^{3} - 2(-2) + 4 = -8 + 4 + 4 = 0$. Hence using polynomial long division, 
			\begin{equation}
				f(x) = (x + 2)(x^{2} - 2x + 2). 
			\end{equation}
		By the rational root test, $x^{2} - 2x + 2$ is irreducible over $\mathbb{Q}$ with complex roots $1 \pm i$. Therefore, the splitting field of $f(x)$ is $\mathbb{Q}(i)$, which has degree 2 since the minimal polynomial of $i$ is $x^{2} + 1$. Therefore, the Galois group $\operatorname{Gal}(\mathbb{Q}(i)/\mathbb{Q})$ has order 2; hence the Galois group is $\mathbb{Z}/2\mathbb{Z}$. 
		\item Let $f(x) = x^{3} - x + 1$; by the rational root test $f(x)$ is irreducible over $\mathbb{Q}$. However, since $f$ is already a depressed cubic, we note that its discrimant is $-4p^{3} - 27q^{2} = 4 - 27 = -23$. Since $-23$ is not a perfect square, we conclude that the Galois group is $S_{3}$. In fact, the splitting field for this cubic is $\mathbb{Q}(\alpha, \sqrt{-23})$, where $\alpha$ is a root of $x^{3} - x + 1$. 
		\item Let $f(x) = x^{3} + x^{2} - 2x - 1$; by the rational root test $f(x)$ is irreducible over $\mathbb{Q}$. Therefore, we will now depress the cubic. Let $x = y - 1/3$. Then 
			\begin{align}
				\begin{split}
					x^{3} + x^{2} - 2x - 1 &= y^{3} - \frac{7}{3}y - \frac{7}{27}. 
				\end{split}
			\end{align}
		The discriminant of the depressed cubic is, 
			\begin{equation}
				D = -4p^{3} - 27q^{2} = 4\left(\frac{7^{3}}{27}\right) - 27\left(\frac{7^{2}}{27^{2}}\right) = \frac{7^{2}}{27}\left(4 \cdot 7 - 1\right) = 7^{2}. 
			\end{equation}
		Since the discriminant is a square, we see that the Galois group of the polynomial is $A_{3}$. 
	\end{enumerate}
\end{solutions}
\begin{prb}{D\&F-14.6.4}
	Determine the Galois group of $x^{4} - 25$. 
\end{prb}
\begin{solutions}
	Let $f(x) = x^{4} - 25$. The roots of $f(x)$ are $\zeta_{4}^{0}\sqrt[4]{25}, \zeta_{4}^{1}\sqrt[4]{25}, \zeta_{4}^{2}\sqrt[4]{25}$, and $\zeta_{4}^{3}\sqrt[4]{25}$, where $\zeta_{4}$ is the primitive 4th root of unity. Here, we recall that the automorphisms in the Galois group of $f$ act transitively on the roots of $f(x)$. Hence, the Galois group of $f(x)$ must contain the automorphism that maps $\sqrt[4]{25} \mapsto -\sqrt[4]{25}$ (i.e., a reflection) and $\sqrt[4]{25} \mapsto \zeta_{4}^{j}\sqrt[4]{25}$ (i.e., a rotation). Hence, the Galois group is $D_{8}$. 
\end{solutions}

\begin{prb}{D\&F-14.6.5}
	Determine the Galois group of $x^{4} + 4$. 
\end{prb}
\begin{solutions}
	Let $f(x) = x^{4} + 4$, which is irreducible over $\mathbb{Q}$. However, the four roots of $f(x)$ are $\pm 1 \pm i$. This means that the splitting field of $f(x)$ is $\mathbb{Q}(i)$, which is a degree 2 extension over $\mathbb{Q}$. Hence, the Galois group is of order 2, which implies that the Galois group is the cyclic group $\mathbb{Z}/2\mathbb{Z}$. 
\end{solutions}


\end{document}