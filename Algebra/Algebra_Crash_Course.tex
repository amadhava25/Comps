%om
\documentclass{article}
\usepackage{../header}
\usepackage{amssymb}
\title{Om Algebra Crash Course}
\author{\me}
\date{December 31, 2025}

\begin{document}
	\maketitle
	\fullline 
	\tableofcontents
	\halfline
	\newpage 
	
\section{Quotient Groups and Homomorphisms}
\subsection{The Isomorphism Theorems}
\begin{itemize}
	\item \textbf{Thm. 16. (First Isomorphism Theorem)} If $\varphi: G \to H$ is a group homomorphism, then $\ker{\varphi} \trianglelefteq G$ and $G/\ker{\varphi} \cong \varphi(G)$. 
	
	{\color{orange}
		Let $\varphi: G \to H$ be a group homomorphism with kernel $K$. First, we will prove that the kernel is a normal subgroup. Let $g \in G$ and $k \in \ker{\varphi}$. Then 
			\begin{equation}
				\varphi(gkg^{-1}) = \varphi(g)\varphi(k)\varphi(g)^{-1} = \varphi(g)\varphi(g)^{-1} = 1 \implies gkg^{-1} \in K. 
			\end{equation}
		This implies that $gKg^{-1} \subseteq K$ for all $g \in G$, and so $K$ is a normal subgroup of $G$. Now let $\ti{\varphi}: G/\ker{\varphi} \to\varphi(G)$ as follows: $\ti{\varphi}(gK) = \varphi(g)$. First, we start by showing that $\ti{\varphi}$ is well-defined. Suppose $g_{1}K = g_{2}K$, which means that $g_{1}g_{2}^{-1} \in K$. Therefore, 
			\begin{equation}
				\varphi(g_{1}g_{2}^{-1}) = 1 \implies \varphi(g_{1}) = \varphi(g_{2}) \implies \ti{\varphi}(g_{1}K) = \ti{\varphi}(g_{2}K). 
			\end{equation}
		
		We need to show that $\ti{\varphi}$ is an isomorphism. Suppose $\ti{\varphi}(g) = \ti{\varphi}(h)$. Then $\varphi(g) = \varphi(h) \iff gh^{-1} \in K \iff gK = hK$. This proves injectivity. Now let $\varphi(g) \in \varphi(G)$. Hence clearly $gK \mapsto g$ so that $\ti{\varphi}$ is surjective. Finally, if $g_{1}K, g_{2}K \in G/K$, then 
			\begin{equation}
				\ti{\varphi}(g_{1}K \cdot g_{2}K) = \ti{\varphi}(g_{1}g_{2}K) = \varphi(g_{1}g_{2}) = \varphi(g_{1})\varphi(g_{2}) = \ti{\varphi}(g_{1}K)\ti{\varphi}(g_{2}K). 
			\end{equation}
		Hence, $\ti{\varphi}$ is indeed an isomorphism of groups. 
	}
	
	\item \textbf{Thm. 18. (Diamond Isomorphism Theorem)} Let $G$ be a group, $A$ and $B$ be subgroups of $G$, and assume that $A \leq N_{G}(B)$. Then (1) $AB$ is a subgroup of $G$, (2) $B \trianglelefteq AB$, (3) $A \cap B \trianglelefteq A$, and (4) $AB/B \cong A/A \cap B$. 
	
	{\color{blue}
		(1) Since $A \leq N_{G}(B)$, it automatically follows that $AB$ is a subgroup of $G$. (2) Since $A \leq N_{G}(B)$ and $B \leq N_{G}(B)$, then $AB \leq N_{G}(B)$, which is to say that $B$ is a normal subgroup of $AB$. (3 - 4) Consider the map $\varphi: A \to AB/B$ defined by $\varphi(a) = aB$. It is straightforward to see that $\varphi$ is a surjective group homomorphism, which means that $\varphi(A) = AB/B$. Now we will determine the kernel: 
			\begin{equation}
				A \ni a \in \ker{\varphi} \iff aB = 1B \iff a \in B. 
			\end{equation}
		Hence, $\ker{\varphi} = A \cap B$. By the first Isomorphism Theorem, $A \cap B \trianglelefteq A$ and $A/A \cap B \cong AB/B$. 
	}
	
	\item \textbf{Thm. 19. (Third Isomorphism Theorem)} Let $G$ be a group and let $H$ and $K$ be normal subgroups of $G$ with $H \leq K$. Then $K/H \trianglelefteq G/H$ and 
		\begin{equation}
			(G/H)/(K/H) \cong G/K. 
		\end{equation}
	
	{\color{orange}
		First, we will show that $K/H \trianglelefteq G/H$. Define the map $\varphi: G/H \to G/K$ by $\varphi(gH) = gK$. First, we need to show that this map is well-defined. Suppose $g_{1}H = g_{2}H$. Then $g_{1} = g_{2}h$ for some $h \in H$. Since $H \leq K$, $h \in K$, which shows that $g_{1}K = g_{2}K$. Hence, $\varphi$ is well-defined. It is straightforward to see that $\varphi$ is a surjective homomorphism. Finally, we can easily show that $\ker{\varphi} = K/H$. This means that (1) $K/H$ is a normal subgroup of $G/H$, and (2) by the First Isomorphism Theorem, $(G/H)/(K/H) \cong G/K$. 
	}
	
	\item \textbf{Thm. 20. (Lattice Isomorphism Theorem)} Let $G$ be a group and $N$ a normal subgroup of $G$. Every subgroup of $G/N$ is of the form $A/N$, where $A$ is a subgroup of $G$ containing $N$. Moreover, for all $A, B \leq G$, with $N\leq A$ and $N \leq B$, \vspace{-0.25cm}
		\begin{enumerate}[itemsep =-2pt,label = (\textbf{\arabic{*}})]
			\item $A \leq B$ if and only if $A/N \leq B/N$. 
			\item If $A \leq B$, then $\abs{B: A} = \abs{B/N: A/N}$. 
			\item $\braket{A,B}/N = \braket{A/N, B/N}$. 
			\item $(A \cap B)/N = (A/N) \cap (B/N)$. 
			\item $A \trianglelefteq G$ if and only if $A/N \trianglelefteq G/N$. 
		\end{enumerate}
		
	\item \textbf{Def. (Factoring Through)} In some of the above proofs of the isomorphism theorems, we have had to define a map $\varphi$ on quotient groups $G/N$ defined by giving the value of $\varphi$ on the coset $gN$ in terms of the representative $g$ alone. In essence, this defines a homomorphism  $\Phi$ on $G$, itself, by specifying the value of $\varphi$ at $g$. Hence, a map on a quotient group $G/N$ is well-defined if and only if $N \leq \ker{\Phi}$. In this case, we say that the homomorphism $\Phi$ \textit{factors through} $N$ and $\varphi$ is the induced homomorphism on $G/N$. Pictorially, 
		\begin{center}
			\begin{tikzcd}
				G \arrow[r, "\pi"] \arrow[dr,"\Phi"] & G/N \arrow[d, "\varphi"] \\
				& H
			\end{tikzcd}
		\end{center}
\end{itemize}
\subsubsection{Exercises}
\begin{exc}{3}
	Prove that if $H$ is a normal subgroup of $G$ of prime index $p$ then for all $K \leq G$ either \vspace{-0.25cm}
	\begin{enumerate}[itemsep =-2pt,label = \textbf{(\roman{*})}]
		\item $K \leq H$ or 
		\item $G = HK$ and $|K: K \cap H| = p$. 
	\end{enumerate}
\end{exc}
\begin{solutions}
	Let $H$ be a normal subgroup of $G$ of prime index $p$, and let $K$ be an arbitrary fixed \textit{nontrivial} subgroup of $G$. If $K \leq H$, we are done; so assume that $K$ is not a subgroup of $H$. Since $K \leq G = N_{G}(H)$, we conclude by the Second Isomorphism Theorem that $HK$ is a subgroup of $G$ and that $K \cap H$ is a normal subgroup of $K$. Hence, we have the chain $H \leq HK \leq G$. Therefore, 
		\begin{equation}
			p = |G:H| = |G:HK| \cdot |HK:H|. 
		\end{equation}
	Since $p$ is a prime, either $|G:HK| = 1$ (in which case $G = HK$), or $|HK:H| = 1 \implies HK = H \implies K \leq H$. By our hypothesis, the latter is not possible. Therefore, $G = HK$. From this, we observe that 
		\begin{equation}
			1 = \frac{|HK|}{|G|} = \frac{|H||K|}{|G||K \cap H|} = \frac{p^{-1}|K|}{|K \cap H|} \implies p = \frac{|K|}{|K \cap H|} = |K: K \cap H|. 
		\end{equation}
\end{solutions}
\begin{exc}4
	Let $C$ be a normal subgroup of the group $A$ and let $D$ be a normal subgroup of the group $B$. Prove that $(C \times D) \trianglelefteq (A \times B)$ and $(A \times B)/(C \times D) \cong (A/C) \times (B/D)$. 
\end{exc}
\begin{solutions}
	Let $C$ be a normal subgroup of the group $A$ and $D$ be a normal subgroup of the group $B$. Define the map, 
		\begin{align*}
			\begin{split}
				\varphi: &A \times B \longrightarrow C \times D \\
				&(a, b) \longmapsto (aC, bD). 
			\end{split}
		\end{align*}	
	First, we will show that $\varphi$ is a group homomorphism. Let $a_{1},a_{2} \in A$ and $b_{1}, b_{2} \in B$. Then 
		\begin{align}
			\begin{split}
				\varphi(a_{1}a_{2},b_{1}b_{2}) &= (a_{1}a_{2}C, b_{1}b_{2}D) \\
				&= (a_{1}C, b_{1}D) \cdot (a_{2}C, b_{2}D) \\
				&= \varphi(a_{1}, b_{1}) \cdot \varphi(a_{2}, b_{2}). 
			\end{split}
		\end{align}
	This confirms that $\varphi$ is a group homomorphism; $\varphi$ is clearly surjective since for any $(aC, bD) \in (A/C) \times (B/D)$, $\varphi: (a, b) \mapsto (aC, bD)$. Now we identify the kernel of this map: 
		\begin{align}
			\begin{split}
				\ker{\varphi} &= \brac*{(a, b) \in A \times B: aC = 1C \text{ and } bD = 1D} \\
				&= \brac*{(a, b) \in A \times B: a \in C \text{ and } b \in D} \\
				&= C \times D. 
			\end{split}
		\end{align}
	Hence, the conclusion proceeds from the First Isomorphism Theorem. 
\end{solutions}
\begin{exc}{9}
	Let $p$ be a prime and let $G$ be a group of order $p^{a}m$, where $p$ does not divide $m$. Assume $P$ is a subgroup of $G$ of order $p^{a}$ and $N$ is a normal subgroup of $G$ of order $p^{b}n$, where $p$ does not divide $n$. Prove that $|P \cap N| = p^{b}$ and $|PN/N| = p^{a - b}$. (The subgroup $P$ of $G$ is called a Sylow $p$-subgroup of $G$. This exercise shows that the intersection of any Sylow $p$-subgroup of $G$ with a normal subgroup $N$ is a Sylow $p$-subgroup of $N$.)
\end{exc}
\begin{solutions}
	Assume all of the given hypotheses. We have the following results: \vspace{-0.25cm} 
	\begin{enumerate}[itemsep =-2pt,label = \textbf{(\roman{*})}]
		\item since $P \leq G = N_{G}(N)$, $PN \leq G$ by the Diamond Isomorphism Theorem;
		\item $P \cap N \leq P$, which means that $|P \cap N| = p^{j}$ for some nonnegative integer $j \leq a$;
		\item $PN \leq G$ implies, by Lagrange's Theorem, that there exists a positive integer $k$ such that 
			\begin{equation}
				|G| = p^{a}m = k \cdot |PN| = k \cdot \frac{|P||N|}{|P \cap N|} = k \cdot \frac{p^{a + b}n}{p^{j}} \implies m = k \cdot p^{b - j}n. 
			\end{equation}
		This shows that $p^{b - j} \mid m$. Since $p \nmid m$, we must necessarily have $p^{b - j} = 1 \implies b - j = 0\implies j = b$. Therefore, $|P \cap N| = p^{b}$. Next, by the Diamond Isomorphism Theorem, since $P/(P \cap N) \cong PN/N$, $|PN/N| = |P|/|P \cap N| = p^{a - b}$. 
	\end{enumerate}
\end{solutions}
\subsection{Composition Series and the H\"older Program}
\begin{itemize}
	\item \textbf{Prop. 21. (Element of Prime Order)} If $G$ is a finite abelian group and $p$ is a prime dividing $\abs{G}$, then $G$ contains an element of order $p$. 
	\item \textbf{Def. (Simple Group)} A (finite or infinite) group $G$ is called \textit{simple} if $|G| > 1$ and the only normal subgroups of $G$ are $1$ and $G$. 
	
	\item \textbf{Def. (Composition Series)} In a group $G$ a sequence of subgroups 
		\begin{equation}
			1 = N_{0} \leq N_{1} \leq N_{2} \leq \dotsm \leq N_{k-1} \leq N_{k} = G
		\end{equation}
	is called a \textit{composition series} if $N_{i} \trianglelefteq N_{i + 1}$ and $N_{i + 1}/N_{i}$ is a simple group for all $0 \leq i \leq k - 1$. The quotient groups $N_{i + 1}/N_{i}$ are called \textit{composition factors} of $G$. 
	\item \textbf{Thm. 22. (Jordan-H\"older)} Let $G$ be a finite group with $G \neq 1$. Then \vspace{-0.25cm}
		\begin{enumerate}[itemsep =-2pt,label = (\textbf{\arabic{*}})]
			\item $G$ has a composition series and 
			\item the composition factors in a composition series are unique, namely, if $1 = N_{0} \leq N_{1} \leq \dotsm \leq N_{r} = G$ and $1 = M_{0} \leq M_{1} \leq \dotsm \leq M_{s} = G$ are two composition series for $G$, then $r = s$ and there is some permutation, $\pi$, of $\{1, 2, \ldots, r\}$ such that 
				\begin{equation}
					M_{\pi(i)}/M_{\pi(i) - 1} \cong N_{i}/N_{i - 1}, \quad 1 \leq i \leq r. 
				\end{equation}
		\end{enumerate}
	\item \textbf{Thm. (Feit-Thompson)} If $G$ is a simple group of odd order, then $G \cong Z_{p}$ for some prime $p$. 
	\item \textbf{Def. (Solvable Group)} A group $G$ is \textit{solvable} if there is a chain of subgroups 
		\begin{equation}
			1 = G_{0} \trianglelefteq G_{1} \trianglelefteq G_{2} \trianglelefteq \dotsm \trianglelefteq G_{s} = G
		\end{equation}
	such that $G_{i + 1}/G_{i}$ is abelian for $i = 0, 1, \ldots, s - 1$. 
	\item \textbf{Obs. (Solvability of Groups in Terms of Subgroups)} Let $G$ be a group and $N$ a normal subgroup of $G$. If $N$ and $G/N$ are solvable, then $G$ is solvable. 
	
	{\color{orange}
		Let $\overline{G} = G/N$, $1 = N_{0} \trianglelefteq N_{1} \trianglelefteq \dotsm \trianglelefteq N_{n} = N$ be a chain of subgroups of $N$ such that $N_{i + 1}/N_{i}$ is abelian for all $0 \leq i \leq n - 1$ and let $\overline{1} = \overline{G}_{0} \trianglelefteq \overline{G}_{1} \trianglelefteq \dotsm \trianglelefteq \overline{G}_{m} = \overline{G}$ be a chain of subgroups such that $\overline{G}_{i + 1}/\overline{G}_{i}$ is abelian for $0 \leq i \leq m - 1$. By the Lattice Isomorphism Theorem, there are subgroups $G_{i}$ of $G$ with $N\leq G_{i}$ such that $G_{i}/N = \overline{G}_{i}$ and $G_{i} \trianglelefteq G_{i + 1}$, $0 \leq i \leq m - 1$. By the Third Isomorphism Theorem, 
			\begin{equation}
				\overline{G}_{i + 1}/\overline{G}_{i} = (G_{i + 1}/N)/(G_{i}/N) \cong G_{i + 1}/G_{i}. 
			\end{equation}
		Hence, the chain 
			\begin{equation}
				1 = N_{0} \trianglelefteq N_{1} \trianglelefteq \dotsm \trianglelefteq N_{n} = N = G_{0} \trianglelefteq \dotsm \trianglelefteq G_{m} = G
			\end{equation}
		is a composition series for $G$, which proves that $G$ is solvable. 
	}
\end{itemize}
\subsubsection{Exercises}
\begin{exc}{1}
	Prove that if $G$ is an abelian simple group then $G \cong Z_{p}$ for some prime $p$ (do not assume $G$ is a finite group). 
\end{exc}
\begin{solutions}
	Let $G$ be a nontrivial, abelian, simple group. Since $G$ is nontrivial, it must contain some nonidentity element $x \in G$. Consider the subgroup $\braket{x}$ generated by this element. Since $G$ is abelian, $\braket{x}$ is a normal subgroup of $G$. And since $G$ is simple, $\braket{x} = G$. Therefore, $G$ is a cyclic group. 
	
	Suppose $G$ is an infinite group. Then $G \cong \mathbb{Z}$ via the isomorphism $\varphi: \mathbb{Z} \to G$ that maps $n \mapsto x^{n}$. However, $\mathbb{Z}$ is not a simple group, since for example, the subgroup $4 \mathbb{Z}$ is a proper normal subgroup of $\mathbb{Z}$. Hence, by contradiction, $G$ must be a finite group. 
	
	Assume $|G| = pm$ for some prime $p$. By Cauchy's Theorem, $G$ contains an element $y$ of order $p$; since $G$ is abelian, the subgroup $\braket{y}$ of index $m$ generated by this element is proper unless $m = 1$. But if $m = 1$, $G$ is a cyclic group of prime order $p$. Then it is easily shown that the map $\varphi: \mathbb{Z}/p\mathbb{Z} \to G$ defined by $\varphi(n) = x^{n}$ is an isomorphism. Therefore, $G$ is isomorphic to $\mathbb{Z}_{p}$ for some prime $p$. 
\end{solutions}

\begin{exc}{4}
	Use Cauchy's Theorem and induction to show that a finite abelian group has a subgroup of order $n$ for each positive divisor $n$ of its order. 
\end{exc}
\begin{solutions}
	Let $G$ be a finite abelian group of order $n$. Assume that the result holds for all groups of order less than $n$. Let $d$ be a divisor of $n$. Decompose $d$ into the product $kp$, where $p$ is some prime; by Cauchy's Theorem, $G$ contains a subgroup of order $p$. Since $G$ is finite abelian, this subgroup, $P$, is normal so that we can examine the quotient group $G/P$. Since $|G/P| < n$, the inductive hypothesis holds for this quotient group. Since $k \mid |G/P|$, by the hypothesis and the Lattice Isomorphism Theorem, there exists a subgroup $P \leq K \leq G$ such that $K/P$ has order $k$. Hence, $|K| = k|P| = kp = d$. Hence, this concludes the proof.  
\end{solutions}
\newpage 
\section{Group Actions}
\subsection{Group Actions and Permutation Representations}
\subsubsection{Exercises}
\begin{exc}{1}
	Let $G$ act on the set $A$. Prove that if $a, b \in A$ and $b = g \cdot a$ for some $g \in G$, then $G_{b} = gG_{a}g^{-1}$ ($G_{a}$ is the stabilizer of $a$). Deduce that if $G$ acts transitively on $A$, then the kernel of the action is $\bigcap_{g \in G}gG_{a}g^{-1}$. 
\end{exc}
\begin{solutions}
	Let $G$ be a group acting on the set $A$, and assume that $b = g \cdot a$ for some $g \in G$. Then 
		\begin{align}
			\begin{split}
				h \in G_{b} &\iff h \cdot b = b \iff h \cdot (g \cdot a) = (g \cdot a) \iff (g^{-1}hg) \cdot a = a \iff g^{-1}hg \in G_{a} \\
				&\iff h \in gG_{a}g^{-1}.  
			\end{split}
		\end{align}
	Now suppose that $G$ acts transitively on $A$. Fix $a \in A$; by transitivity, for each $b \in A$, there exists some $g \in G$ such that $b  = g\cdot a$. This means that for each $b \in A$, there exists some $g \in G$ such that $G_{b} = gG_{a}g^{-1}$. Now, we observe that a group element is contained in the kernel of the group action if and only if the element stabilizes every $b \in A$. Therefore, if $\alpha: G \times A \to A$ denotes the group action,
		\begin{equation}
			h \in \ker{\alpha} \iff h \in \bigcap_{b \in A}G_{b} \iff h \in \bigcap_{g \in G}gG_{a}g^{-1}. 
		\end{equation}
\end{solutions}
\begin{exc}{2}
	Let $G$ be a \textit{permutation group} on the set $A$ (i.e., $G \leq S_{A}$), let $\sigma \in G$, and let $a \in A$. Prove that $\sigma G_{a} \sigma^{-1} = G_{\sigma(a)}$. Deduce that if $G$ acts transitively on the set $A$, then 
		\begin{equation}
			\bigcap_{\sigma \in G}\sigma G_{a}\sigma^{-1} = 1. 
		\end{equation}
\end{exc}
\begin{solutions}
	Let $G$ be a permutation group on the set $A$, and let $\sigma \in G$, $a \in A$. Then 
		\begin{align}
			\begin{split}
				\tau \in G_{\sigma(a)} &\iff \tau \cdot \sigma(a) = \sigma(a) \iff (\sigma^{-1}\tau\sigma)(a) = a \iff \sigma^{-1}\tau\sigma \in G_{a} \\
				&\iff \tau \in \sigma G_{a}\sigma^{-1}. 
			\end{split}
		\end{align}
	This proves the first claim. Now assume that $A$ acts transitively on the set $A$. Fix $a \in A$; by transitivity, for every $b \in B$, there exists some $\sigma \in G$ such that $b = \sigma(a)$. But then, this implies that $G_{b} = G_{\sigma(b)} = gG_{a}g^{-1}$. Therefore, if $\alpha: G \times A \to A$ denotes the group action, then 
		\begin{equation}
			\tau \in \ker{\alpha} \iff h \in \bigcap_{b \in A}G_{b} = \bigcap_{\sigma \in G}G_{\sigma(a)3} = \bigcap_{\sigma \in G}\sigma G_{a}\sigma^{-1}. 
		\end{equation}
	On the other hand, by uniqueness of the identity in a group, it follows that the only permutation that fixes \textit{every} element of $A$ is the identity. This means that $\ker{\alpha} = 1$. Hence, the proof concludes. 
\end{solutions}
\begin{exc}{9}
	Assume $G$ acts transitively on the finite set $A$ and let $H$ be a normal subgroup of $G$. Let $\mathscr{O}_{1}, \ldots, \mathscr{O}_{r}$ be the distinct orbits of $H$ on $A$. \vspace{-0.25cm}
		\begin{enumerate}[itemsep =-2pt,label = (\textbf{\alph{*}})]
			\item Prove that $G$ permutes the sets $\mathscr{O}_{1}, \ldots, \mathscr{O}_{r}$ in the sense that for each $g \in G$ and each  $i \in \{1, \ldots, r\}$ there is a $j$ such that $g\mathscr{O}_{i} = \mathscr{O}_{j}$, where $g\mathscr{O} = \brac*{g\cdot a: a \in \mathscr{O}}$. Prove that $G$ is transitive on $\{\mathscr{O}_{1}, \ldots, \mathscr{O}_{r}\}$. Deduce that all orbits of $H$ on $A$ have the same cardinality. 
		\end{enumerate}
\end{exc}
\begin{solutions}
	\begin{enumerate}[itemsep =-2pt,label = (\alph{*})]
		\item Remember that orbits of an action are equivalence classes under the equivalence relation $b \sim a$ iff $b = h \cdot a$ for some $h \in H$. For each of the $r$ orbits of $H$ on $A$, let $a_{j} \in A$ be a representative element; that is, for each $j = 1, \ldots, r$, suppose that 
		\begin{equation}
			\mathscr{O}_{j} = \brac*{h \cdot a_{j}: h \in H}. 
		\end{equation}
		
		Since the orbits of $H$ on $A$ partition $A$, for each $i \in \{1, \ldots, r\}$ and $g \in G$, $g \cdot a_{i}$ lies in some orbit $\mathscr{O}_{j}$. We claim that $g\mathscr{O}_{i} = \mathscr{O}_{j}$. Suppose $g \cdot a_{i} = h' \cdot a_{j}$ for some $h' \in H$. Then 
		\begin{align}
			\begin{split}
				g\mathscr{O}_{i} &= \brac*{g \cdot (h \cdot a_{i}): h \in H} = \brac*{(gh) \cdot a_{i}: h \in H} \\
				&= \brac*{h'' \cdot (g \cdot a_{i}): h'' \in H} \qquad (\text{by normality of $H$ in $G$}) \\
				&= \brac*{h'' \cdot (h' \cdot a_{j}): h'' \in H} = \brac*{h \cdot a_{j}: h \in H} \\
				&= \mathscr{O}_{j}. 
			\end{split}
		\end{align}
		Hence, this concludes the proof that $G$ permutes the orbits of $H$ on $A$. Now, since $G$ acts transitively on $A$, for each pair $(i, j) \in \{1, \ldots, r\}$, there exists some $g \in G$ such that $g \cdot a_{i} = a_{j}$. By our previous observation, this implies that for each pair of orbits $(\mathscr{O}_{i}, \mathscr{O}_{j})$, there exists some $g \in G$ such that $g \mathscr{O}_{i} = \mathscr{O}_{j}$. Hence, $G$ acts transitively on the set of orbits of $H$ on $A$. Finally, given any pair $\mathscr{O}_{i}, \mathscr{O}_{j}$ of orbits of $H$ on $A$, the map $\varphi: \mathscr{O}_{i} \to \mathscr{O}_{j}$ given by $\varphi(a) = g \cdot a$ for all $a \in \mathscr{O}_{i}$ and where $g \in G$ is the group element such that $g\mathscr{O}_{i} = \mathscr{O}_{j}$ can be easily shown to be a bijection by the above reasoning. 
	\end{enumerate}
\end{solutions}
\subsection{Groups Acting on Themselves by Left Multiplication}
\begin{itemize}
	\item \textbf{Thm. 3. (Action on Set of Left Cosets)} Let $G$ be a group, $H$ a subgoup of $G$, and let $G$ act by left multiplication on the set $A$ of left cosets of $H$ in $G$. Let $\pi_{H}$ be the associated permutation representation afforded by this action. Then \vspace{-0.25cm}
		\begin{enumerate}[itemsep =-2pt,label = (\textbf{\arabic{*}})]
			\item $G$ acts transitively on $A$
			\item the stabilizer in $G$ of the point $1H \in A$ is the subgroup $H$. 
			\item the kernel of the action (i.e., the kernel of $\pi_{H}$) is $\bigcap_{x \in G}xHx^{-1}$, and $\ker{\pi_{H}}$ is the largest normal subgroup of $G$ contained in $H$. 
		\end{enumerate}
		
	{\color{orange}
		Assume the given hypotheses. \vspace{-0.25cm}
		\begin{enumerate}[itemsep =-2pt,label = (\textbf{\arabic{*}})]
			\item Let $aH, bH \in A$, where $a,b\in G$. Then $ba^{-1} \in G$. Hence, $(ba^{-1})aH = bH$. Therefore, the arbitrary cosets $aH$ and $bH$ lie in the same orbit, which proves that $G$ acts transitively on $A$.
			\item $g \in G_{1H} \iff g \cdot 1H = 1H \iff gH = 1H \iff g \in H$. Hence, $G_{1H} = H$. 
			\item By definition of $\pi_{H}$, we must have 
				\begin{align}
					\begin{split}
						\ker{\pi_{H}} &= \brac*{g \in G: gxH = xH \forall x \in G} \\
						&= \brac*{g \in G: (x^{-1}gx)H = H \forall x \in G} \\
						&= \brac*{g \in G: x^{-1}gx \in H \forall x \in G} \\
						&= \brac*{g \in G: g \in xHx^{-1} \forall x \in G} = \bigcap_{x \in G}xHx^{-1}. 
					\end{split}
				\end{align}
			Now, we need to prove that $\ker{\pi_{H}}$ is the largest normal subgroup of $G$ contained in $H$. First observe that $\ker{\pi_{H}} \trianglelefteq G$ and $\ker{\pi_{H}} \leq H$. Let $N$ be a normal subgroup of $G$ contained in $H$, which means that $N = xNx^{-1} \leq xHx^{-1}$ for all $x \in G$. Hence, 
				\begin{equation}
					N \leq \bigcap_{x \in G}xHx^{-1} = \ker{\pi_{H}}. 
				\end{equation}
		\end{enumerate}
	}
	
	\item \textbf{Cor. 4. (Cayley's Theorem)} Every group is isomorphic to a subgroup of some symmetric group. If $G$ is a group of order $n$, then $G$ is isomorphic to a subgroup of $S_{n}$. 
	
	{\color{orange}
		Let $H = 1$ and apply the preceding theorem to obtain a homomorphism of $G$ into $S_{G}$ (here, we identify the cosets of the identity subgoup with the elements of $G$). Since the kernel of this homomorphism is contained in $H = 1$, $G$ is isomorphic to its image in $S_{G}$. 
	}
	
	\item \textbf{Cor. 5. (Subgroups of Smallest Prime Index)} If $G$ is a finite group of order $n$ and $p$ is the smallest prime dividing $|G|$, then any subgroup of index $p$ is normal. 
	
	{\color{orange}
		Suppose $H \leq G$ and $|G:H| = p$. Let $\pi_{H}$ be the permutation representation afforded by multiplication on the set of left cosets of $H$ in $G$, let $K = \ker{\pi_{H}}$, and  let $|H:K| = k$. Then $|G:K| = |G:H||H:K| = pk$. Since $H$ has $p$ left cosets, $G/K$ is isomorphic to a subgroup of $S_{p}$ by the First Isomorphism Theorem. By Lagrange's Theorem, $pk = |G/K|$ divides $p!$. Therefore, $k \mid (p - 1)!$. But all of the prime divisors of $(p - 1)!$ are less than $p$, and by the minimality of $p$, every prime divisor of $k$ is greater than or equal to $p$. This forces $k = 1$ so that $H = K \trianglelefteq G$, completing the proof. 
	}
\end{itemize}
\subsubsection{Exercises}
\begin{exc}{8}
	Prove that if $H$ has finite index $n$ then there is a normal subgroup $K$ of $G$ with $K \leq H$ and $|G: K| \leq n!$. 
\end{exc}
\begin{solutions}
	Let $G$ be an arbitrary group, and $H$ a subgroup of $G$ of finite index $n$. Let $G$ act on the set $A$ of left cosets of $H$ by left multiplication, and denote the afforded permutation representation as $\pi_{H}$. Define $K = \ker{\pi_{H}} \trianglelefteq G$ such that $K \leq H$. By the First Isomorphism Theorem, $G/K$ is isomorphic to the subgroup $\pi_{H}(G) \leq S_{A}$. Since $H$ has $n$ left cosets, $|S_{A}| = n!$ so that $|\pi_{H}(G)| \mid n!$. In particular, this implies that $|\pi_{H}(G)| \leq n!$, which then implies that $|G/K| \leq n!$. 
\end{solutions}
\begin{exc}{9}
	Prove that if $p$ is a prime and $G$ is a group of order $p^{\alpha}$ for some $\alpha \in \mathbb{Z}^{+}$, then every subgroup of index $p$ is normal in $G$. Deduce that every group of order $p^{2}$ has a normal subgroup of order $p$. 
\end{exc}
\begin{solutions}
	Let $p$ be a prime and $G$ a group of order $p^{\alpha}$ for some $\alpha \in \mathbb{Z}^{+}$. Since $p$ is the smallest prime dividing the order of $G$, we conclude by Corollary 5 that any subgroup of index $p$ must be normal in $G$. Now let $G$ be a group of order $p^{2}$. \textit{If} $G$ has a subgroup of order $p$, then since $p^{2}/p = p$, the index of the subgroup is $p$; by the previous observation, this subgroup must be normal in $G$. Therefore, it suffices to show that such subgroups necessarily exist. But existence is straightforward: since $p$ divides $|G|$, then by Cauchy's Theorem, $G$ has to contain an element of order $p$. Then the subgroup generated by this element has to have order $p$, which then concludes the claim. 
\end{solutions}
\begin{exc}{10}
	Prove that every non-abelian group of order 6 has a nonnormal subgroup of order $2$. Use this to classify groups of order 6. [Produce an injective homomorphism into $S_{3}$.]
\end{exc}
\begin{solutions}
	 We argue by contradiction; let $G$ be a non-abelian group of order 6. By Cauchy's Theorem, $G$ must contain \textit{at least one} element of order 2. Considering the subgroup generated by this element, $G$ must contain a subgroup of order $2$. Assume to the contrary that every subgroup of order 2 is normal in $G$, and let $P = \{1, a\}$ be such a subgroup. By definition of normality, $gag^{-1} = a$ for all $g \in G$, which implies $ga = ag$ for all $g \in G$, which then implies that $a \in Z(G)$. I.e., $|Z(G)| \geq 2$. \vspace{-0.25cm}
	 	\begin{enumerate}[itemsep =-2pt,label = (\textbf{\roman{*}})]
	 		\item Suppose $|Z(G)| = 2$. Then $|G/Z| = 3$, which means $G/Z$ is cyclic, and so $G$ is abelian - contradicting our assumption that $|Z(G)| = 2$. 
	 		\item Suppose $|Z(G)| = 3$. Then $|G/Z| = 2$, which means $G/Z$ is cyclic, and so $G$ is abelian - contradicting our assumption that $|Z(G)| = 3$. 
	 		\item Suppose $|Z(G)| = 6$. Then $G$ is abelian, which contradicts our hypothesis that $G$ is non-abelian. 
	 	\end{enumerate}
	 Hence, we must have $|Z| = 1$, but this contradicts our hypothesis that every subgroup of order $2$ is normal. Hence, $G$ must contain a nonnormal subgroup of order 2. 
\end{solutions}
\begin{exc}{14}
	Let $G$ be a finite group of composite order $n$ with the property that $G$ has a subgroup of order $k$ for each positive integer $k$ dividing $n$. Prove that $G$ is not simple. 
\end{exc}
\begin{solutions}
	Let $G$ be a finite group of composite order $n$ with the property that $G$ has a subgroup of order $k$ for each positive integer $k$ dividing $n$. Let $n = p_{1}^{\alpha_{1}}p_{2}^{\alpha_{2}}\dotsm p_{m}^{\alpha_{m}}$ be the prime factorization of $n$, and $p_{1}$ be the smallest prime in the factorization (possibly after rearranging and renumbering). Then since $j = p_{1}^{\alpha_{1} - 1}p_{2}^{\alpha_{2}}\dotsm p_{m}^{\alpha_{m}} \mid n$, $G$ contains a subgroup $J$ of order $j$. By Lagrange's Theorem, $[G: J] = p_{1}$. Hence, by Corollary 5, this subgroup must be a proper normal, nontrivial, subgroup of $G$ which means that $G$ cannot be simple. 
\end{solutions}
\subsection{Groups Acting on Themselves by Conjugation}
\begin{itemize}
	\item \textbf{Prop. 6. (Number of Conjugates of a Subset)} Let $G$ be a group and $S$ a subset of $G$. Then the number of conjugates of $S$ is equal to $|G: N_{G}(S)| = |G: G_{S}|$. In particular, the number of conjugates of an element $s$ is the index of the centralizer of $s$, $|G: C_{G}(S)|$. 
	
	\item \textbf{Thm. 7. (Class Equation)} Let $G$ be a finite group and let $g_{1}, g_{2}, \ldots, g_{r}$ representatives of the distinct conjugacy classes of $G$ not contained in the center $Z(G)$ of $G$. Then 
		\begin{equation}
			|G| = |Z(G)| + \sum_{i = 1}^{r}|G: C_{G}(g_{i})|. 
		\end{equation}
		
	\item \textbf{Thm. 8. (Groups of Order $p^{2}$)} If $p$ is a prime and $P$ is a group of prime power $p^{\alpha}$ for some $\alpha \geq 1$, then $P$ has a nontrivial center: $Z(P) \neq 1$. 
	
	{\color{orange}
		Consider the class equation: 
			\begin{equation}
				|P| = |Z(P)| + \sum_{i = 1}^{r}|P: C_{P}(g_{i})|, 
			\end{equation}
		where $g_{1}, \ldots, g_{r}$ are representatives of the distinct non-central conjugacy classes. By definition, $C_{P}(g_{i}) \neq P$ for $i = 1, 2, \ldots, r$ so that $p \mid |P: C_{p}(g_{i})|$. Since $p \mid |P|$, it follows that $p \mid |Z(P)|$. Hence, $|Z(P)|$ cannot be trivial. 
	}
	
	\item \textbf{Thm. 9. (Groups of Order $p^{2}$)} If $|P| = p^{2}$ for some prime $p$, then $P$ is abelian. More precisely, $P$ is isomorphic to either $\mathbb{Z}_{p^{2}}$ or $\mathbb{Z}_{p} \times \mathbb{Z}_{p}$. 
	
	{\color{orange}
		By the previous theorem, $Z(P)$ is nontrivial so that $P/Z(P)$ is cyclic. Hence, $P$ is abelian. If $P$ contains an element of order $p^{2}$, then $P$ is cyclic so that $P \cong \mathbb{Z}_{p^{2}}$. So suppose that \textit{every} nontrivial element of $P$ has order $p$. Let $x, y$ be distinct nonidentity elements of $P$. Since $|\braket{x, y}| > |\braket{x}| = p$, we must have that $P = \braket{x, y}$. Since $x$ and $y$ have order $p$, $\braket{x} \times \braket{y} \cong \mathbb{Z}_{p} \times \mathbb{Z}_{p}$. Hence, the map $\varphi: (x^{a}, y^{b}) \mapsto x^{a}y^{b}$ is an isomorphism from $\braket{x} \times \braket{y} \to P$, which completes the proof. 
	}
\end{itemize}



\newpage 
\section{Direct and Semidirect Products and Abelian Groups}

\end{document}